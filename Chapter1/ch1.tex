\chapter{Introduction}
\section{Problem statement}
Due to the damaging environmental effects of using fossil fuels in the transport sector, national and international targets have been set in order to reduce global CO\textsubscript{2} emissions. In the UK for example, there is a plan to completely ban the sales of new conventional petroleum vehicles by as early as 2040. \cite{DepartmentforEnvironment2017} One proposed solution is further adoption of fuel cells and other energy generation methods which utilize hydrogen as a carbon free energy source. 

Despite the fact that the technology for hydrogen powered fuel cells, in particular proton exchange membrane fuel cells, has existed since the early 1960’s their application has been limited to providing power for space missions and other niche applications. It wasn’t until the late 90’s where developments in lowering platinum catalyst loading and the production of thin film electrodes drove the cost of fuel cells down to a level where they were a realistic option for transportation. As of 2017, a number of auto mobile manufacturers including Toyota,\cite{Toyota2015} Hyundai, \cite{Hyundai2015} Honda \cite{Honda} and Daimler \cite{Mohrdieck2014} now offer hydrogen vehicles commercially and it is becoming increasingly possible to retrofit a petroleum vehicle to run off hydrogen.\cite{FCell2016} Many countries in the EU and globally have ambitious hydrogen infrastructure plans over the next 10 years in an effort to become less reliant on importing fossil fuels, increase their energy security, and transition to a carbon free energy system.

The development of the hydrogen economy is still in its infancy in Europe, but several countries are aiming to employ sizable hydrogen fuelling infrastructures over the next few decades. National reports state that Europe’s position in 2030 will be: UK - 1,100 hydrogen refuelling stations and 1.6 million fuel cell vehicles 7, France – 600 hydrogen refuelling stations and 0.8 million fuel cell vehicles 8, Germany – 1,180 hydrogen refuelling stations and 1.8 million fuel cell vehicles 9 and the Netherlands – 200 hydrogen refuelling stations and 0.2 million fuel cell vehicles. 9 The fuel cell system in a hydrogen vehicle can easily degrade if even parts-per-billion to parts-per-million level of some impurities are present in the hydrogen. Therefore, it is imperative that hydrogen purity, and techniques for verifying the purity, are adequate to ensure customers vehicles are not inadvertently damaged by fluctuations in hydrogen composition. 

International standards dictate that it is mandatory for all hydrogen suppliers to prove that their product is pure enough to prevent degradation of fuel cell components. The international standard ISO 14687-2:2012 10 shown in Table 1 specifies the maximum impurity levels of 13 impurities that are permissible in fuel cell hydrogen. ISO 14687-2:2012 include some challenging hydrogen purity specifications mainly due to the low limits of detection of standard techniques used to measure the compounds included in the standard. 

Existing hydrogen purity laboratories are unable to perform traceable analysis to ISO 14687 specifications because appropriate methods and standards have not been developed. The consequence of this is that hydrogen suppliers cannot provide evidence that their fuel meets the International Standard and therefore are not permitted to supply hydrogen. Of the 13 gaseous impurities listed in ISO 14687-2, there is no single method for measuring all impurities. Laboratories must therefore use several instruments to perform such an analysis.  In 2015 Murugan et al published a review of methods for analysing the purity of fuel grade hydrogen 11. They concluded that in order for a single laboratory to provide full hydrogen analysis to ISO 14687-2 specifications it would need to comprise a variety of instruments including GCs, FTIR and CRDS. The capital cost of purchasing the gas analysers to perform analysis on the measurable impurities in a hydrogen sample can amount to >€500,000 11 and hence performing analysis would be out of reach for many of the smaller laboratories. 

While the impurities listed in ISO 14687-2 are specified at extremely low amount fractions, many can be analysed at higher amount fractions through the use of cheap and routine gas analysers such as GC-MS. A potential solution to this would be to increase the concentration above the limit of detection of one of these cheaper analysers. These techniques are referred to as enrichment or pre-concentration. The most commonly used technique for pre-concentration of hydrogen fuel samples is referred to as ‘Hydrogen Impurity Enrichment’.  This method involves passing the sample through a palladium or palladium alloy membrane which is heated to 400\textdegree C. Palladium as a membrane material only allows the passage of hydrogen, and as hydrogen leaves the system, the impurities remain, increasing in time as more hydrogen permeates through the membrane.  This increase in concentration is referred to as the enrichment factor and. Once the enrichment is complete the sample can then be analysed at these higher concentrations, and using the enrichment factor, the original composition of the sample can be found. 

In order for these devices to provide accurate results the behaviour of the membrane material, and its interaction with any impurities present in the hydrogen same, must be properly understood.  


\section{Research Background}
\subsection{Hydrogen Production}
Hydrogen production refers to a range of industrial processes for generating hydrogen. Since there are no natural reserves of hydrogen all hydrogen must be obtained through one of these methods. The most important factor for determining the feasibility of a hydrogen production process is the primary source of energy that is used. Currently the options for this are nuclear energy in the form of heat, renewable energy in the form of heat, electricity, or light, or fossil fuels. Currently the primary sources of hydrogen are from steam reforming of methane and other hydrocarbons which in total accounts for 96\% of global hydrogen production, with electrolysis of water accounting for the remaining 4\%.
\subsubsection{Hydrogen from fossil fuels and hydrocarbons}
Fossil fuels are the most dominating source of hydrogen production and there are a number of processes which utilize fossil fuels to produce hydrogen. The most popular and therefore the ones which will be discussed are steam methane reforming, hydrocarbon decomposition
\paragraph{Steam Methane reforming}
is the conventional and most economical method for producing hydrogen, and it has been predicated by the IEA that this trend will continue despite the emergence of other hydrogen production methods. Steam methane reforming occurs through a two-step chemical process. If another hydrocarbon other than methane is being used it must first be pre-reformed into methane as shown in equation \ref{eq:1}

\begin{equation} \label{eq:1}
    C_n H_m + H_2 O \rightarrow nCO +(\frac{n+m}{2})H_2 
\end{equation}
\begin{equation}\label{eq:2}
    CH_4 + H_2 O \rightarrow CO + 3H_2 \quad \Delta H_{298K}^o = +205 kJ/mol
\end{equation}
\begin{equation}\label{eq:3}
    CO+ H_2 O \rightarrow CO_2 + H_2 \quad \Delta H_{298K}^o = -41 kJ/mol
\end{equation}

Equation \ref{eq:2} takes place in a reactor operating at 700-850\textdegree C, at pressures of 3-25 bar, 
and in the presence of a nickel based catalyst. 
The result of this step is a mixture of CO and H\textsubscript{2}, commonly referred to as syngas. 
This syngas is used as a feedstock for the reaction shown in equation \ref{eq:3} known as water gas shift in order to produce greater hydrogen yields.  
This step is carried out in a two-step reaction. 
An initial high temperature stage at 350\textdegree C which converts majority of the syngas to CO\textsubscript{2} and hydrogen, and a final low-temperature step which operates at 250\textdegree{O} C which utilizes a catalyst with higher activity to minimise the remaining CO\textsubscript{2}. The final product will be a mixture of CO\textsubscript{2} and H\textsubscript{2}.  

A number of separation steps are utilised in order to prevent impurities from contaminating the resulting gas mixture. The traditional separation step is pressure swing adsorption (PSA) which takes advantage of adsorption of gaseous molecules onto a molecular sieve at high pressures. Hydrogen purities of ~99.9\% are achievable using this method however the cost is high and typically contributes to around 20-30\% of the total production cost. The other main separation step is desulphurization which uses a combination of CoMo and ZnO catalysts in series at 450-550\textdegree C to remove sulphur. This step is essential to ensure sulphur is not present in the gas exit stream and also to ensure catalyst poisoning does not occur at any point in the process. 

\paragraph{Hydrocarbon decomposition}
is a process by which hydrocarbon molecules are converted into solid carbon and hydrogen. This reaction is typically operated either thermally or by creating a plasma. Both methods require a metallic catalyst such as nickel or iron. The reaction is shown in equation \ref{eq:4}

\begin{equation}\label{eq:4}
    C_x H_2x+2 \rightarrow xC + (x+1)H_2
\end{equation}

An advantage of this process is that the only feedstock is the hydrocarbon, so presuming that the feedstock is sufficiently pure this method of hydrogen production should remove the needs for further downstream processing. The main disadvantage of this method is the since solid carbon is the main by-product the catalyst will easily be deactivated and will require regular maintenance.
\subsubsection{Hydrogen from water}
\paragraph{Thermal decomposition of water}
is the process of splitting water into hydrogen and oxygen at temperatures of 2000\textdegree C, this can be lowered under the presence of a nickel or iron based catalyst. Due to the high energy demand for this production method water splitting is not a feasible method of commercial hydrogen production.

\begin{equation}
    H_2 O \rightarrow H_2 + \frac{1}{2} O_2  \quad \Delta H_{298K}^o = +286 kJ/mol
\end{equation}

\paragraph{Electrolysis}
is the second most popular method for producing pure hydrogen after SMR. This method uses an electric current to split water into hydrogen and oxygen. The main competitive advantage of electrolysis is that they are modular and highly scalable, allowing hydrogen to be produced in a distributed manner. The main input to the process is electricity and if this electricity is produced using renewable sources then the process can be considered carbon neutral. This is further incentivised by the increasing price of natural gas and the decreasing price of electricity, which some predict will result in electrolysis becoming more economically feasible than SMR in the future. 
\begin{equation}
2H_2 O +  2e^- \rightarrow H_2 + 2OH^-
\end{equation}
\begin{equation}
2OH^- \rightarrow \frac{1}{2}O_2+ H_2 O + 2e^-
\end{equation}


\subsection{Hydrogen impurities in the supply chain}

\subsubsection{Effect on operation of a fuel cell}

\paragraph{Water}
generally does not affect the function of a fuel cell, however; it provides a transport mechanism for water-soluble contaminants such as K+ and Na+ to pass through the electrolyte and have a negative long-term effect on the conductivity of the cathode side of the membrane. In addition, water may increase the risk of ice formation within vehicle fuel storage and hydrogen dispensing systems under certain conditions. 

Water can be present from both SMR and electrolysis due to it being a main by-product of SMR reactions, and the main reactant in electrolysis. The PSA process used in SMR is considered an appropriate barrier to water in the end product due to the high selectivity to removing water of the molecular sieves used. When a PSA system is designed to produce an output of CO below 0.2 \textmu mol/mol, the concentration of water will be less than 0.1 \textmu mol/mol. This makes it unlikely for H\textsubscript{2}O to be present in hydrogen produced using this method.


There are three potential pathways for water to contaminate hydrogen through electrolysis. These are:
\begin{itemize}
    \item Electro-osmosis through the proton exchange membrane
    \item Hydrogen water saturated at 60\textdegree C
    \item Drier malfunction
\end{itemize}
The drier should remove most of the water from the produced hydrogen. In the event of drier failure most systems are fit with a dew point analyser that will trip, shutting off production until the issue can be fixed. 

\paragraph{Total hydrocarbon content}
Different hydrocarbons have different effects on fuel cell performance. Generally aromatic hydrocarbons adsorb more strongly on the catalyst surface than other hydrocarbons inhibiting access to hydrogen. Methane (CH\textsubscript{4}) is considered an inert constituent since its effect on fuel cell performance is to dilute the hydrogen fuel stream.

The presence of hydrocarbons are most likely to result from the SMR process. Hydrocarbons are not expected to be present at all in electrolysis. Similar to water contamination through SMR, the most likely reason for hydrocarbon contamination is due to malfunction of the PSA system used to purify the product hydrogen. A PSA system designed to deliver hydrogen with a CO concentration <0.2 \textmu mol/mol should be sufficient to reduce the amount fraction of hydrocarbons to below the 5 \textmu mol/mol required by ISO 14687. Therefore the probability of hydrocarbons being present is rare. 

\paragraph{Oxygen}
Oxygen (O2) in low concentrations does not adversely affect the function of the fuel cell system; however, it may be a concern for some on-board vehicle storage systems, for example, by reaction with metal hydride storage materials.
In SMR processes oxygen is not used as a raw material, nor is it stable during the process conditions, readily reacting with hydrogen to produce water. In addition to this the oxygen content of the feedstock to the PSA separation stage must be below a certain level for safety reasons. Therefore oxygen contamination from SMR is unlikely. 
Oxygen is a main by-product of electrolysis, although is generated at the anode side of the electrolysis stack. Likely methods of contamination are through cross over through the PEM membrane. Due to the danger of high oxygen levels in hydrogen streams most electrolysis systems are fit with an oxygen sensor that trips the system if the concentration of oxygen in the hydrogen stream surpasses 5 umol/mol. 

\paragraph{Helium, nitrogen and argon}
Inert constituents, such as helium (He), nitrogen (N2) and argon (Ar) do not adversely affect the function of fuel cell components or a fuel cell system. However, they dilute the hydrogen gas. N2 and Ar especially can affect system operation and efficiency and can also affect the accuracy of mass metering instruments for hydrogen dispensing.
Helium is not present as a feed material in any of the discussed processes, however there is also no barrier to Helium in the exit stream and therefore any helium that enters a SMR or electrolysis process will not be removed. Despite this it is unlikely that helium will be present in a hydrocarbon feedstock, or water.
Argon is similar to helium, however it is more likely for Argon to be present in natural gas. Unlike helium, the PSA step in SMR can act as a barrier for Argon, however this will depend on the specific molecular sieve used in the system.
Nitrogen is the most likely inert impurity to be present in fuel cell hydrogen, this is due to the abundance of nitrogen in the air which the system could be exposed to, and the frequency at which nitrogen is used as a functional gas in processes for purging chambers, actuating valves etc.

\paragraph{Carbon dioxide} 
does not typically affect the function of fuel cells. However, CO2 may adversely effect on board hydrogen storage systems using metal hydride alloys. With CO2, at levels very much higher than the specification, a reverse water gas shift reaction can occur under certain conditions in fuel cell systems to create carbon monoxide.
Like most other impurities CO2 is likely to be removed from the SMR process at the PSA step, with most commonly used molecular sieves being able to remove carbon dioxide during normal operation.
CO2 can be present in the water used for electrolysis although there are several interlocks to prevent it reaching the exit stream. Most electrolysis systems have a CO2 filter on the inlet and a reverse osmosis purification unit to ensure the purity of the inlet water.  An anodic separation tank which features an ion exchange resin in a closed water loop also acts as an additional barrier, and finally CO2 has a low crossover potential through the PEM membrane and therefore is unlikely to cross into the cathode side of the system.

\paragraph{Carbon monoxide}
Carbon monoxide (CO) is a severe catalyst poison that adversely affects fuel cell performance and needs to be kept at very low levels in hydrogen fuel. Although its effect can be reversed through mitigating strategies, such as material selection of membrane electrode assembly (MEA), system design and operation, the life time effects of CO on performance is a strong concern. Lower catalyst loadings are particularly susceptible to catalyst poisoning contaminants.
Carbon monoxide can be present in gas produced from SMR through PSA malfunction, which is the main barrier to CO contamination. It is unlikely for CO to be present from electrolysis.

\paragraph{Total sulfur compounds}
Sulfur containing compounds are severe catalyst poisons that at even very low levels can cause irreversible degradation of fuel cell performance. The specific sulfur compounds that are addressed are in particular: hydrogen sulfide (H2S), carbonyl sulfide (COS), carbon disulfide (CS2), methyl mercaptan (CH3SH). Lower catalyst loadings are particularly susceptible to catalyst poisoning contaminants.
Sulphur contamination is most likely to come from hydrogen produced from hydrocarbon sources. Since the SMR process also uses catalysts that are susceptible to poisoning from sulphur compounds all plants are fit with a desulphurisation unit upstream from the main process. This is designed to reduce the concentration of sulphurous compounds to <50 nmol/mol. Should the desulphurisation unit fail the catalysts used in both reforming steps will be deactivated, preventing the process from operating and will likely result in shut down of the plant. PSA also acts as a final barrier, since H2S will adsorb onto the molecular sieves more strongly than CO.
The other potential source of sulphur contamination is the potential release from any gasket materials used in the process. This can be easily prevented by ensuring only materials that do not contain sulphur are used. 
It is unlikely that sulphur contamination will arise from electrolysis. 

\paragraph{Formaldehyde and formic acid}
Formaldehyde (HCHO) and formic acid (HCOOH) have a similar effect on fuel cell performance as CO and are thus considered as reversible contaminants. The effect of HCHO and HCOOH on fuel cell performance can be more severe than that of CO due to slower recovery kinetics and their specifications are lower than that for CO. Lower catalyst loadings are particularly susceptible to catalyst poisoning contaminants.
Formaldehyde is a by-product from the reform steps in SMR and depending on the specific operating conditions of the process however PSA should act as a sufficient barrier to formaldehyde contaminating the product.
\paragraph{Ammonia}
Ammonia (NH3) causes some irreversible fuel cell performance degradation by affecting the ion exchange capacity of the ionomer of the proton exchange membrane and/or electrode.
Hydrogen could be contaminated with Ammonia either through SMR, ammonia can be a by-product of the reforming steps and should be removed by PSA. It can also be present in water used in electrolysis however the reverse osmosis step should be sufficient in removing all ammonia before it is used in the process 

\paragraph{Total halogenated compounds} 
Halogenated compounds cause irreversible performance degradation. Potential sources include chlor- alkali production processes, refrigerants used in processing, and cleaning agents.

\paragraph{Particulates}
A maximum particulate concentration is specified to ensure that filters are not clogged and/or particulates do not enter the fuel system and affect operation of valves and fuel cell stacks. A maximum particulate size diameter is not specified but should be addressed in fuelling station and/or component standards. Particulate sizes should be kept as small as possible. It is noted that a specific threshold for particulate size which causes degradation has not been made clear and it is influenced by the particulate in ambient air while sampling and refuelling process.

\section{Research Objectives}
This thesis will focus on developing hydrogen impurity enrichment as a low-cost technique for measuring the impurities in fuel grade hydrogen to ISO 14687-2 specification. This study will revolve around the membrane materials used to concentrate the impurities in hydrogen samples and will aim to determine the best material, and conditions for the hydrogen impurity enrichment device. 
The thesis aims are as follows:
\begin{itemize}
\item Identify the best material for enriching impurities based on the degree of interaction and reactivity with the impurities shown in Table 1 
\item Convert the experimental set up in to a commercially viable prototype which could be used in analytical laboratories
\item Finalise a protocol for national measurement institutions to follow when enriching a hydrogen sample. 
\item Perform full enrichment using these three conclusions on a real sample taken from a hydrogen refuelling station
\end{itemize}
In order to determine suitable enrichment material ‘Density Functional Theory (DFT) will be used to screen a number of materials for their suitability as an impurity enrichment membrane on their simulated interaction strength with ISO 14687 impurities. 
The best performing membrane materials simulated in Chapter 3 will then be synthesised in Chapter 4. The hydrogen permeability of each material under a number of ISO 14687-2 impurities will be measured to validate the simulation results and further narrow down the most suitable membrane composition. 
Following from this the best membrane will be used in Chapter 5 which will describe the design and commercialisation of the final hydrogen impurity enrichment device. The design of the enrichment device will include an uncertainty budget of the technique, automation of the device, and compliance to European standards.
Finally the new device, featuring the most suitable membrane, redesigned process, and protocols for krypton spiking will be tested using a real sample taken from a hydrogen refuelling station.


\section{Thesis structure and presentation}
This thesis consists of 6 chapters, which includes the ‘Introduction’, ‘Literature Review’, experimental chapters and ‘Conclusion’. The thesis structure is visualised in Figure 1-1. The experimental chapters address different aspects of development of hydrogen metrology techniques as described above. 

 \renewcommand{\bibname}{References}
\bibliographystyle{unsrtnat}
\bibliography{library.bib}
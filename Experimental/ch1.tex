\chapter{Experimental methods}

\section{Simulations}
Density functional theory calculations were performed using the Quantum Espresso (QE) ab initio simulation package using the generalized gradient approximation with the PW91 functional to describe electron-correlation effects. Ion-electron interactions were described using ultra-soft pseudopotentials. A plane-wave expansion with a cut-off of 233.73 eV was used in all calculations. Geometry relaxations were performed with a conjugate gradient method until the forces on all unconstrained atoms were less than 0.03 eV/A. A Monkhorst Pack mesh with 4x4x4 k-points was used for all calculations.

The supercell used contained 20 metal atoms and one gaseous molecule located on one of four available sites on the metal lattice. It was assumed that all palladium systems adopt the substitutional; random fcc structure. Metal atoms were randomly distributed among the fcc lattice in the supercell. All atoms were allowed to relax during the calculation, with the volume of the super cell fixed at the optimised volume of the super cell without adsorbed molecules. 

Geometry optimization was performed to get the lattice constant and total energy of each alloy prior to adsorption of gaseous molecules. 

\section{Membrane manufacture}

\subsection{Materials used}
\subsection{Support fabritcation}
The YSZ 3\% hollow fibre substrates with a desired micro-structure were fabricated by a fingering induced phase-inversion process, followed by high temperature sintering. A uniform ceramic suspension, with 60 wt.\% solid loading, was prepared by ball milling. After degassing, the ceramic suspension was transferred into 200 mL stainless steel syringes and extruded through a tube-in-orifice spinneret (outer diameter 3 mm, inner diameter 1.2 mm) into a coagulation bath with no air gap. An extrusion rate of 7 and 5 mL min\textsuperscript{-1} was adopted for ceramic suspension and bore fluid (15 wt.\% 1,4- dioxane in n-hexane) respectively. The formed precursor fibres were kept in deionized water for a minimum of 12 h, in order to remove the excess solvent. After being gently washed with deionized water, the precursor fibres were dried at room temperature and sintered at 1400\textdegree C in a tubular furnace (Elite, Model TSH 17/75/450).
\subsection{Membrane deposition}
\subsubsection{Electoless Plating}
Palladium silver, copper, gold, and ternary alloy compositions (expect for PdCuZr) were deposited onto the surface of the porous YSZ substrate through electroless plating. The process was performed in two steps; the first involves ‘activating’ the surface of the material intended for deposition by seeding the surface with particles of a metal with a 
higher electro positivity than the metal intended to be plated. Palladium was used due to its high electro positivity compared to other metals commonly plated through electroless deposition. This activation step is required for non-conductive supports such as ceramics and glass but is usually not required for conductive supports depending on the specific support material and intended plating layer. The ‘activation’ step is followed by the ‘plating’ step where a solution containing a metallic salt complexing agent, and stabilising agent is reduced through the use of a reducing agent, causing solid metal to be displaced from the 
solution, and due to the catalytic activity of the seeds placed in the prior step, forming a dense metal layer on the intended surface. 

Electroless plating results in a strongly adhered, dense metallic layer which can be deposited easily on a large range of morphologies.

Prior to deposition, the outer surface of the fibre was cleaned by sequential washings with a 1:1 mixture of ethanol and water for 10 min in an ultrasonic bath, and were then dried overnight at 120\textdegree C.

Preceding electroless plating, the substrates were coated at one end with a gas tight glaze and sintered at 900\textdegree C for 1 hour. Prior to deposition, the outer surface of the fibre was cleaned by sequential washings with a 1:1 mixture of ethanol and water for 10 min in an 
ultrasonic bath, and were then dried overnight at 120 °C.

\begin{table}[]
    \centering
    \caption{Compositions used for preparation of palladium based membranes on YSZ substrate through electroless plating and immersion plating}
    \label{ELP}
    \begin{tabular}{@{}ccccc@{}}
    \toprule
    \multirow{2}{*}{Compound} & \multicolumn{4}{c}{Deposited metal} \\
                              & Pd      & Ag     & Cu      & Au     \\ \midrule
    Metal Source (g/L)        &         &        &         &        \\ \midrule
    PdCl\textsubscript{2}                     & 4       & -      & -       & -      \\
    AgNO\textsubscript{3}                     & -       & 3.4    & -       & -      \\
    CuSO\textsubscript{4}                     & -       & -      & 10      & -      \\
    AuCl\textsubscript{3}                     & -       & -      & -       & 0.1    \\ \midrule
    \multicolumn{5}{c}{Stabilising agent}                           \\ \midrule
    NH\textsubscript{3}-H\textsubscript{2}O (mL/L)            & 198     & 200    & -       & -      \\
    NaOH (g/L)                & -       & -      & 8.63    & 1      \\ \midrule
    \multicolumn{5}{c}{Complexing Agent (g/L)}                      \\ \midrule
    Na\textsubscript{2}EDTA-2H\textsubscript{2}O              & 40      & 35     & -       & -      \\
    Persulfate                & -       & -      & 30      & -      \\ \midrule
    \multicolumn{5}{c}{Reducing agent (mL/L)}                       \\ \midrule
    N\textsubscript{2}H\textsubscript{4}                      & 5.6     & 4.2    & -       & -      \\
    Formaldehyde              & -       & -      & 14      & -      \\ \bottomrule
    \end{tabular}
    \end{table}

The substrates were then activated with Pd nuclei via sensitisation in an acidic SnCl\textsubscript{2} solution, followed by activation in an acidic PdCl\textsubscript{2} solution. The sensitisation/activation 
process was carried out by immersing the glazed hollow fibre substrates sequentially in five chemical baths, i.e. acidic SnCl\textsubscript{2} solution for 5 min; deionised water for 5 min, acidic PdCl\textsubscript{2} solution for 5 min; 0.01 M HCl solution for 2 min; and finally deionised water for 3 mins. 
All chemical baths were homogenised by stirring. The sensitisation/activation process was repeated for 6 cycles. The composition of each bath is shown in Table \ref{ELP}.

The substrates were then immersed in a Pd electroless plating (ELP) solution, at 60\textdegree C, in order to deposit metallic Pd layers onto the activated surface. The Pd ELP solution was prepared according to the composition presented in Table \ref{ELP} and left to stabilize for 1 h in an ultrasonic bath prior to use. The volume of Pd ELP solution was fixed at 4 mL per cm\textsubscript{2} of 
substrate surface area. The electroless plating procedure was performed twice, with a total plating time of 60 mins.

After the palladium coating the membranes were then subjected to one, or multiple, other plating steps of silver, gold, or copper. The plating time for silver was 30 minutes for one cycle and the volume of plating solution to substrate was the same as the palladium steps.

It should be noted that the deposition of gold is through immersion plating rather than electroless plating. Immersion plating is the process of applying adhering layers of nobler metals to another metal's surface by dipping the material in a heated nobler metal solution ion to produce a replacement reaction. This causes the deposition of a metallic coating on a base metal from solutions that contain coating metal. One metal is typically displaced by metal ions that have lower levels of oxidation potential, relative to the metal ion being displaced. The plating time for gold was 3 hours and the volume of plating solution to substrate was 4mL per cm\textsuperscript{3}. 

The resulting composite membranes consisting of multiple metal layers stacked were then heat treated at 500\textdegree C under an environment containing 25\% H\textsubscript{2} in Ar balance for 24 hours in order to alloy the layers into a homogenous membrane and reduce any oxides that were present on the surface.

\subsubsection{Magnetron Sputtering}
Membranes were deposited using a closed field unbalanced magnetron sputter ion plating system produced by Teer Coatings Ltd. The thin film membranes were deposited onto the YSZ 3\% hollow fibres by mounting them vertically inside the sputtering system. The system was then evacuated to 1x10\textsuperscript{-6} mBar and subjected to an ion cleaning process with Ar plasma prior to sputtering. Pd, Cu, and Zr targets (99.9\% purity) were used to sputter the chosen alloy composition onto the membranes at the target currents shown in. A bias voltage of 50 V was applied to the magnetron during deposition runs. Samples were deposited using pulsed DC, with a constant target to substrate distance and a sample rotation speed of 16 rpm. An Ar flux of 25 (SCC/m) was used during deposition. PdCu membranes in both BCC and FCC phase were manufactured through magnetron sputtering along with PdCuZr ternary alloys. 

\subsection{Materials testing}\label{MatTest}
The thickness of the plated layers was characterised by first using a Focused Ion Beam (FIB) to mill through a section of the hollow fibre to provide a flat, cross sectional surface for analysis. The thickness of the metal later was then measured using high resolution Scanning Electron Microscopy (SEM) and composition analysed using Energy Dispersive x-ray Spectrometry (EDS) on the same sample. 

The surface composition of the membrane was further characterised using X-ray Photoelectron Spectroscopy to provide a more accurate compositional analysis for the top 10 nm of the membrane, the depth which is most relevant for catalytic dissociation of hydrogen and adsorption of impurities.

Prior to the H\textsubscript{2} permeation tests, the integrity of the hollow fibre membranes was evaluated by testing the gas-tightness of the membrane under N\textsubscript{2} atmosphere, up to 10 bar and room temperature and using a gas-tightness apparatus developed in house. 10

\section{Membrane testing}
\subsection{Preparation of gas standards}
Gas standards of hydrogen were prepared gravimetrically in 10 litre cylinders (BOC, UK) in accordance with ISO 6142-1 from pure hydrogen (Air Products, UK), nitrogen (Air Products), carbon monoxide (Scott Speciality Gases, UK), methane (CK Gases, UK) and krypton (BOC, UK). Any impurities that were detected in these pure gases were quantified and these values were then incorporated into the final determination of the gas mixture compositions and uncertainties. For the purpose of this paper the gas standards that were used to perform the permeation tests will be referred to as gas mixtures, and the gas standards that were used to calibrate the analytical instruments will be referred to as calibration gas standards. Before use, the gas mixtures were verified against traceable primary reference materials using Gas Chromatography with either a pulsed helium discharge ionisation detector (PDHID) for samples not containing sulphur, and sulphur chemiluminescence detector (SCD) for sulphur containing samples.
\subsection{Membrane testing rig}
After the membranes were ensured to be gas tight, hydrogen permeation measurements were performed using the experimental apparatus shown in Figure 1. The palladium alloy composite hollow fibre membranes were sealed on to a stainless steel ¼” NPT fitting. The membrane was then placed in a Sulfinert®-treated sample vessel (Thames Restek, UK) with nominal volume of 300 cm\textsuperscript{3}. The vessel was then heated using heavy insulated heating tape (OMEGA STH051-020). The heating was controlled using the temperature of the membrane using a PID temperature controller (OMRON). The feed was supplied from a cylinder either containing BIP+ hydrogen (Air Products) for pure hydrogen permeability tests or one of two gravimetrically prepared gas mixtures for impurity testing discussed in the previous section. Prior to introducing gas to the membrane the system was evacuated down to 1 x 10\textsuperscript{-6} mbar.

The flux and permeability of the membrane were automatically calculated using software developed in house. Each membrane sample was made using the same batch of substrate and cut to the same length prior to deposition. The permeability (P)  of a dense metal membrane is given by Eqn (1) and is a function of the hydrogen flux through the membrane (J), the concentration and pressure gradient across the membrane ($P^{0.5}_{ret}-P^{0.5}_{perm}$), and the thickness of the metallic membrane (l). 

\begin{equation} \label{eq:1}
    P = \frac{J l}{P^{0.5}_{ret}-P^{0.5}_{perm}}
\end{equation}


\section{Hydrogen impurity enrichment}
\subsection{Device design}

\renewcommand{\bibname}{References}
\bibliographystyle{unsrtnat}
\bibliography{library.bib}
\chapter{Experimental methods}

\section{Simulations}
Density functional theory calculations were performed using the Quantum Espresso (QE) ab 
initio simulation package using the generalized gradient approximation with the PW91 
functional to describe electron-correlation effects. Ion-electron interactions were described 
using ultra-soft pseudopotentials. A plane-wave expansion with a cut-off of 233.73 eV was used 
in all calculations. Geometry relaxations were performed with a conjugate gradient method until 
the forces on all unconstrained atoms were less than 0.03 eV/A. A Monkhorst Pack mesh with 
4x4x4 k-points was used for all calculations.

The supercell used contained 20 metal atoms and one gaseous molecule located on one of four 
available sites on the metal lattice. It was assumed that all palladium systems adopt the 
substitutional; random fcc structure. Metal atoms were randomly distributed among the fcc 
lattice in the supercell. All atoms were allowed to relax during the calculation, with the 
volume of the super cell fixed at the optimised volume of the super cell without adsorbed 
molecules. 

Geometry optimization was performed to get the lattice constant and total energy of each 
alloy prior to adsorption of gaseous molecules. 

\section{Membrane manufacture}

\subsection{Materials used}
\subsection{Support fabritcation}
The YSZ 3\% hollow fibre substrates with a desired micro-structure were fabricated by a 
fingering induced phase-inversion process, followed by high temperature sintering. A uniform 
ceramic suspension, with 60 wt.\% solid loading, was prepared by ball milling. After degassing, 
the ceramic suspension was transferred into 200 mL stainless steel syringes and extruded 
through a tube-in-orifice spinneret (outer diameter 3 mm, inner diameter 1.2 mm) into a 
coagulation bath with no air gap. An extrusion rate of 7 and 5 mL min-1 was adopted for 
ceramic suspension and bore fluid (15 wt.\% 1,4- dioxane in n-hexane), respectively. The formed 
precursor fibres were kept in deionized water for a minimum of 12 h, in order to remove the 
excessive solvent. After being gently washed with deionized water, the precursor fibres were 
dried at room temperature and sintered at 1400\textdegree C in a tubular furnace (Elite, Model TSH 
17/75/450).
\subsection{Membrane deposition}
\subsubsection{Electoless Plating}
Palladium silver, copper, gold, and ternary alloy compositions (expect for PdCuZr) were 
deposited onto the surface of the porous YSZ substrate through electroless plating. 
The process was performed in two steps; the first involves ‘activating’ the surface of the 
material intended for deposition by seeding the surface with particles of a metal with a 
higher electro positivity than the metal intended to be plated. Normally palladium is used due 
to its high electro positivity compared to other metals commonly plated through electroless 
deposition. This activation step is required for non-conductive supports such as ceramics and 
glass but may not be required for conductive supports depending on the specific support 
material and intended plating layer. The ‘activation’ step is followed by the ‘plating’ step 
where a solution containing a metallic salt, complexing agent, and stabilising agent is 
reduced through the use of a reducing agent, causing solid metal to be displaced from the 
solution, and due to the catalytic activity of the seeds placed in the prior step, forming a 
dense metal layer on the intended surface. Electroless plating results in a strongly adhered, 
dense metallic layer which can be deposited easily on a large range of morphologies.

Prior to deposition, the outer surface of the fibre was cleaned by sequential washings with a 
1:1 mixture of ethanol and water for 10 min in an ultrasonic bath, and were then dried overnight 
at 120\textdegree C.

Preceding electroless plating, the substrates were coated at one end with a gas tight glaze 
and sintered at 900\textdegree C for 1 hour. Prior to deposition, the outer surface of the fibre was 
cleaned by sequential washings with a 1:1 mixture of ethanol and water for 10 min in an 
ultrasonic bath, and were then dried overnight at 120 °C.

The substrates were then activated with Pd nuclei via sensitisation in an acidic SnCl2 
solution, followed by activation in an acidic PdCl2 solution. The sensitisation/activation 
process was carried out by immersing the glazed hollow fibre substrates sequentially in five 
chemical baths, i.e. acidic SnCl2 solution for 5 min; deionised water for 5 min, acidic PdCl2 
solution for 5 min; 0.01 M HCl solution for 2 min; and finally deionised water for 3 mins. 
All chemical baths were homogenised by stirring. The sensitisation/activation process was 
repeated for 6 cycles. The composition of each bath is shown in Table 2.

The substrates were then immersed in a Pd electroless plating (ELP) solution, at 60\textdegree C, 
in order to deposit metallic Pd layers onto the activated surface. The Pd ELP solution was 
prepared according to the composition presented in Table 2 and left to stabilize for 1 h in an 
ultrasonic bath prior to use. The volume of Pd ELP solution was fixed at 4 mL per cm2 of 
substrate surface area. The electroless plating procedure was performed twice, with a total 
plating time of 60 mins.

After the palladium coating the membranes were then subjected to one, or multiple, other 
plating steps of silver, gold, or copper. The plating time for silver was 30 minutes for one 
cycle and the volume of plating solution to substrate was the same as the palladium steps.

It should be noted that the deposition of gold is through immersion plating rather than 
electroless plating. Immersion plating is the process of applying adhering layers of nobler 
metals to another metal's surface by dipping the material in a heated nobler metal solution 
ion to produce a replacement reaction. This causes the deposition of a metallic coating on a 
base metal from solutions that contain coating metal. One metal is typically displaced by 
metal ions that have lower levels of oxidation potential, relative to the metal ion being 
displaced. The plating time for gold was 3 hours and the volume of plating solution to 
substrate was 4mL per cm3.

The resulting composite membranes consisting of multiple metal layers stacked were then heat 
treated at 500oC under an environment containing 25\% H2 in Ar balance for 24 hours in order 
to alloy the layers into a homogenous metal membrane.

\subsection{Materials testing}

\section{Membrane testing}
\subsection{Preparation of gas standards}
\subsection{Membrane testing rig}

\section{Hydrogen impurity enrichment}
\subsection{Device design}

\renewcommand{\bibname}{References}
\bibliographystyle{unsrtnat}
\bibliography{library.bib}
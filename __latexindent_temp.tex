Automatically generated by Mendeley Desktop 1.19.4
Any changes to this file will be lost if it is regenerated by Mendeley.

BibTeX export options can be customized via Preferences -> BibTeX in Mendeley Desktop

@techreport{Summerton2015,
author = {Summerton, Phil and Billington, Sophie and Stewart, Alex and Energy, Element and Bidet, Pierre and Faure, Nathalie},
pages = {18},
title = {{Fuelling France}},
year = {2015}
}
@article{Burkhanov2011,
author = {Burkhanov, By Gennady S and Gorina, Nelli B and Kolchugina, Natalia B and Roshan, Nataliya R and Slovetsky, Dmitry I and Chistov, Evgeny M},
doi = {10.1595/147106711x540346},
file = {:Users/marc/Library/Application Support/Mendeley Desktop/Downloaded/Burkhanov et al. - 2011 - Palladium-Based Alloy Membranes for Separation of High Purity Hydrogen from Hydrogen-Containing Gas Mixtures.pdf:pdf},
isbn = {00321400
14710676},
journal = {Platinum Metals Review},
number = {1},
pages = {3--12},
title = {{Palladium-Based Alloy Membranes for Separation of High Purity Hydrogen from Hydrogen-Containing Gas Mixtures}},
volume = {55},
year = {2011}
}
@article{Tanc2019,
abstract = {Last three decades, costumers and manufacturers of automotive sector have been influenced positively by Hydrogen and fuel cells (FCs). The main goal of automakers can be pointed as minimizing the fuel consumption and exhaust emissions while improving the range limits, energy efficiency and latest technology adaptation. Therewithal, electric assisted propulsion systems added to vehicles and are called as electric vehicles (EVs). For that matter, Battery Electric Vehicles (BEVs) and hydrogen Fuel Cell Electric Vehicles (FCEVs) have become the focus of researchers and producers. In this mini foreseen review, overview of the next quarter century vision of FCEVs are expressed and discussed by the helped of previous researches and with future forecast reports. The introduction part is summarized the general approach and future expectations of FCs in detailed. Technical overview is represented for FCs and FCEVs in terms of current state of technology to foreseen expectancy. Infrastructure analysis and future aspects overview part is also discussed for sector's perspective on FCEVs. The near future perspective of the FCEVs, which is seen as the next step in EVs, is discussed in detail in the next quarter century vision. Authors concluded that, between the 2030s-2050s, hydrogen FCEVs will continue their rising demand scale under the circumstances of decreasing expensive technology; enhanced energy optimization; extended range limits and increasing hydrogen refueling stations.},
author = {Tan{\c{c}}, Bahattin and Arat, H{\"{u}}seyin Turan and Baltacıoğlu, Ertuğrul and Aydın, Kadir},
doi = {https://doi.org/10.1016/j.ijhydene.2018.10.112},
issn = {0360-3199},
journal = {International Journal of Hydrogen Energy},
keywords = {Fuel cell electric vehicle,Future prospects,Hydrogen,Hydrogen refueling stations},
number = {20},
pages = {10120--10128},
title = {{Overview of the next quarter century vision of hydrogen fuel cell electric vehicles}},
url = {http://www.sciencedirect.com/science/article/pii/S0360319918333111},
volume = {44},
year = {2019}
}
@article{Tian2016,
abstract = {Zinc oxide (ZnO) and zeolitic imidazolate framework-8 (ZIF?8) core?shell heterostructures were obtained by using the self-template strategy where ZnO nanorods not only act as the template, but also provide Zn2+ ions for the formation of ZIF?8 shell. The ZIF?8 shell was uniformly deposited to form ZnO@ZIF?8 nanorods with core?shell heterostructures at 70 °C for 24 h as the optimum reaction time by the hydrothermal synthesis. Transmission electron microscopy (TEM) images revealed that the ZnO@ZIF?8 heterostructures are composed of ZnO as core and ZIF?8 as shell. Nitrogen (N2) sorption isotherms demonstrated that the as-prepared ZnO@ZIF?8 nanorods are a typical microporous material. Additionally, the ZnO@ZIF?8 nanorods sensor exhibited distinct gas response for reducing gases with different molecule sizes. The selectivity of the ZnO@ZIF?8 nanorods sensor was obviously improved for the detection of formaldehyde owing to the limitation effect of the aperture of ZIF?8 shell. This study demonstrated that semiconductor@MOF core?shell heterostructures may be a novel way to enhance the selectivity of the gas sensing materials.},
author = {Tian, Hailin and Fan, Huiqing and Li, Mengmeng and Ma, Longtao},
doi = {10.1021/acssensors.5b00236},
file = {:Users/marc/Library/Application Support/Mendeley Desktop/Downloaded/Tian et al. - 2016 - Zeolitic Imidazolate Framework Coated ZnO Nanorods as Molecular Sieving to Improve Selectivity of Formaldehyde Gas.pdf:pdf},
issn = {23793694},
journal = {ACS Sensors},
keywords = {ZIF-8,ZnO,core-shell heterostructures,formaldehyde,gas sensor},
number = {3},
pages = {243--250},
title = {{Zeolitic Imidazolate Framework Coated ZnO Nanorods as Molecular Sieving to Improve Selectivity of Formaldehyde Gas Sensor}},
volume = {1},
year = {2016}
}
@article{Jin2016,
author = {Jin, Hua and Wollbrink, Alexander and Yao, Rui and Li, Yanshuo and Caro, Juergen and Yang, Weishen},
doi = {10.1016/j.memsci.2016.04.017},
file = {:Users/marc/Library/Application Support/Mendeley Desktop/Downloaded/Jin et al. - 2016 - A novel CAU-10-H MOF membrane for hydrogen separation under hydrothermal conditions.pdf:pdf},
isbn = {03767388},
journal = {Journal of Membrane Science},
pages = {40--46},
title = {{A novel CAU-10-H MOF membrane for hydrogen separation under hydrothermal conditions}},
volume = {513},
year = {2016}
}
@article{Yavari2012,
abstract = {Pioneering research in 2004 by Geim and Novoselov (2010 Nobel Prize winners in Physics) of the University of Manchester led to the isolation of a monolayer graphene sheet. Graphene is a single-atom-thick sheet of sp2 hybridized carbon atoms that are packed in a hexagonal honeycomb crystalline structure. Graphene is the fundamental building block of all sp2 carbon materials including single-walled carbon nanotubes, mutliwalled carbon nanotubes, and graphite and is therefore interesting from the fundamental standpoint as well as for practical applications. One of the most promising applications of graphene that has emerged so far is its utilization as an ultrasensitive chemical or gas sensor. In this article, we review some of the significant work performed with graphene and its derivatives for gas detection and provide a perspective on the challenges that need to be overcome to enable commercially viable graphene chemical sensor technologies.},
author = {Yavari, Fazel and Koratkar, Nikhil},
doi = {10.1021/jz300358t},
file = {:Users/marc/Library/Application Support/Mendeley Desktop/Downloaded/Yavari, Koratkar - 2012 - Graphene-based chemical sensors.pdf:pdf},
isbn = {1948-7185},
issn = {19487185},
journal = {Journal of Physical Chemistry Letters},
number = {13},
pages = {1746--1753},
pmid = {21995723},
title = {{Graphene-based chemical sensors}},
volume = {3},
year = {2012}
}
@article{Yukawa2003,
abstract = {The alloying effects have been investigated experimentally on the hydriding properties of vanadium at low hydrogen pressures. The PCT curves for the $\beta$ phase (V2H or VH) are measured using an electrochemical method. It is found that the logarithm of the plateau pressure for the V2H and the VH phases change almost linearly with the amount of alloying element in vanadium metal. Also, when the Ti/Cr compositional ratio is fixed at 1, the equilibrium hydrogen pressure increases for the V2H phase but decreases for the VH phase with increasing total content of Ti and Cr. However, both the V2H and the VH phases become unstable with decreasing Ti/Cr compositional ratio, so that the PCT curve shifts towards the lower H/M side and the second plateau region for the VH2phase existing at high hydrogen pressures spreads to some extent, which leads to the improvement in the effective hydrogen capacity of the alloy. {\textcopyright} 2003 Elsevier B.V. All rights reserved.},
author = {Yukawa, Hiroshi and Yamashita, Daisuke and Ito, Shigeyuki and Morinaga, Masahiko and Yamaguchi, Shu},
doi = {10.1016/S0925-8388(03)00099-9},
file = {:Users/marc/Library/Application Support/Mendeley Desktop/Downloaded/Yukawa et al. - 2003 - Compositional dependence of hydriding properties of vanadium alloys at low hydrogen pressures.pdf:pdf},
isbn = {0925-8388},
issn = {09258388},
journal = {Journal of Alloys and Compounds},
keywords = {Alloying effects,Hydride stability,Hydrogen storage materials,V2H,VH,Vanadium hydride},
pages = {45--49},
title = {{Compositional dependence of hydriding properties of vanadium alloys at low hydrogen pressures}},
volume = {356-357},
year = {2003}
}
@article{OBrien2010,
author = {O'Brien, Casey P and Howard, Bret H and Miller, James B and Morreale, Bryan D and Gellman, Andrew J},
doi = {10.1016/j.memsci.2009.11.070},
isbn = {03767388},
journal = {Journal of Membrane Science},
number = {1-2},
pages = {380--384},
title = {{Inhibition of hydrogen transport through Pd and Pd47Cu53 membranes by H2S at 350°C}},
volume = {349},
year = {2010}
}
@article{Adatoz2015,
author = {Adatoz, Elda and Avci, Ahmet K and Keskin, Seda},
doi = {10.1016/j.seppur.2015.08.020},
isbn = {13835866},
journal = {Separation and Purification Technology},
pages = {207--237},
title = {{Opportunities and challenges of MOF-based membranes in gas separations}},
volume = {152},
year = {2015}
}
@article{Zhou2015a,
abstract = {A seeding-free synthesis strategy was developed for the preparation of dense and phase-pure zeolite FAU membranes through mussel-inspired polydopamine (PDA) modification of porous Al{\textless}inf{\textgreater}2{\textless}/inf{\textgreater}O{\textless}inf{\textgreater}3{\textless}/inf{\textgreater} tubes. Zeolite FAU nutrients can be attracted and bound to the support surface via the formation of strong non-covalent and covalent chemical bonds, thus promoting the nucleation and growth of uniform, well-intergrown and phase-pure zeolite FAU membranes. The SEM and XRD characterizations demonstrate that a relative thin but dense and pure-phase zeolite FAU membrane with a thickness of about 2.3$\mu$m can be obtained on the PDA-modified Al{\textless}inf{\textgreater}2{\textless}/inf{\textgreater}O{\textless}inf{\textgreater}3{\textless}/inf{\textgreater} tube after crystallization at 75°C for 24h, and no visible cracks, pinholes or other defects are observed on the membrane layer. The zeolite FAU membrane prepared at 75°C for 24h was evaluated in single gas permeation and mixed gas separation. For binary mixtures at 50°C and 1bar, the mixture separation factors of H{\textless}inf{\textgreater}2{\textless}/inf{\textgreater}/CH{\textless}inf{\textgreater}4{\textless}/inf{\textgreater} and H{\textless}inf{\textgreater}2{\textless}/inf{\textgreater}/C{\textless}inf{\textgreater}3{\textless}/inf{\textgreater}H{\textless}inf{\textgreater}8{\textless}/inf{\textgreater} are 9.9 and 127.7, respectively, which are much higher than the corresponding Knudsen coefficients. And relative high H{\textless}inf{\textgreater}2{\textless}/inf{\textgreater} permeance of about 1.9×10{\textless}sup{\textgreater}-7{\textless}/sup{\textgreater}molm{\textless}sup{\textgreater}-2{\textless}/sup{\textgreater}s{\textless}sup{\textgreater}-1{\textless}/sup{\textgreater}Pa{\textless}sup{\textgreater}-1{\textless}/sup{\textgreater} can be obtained through the FAU membrane due to the thin layer and the relative wide pore size of 0.74nm, demonstrating a viable direction for promising application of FAU membranes in hydrogen purification and separation.},
archivePrefix = {arXiv},
arxivId = {Zhou, Chen, 2015, Facile},
author = {Zhou, Chen and Yuan, Chenfang and Zhu, Yaqiong and Caro, J{\"{u}}rgen and Huang, Aisheng},
doi = {10.1016/j.memsci.2015.07.045},
eprint = {Zhou, Chen, 2015, Facile},
file = {:Users/marc/Library/Application Support/Mendeley Desktop/Downloaded/Zhou et al. - 2015 - Facile synthesis of zeolite FAU molecular sieve membranes on bio-adhesive polydopamine modified Alinf2infOinf3inf t.pdf:pdf},
isbn = {0376-7388},
issn = {18733123},
journal = {Journal of Membrane Science},
keywords = {Gas separation,Molecular sieve membrane,Polydopamine modification,Zeolite FAU membrane},
pages = {174--181},
publisher = {Elsevier},
title = {{Facile synthesis of zeolite FAU molecular sieve membranes on bio-adhesive polydopamine modified Al{\textless}inf{\textgreater}2{\textless}/inf{\textgreater}O{\textless}inf{\textgreater}3{\textless}/inf{\textgreater} tubes}},
url = {http://dx.doi.org/10.1016/j.memsci.2015.07.045},
volume = {494},
year = {2015}
}
@article{Xu2018,
abstract = {{\textless}p{\textgreater}A kaolin modification layer, which makes the surface of the support smooth and covers the defects on the support, not only attracts Si/Al active materials to the surface of the ceramic tube, but also dissolves under alkaline conditions to assist the growth of zeolite membranes.{\textless}/p{\textgreater}},
author = {Xu, Yao Yi and Wei, Xue Ling and Liang, Shuai and Sun, Ya Li and Chao, Zi Sheng},
doi = {10.1039/c7nj04953f},
file = {:Users/marc/Library/Application Support/Mendeley Desktop/Downloaded/Xu et al. - 2018 - Synthesis of a ZSM-5NaA hybrid zeolite membrane using kaolin as a modification layer.pdf:pdf},
isbn = {8673188713257},
issn = {13699261},
journal = {New Journal of Chemistry},
number = {9},
pages = {6664--6672},
publisher = {Royal Society of Chemistry},
title = {{Synthesis of a ZSM-5/NaA hybrid zeolite membrane using kaolin as a modification layer}},
volume = {42},
year = {2018}
}
@phdthesis{Rodtaz2003,
address = {Switzerland},
author = {Rodtaz, Paul Hendrik},
booktitle = {Institute of Energy Technology, Measurement and Control Laboratory},
file = {:Users/marc/Library/Application Support/Mendeley Desktop/Downloaded/Rodtaz - 2003 - Dynamics of the Polymer Electrolyte Fuel Cell Experiments and Model-Based Analysis.pdf:pdf},
number = {15320},
publisher = {ETH Zurich},
title = {{Dynamics of the Polymer Electrolyte Fuel Cell: Experiments and Model-Based Analysis}},
volume = {Doctor of },
year = {2003}
}
@article{Ambrosetti2014,
abstract = {Thanks to its single atom thickness and its mechanical strength, nanoporous graphene is currently being regarded as a promising candidate for efficient and reliable gas separation applications. Clearly, the accurate energetic characterization of the penetration processes involving relevant gas-phase molecules is a fundamental prerequisite for any possible application. Here we evaluate permeation barriers and adsorption energies of the H2O, CH4, CO, CO2, O2, and H2 molecules and of the Ar atom on two types of hydrogen saturated pores by means of ab initio simulations, based on the density functional theory (DFT), able to include dispersion corrections too. We find that, although the qualitative trend followed by the values of the permeation barriers of the considered molecules is independent of the adopted DFT functional, at a quantitative level the results are noticeably affected by the dispersion corrections and the chosen exchange contribution characterizing the different functionals, as well as by the allowed graphene sheet distortions. Interestingly, we observe that, due to the occurrence of nontrivial H-bond interactions with the pore-saturating H atoms, the permeation barrier of water remains low even considering a small-size pore. The barrier is further diminished when considering the interaction with a second water molecule on the opposite side of the pore. These observations, combined with the relatively strong binding of the water molecule with the defected surface, suggests that porous graphene could also represent a promising membrane for water filtration.$\backslash$nThanks to its single atom thickness and its mechanical strength, nanoporous graphene is currently being regarded as a promising candidate for efficient and reliable gas separation applications. Clearly, the accurate energetic characterization of the penetration processes involving relevant gas-phase molecules is a fundamental prerequisite for any possible application. Here we evaluate permeation barriers and adsorption energies of the H2O, CH4, CO, CO2, O2, and H2 molecules and of the Ar atom on two types of hydrogen saturated pores by means of ab initio simulations, based on the density functional theory (DFT), able to include dispersion corrections too. We find that, although the qualitative trend followed by the values of the permeation barriers of the considered molecules is independent of the adopted DFT functional, at a quantitative level the results are noticeably affected by the dispersion corrections and the chosen exchange contribution characterizing the different functionals, as well as by the allowed graphene sheet distortions. Interestingly, we observe that, due to the occurrence of nontrivial H-bond interactions with the pore-saturating H atoms, the permeation barrier of water remains low even considering a small-size pore. The barrier is further diminished when considering the interaction with a second water molecule on the opposite side of the pore. These observations, combined with the relatively strong binding of the water molecule with the defected surface, suggests that porous graphene could also represent a promising membrane for water filtration.},
author = {Ambrosetti, Alberto and Silvestrelli, Pier Luigi},
doi = {10.1021/jp504914u},
file = {:Users/marc/Library/Application Support/Mendeley Desktop/Downloaded/Ambrosetti, Silvestrelli - 2014 - Gas separation in nanoporous graphene from first principle calculations.pdf:pdf},
isbn = {1932-7447},
issn = {19327455},
journal = {Journal of Physical Chemistry C},
number = {33},
pages = {19172--19179},
title = {{Gas separation in nanoporous graphene from first principle calculations}},
volume = {118},
year = {2014}
}
@article{Xu1994,
author = {Xu, J. and Sun, X. K. and Liu, Q. Q. and Chen, W. X.},
doi = {10.1007/BF02651595},
file = {:Users/marc/Library/Application Support/Mendeley Desktop/Downloaded/Xu et al. - 1994 - Hydrogen permeation behavior in IN718 and GH761 superalloys.pdf:pdf},
issn = {10735623},
journal = {Metallurgical and Materials Transactions A},
number = {3},
pages = {539--544},
title = {{Hydrogen permeation behavior in IN718 and GH761 superalloys}},
volume = {25},
year = {1994}
}
@article{Koenig2012,
abstract = {Membranes act as selective barriers and play an important role in processes such as cellular compartmentalization and industrial-scale chemical and gas purification. The ideal membrane should be as thin as possible to maximize flux, mechanically robust to prevent fracture, and have well-defined pore sizes to increase selectivity. Graphene is an excellent starting point for developing size-selective membranes because of its atomic thickness, high mechanical strength, relative inertness and impermeability to all standard gases. However, pores that can exclude larger molecules but allow smaller molecules to pass through would have to be introduced into the material. Here, we show that ultraviolet-induced oxidative etching can create pores in micrometre-sized graphene membranes, and the resulting membranes can be used as molecular sieves. A pressurized blister test and mechanical resonance are used to measure the transport of a range of gases (H(2), CO(2), Ar, N(2), CH(4) and SF(6)) through the pores. The experimentally measured leak rate, separation factors and Raman spectrum agree well with models based on effusion through a small number of angstrom-sized pores.},
annote = {Koenig, Steven P
Wang, Luda
Pellegrino, John
Bunch, J Scott
ENG
Research Support, Non-U.S. Gov't
Research Support, U.S. Gov't, Non-P.H.S.
England
2012/10/09 06:00
Nat Nanotechnol. 2012 Nov;7(11):728-32. doi: 10.1038/nnano.2012.162. Epub 2012 Oct 7.},
author = {Koenig, S P and Wang, L and Pellegrino, J and Bunch, J S},
doi = {10.1038/nnano.2012.162},
isbn = {1748-3395 (Electronic)
1748-3387 (Linking)},
journal = {Nat Nanotechnol},
keywords = {*Membranes, Artificial,Gases/*isolation {\&} purification,Graphite/*chemistry,Oxidation-Reduction,Porosity},
number = {11},
pages = {728--732},
pmid = {23042491},
title = {{Selective molecular sieving through porous graphene}},
url = {http://www.ncbi.nlm.nih.gov/pubmed/23042491},
volume = {7},
year = {2012}
}
@article{Toda2015,
abstract = {This paper is a review of the recent progress on gas sensors using graphene oxide (GO). GO is not a new material but its unique features have recently been of interest for gas sensing applications, and not just as an intermediate for reduced graphene oxide (RGO). Graphene and RGO have been well known gas-sensing materials, but GO is also an attractive sensing material that has been well studied these last few years. The functional groups on GO nanosheets play important roles in adsorbing gas molecules, and the electric or optical properties of GO materials change with exposure to certain gases. Addition of metal nanoparticles and metal oxide nanocomposites is an effective way to make GO materials selective and sensitive to analyte gases. In this paper, several applications of GO based sensors are summarized for detection of water vapor, NO{\textless}inf{\textgreater}2{\textless}/inf{\textgreater}, H{\textless}inf{\textgreater}2{\textless}/inf{\textgreater}, NH{\textless}inf{\textgreater}3{\textless}/inf{\textgreater}, H{\textless}inf{\textgreater}2{\textless}/inf{\textgreater}S, and organic vapors. Also binding energies of gas molecules onto graphene and the oxygenous functional groups are summarized, and problems and possible solutions are discussed for the GO-based gas sensors.},
author = {Toda, Kei and Furue, Ryo and Hayami, Shinya},
doi = {10.1016/j.aca.2015.02.002},
file = {:Users/marc/Library/Application Support/Mendeley Desktop/Downloaded/Toda, Furue, Hayami - 2015 - Recent progress in applications of graphene oxide for gas sensing A review.pdf:pdf},
isbn = {0003-2670},
issn = {18734324},
journal = {Analytica Chimica Acta},
keywords = {Applications to hazardous gas sensing,Gas molecule binding energy,Gas sensor,Graphene oxide,Reduced graphene oxide},
pages = {43--53},
pmid = {26002325},
publisher = {Elsevier B.V.},
title = {{Recent progress in applications of graphene oxide for gas sensing: A review}},
url = {http://dx.doi.org/10.1016/j.aca.2015.02.002},
volume = {878},
year = {2015}
}
@article{Wang2015,
abstract = {In this study, water stable zirconium metal-organic framework (UiO-66) has been synthesized and for the first time applied as an adsorbent to remove aquatic arsenic contamination. The as-synthesized UiO-66 adsorbent functions excellently across a broad pH range of 1 to 10, and achieves a remarkable arsenate uptake capacity of 303 mg/g at the optimal pH, i.e., pH = 2. To the best of our knowledge, this is the highest arsenate As(V) adsorption capacity ever reported, much higher than that of currently available adsorbents (5-280 mg/g, generally less than 100 mg/g). The superior arsenic uptake performance of UiO-66 adsorbent could be attributed to the highly porous crystalline structure containing zirconium oxide clusters, which provides a large contact area and plenty of active sites in unit space. Two binding sites within the adsorbent framework are proposed for arsenic species, i.e., hydroxyl group and benzenedicarboxylate ligand. At equilibrium, seven equivalent arsenic species can be captured by one Zr6 cluster through the formation of Zr-O-As coordination bonds.},
annote = {Wang, Chenghong
Liu, Xinlei
Chen, J Paul
Li, Kang
ENG
Research Support, Non-U.S. Gov't
England
2015/11/13 06:00
Sci Rep. 2015 Nov 12;5:16613. doi: 10.1038/srep16613.},
author = {Wang, C and Liu, X and Chen, J P and Li, K},
doi = {10.1038/srep16613},
isbn = {2045-2322 (Electronic)
2045-2322 (Linking)},
journal = {Sci Rep},
keywords = {*Water Pollutants, Chemical,Adsorption,Arsenic/*chemistry,Organic Chemicals/*chemistry,Temperature,Water Purification/*methods,Zirconium/*chemistry},
pages = {16613},
pmid = {26559001},
title = {{Superior removal of arsenic from water with zirconium metal-organic framework UiO-66}},
url = {http://www.ncbi.nlm.nih.gov/pubmed/26559001},
volume = {5},
year = {2015}
}
@article{Bai2016a,
abstract = {We prepared a heterostructure hybrid of PANI and SnO2 by a rapid and facile in situ chemical oxidation polymerization under very low monomer concentration, and the hybrid was loaded on a flexible PET thin film to structure a smart NH3 sensor. The structure, morphology and thermal stability of the hybrid were characterized by various analysis methods. Compared with those reported in literatures, the hybrid-based sensor not only has high sensitivity, good selectivity and wide linear response to NH3 at room temperature of 21 ??C but also has flexible, structure simple and wearable performance. The room temperature operating of the sensor is particularly interesting, which leads to low-power consumption, environmental safety and long life times of the sensing materials. The improvement of sensing properties is attributed to the complementary and synergistic effect between SnO2 and PANI, and formation of p-n heterojunction at the interface in hybrid.},
author = {Bai, Shouli and Tian, Yanli and Cui, Meng and Sun, Jianhua and Tian, Ye and Luo, Ruixian and Chen, Aifan and Li, Dianqing},
doi = {10.1016/j.snb.2015.12.007},
file = {:Users/marc/Library/Application Support/Mendeley Desktop/Downloaded/Bai et al. - 2016 - Polyaniline@SnO2 heterojunction loading on flexible PET thin film for detection of NH3 at room temperature.pdf:pdf},
issn = {09254005},
journal = {Sensors and Actuators, B: Chemical},
keywords = {NH3 sensor,Polyaniline,Polyethylene terephthalate substrate,SnO2,p-n junction},
pages = {540--547},
publisher = {Elsevier B.V.},
title = {{Polyaniline@SnO2 heterojunction loading on flexible PET thin film for detection of NH3 at room temperature}},
url = {http://dx.doi.org/10.1016/j.snb.2015.12.007},
volume = {226},
year = {2016}
}
@article{Zuo2006a,
author = {Zuo, Chendong and Lee, T H and Dorris, S E and Balachandran, U and Liu, Meilin},
doi = {10.1016/j.jpowsour.2005.12.042},
isbn = {03787753},
journal = {Journal of Power Sources},
number = {2},
pages = {1291--1295},
title = {{Composite Ni–Ba(Zr0.1Ce0.7Y0.2)O3 membrane for hydrogen separation}},
volume = {159},
year = {2006}
}
@article{Lewis2014,
author = {Lewis, Amanda E and Zhao, Hongbin and Syed, Haseeba and Wolden, Colin A and Way, J Douglas},
doi = {10.1016/j.memsci.2014.04.022},
isbn = {03767388},
journal = {Journal of Membrane Science},
pages = {167--176},
title = {{PdAu and PdAuAg composite membranes for hydrogen separation from synthetic water-gas shift streams containing hydrogen sulfide}},
volume = {465},
year = {2014}
}
@article{Sjoberg2013,
abstract = {Membrane separation of CO2from synthesis gas could be an energy efficient and simple alternative to other separation techniques. In this work, a membrane comprised of an about 0.7$\mu$m thick MFI film on a graded alumina support was used to separate CO2from synthesis gas produced by pilot scale gasification of black liquor. The separation of CO2from the synthesis gas was carried out at a feed pressure of 2.25MPa, a permeate pressure of 0.3MPa and room temperature. In the beginning of the experiment, when the H2S concentration in the feed was 0.5{\%} and the concentration of water in the feed was 0.07{\%}, a CO2/H2separation factor of 10.4 and a CO2flux of 67.0kgm-2h-1were observed. However, as the H2S concentration in the feed to the membrane increased to 1.7{\%}, the CO2/H2separation factor and the CO2flux decreased to 5 and 61.4kgm-2h-1, respectively. The results suggest that MFI membranes are promising candidates for the separation of CO2from synthesis gas. {\textcopyright} 2013.},
author = {Sj{\"{o}}berg, Erik and Sandstr{\"{o}}m, Linda and {\"{O}}hrman, Olov G W and Hedlund, Jonas},
doi = {10.1016/j.memsci.2013.05.008},
isbn = {0376-7388},
issn = {03767388},
journal = {Journal of Membrane Science},
keywords = {Black liquor gasification,Carbon dioxide,Hydrogen sulphide,Separation,Zeolite membrane},
pages = {131--137},
publisher = {Elsevier},
title = {{Separation of CO2 from black liquor derived syngas using an MFI membrane}},
url = {http://dx.doi.org/10.1016/j.memsci.2013.05.008},
volume = {443},
year = {2013}
}
@article{Jeon2011,
author = {Jeon, S Y and Lim, D K and Choi, M B and Wachsman, E D and Song, S J},
doi = {10.1016/j.seppur.2011.03.018},
file = {:Users/marc/Library/Application Support/Mendeley Desktop/Downloaded/Jeon et al. - 2011 - Hydrogen separation by Pd–CaZr0.9Y0.1O3−$\delta$ cermet composite membranes.pdf:pdf},
isbn = {13835866},
journal = {Separation and Purification Technology},
number = {3},
pages = {337--341},
title = {{Hydrogen separation by Pd–CaZr0.9Y0.1O3−$\delta$ cermet composite membranes}},
volume = {79},
year = {2011}
}
@article{Shen2009,
author = {Shen, Qiang and Hou, Ming and Liang, Dong and Zhou, Zhimin and Li, Xiaojin and Shao, Zhigang and Yi, Baolian},
doi = {10.1016/j.jpowsour.2008.12.075},
isbn = {03787753},
journal = {Journal of Power Sources},
number = {2},
pages = {1114--1119},
title = {{Study on the processes of start-up and shutdown in proton exchange membrane fuel cells}},
volume = {189},
year = {2009}
}
@article{Cardoso2017,
author = {Cardoso, Sim{\~{a}}o P and Azenha, Ivo S and Lin, Zhi and Portugal, In{\^{e}}s and Rodrigues, Al{\'{i}}rio E and Silva, Carlos M},
doi = {10.1080/15422119.2017.1383917},
file = {:Users/marc/Library/Application Support/Mendeley Desktop/Downloaded/Cardoso et al. - 2017 - Inorganic Membranes for Hydrogen Separation.pdf:pdf},
issn = {1542-2119},
journal = {Separation {\&} Purification Reviews},
number = {3},
pages = {1--38},
publisher = {Taylor {\&} Francis},
title = {{Inorganic Membranes for Hydrogen Separation}},
url = {https://www.tandfonline.com/doi/full/10.1080/15422119.2017.1383917},
volume = {47},
year = {2017}
}
@article{Chong2015,
abstract = {Graphene oxide (GO) membranes have demonstrated great potential in gas separation and liquid filtration. For upscale applications, GO membranes in a hollow fibre geometry are of particular interest due to the high-efficiency and easy-assembly features at module level. However, GO membranes were found unstable in dry state on ceramic hollow fibre substrates, mainly due to the drying-related shrinkage, which has limited the applications and post-treatments of GO membranes. We demonstrate here that GO hollow fibre membranes can be stabilised by using a porous poly(methyl methacrylate) (PMMA) sacrificial layer, which creates a space between the hollow fibre substrate and the GO membrane thus allowing stress-free shrinkage. Defect-free GO hollow fibre membrane was successfully determined and the membrane was stable in a long term (1200 hours) gas-tight stability test. Post-treatment of the GO membranes with UV light was also successfully accomplished in air, which induced the creation of controlled microstructural defects in the membrane and increased the roughness factor of the membrane surface. The permeability of the UV-treated GO membranes was greatly enhanced from 0.07 to 2.8 L m(-2) h(-1) bar(-1) for water, and 0.14 to 7.5 L m(-2) h(-1) bar(-1) for acetone, with an unchanged low molecular weight cut off ({\~{}}250 Da).},
annote = {Chong, J Y
Aba, N F D
Wang, B
Mattevi, C
Li, K
ENG
Research Support, Non-U.S. Gov't
England
2015/11/04 06:00
Sci Rep. 2015 Nov 3;5:15799. doi: 10.1038/srep15799.},
author = {Chong, J Y and Aba, N F and Wang, B and Mattevi, C and Li, K},
doi = {10.1038/srep15799},
isbn = {2045-2322 (Electronic)
2045-2322 (Linking)},
journal = {Sci Rep},
pages = {15799},
pmid = {26527173},
title = {{UV-Enhanced Sacrificial Layer Stabilised Graphene Oxide Hollow Fibre Membranes for Nanofiltration}},
url = {http://www.ncbi.nlm.nih.gov/pubmed/26527173},
volume = {5},
year = {2015}
}
@article{Mejdell2009,
author = {Mejdell, A L and J{\o}ndahl, M and Peters, T A and Bredesen, R and Venvik, H J},
doi = {10.1016/j.seppur.2009.04.025},
isbn = {13835866},
journal = {Separation and Purification Technology},
number = {2},
pages = {178--184},
title = {{Effects of CO and CO2 on hydrogen permeation through a ∼3{\$}\mu{\$}m Pd/Ag 23wt.{\{}{\%}{\}} membrane employed in a microchannel membrane configuration}},
volume = {68},
year = {2009}
}
@article{Shao2009,
author = {Shao, Lu and Low, Bee Ting and Chung, Tai-Shung and Greenberg, Alan R},
doi = {10.1016/j.memsci.2008.11.019},
isbn = {03767388},
journal = {Journal of Membrane Science},
number = {1-2},
pages = {18--31},
title = {{Polymeric membranes for the hydrogen economy: Contemporary approaches and prospects for the future}},
volume = {327},
year = {2009}
}
@article{Zhou2014,
abstract = {Porous graphene, which features nano-scaled pores on the sheets, is mostly investigated by computational studies. The pores on the graphene sheets may contribute to the improved mass transfer and may show potential applications in many fields. To date, the preparation of porous graphene includes chemical bottom-up approach via the aryl-aryl coupling reaction and physical preparation by high-energy techniques, and is generally conducted on substrates with limited yields. Here we show a general and scalable synthesis method for porous graphene that is developed through the carbothermal reaction between graphene and metal oxide nanoparticles produced from oxometalates or polyoxometalates. The pore formation process is observed in situ with the assistance of an electron beam. Pore engineering on graphene is conducted by controlling the pore size and/or the nitrogen doping on the porous graphene sheets by varying the amount of the oxometalates or polyoxometalates, or using ammonium-containing oxometalates or polyoxometalates.},
archivePrefix = {arXiv},
arxivId = {arXiv:1011.1669v3},
author = {Zhou, Ding and Cui, Yi and Xiao, Pei-Wen and Jiang, Mei-Yang and Han, Bao-Hang},
doi = {10.1038/ncomms5716},
eprint = {arXiv:1011.1669v3},
file = {:Users/marc/Library/Application Support/Mendeley Desktop/Downloaded/Zhou et al. - 2014 - A general and scalable synthesis approach to porous graphene.pdf:pdf},
isbn = {2041-1723 (Electronic)$\backslash$r2041-1723 (Linking)},
issn = {2041-1723},
journal = {Nature Communications},
pages = {4716},
pmid = {25178835},
publisher = {Nature Publishing Group},
title = {{A general and scalable synthesis approach to porous graphene}},
url = {http://www.nature.com/doifinder/10.1038/ncomms5716},
volume = {5},
year = {2014}
}
@article{Lee2016b,
abstract = {Carbon molecular sieve (CMS) membranes on the inside of porous composite stainless steel supports were developed for gas separation in this research effort. The intermediate alumina layer was introduced to reduce the pore size of the porous stainless steel tube and subsequently provide uniform surface roughness. Viscosity of the phenolic polymer solution was varied from 10 to 30 centipoises (cP) to maximize performance of the CMS membranes. Pyrolysis temperature was also varied from 700 °C to 900 °C to optimize the fabrication of uniform CMS membranes on porous composite stainless steel supports. High performance CMS membranes were obtained from triple coatings and subsequent pyrolysis at 700 °C. The viscosity of precursor solutions played a critical role to determine the performance of CMS membranes in terms of gas permeance and ideal gas separation factor. The highest separation performance of the CMS membranes was shown with viscosity of 20 cP, resulting in gas separation factor of 462 for He/N2, 97 for CO2/N2, and 15.4 for O2/N2.},
author = {Lee, Pyung Soo and Kim, Daejin and Nam, Seung Eun and Bhave, Ramesh R.},
doi = {10.1016/j.micromeso.2015.12.054},
file = {:Users/marc/Library/Application Support/Mendeley Desktop/Downloaded/Lee et al. - 2016 - Carbon molecular sieve membranes on porous composite tubular supports for high performance gas separations.pdf:pdf},
issn = {13871811},
journal = {Microporous and Mesoporous Materials},
keywords = {Carbon molecular sieve membrane,Gas separation,Intermediate alumina layer,Porous stainless steel support},
pages = {332--338},
publisher = {Elsevier Ltd},
title = {{Carbon molecular sieve membranes on porous composite tubular supports for high performance gas separations}},
url = {http://dx.doi.org/10.1016/j.micromeso.2015.12.054},
volume = {224},
year = {2016}
}
@article{Lang1985,
abstract = {The electrostatic interaction between two adsorbates, and, in particular, between an adsorbed atom and an adsorbed or adsorbing molecule is studied. Based on self-consistent calculations of the electrostatic potential around a series of atoms outside a jellium surface, it is shown that a simple electrostatic interaction can explain a large number of experimental observations concerning the influence of pre-adsorbed atoms on the adsorption rate, stability and adsorption configuration of simple molecules on metal surfaces. The role of pre-adsorbed alkalis as promoters and of electronegative atoms like P, S, Cl and O as poisons for the adsorption of electron acceptor molecules like H2, O2, N2 and CO is discussed, as well as the relative magnitude of the influence of the alkalis and the electronegative atoms. The peculiar effects that pre-adsorbed atoms have on molecules like H2O and NH3 are ascribed to the large intra-molecular electron transfer in these molecules. {\textcopyright} 1985.},
author = {Lang, N. D. and Holloway, S. and N{\o}rskov, J. K.},
doi = {10.1016/0039-6028(85)90208-0},
file = {:Users/marc/Library/Application Support/Mendeley Desktop/Downloaded/Lang, Holloway, N{\o}rskov - 1985 - Electrostatic adsorbate-adsorbate interactions The poisoning and promotion of the molecular adsorption.pdf:pdf},
isbn = {00396028 (ISSN)},
issn = {00396028},
journal = {Surface Science},
number = {1},
pages = {24--38},
title = {{Electrostatic adsorbate-adsorbate interactions: The poisoning and promotion of the molecular adsorption reaction}},
volume = {150},
year = {1985}
}
@article{Yukawa2002a,
abstract = {The alloying effects on the hydrogen absorption properties of vanadium at low hydrogen pressures have been investigated systematically with various binary V-3 mol{\%} M alloys where M is a 3d transition metal. The PCT curves were measured using an electrochemical method devised newly for this investigation. It was found that the hydriding properties of vanadium at low hydrogen pressures are affected largely by the presence of a small amount of alloying elements, M. For example, the equilibrium hydrogen pressure at low pressures changes monotonously following the order of elements, M, in the periodic table. This change at low hydrogen pressures is different from that at high hydrogen pressures. {\textcopyright} 2002 Elsevier Science B.V. All rights reserved.},
author = {Yukawa, Hiroshi and Teshima, Akira and Yamashita, Daisuke and Ito, Shigeyuki and Morinaga, Masahiko and Yamaguchi, Shu},
doi = {10.1016/S0925-8388(01)01936-3},
file = {:Users/marc/Library/Application Support/Mendeley Desktop/Downloaded/Yukawa et al. - 2002 - Alloying effects on the hydriding properties of vanadium at low hydrogen pressures.pdf:pdf},
issn = {09258388},
journal = {Journal of Alloys and Compounds},
keywords = {Alloying effects,Electrochemical reactions,Gas-solid reactions,Hydride stability,Hydrogen absorbing materials,Vanadium hydride},
number = {1-2},
pages = {264--268},
title = {{Alloying effects on the hydriding properties of vanadium at low hydrogen pressures}},
volume = {337},
year = {2002}
}
@incollection{PachecoTanaka2015,
author = {{Pacheco Tanaka}, D A and Okazaki, J and {Llosa Tanco}, M A and Suzuki, T M},
booktitle = {Palladium Membrane Technology for Hydrogen Production, Carbon Capture and Other Application},
doi = {10.1533/9781782422419.1.83},
editor = {Doukelis, A and Panopoulos, K and Koumanakos, A and Kakaras, E},
pages = {83--99},
publisher = {Woodhead Publishing},
title = {{Fabrication of supported palladium alloy membranes using electroless plating techniques}},
year = {2015}
}
@article{Xie2012,
abstract = {Through deposition of APTES-functionalized Al(2)O(3) particles onto a coarse macroporous support, a new strategy to reduce the pore size and simutaneously promote a high density of heterogeneous nucleation sites was developed, and a continuous and thin ZIF-8 membrane exhibiting remarkably high H(2) permeance of 5.73 × 10(-5) mol m(-2) s(-1) Pa(-1) and H(2)/N(2) ideal selectivity of 15.4 was achieved.},
author = {Xie, Zhong and Yang, Jianhua and Wang, Jinqu and Bai, Ju and Yin, Huimin and Yuan, Bing and Lu, Jinming and Zhang, Yan and Zhou, Liang and Duan, Chunying},
doi = {10.1039/c2cc17607f},
file = {:Users/marc/Library/Application Support/Mendeley Desktop/Downloaded/Xie et al. - 2012 - Deposition of chemically modified $\alpha$-Al2O3 particles for high performance ZIF-8 membrane on a macroporous tube.pdf:pdf},
isbn = {1359-7345 1364-548X},
issn = {1359-7345},
journal = {Chemical Communications},
number = {48},
pages = {5977},
pmid = {22314504},
title = {{Deposition of chemically modified $\alpha$-Al2O3 particles for high performance ZIF-8 membrane on a macroporous tube}},
url = {http://xlink.rsc.org/?DOI=c2cc17607f},
volume = {48},
year = {2012}
}
@article{Tarditi2014a,
author = {Tarditi, AM and Imhoff, C and Braun, F},
file = {:Users/marc/Library/Application Support/Mendeley Desktop/Downloaded/Tarditi, Imhoff, Braun - 2014 - PdCuAu ternary alloy membranes Hydrogen permeation properties in the presence of H 2 S.pdf:pdf},
journal = {Journal of Membrane  {\ldots}},
pages = {246--255},
title = {{PdCuAu ternary alloy membranes: Hydrogen permeation properties in the presence of H 2 S}},
volume = {479},
year = {2014}
}
@misc{Hohenberg1964,
abstract = {This paper deals with the ground state of an interacting electron gas in an external potential v(r). It is proved that there exists a universal functional of the density, F[n(r)], independent of v(r), such that the expression E≡∫v(r)n(r)dr+F[n(r)] has as its minimum value the correct ground-state energy associated with v(r). The functional F[n(r)] is then discussed for two situations: (1) n(r)=n0+ñ(r), ñ/n0≪1, and (2) n(r)=ϕ(r/r0) with ϕ arbitrary and r0→∞. In both cases F can be expressed entirely in terms of the correlation energy and linear and higher order electronic polarizabilities of a uniform electron gas. This approach also sheds some light on generalized Thomas-Fermi methods and their limitations. Some new extensions of these methods are presented.},
author = {Hohenberg, P and Kohn, W},
booktitle = {Physical Review},
doi = {10.1103/PhysRev.136.B864},
number = {3B},
title = {{Inhomogeneous electron gas}},
volume = {136},
year = {1964}
}
@misc{Christophersen2003,
abstract = {Scallop Pecten maximus spat (1.3-2.1mmshell height) from di¡erent settlement groups were transferred from hatchery to land-based nursery at different ages and sizes. Chemical content, growth and survival were compared at transfer time and after 1 and 8 weeks of nursery growth. Growth was lowest and mortality highest in the ¢rst week after transfer. Mean shell height growth was 21.5-71.4 mm day 1and ash-free dry weight (AFDW) growth 2.7 to 10.3 mg day-1. Spat from the ¢rst settlement group attained a larger size and weight than spat from larvae settled 3 days later, but had a lower daily growth rate ({\%}). Keeping the late-settled spat a longer time in the hatchery to reach a bigger size before transfer seemed not to improve subsequent nursery growth. Survival showed a large variation with mean survival ranging from 32{\%} to 74{\%}. A substantial reduction in lipid content was found after transfer to the nursery. Sterol content at transfer was the only lipid class correlating with survival in the nursery. Based on the results, it is justi¢ed that spat groups of different settlement age are included in production of 15-mm great scallop spat if they are transferred from the hatchery at the same age.},
author = {Christophersen, Gyda and Lie, {\O}yvind},
booktitle = {Aquaculture Research},
doi = {10.1046/j.1365-2109.2003.00873.x},
number = {8},
title = {{Nursery growth, survival and chemical composition of great scallop Pecten maximus (L.) spat from different larval settlement groups}},
volume = {34},
year = {2003}
}
@article{Schnorr2011,
abstract = {On the basis of their unique electrical and mechanical properties, carbon nanotubes (CNTs) have attracted great attention in recent years. A diverse array of methods has been developed to modify CNTs and to assemble them into devices. On the basis of these innovations, many applications that include the use of CNTs have been demonstrated. Transparent electrodes for organic light-emitting diodes (OLEDs), lithium-ion batteries, supercapacitors, and CNT-based electronic components such as field-effect transistors (FETs) have been demonstrated. Furthermore, CNTs have been employed in catalysis and sensing as well as filters and mechanical and biomedical applications. This review highlights illustrative examples from these areas to give an overview of applications of CNTs.},
author = {Schnorr, Jan M. and Swager, Timothy M.},
doi = {10.1021/cm102406h},
file = {:Users/marc/Library/Application Support/Mendeley Desktop/Downloaded/Schnorr, Swager - 2011 - Emerging applications of carbon nanotubes.pdf:pdf},
isbn = {0897-4756$\backslash$r1520-5002},
issn = {08974756},
journal = {Chemistry of Materials},
number = {3},
pages = {646--657},
title = {{Emerging applications of carbon nanotubes}},
volume = {23},
year = {2011}
}
@article{Sakamoto1996,
author = {Sakamoto, Y and Chen, F L and Kinari, Y and Sakamoto, F},
file = {:Users/marc/Library/Application Support/Mendeley Desktop/Downloaded/Sakamoto et al. - 1996 - Effect of Carbon Monoxide on Hydrogen Permeation in some Palladium Based Alloy Membranes.pdf:pdf},
journal = {International Journal of Hydrogen Energy},
number = {11/12},
pages = {1017--1024},
title = {{Effect of Carbon Monoxide on Hydrogen Permeation in some Palladium Based Alloy Membranes}},
volume = {210},
year = {1996}
}
@misc{Kong2011,
archivePrefix = {arXiv},
arxivId = {arXiv:1208.5721},
author = {{Shane E. Roark, Boulder, CO (US); Richard MacKay, Lafayette}, CO (US); and S, Michael V. Mundschau Longmont and (US), s CO},
doi = {10.1016/j.(73)},
eprint = {arXiv:1208.5721},
file = {:Users/marc/Library/Application Support/Mendeley Desktop/Downloaded/Shane E. Roark, Boulder, CO (US) Richard MacKay, Lafayette, S, (US) - 2006 - DENSE, LAYERED MEMBRANES FOR HYDROGEN SEPARATION.pdf:pdf},
isbn = {2004001828},
issn = {2004001828},
number = {12},
pages = {12--15},
pmid = {1000182772},
title = {{DENSE, LAYERED MEMBRANES FOR HYDROGEN SEPARATION}},
volume = {2},
year = {2006}
}
@article{Flanagan2011a,
author = {Flanagan, Ted B and Wang, D and Shanahan, Kirk},
doi = {10.1016/j.seppur.2011.03.027},
isbn = {13835866},
journal = {Separation and Purification Technology},
number = {3},
pages = {385--392},
title = {{The effect of CO on hydrogen permeation through Pd and through internally oxidized and un-oxidized Pd alloy membranes}},
volume = {79},
year = {2011}
}
@phdthesis{Lloyd2004,
address = {England},
author = {Lloyd, Robin Jonathan},
booktitle = {Department of Engineering Science},
publisher = {University of Oxford},
title = {{Fabrication, Testing and Modelling of Palladium Membranes for Fuel Cell Applications}},
volume = {PhD in Eng},
year = {2004}
}
@article{Sohrabnezhad2007,
abstract = {Our experimental observation shows that new methylene blue (NMB) incorporated in H-mordenite zeolite can serve as an excellent candidate for optical humidity sensing. New methylene blue was incorporated into H-mordenite zeolite by ion exchange reaction in aqueous phase. The dye solid was characterized by diffuse reflectance spectroscopy (DRS).The mechanism of the sensor is based on the protonation and deprotonation of the dye molecules which are associated with the desorption and adsorption of water molecules by zeolite, respectively. The measurement was carried out at seven fixed humidity points in the range of 6-98{\%} relative humidity. The sensor showed a linear response range from 6 to 95{\%} relative humidity, good stability and reversibility. The sensor operates both in 632??nm and 590??nm bands. The sensor demonstrates relatively fast response and recovery times about 1??min in the direction of adsorption and about 2??min in the direction of desorption of water. ?? 2006 Elsevier B.V. All rights reserved.},
author = {Sohrabnezhad, Shabnam and Pourahmad, Afshin and Sadjadi, Mir Abdollah},
doi = {10.1016/j.matlet.2006.09.006},
file = {:Users/marc/Library/Application Support/Mendeley Desktop/Downloaded/Sohrabnezhad, Pourahmad, Sadjadi - 2007 - New methylene blue incorporated in mordenite zeolite as humidity sensor material.pdf:pdf},
isbn = {0167-577X},
issn = {0167577X},
journal = {Materials Letters},
keywords = {Humidity sensor,Mordenite,New methylene blue,Optical humidity sensing},
number = {11-12},
pages = {2311--2314},
title = {{New methylene blue incorporated in mordenite zeolite as humidity sensor material}},
volume = {61},
year = {2007}
}
@article{Kreno2012,
author = {Kreno, Lauren E and Leong, Kirsty and Farha, Omar K and Allendorf, Mark and Duyne, Richard P Van and Hupp, Joseph T},
doi = {10.1021/cr200324t},
file = {:Users/marc/Library/Application Support/Mendeley Desktop/Downloaded/Kreno et al. - 2012 - Metal- Organic Framework Materials as Chemical Sensors.pdf:pdf},
isbn = {0009-2665},
issn = {1520-6890},
journal = {Chemical Reviews},
pages = {1105--1125},
pmid = {22070233},
title = {{Metal- Organic Framework Materials as Chemical Sensors}},
volume = {112},
year = {2012}
}
@article{Phair2006,
abstract = {Hydrogen separation membrane devices are attracting increasing interest for the industrial production of hydrogen. Metal membranes, in particular, are promising due to their resilience to the demands of a typical hydrogen purification process. However, the use of too much Pd within the membranes limits their wide-scale industrial use due to excessive costs. The present article reviews the design, preparation, operation, and critical performance features of novel non-Pd-based alloys. The theory behind the permeation of hydrogen through metal membranes is presented as well as the materials and methods central to their design and improvement. Crystalline, amorphous, and thin layer metal membranes are contrasted, while the advanced experimental techniques and mechanical tests for their characterization are discussed. The review considers the design of novel metal membranes from first principles and assesses catalytic and protective surface layers which may enhance their hydrogen separation capabilities.},
author = {Phair, John W. and Donelson, Richard},
doi = {10.1021/ie051333d},
file = {:Users/marc/Library/Application Support/Mendeley Desktop/Downloaded/Phair, Donelson - 2006 - Developments and design of novel (non-palladium-based) metal membranes for hydrogen separation.pdf:pdf},
isbn = {0888-5885},
issn = {08885885},
journal = {Industrial and Engineering Chemistry Research},
number = {16},
pages = {5657--5674},
title = {{Developments and design of novel (non-palladium-based) metal membranes for hydrogen separation}},
volume = {45},
year = {2006}
}
@article{Li2000,
author = {Li, A and Liang, W and Highes, R},
file = {:Users/marc/Library/Application Support/Mendeley Desktop/Downloaded/Li, Liang, Highes - 2000 - The effect of carbon monoxide and steam on the hydrogen permeability of a Pdstainless steel membrane.pdf:pdf},
journal = {Journal of Membrane Science},
pages = {135--141},
title = {{The effect of carbon monoxide and steam on the hydrogen permeability of a Pd/stainless steel membrane}},
volume = {165},
year = {2000}
}
@article{Ryi2011,
abstract = {This study presents a new non-alloy Ru/Pd composite membrane fabricated by electroless plating for hydrogen separation. It shows that palladium and ruthenium can be deposited on an aluminum-oxide-modified porous Hastalloy by using our new EDTA-free plating bath at room temperature and 358 K, respectively. A 6.8 ??m thick non-alloy Ru/Pd membrane film could be plated and helium leak test confirmed that the membrane was free of defects. Hydrogen permeation test showed that the membrane had a hydrogen permeation flux of 4.5 ?? 10-1 mol m-2 s-1 at a temperature of 773 K and a pressure difference of 100 kPa. The hydrogen permeability normalized value with thickness of the membrane was 1.4 times higher than our pure Pd membrane having similar structure. The EDX profiles of the front and back side membrane, cross-sectional EDX line scanning and XRD profile show that there was no alloying progress between the palladium and ruthenium layer after hydrogen permeation test at 773 K. ?? 2010, Hydrogen Energy Publications, LLC. Published by Elsevier Ltd. All rights reserved.},
author = {Ryi, Shin Kun and Li, Anwu and Lim, C. Jim and Grace, John R.},
doi = {10.1016/j.ijhydene.2010.06.014},
file = {:Users/marc/Library/Application Support/Mendeley Desktop/Downloaded/Ryi et al. - 2011 - Novel non-alloy RuPd composite membrane fabricated by electroless plating for hydrogen separation.pdf:pdf},
isbn = {0360-3199},
issn = {03603199},
journal = {International Journal of Hydrogen Energy},
keywords = {Electroless plating,Hydrogen,Membrane,Pd,Porous substrate,Ru},
number = {15},
pages = {9335--9340},
publisher = {Elsevier Ltd},
title = {{Novel non-alloy Ru/Pd composite membrane fabricated by electroless plating for hydrogen separation}},
url = {http://dx.doi.org/10.1016/j.ijhydene.2010.06.014},
volume = {36},
year = {2011}
}
@unpublished{M.DownerA.Murugan,
author = {{M. Downer A. Murugan}, A Brown},
file = {:Users/marc/Library/Application Support/Mendeley Desktop/Downloaded/M. Downer A. Murugan - Unknown - Hydrogen Purity Analysis for Fuel Cell Vehicles - Poster.pdf:pdf},
title = {{Hydrogen Purity Analysis for Fuel Cell Vehicles - Poster}}
}
@article{Nayebossadri2014,
author = {Nayebossadri, Shahrouz and Speight, John and Book, David},
doi = {10.1016/j.memsci.2013.10.002},
file = {:Users/marc/Library/Application Support/Mendeley Desktop/Downloaded/Nayebossadri, Speight, Book - 2014 - Effects of low Ag additions on the hydrogen permeability of Pd–Cu–Ag hydrogen separation membra.pdf:pdf},
isbn = {03767388},
journal = {Journal of Membrane Science},
pages = {216--225},
title = {{Effects of low Ag additions on the hydrogen permeability of Pd–Cu–Ag hydrogen separation membranes}},
volume = {451},
year = {2014}
}
@article{Escolastico2015,
abstract = {Lanthanide tungstates (Ln6WO12) are promising candidates for the development of ceramic hydrogen transport membranes since they exhibit mixed ionic (proton and oxygen ion transport) and electronic conductivity and remarkable stability in a moist CO2 environment at high temperatures. This work presents the structural and electrochemical characterization of mixed conducting materials for the specific system Nd5.5W1−xMoxO11.25−$\delta$ (x = 0, 0.1, 0.5 and 1). Evolution of the crystalline structure is studied as a function of the sintering temperature. Shrinkage behavior is analyzed for all compositions in the temperature range from 1000 °C to 1500 °C and these compounds show high sintering activity even at relatively low temperatures. The total conductivity in different environments is studied systematically for samples sintered at 1350 °C. The H/D isotopic effect is also studied by DC-electrochemical measurements. H2 permeation is investigated for the selected compound Nd5.5W0.5Mo0.5O11.25−$\delta$ in the range of 700–1000 °C achieving values of 0.3 mL min−1 cm−2 for a 0.9 mm thick disc membrane. Finally, the stability of this material under different CO2 and H2S-rich atmospheres at high temperatures is proven.},
author = {Escol{\'{a}}stico, Sonia and Somacescu, Simona and Serra, Jos{\'{e}} M.},
doi = {10.1039/C4TA03699A},
file = {:Users/marc/Library/Application Support/Mendeley Desktop/Downloaded/Escol{\'{a}}stico, Somacescu, Serra - 2015 - Tailoring mixed ionic–electronic conduction in H 2 permeable membranes based on the syste.pdf:pdf},
issn = {2050-7488},
journal = {J. Mater. Chem. A},
number = {2},
pages = {719--731},
title = {{Tailoring mixed ionic–electronic conduction in H 2 permeable membranes based on the system Nd 5.5 W 1−x Mo x O 11.25−$\delta$}},
url = {http://xlink.rsc.org/?DOI=C4TA03699A},
volume = {3},
year = {2015}
}
@article{Conde2016,
author = {Conde, Julio J and Maro{\~{n}}o, Marta and S{\'{a}}nchez-Herv{\'{a}}s, Jos{\'{e}} Mar{\'{i}}a},
doi = {10.1080/15422119.2016.1212379},
isbn = {1542-2119 1542-2127},
journal = {Separation {\{}{\&}{\}} Purification Reviews},
number = {2},
pages = {152--177},
title = {{Pd-Based Membranes for Hydrogen Separation: Review of Alloying Elements and Their Influence on Membrane Properties}},
volume = {46},
year = {2016}
}
@article{Yacou2012,
abstract = {Here we show the long term performance at high temperatures of a multi-tube module containing 8 membranes in 4 parallel lines with a total of 545 cm2 area. The membranes were prepared via thin film dip coating of cobalt oxide silica (CoOxSi) sol-gel on tubular alumina supports. A preliminary study found that the sol-gel containing 20 mol{\%} cobalt oxide formed the best microporous structure with the highest surface area and pore volume. All resulting membranes delivered permeances of [similar]1 [times] 10-7 mol m-2 s-1Pa-1 at 500 [degree]C, indicating a high repeatability for the membrane fabrication process. The permselectivities of helium (He) and hydrogen (H2) over carbon dioxide (CO2) and nitrogen (N2) increased from 10-20 at 100 [degree]C to values close to 1000 at 500 [degree]C. Additionally, the apparent energies of activation (Eact) for the smaller kinetic diameter gases He and H2 at 12.2 and 19.5 kJ mol-1 were high and contrary to the negative values for larger gases N2 and CO2 at -1.8 and -7.4 kJ mol-1. These remarkable results were attributed to the molecular sieving mechanism of the microporous silica which was enhanced by the embedding of cobalt oxide into the matrix, delivering structural control with an average pore size of 3 A. The Eact for H2 permeance was higher than that of He, indicating that the cobalt oxide played an important role in H2transport. Two membrane lines performed exceptionally well for binary gas mixture processing with H2 purity reaching values close to 100{\%} in the permeate stream for argon (Ar) concentrations of up to 80{\%} in the retentate stream. A major finding here is that the binary gas selectivity was independent of temperature, contrary to the permselectivity observed for single gas permeance. Further, the H2 flow rate was greatly affected by the concentration of Ar in the mixture, while the temperature dependency played only a marginal role. In particular, competitive adsorption in the percolative pathways containing pore constrictions or bottlenecks of the anisotropic CoOxSi matrix allowed Ar to impede H2 diffusion. Finally, the CoOxSi membranes proved thermally stable and robust for 2000 h of testing for various thermal cycles up to 500 [degree]C.},
author = {Yacou, Christelle and Smart, Simon and {Diniz da Costa}, Jo{\~{a}}o C.},
doi = {10.1039/c2ee03247c},
file = {:Users/marc/Library/Application Support/Mendeley Desktop/Downloaded/Yacou, Smart, Diniz da Costa - 2012 - Long term performance cobalt oxide silica membrane module for high temperature H2 separation.pdf:pdf},
isbn = {1754-5692},
issn = {1754-5692},
journal = {Energy {\&} Environmental Science},
number = {2},
pages = {5820},
title = {{Long term performance cobalt oxide silica membrane module for high temperature H2 separation}},
url = {http://xlink.rsc.org/?DOI=c2ee03247c},
volume = {5},
year = {2012}
}
@article{Macchione2007,
abstract = {The present paper discusses the tendency of solution-cast Hyflon{\textregistered}AD membranes to retain unexpectedly high amounts of solvent, the possible reasons of this phenomenon and its effect on the membrane performance. Dense membranes, prepared by solution-casting and subsequent evaporation, showed large differences in their thermal, mechanical and transport properties, depending on the residual solvent content. Complete solvent removal required heating under vacuum up to well above the glass transition temperature. Analysis of the permeability, diffusion and solubility coefficients of six permanent gases showed that plasticization by the residual solvent reduces the permselectivity and increases the permeability. Data of solution-cast membranes after complete solvent removal compare well with those of a melt-pressed sample. Experimental gas transport parameters were confronted with simulated data, obtained by the Gusev-Suter Transition-State Theory (TST) method and by molecular dynamics (MD) simulations.1H High Resolution Magic Angle Spinning Nuclear Magnetic Resonance spectroscopic analysis of the residual solvent in the polymer matrix did not reveal a particular interaction between polymer and solvent, suggesting that the solvent retention is mainly diffusion controlled. {\textcopyright} 2007 Elsevier Ltd. All rights reserved.},
author = {Macchione, Marialuigia and Jansen, Johannes Carolus and {De Luca}, Giuseppina and Tocci, Elena and Longeri, Marcello and Drioli, Enrico},
doi = {10.1016/j.polymer.2007.02.068},
issn = {00323861},
journal = {Polymer},
keywords = {Gas separation membrane,Hyflon AD60X,Residual solvent},
number = {9},
pages = {2619--2635},
title = {{Experimental analysis and simulation of the gas transport in dense Hyflon{\textregistered}AD60X membranes: Influence of residual solvent}},
volume = {48},
year = {2007}
}
@article{Unemoto2007a,
author = {Unemoto, A and Kaimai, A and Sato, K and Otake, T and Yashiro, K and Mizusaki, J and Kawada, T and Tsuneki, T and Shirasaki, Y and Yasuda, I},
doi = {10.1016/j.ijhydene.2007.04.030},
file = {:Users/marc/Library/Application Support/Mendeley Desktop/Downloaded/Unemoto et al. - 2007 - Surface reaction of hydrogen on a palladium alloy membrane under co-existence of H2OH2O, CO, CO2CO2 or CH4CH4☆.pdf:pdf},
isbn = {03603199},
journal = {International Journal of Hydrogen Energy},
number = {16},
pages = {4023--4029},
title = {{Surface reaction of hydrogen on a palladium alloy membrane under co-existence of H2OH2O, CO, CO2CO2 or CH4CH4☆}},
volume = {32},
year = {2007}
}
@article{Wouda1998,
author = {Wouda, P T and Schmid, M and Nieuwenhuys, B E and Varga, P},
file = {:Users/marc/Library/Application Support/Mendeley Desktop/Downloaded/Wouda et al. - 1998 - STM study of the ( 111 ) and ( 100 ) surfaces of PdAg.pdf:pdf},
journal = {Surface Science},
keywords = {alloys,low-index single crystal surfaces,palladium,scanning tunneling microscopy,silver,surface segregation},
pages = {292--300},
title = {{STM study of the ( 111 ) and ( 100 ) surfaces of PdAg}},
volume = {417},
year = {1998}
}
@article{Howarth2016,
abstract = {The construction of thousands of well-defined, porous, metal–organic framework (MOF) structures, spanning a broad range of topologies and an even broader range of pore sizes and chemical functionalities, has fuelled the exploration of many applications. Accompanying this applied focus has been a recognition of the need to engender MOFs with mechanical, thermal and/or chemical stability. Chemical stability in acidic, basic and neutral aqueous solutions is important. Advances over recent years have made it possible to design MOFs that possess different combinations of mechanical, thermal and chemical stability. Here, we review these advances and the associated design principles and synthesis strategies. We focus on how these advances may render MOFs effective as heterogeneous catalysts, both in chemically harsh condensed phases and in thermally challenging conditions relevant to gas-phase reactions. Finally, we briefly discuss future directions of study for the production of highly stable MOFs.},
author = {Howarth, Ashlee J. and Liu, Yangyang and Li, Peng and Li, Zhanyong and Wang, Timothy C. and Hupp, Joseph T. and Farha, Omar K.},
doi = {10.1038/natrevmats.2015.18},
file = {:Users/marc/Library/Application Support/Mendeley Desktop/Downloaded/Howarth et al. - 2016 - Chemical, thermal and mechanical stabilities of metal–organic frameworks.pdf:pdf},
isbn = {2058-8437},
issn = {2058-8437},
journal = {Nature Reviews Materials},
number = {3},
pages = {15018},
title = {{Chemical, thermal and mechanical stabilities of metal–organic frameworks}},
url = {http://www.nature.com/articles/natrevmats201518},
volume = {1},
year = {2016}
}
@article{Wu2018,
abstract = {Because of poor heterogeneous nucleation of MOF on the ceramic support surface, it is usually difficult to prepare defect-free metal-organic framework (MOF) layers as shape-selective membrane on the native ceramic supports by a direct solvothermal synthesis route. In the present work, inspired by “like dissolves like” principle, a simple and facile synthesis strategy was developed to prepare highly reproducible and permselective zeolitic imidazolate framework ZIF-8 membranes on the 1H,1H,2H,2H-perfluoroalkyltriethoxysilanes (POTS) modified $\alpha$-Al2O3disks. Attributing to the enhancement of surface hydrophobicity of the $\alpha$-Al2O3supports after POTS modification, hydrophobic ZIF-8 membranes are tending to grow well on the hydrophobic surface following the “like grows like” principle. Thus a well-intergrown ZIF-8 membrane with a thickness of about 20 µm has formed on the POTS-modified $\alpha$-Al2O3disks. The developed ZIF-8 membranes show high hydrogen selectivity and thermal stability. At 200 °C and 1 bar, the mixture separation factors of H2/CO2, H2/N2, H2/CH4, and H2/C3H8are 15.8, 22.6, 40.6, and 549.3, with H2permeances higher than 2.1 × 10−7mol{\textperiodcentered}m−2{\textperiodcentered}s−1{\textperiodcentered}Pa−1, which is promising for hydrogen separation and purification.},
author = {Wu, Xiaocao and Liu, Chuanyao and Caro, J{\"{u}}rgen and Huang, Aisheng},
doi = {10.1016/j.memsci.2018.04.053},
file = {:Users/marc/Library/Application Support/Mendeley Desktop/Downloaded/Wu et al. - 2018 - Facile synthesis of molecular sieve membranes following “like grows like” principle.pdf:pdf},
isbn = {0376-7388},
issn = {18733123},
journal = {Journal of Membrane Science},
keywords = {Hydrogen separation,Like grows like,Metal-organic frameworks membranes,Molecular sieve membrane,ZIF-8 membranes},
number = {May},
pages = {1--7},
publisher = {Elsevier B.V.},
title = {{Facile synthesis of molecular sieve membranes following “like grows like” principle}},
url = {https://doi.org/10.1016/j.memsci.2018.04.053},
volume = {559},
year = {2018}
}
@article{Stassen2017,
abstract = {Metal–organic frameworks (MOFs) are typically highlighted for their potential application in gas storage, separations and catalysis. In contrast, the unique prospects these porous and crystalline materials offer for application in electronic devices, although actively developed, are often underexposed. This review highlights the research aimed at the implementation of MOFs as an integral part of solid-state microelectronics. Manufacturing these devices will critically depend on the compatibility of MOFs with existing fabrication protocols and predominant standards. Therefore, it is important to focus in parallel on a fundamental understanding of the distinguishing properties of MOFs and eliminating fabrication-related obstacles for integration. The latter implies a shift from the microcrystalline powder synthesis in chemistry labs, towards film deposition and processing in a cleanroom environment. Both the fundamental and applied aspects of this two-pronged approach are discussed. Critical directions for future research are proposed in an updated high-level roadmap to stimulate the next steps towards MOF-based microelectronics within the community.},
author = {Stassen, Ivo and Burtch, Nicholas and Talin, Alec and Falcaro, Paolo and Allendorf, Mark and Ameloot, Rob},
doi = {10.1039/C7CS00122C},
file = {:Users/marc/Library/Application Support/Mendeley Desktop/Downloaded/Stassen et al. - 2017 - An updated roadmap for the integration of metal–organic frameworks with electronic devices and chemical sensor.pdf:pdf},
issn = {0306-0012},
journal = {Chem. Soc. Rev.},
number = {11},
pages = {3185--3241},
publisher = {Royal Society of Chemistry},
title = {{An updated roadmap for the integration of metal–organic frameworks with electronic devices and chemical sensors}},
url = {http://xlink.rsc.org/?DOI=C7CS00122C},
volume = {46},
year = {2017}
}
@inproceedings{Mohrdieck2014,
address = {Brussels},
author = {Mohrdieck, C},
booktitle = {7th Stakeholder Forum of the FCH JU},
title = {{Daimler's road to FCEV market introduction}},
year = {2014}
}
@article{Heras1997,
abstract = {The adsorption–thermodesorption behavior of H2O molecules on clean, thin polycrystalline Pd-films has been studied by means of work function (WF) and mass resolved temperature programmed desorption (TPD). Film morphology was varied and stabilized by annealing in the range 77–473 K. As deduced from WF changes during adsorption and TPD, unannealed films formed on glass at 77 K are highly porous. On the films at 77 K, H2O adsorbs molecularly and binds to the surface through the O-atom as suggested by the decrease in work function. Data can be rationalized assuming that during TPD, the adsorbed H2O molecules undergo decomposition. H2 never evolved from the films, not even at 473 K, suggesting that the observed H2O desorption peak at T{\textgreater}220 K should involve molecule rebuilding, as found in Pd single crystals.},
author = {Heras, J.M and Esti{\'{u}}, G and Viscido, L},
doi = {10.1016/S0169-4332(96)00686-1},
file = {:Users/marc/Library/Application Support/Mendeley Desktop/Downloaded/Heras, Esti{\'{u}}, Viscido - 1997 - The interaction of water with clean palladium films A thermal desorption and work function study.pdf:pdf},
issn = {01694332},
journal = {Applied Surface Science},
number = {4},
pages = {455--464},
title = {{The interaction of water with clean palladium films: A thermal desorption and work function study}},
url = {http://www.sciencedirect.com/science/article/pii/S0169433296006861},
volume = {108},
year = {1997}
}
@article{Liu2013,
abstract = {The adsorption of benzene on metal surfaces is an important benchmark system for hybrid inorganic/organic interfaces. The reliable determination of the interface geometry and binding energy presents a significant challenge for both theory and experiment. Using the Perdew–Burke–Ernzerhof (PBE), PBE + vdW (van der Waals) and the recently developed PBE + vdW surf (density-functional theory with vdW interactions that include the collective electronic response of the substrate) methods, we calculated the structures and energetics for benzene on transition-metal surfaces: Cu, Ag, Au, Pd, Pt, Rh and Ir. Our calculations demonstrate that vdW interactions increase the binding energy by more than 0.70 eV for physisorbed systems (Cu, Ag and Au) and by an even larger amount for strongly bound systems (Pd, Pt, Rh and Ir). The collective response of the substrate electrons captured via the vdW surf method plays a significant role for most substrates, shortening the equilibrium distance by 0.25 {\AA} for Cu and decreasing the binding energy by 0.27 eV for Rh. The reliability of our results is assessed by comparison with calculations using the random-phase approximation including renormalized single excitations, and the experimental data from temperature-programmed desorption, microcalorimetry measurements and low-energy electron diffraction.},
author = {Liu, Wei and Ruiz, Victor G. and Zhang, Guo Xu and Santra, Biswajit and Ren, Xinguo and Scheffler, Matthias and Tkatchenko, Alexandre},
doi = {10.1088/1367-2630/15/5/053046},
file = {:Users/marc/Library/Application Support/Mendeley Desktop/Downloaded/Liu et al. - 2013 - Structure and energetics of benzene adsorbed on transition-metal surfaces Density-functional theory with van der Waa.pdf:pdf},
isbn = {1367-2630},
issn = {13672630},
journal = {New Journal of Physics},
number = {111},
title = {{Structure and energetics of benzene adsorbed on transition-metal surfaces: Density-functional theory with van der Waals interactions including collective substrate response}},
volume = {15},
year = {2013}
}
@article{Barea2014,
abstract = {The release of anthropogenic toxic pollutants into the atmosphere is a worldwide threat of growing concern. In this regard, it is possible to take advantage of the high versatility of MOFs materials in order to develop new technologies for environmental remediation purposes. Consequently, one of the main scientific challenges to be achieved in the field of MOF research should be to maximize the performance of these solids towards the sensing, capture and catalytic degradation of harmful gases and vapors by means of a rational control of size and reactivity of the pore walls that are directly accessible to guest molecules.},
author = {Barea, Elisa and Montoro, Carmen and Navarro, Jorge A. R.},
doi = {10.1039/C3CS60475F},
file = {:Users/marc/Library/Application Support/Mendeley Desktop/Downloaded/Barea, Montoro, Navarro - 2014 - Toxic gas removal – metal–organic frameworks for the capture and degradation of toxic gases and vap.pdf:pdf},
isbn = {0306-0012},
issn = {0306-0012},
journal = {Chem. Soc. Rev.},
number = {16},
pages = {5419--5430},
pmid = {24705539},
title = {{Toxic gas removal – metal–organic frameworks for the capture and degradation of toxic gases and vapours}},
url = {http://xlink.rsc.org/?DOI=C3CS60475F},
volume = {43},
year = {2014}
}
@article{Kim2008,
author = {Kim, Hyoung-Juhn and Lim, Sang Jin and Lee, Jeung Woo and Min, In-Gyu and Lee, Sang-Yeop and Cho, EunAe and Oh, In-Hwan and Lee, Jong Hyun and Oh, Seung-Chan and Lim, Tae-Won and Lim, Tae-Hoon},
doi = {10.1016/j.jpowsour.2007.12.112},
file = {:Users/marc/Library/Application Support/Mendeley Desktop/Downloaded/Kim et al. - 2008 - Development of shut-down process for a proton exchange membrane fuel cell.pdf:pdf},
isbn = {03787753},
journal = {Journal of Power Sources},
number = {2},
pages = {814--820},
title = {{Development of shut-down process for a proton exchange membrane fuel cell}},
volume = {180},
year = {2008}
}
@article{Sun2006,
author = {Sun, G B and Hidajat, K and Kawi, S},
doi = {10.1016/j.memsci.2006.07.015},
file = {:Users/marc/Library/Application Support/Mendeley Desktop/Downloaded/Sun, Hidajat, Kawi - 2006 - Ultra thin Pd membrane on $\alpha$-Al2O3 hollow fiber by electroless plating High permeance and selectivity.pdf:pdf},
isbn = {03767388},
journal = {Journal of Membrane Science},
number = {1-2},
pages = {110--119},
title = {{Ultra thin Pd membrane on $\alpha$-Al2O3 hollow fiber by electroless plating: High permeance and selectivity}},
volume = {284},
year = {2006}
}
@article{Constable1936,
author = {Constable, R. W. Bost and E. W.},
doi = {10.15227/orgsyn.016.0081},
issn = {00786209},
journal = {Organic Syntheses},
number = {September},
pages = {81},
title = {{sym.-TRITHIANE}},
url = {http://orgsyn.org/demo.aspx?prep=CV2P0610},
volume = {16},
year = {1936}
}
@article{Y.S.Cheng1999,
author = {{Y.S. Cheng}, K L Yeung},
file = {:Users/marc/Library/Application Support/Mendeley Desktop/Downloaded/Y.S. Cheng - 1999 - Palladium-Silver composite membranes by electroless plating technique.pdf:pdf},
journal = {Journal of Membrane Science},
pages = {127--141},
title = {{Palladium-Silver composite membranes by electroless plating technique}},
volume = {158},
year = {1999}
}
@article{Pentyala2016,
abstract = {Novel nanoporous materials known as Metal-Organic Frameworks (MOFs) are currently attracting wide attention due to their potential applications such as gas storage, gas separation and catalysis. Recently, MOF materials were also developed for chemical sensors. In this work, for the first time iso-structural MOFs M-MOF-74 (where M = Mg, Ni, Co and Zn) and ethylenediamine functionalized Mg-MOF-74 were investigated as selective CO2 gas sensing material for the work function read-out based method at ambient temperature and different humidity levels. M-MOF-74s are very interesting MOFs because they contain coordinatively unsaturated metal centers where adsorbates can interact strongly with their affinity towards gas molecules. Interaction of the CO2 molecules with open metal sites or amine-functional groups changes the work function (????) of sensing layer which can serve as a signal for gas sensors. Results show that ethylenediamine functionalized Mg-MOF-74 exhibited enhanced CO2 gas sensing properties compared to other M-MOF-74 with respect to sensitivity, reversibility and stability.},
author = {Pentyala, Venkateswarlu and Davydovskaya, Polina and Ade, Martin and Pohle, Roland and Urban, Gerald},
doi = {10.1016/j.snb.2015.11.071},
file = {:Users/marc/Library/Application Support/Mendeley Desktop/Downloaded/Pentyala et al. - 2016 - Carbon dioxide gas detection by open metal site metal organic frameworks and surface functionalized metal organ.pdf:pdf},
issn = {09254005},
journal = {Sensors and Actuators, B: Chemical},
keywords = {Ethylenediamine-functionalization,Humidity effect,Kelvin probe,MOF},
pages = {363--368},
publisher = {Elsevier B.V.},
title = {{Carbon dioxide gas detection by open metal site metal organic frameworks and surface functionalized metal organic frameworks}},
url = {http://dx.doi.org/10.1016/j.snb.2015.11.071},
volume = {225},
year = {2016}
}
@article{Lundin2016,
author = {Lundin, Sean-Thomas B and Yamaguchi, Taichiro and Wolden, Colin A and Oyama, S Ted and Way, J Douglas},
doi = {10.1016/j.memsci.2016.04.048},
file = {:Users/marc/Library/Application Support/Mendeley Desktop/Downloaded/Lundin et al. - 2016 - The role (or lack thereof) of nitrogen or ammonia adsorption-induced hydrogen flux inhibition on palladium membra.pdf:pdf},
isbn = {03767388},
journal = {Journal of Membrane Science},
pages = {65--72},
title = {{The role (or lack thereof) of nitrogen or ammonia adsorption-induced hydrogen flux inhibition on palladium membrane performance}},
volume = {514},
year = {2016}
}
@article{Padama2011,
author = {Padama, Allan Abraham B. and Ozawa, Nobuki and Budhi, Yogi Wibisono and Kasai, Hideaki},
doi = {10.1143/JJAP.50.045701},
file = {:Users/marc/Library/Application Support/Mendeley Desktop/Downloaded/Padama et al. - 2011 - Density Functional Theory Investigation on the Dissociation and Adsorption Processes of N sub2sub on Pd(111) and.pdf:pdf},
issn = {0021-4922},
journal = {Japanese Journal of Applied Physics},
number = {4},
pages = {045701},
title = {{Density Functional Theory Investigation on the Dissociation and Adsorption Processes of N {\textless}sub{\textgreater}2{\textless}/sub{\textgreater} on Pd(111) and Pd {\textless}sub{\textgreater}3{\textless}/sub{\textgreater} Ag(111) Surfaces}},
url = {http://stacks.iop.org/1347-4065/50/045701},
volume = {50},
year = {2011}
}
@article{Strahl2014,
author = {Strahl, Stephan and Husar, Attila and Riera, Jordi},
doi = {10.1016/j.jpowsour.2013.09.122},
isbn = {03787753},
journal = {Journal of Power Sources},
pages = {474--482},
title = {{Experimental study of hydrogen purge effects on performance and efficiency of an open-cathode Proton Exchange Membrane fuel cell system}},
volume = {248},
year = {2014}
}
@article{Eutemann2012,
abstract = {Formic Acid},
author = {Eutemann, W Erner R and Aktiengesellschaft, Basf and Republic, Federal},
doi = {10.1002/14356007.a12},
isbn = {3527306730},
issn = {1435-6007},
pages = {13--33},
title = {{Formic Acid}},
year = {2012}
}
@article{Bunch2008,
author = {Bunch, J Scott and Verbridge, Scott S and Alden, Jonathan S and van der Zande, Arend M and Parpia, Jeevak M and Craighead, Harold G and McEuen, Paul L},
journal = {Nano Letters},
number = {8},
pages = {2458--2462},
title = {{Impermeable Atomic Membranes from Graphene Sheets}},
volume = {8},
year = {2008}
}
@article{Mitzel2016,
author = {Mitzel, Jens and G{\"{u}}lzow, Erich and Kabza, Alexander and Hunger, J{\"{u}}rgen and Araya, Samuel Simon and Piela, Piotr and Alecha, Iker and Tsotridis, Georgios},
doi = {10.1016/j.ijhydene.2016.08.065},
isbn = {03603199},
journal = {International Journal of Hydrogen Energy},
number = {46},
pages = {21415--21426},
title = {{Identification of critical parameters for PEMFC stack performance characterization and control strategies for reliable and comparable stack benchmarking}},
volume = {41},
year = {2016}
}
@article{He2013,
abstract = {A general technical route of Design-Preparation-Construction-Operation-Integration (DPCOI) platform for hollow fiber carbon membranes (HFCMs) was firstly suggested to promote the development of carbon membrane material to commercial application. A polymer of cellulose acetate (CA) was chosen as the material for spinning of hollow fibers. Cellulosic precursors were regenerated from CA hollow fibers using an optimized deacetylation process, which were further used for preparation of HFCMs using a specific carbonization procedure. Membrane separation performances of the prepared HFCMs were tested by gas permeation measurements for single gas and gas mixtures, while the carbon membrane stability and durability were documented by exposing the membranes to a real flue gas containing water vapor and acid gases of SO2and NOx. HYSYS simulation was also conducted to evaluate the process feasibility of CO2capture by HFCMs in a post combustion process. The investigation results indicated that DPCOI platform could be well used to guide the development of HFCMs, and potentially promote their commercial applications for gas separation. {\textcopyright} 2012 Elsevier B.V.},
author = {He, Xuezhong and H{\"{a}}gg, May Britt},
doi = {10.1016/j.cej.2012.10.051},
file = {:Users/marc/Library/Application Support/Mendeley Desktop/Downloaded/He, H{\"{a}}gg - 2013 - Hollow fiber carbon membranes From material to application.pdf:pdf},
isbn = {0376-7388},
issn = {13858947},
journal = {Chemical Engineering Journal},
keywords = {CO2capture,Cellulose acetate,Gas separation,Hollow fiber carbon membranes,Spinning},
pages = {440--448},
publisher = {Elsevier B.V.},
title = {{Hollow fiber carbon membranes: From material to application}},
url = {http://dx.doi.org/10.1016/j.cej.2012.10.051},
volume = {215-216},
year = {2013}
}
@misc{StructureCommissionoftheInternationalZeoliteAssociation,
author = {{Structure Commission of the International Zeolite Association}},
booktitle = {2017},
title = {{Database of Zeolite Structures}},
url = {http://www.iza-structure.org/databases/}
}
@article{Weil2006,
abstract = {Coal is potentially a very inexpensive source of clean hydrogen fuel for use in fuel cells, turbines, and various process applications. To realize its potential however, efficient low-cost gas separation systems are needed to provide high purity oxygen that will enhance the coal gasification reaction and to extract hydrogen from the resulting gas product stream. Several types of inorganic membranes are being developed for hydrogen or oxygen separation, including porous alumina, transition metal oxide perovskites, and zirconia. Because they form the heart of the working device, numerous advances have been made in the fabrication and performance of these membrane materials. However, less emphasis has been placed on the materials that will be used in the balance of the device; in particular, the seals that bond the functional ceramic to the metallic structural component. In an effort to begin addressing this issue, we have examined ceramic-to-metal brazing as a method of sealing a model set of gas separation component materials: yttria-stabilized zirconia and stainless steel. In comparative high-temperature exposure testing of joints prepared using commercial brazes and a newly conceived braze alloy, the commercial material proved to be unsuitable due to excessive oxidation. On the other hand, the new material not only displayed superior oxidation resistance, but also excellent hermeticity in prototypic membrane testing. ?? 2005 Elsevier Ltd. All rights reserved.},
author = {Weil, K. S. and Hardy, J. S. and Rice, J. P. and Kim, J. Y.},
doi = {10.1016/j.fuel.2005.07.023},
file = {:Users/marc/Library/Application Support/Mendeley Desktop/Downloaded/Weil et al. - 2006 - Brazing as a means of sealing ceramic membranes for use in advanced coal gasification processes.pdf:pdf},
isbn = {0016-2361},
issn = {00162361},
journal = {Fuel},
number = {2},
pages = {156--162},
title = {{Brazing as a means of sealing ceramic membranes for use in advanced coal gasification processes}},
volume = {85},
year = {2006}
}
@phdthesis{Morreale2006,
author = {Morreale, Bryan David},
file = {:Users/marc/Library/Application Support/Mendeley Desktop/Downloaded/Morreale - 2006 - The infleuance of H2S on palladium and palladium-copper alloy membranes.pdf:pdf},
title = {{The infleuance of H2S on palladium and palladium-copper alloy membranes}},
year = {2006}
}
@incollection{Xu2016,
author = {Xu, Quan and Zhang, Wenwen},
booktitle = {Advances in Carbon Nanostructures},
doi = {10.5772/64396},
editor = {{Adrian M.T. SIlvia}, Sonia A C Carabinerio},
publisher = {Intech},
title = {{Next-Generation Graphene-Based Membranes for Gas Separation and Water Purifications}},
year = {2016}
}
@article{Jones2016,
abstract = {{\textless}p{\textgreater}Shedding light on the design strategies used to make structurally photoactive metal–organic frameworks.{\textless}/p{\textgreater}},
author = {Jones, C. L. and Tansell, A. J. and Easun, T. L.},
doi = {10.1039/C5TA09424K},
file = {:Users/marc/Library/Application Support/Mendeley Desktop/Downloaded/Jones, Tansell, Easun - 2016 - The lighter side of MOFs structurally photoresponsive metal–organic frameworks.pdf:pdf},
isbn = {2050-7488},
issn = {2050-7488},
journal = {J. Mater. Chem. A},
number = {18},
pages = {6714--6723},
publisher = {Royal Society of Chemistry},
title = {{The lighter side of MOFs: structurally photoresponsive metal–organic frameworks}},
url = {http://xlink.rsc.org/?DOI=C5TA09424K},
volume = {4},
year = {2016}
}
@misc{PLUNKEYTY2018,
abstract = {hgjvhgjkhkjghkjbn},
author = {PLUNKEYTY, MARC and PLUNKEYTY, MARC and PLUNKEYTY, MARC and PLUNKEYTY, MARC},
booktitle = {JMS},
doi = {sdfdsfsd},
title = {sdfdsfsd},
url = {http://sdfsdfsd},
year = {2018}
}
@article{K.L.YeungA.Varma1999,
author = {{K. L. Yeung  A. Varma}, S C Christiansen},
file = {:Users/marc/Library/Application Support/Mendeley Desktop/Downloaded/K. L. Yeung A. Varma - 1999 - Palladium composite membranes by electroless plating technique Relationships between plating kinetics, fil.pdf:pdf},
journal = {Journal of Membrane Science},
pages = {107--122},
title = {{Palladium composite membranes by electroless plating technique: Relationships between plating kinetics, film microstructure and membrane performance}},
volume = {159},
year = {1999}
}
@article{Rengga2013,
abstract = {Bamboo based activated carbon (AC) was modified by attaching silver (AC-Ag) and copper (AC-Cu) nano-particles to reduce the low-concentration of formaldehyde in air. Batch isotherm tests were performed to determine the adsorption capacity of each activated carbon. At equilibrium concentration of 8 ppm, AC-Cu adsorbed 29{\%} higher amount of formaldehyde as compared to the original AC. The highest loading amount was obtained for AC-Ag, being around 0.425 mg/g AC-Ag. The concentration of formaldehyde removed by the AC-Ag was 1.6 times higher than the virgin AC at the same equilibrium concentration.$\backslash$n},
author = {Rengga, W. D. P. and Sudibandriyo, M. and Nasikin, M.},
doi = {10.7763/IJCEA.2013.V4.320},
file = {:Users/marc/Library/Application Support/Mendeley Desktop/Downloaded/Rengga, Sudibandriyo, Nasikin - 2013 - Adsorption of Low-Concentration Formaldehyde from Air by Silver and Copper Nano-Particles Attache.pdf:pdf},
issn = {20100221},
journal = {International Journal of Chemical Engineering and Applications},
number = {5},
pages = {332--336},
title = {{Adsorption of Low-Concentration Formaldehyde from Air by Silver and Copper Nano-Particles Attached on Bamboo-Based Activated Carbon}},
url = {http://www.ijcea.org/index.php?m=content{\&}c=index{\&}a=show{\&}catid=51{\&}id=641},
volume = {4},
year = {2013}
}
@article{Tarditi2011,
abstract = {The use of the sequential electroless plating method allowed us to obtain the PdAgCu ternary alloy on top of dense stainless steel (SS) 316 L disks. The XRD analysis indicated that initially the nucleation of the two phases of the alloy (FCC and BCC) takes place, but the FCC/BCC ratio increases with the annealing time at 500 ??C in H2 stream. After 162 h, the film contained only the FCC phase, which presents promising properties to be applied in the synthesis of hydrogen selective membranes. SEM cross-section results showed that a dense, continuous, defect-free film was deposited on top of the SS support, and the EDS data indicated that no significant gradient was present on the thickness of the film. XPS and LEIS allowed us to determine that Cu and Ag surface segregation takes place after annealing up to 500 ??C/5 days. In the top-most surface layer, Ag enrichment takes place as determined by ARXPS experiments which can be the result of the lower surface tension of Ag compared to that of Cu and Pd. Increasing the annealing temperature results in an increase of the Ag surface segregation while the Cu concentration in the top-most surface layer decreases. ?? 2010 Elsevier B.V. All rights reserved.},
author = {Tarditi, Ana M. and Cornaglia, Laura M.},
doi = {10.1016/j.susc.2010.10.001},
file = {:Users/marc/Library/Application Support/Mendeley Desktop/Downloaded/Tarditi, Cornaglia - 2011 - Novel PdAgCu ternary alloy as promising materials for hydrogen separation membranes Synthesis and characteri.pdf:pdf},
isbn = {0039-6028},
issn = {00396028},
journal = {Surface Science},
keywords = {Hydrogen separation membrane,Palladium based ternary alloys,Surface segregation},
number = {1-2},
pages = {62--71},
publisher = {Elsevier B.V.},
title = {{Novel PdAgCu ternary alloy as promising materials for hydrogen separation membranes: Synthesis and characterization}},
url = {http://dx.doi.org/10.1016/j.susc.2010.10.001},
volume = {605},
year = {2011}
}
@article{Hu2014,
abstract = {Metal-organic frameworks (MOFs) are a unique class of crystalline solids comprised of metal cations (or metal clusters) and organic ligands that have shown promise for a wide variety of applications. Over the past 15 years, research and development of these materials have become one of the most intensely and extensively pursued areas. A very interesting and well-investigated topic is their optical emission properties and related applications. Several reviews have provided a comprehensive overview covering many aspects of the subject up to 2011. This review intends to provide an update of work published since then and focuses on the photoluminescence (PL) properties of MOFs and their possible utility in chemical and biological sensing and detection. The spectrum of this review includes the origin of luminescence in MOFs, the advantages of luminescent MOF (LMOF) based sensors, general strategies in designing sensory materials, and examples of various applications in sensing and detection.},
author = {Hu, Zhichao and Deibert, Benjamin J. and Li, Jing},
doi = {10.1039/C4CS00010B},
file = {:Users/marc/Library/Application Support/Mendeley Desktop/Downloaded/Hu, Deibert, Li - 2014 - Luminescent metal–organic frameworks for chemical sensing and explosive detection.pdf:pdf},
isbn = {1460-4744 (Electronic)$\backslash$r0306-0012 (Linking)},
issn = {0306-0012},
journal = {Chem. Soc. Rev.},
number = {16},
pages = {5815--5840},
pmid = {24577142},
title = {{Luminescent metal–organic frameworks for chemical sensing and explosive detection}},
url = {http://xlink.rsc.org/?DOI=C4CS00010B},
volume = {43},
year = {2014}
}
@article{Varghese2015,
abstract = {Two-dimensional materials have attracted great scientific attention due to their unusual and fascinating properties for use in electronics, spintronics, photovoltaics, medicine, composites, etc. Graphene, transition metal dichalcogenides such as MoS2, phosphorene, etc., which belong to the family of two-dimensional materials, have shown great promise for gas sensing applications due to their high surface-to-volume ratio, low noise and sensitivity of electronic properties to the changes in the surroundings. Two-dimensional nanostructured semiconducting metal oxide based gas sensors have also been recognized as successful gas detection devices. This review aims to provide the latest advancements in the field of gas sensors based on various two-dimensional materials with the main focus on sensor performance metrics such as sensitivity, specificity, detection limit, response time, and reversibility. Both experimental and theoretical studies on the gas sensing properties of graphene and other two-dimensional materials beyond graphene are also discussed. The article concludes with the current challenges and future prospects for two-dimensional materials in gas sensor applications.},
archivePrefix = {arXiv},
arxivId = {arXiv:cond-mat/0611602v1},
author = {Varghese, Seba and Varghese, Saino and Swaminathan, Sundaram and Singh, Krishna and Mittal, Vikas},
doi = {10.3390/electronics4030651},
eprint = {0611602v1},
file = {:Users/marc/Library/Application Support/Mendeley Desktop/Downloaded/Varghese et al. - 2015 - Two-Dimensional Materials for Sensing Graphene and Beyond.pdf:pdf},
issn = {2079-9292},
journal = {Electronics},
keywords = {electronic properties,gas sensing,sensor performance,two-dimensional materials},
number = {3},
pages = {651--687},
pmid = {25117934},
primaryClass = {arXiv:cond-mat},
title = {{Two-Dimensional Materials for Sensing: Graphene and Beyond}},
url = {http://www.mdpi.com/2079-9292/4/3/651/},
volume = {4},
year = {2015}
}
@article{Stavila2014,
abstract = {Metal–organic frameworks (MOFs) are a class of hybrid materials with unique optical and electronic properties arising from rational self-assembly of the organic linkers and metal ions/clusters, yielding myriads of possible structural motifs. The combination of order and chemical tunability, coupled with good environmental stability of MOFs, are prompting many research groups to explore the possibility of incorporating these materials as active components in devices such as solar cells, photodetectors, radiation detectors, and chemical sensors. Although this field is only in its incipiency, many new fundamental insights relevant to integrating MOFs with such devices have already been gained. In this review, we focus our attention on the basic requirements and structural elements needed to fabricate MOF-based devices and summarize the current state of MOF research in the area of electronic, opto-electronic and sensor devices. We summarize various approaches to designing active MOFs, creation of hybrid material systems combining MOFs with other materials, and assembly and integration of MOFs with device hardware. Critical directions of future research are identified, with emphasis on achieving the desired MOF functionality in a device and establishing the structure–property relationships to identify and rationalize the factors that impact device performance.},
author = {Stavila, V. and Talin, A. A. and Allendorf, M. D.},
doi = {10.1039/C4CS00096J},
file = {:Users/marc/Library/Application Support/Mendeley Desktop/Downloaded/Stavila, Talin, Allendorf - 2014 - MOF-based electronic and opto-electronic devices.pdf:pdf},
isbn = {0306-0012},
issn = {0306-0012},
journal = {Chem. Soc. Rev.},
number = {16},
pages = {5994--6010},
pmid = {24802763},
title = {{MOF-based electronic and opto-electronic devices}},
url = {http://xlink.rsc.org/?DOI=C4CS00096J},
volume = {43},
year = {2014}
}
@article{Fang2014,
author = {Fang, Shumin and Brinkman, Kyle and Chen, Fanglin},
doi = {10.1021/am405169d},
file = {:Users/marc/Library/Application Support/Mendeley Desktop/Downloaded/Fang, Brinkman, Chen - 2014 - Unprecedented CO sub2sub -Promoted Hydrogen Permeation in Ni-BaZr sub0.1sub Ce sub0.7sub Y sub0.1sub Yb.pdf:pdf},
issn = {1944-8244},
journal = {ACS Applied Materials {\&} Interfaces},
keywords = {9 it is therefore,application,barium cerate,chemical stability,composite membrane,conductors possessing good chemical,high temperature proton conductor,high-temperature proton,hydrogen permeation,important to identify potential,stability in a concentrated,very challenging but extremely},
number = {1},
pages = {725--730},
title = {{Unprecedented CO {\textless}sub{\textgreater}2{\textless}/sub{\textgreater} -Promoted Hydrogen Permeation in Ni-BaZr {\textless}sub{\textgreater}0.1{\textless}/sub{\textgreater} Ce {\textless}sub{\textgreater}0.7{\textless}/sub{\textgreater} Y {\textless}sub{\textgreater}0.1{\textless}/sub{\textgreater} Yb {\textless}sub{\textgreater}0.1{\textless}/sub{\textgreater} O {\textless}sub{\textgreater}3−$\delta${\textless}/sub{\textgreater} Membrane}},
url = {http://pubs.acs.org/doi/abs/10.1021/am405169d},
volume = {6},
year = {2014}
}
@article{Cacho-Bailo2016,
abstract = {Double-layered zeolitic imidazolate framework (ZIF) membranes were fabricated inside polyimide P84 hollow fibers by a step-synthesis conducted by microfluidic technology and applied to pre-combustion gas separation. Our hypothesis{\{},{\}} based on the information provided by a combination of molecular simulation and experiments{\{},{\}} is that a CO2 adsorption reduction on the surface of the ZIF-9 would enhance the molecular sieving effect of this ZIF-9 layer and therefore the selectivity in the H2/CO2 mixture separation of the entire membrane. This reduction would be achieved by means of a less CO2-adsorptive methylimidazolate-based ZIF-67 or ZIF-8 layer coating the ZIF-9. ZIF-8/ZIF-9 and ZIF-67/ZIF-9 double-layered membranes were prepared and characterized by XRD{\{},{\}} FTIR{\{},{\}} SEM{\{},{\}} FIB{\{},{\}} TEM and EDS. This unprecedented strategy led to a H2/CO2 separation selectivity of 9.6 together with a 250 GPU H2 permeance at 150 [o]C{\{},{\}} showing a significant improvement with respect to the pure ZIF-9 membrane. Double-layered membranes also showed higher CO2 apparent activation energies than single-layered{\{},{\}} attributable to a diminished adsorption.},
author = {Cacho-Bailo, Fernando and Matito-Martos, Ismael and Perez-Carbajo, Julio and Etxeberr{\'{i}}a-Benavides, Miren and Karvan, Oğuz and Sebasti{\'{a}}n, V{\'{i}}ctor and Calero, Sof{\'{i}}a and T{\'{e}}llez, Carlos and Coronas, Joaqu{\'{i}}n},
doi = {10.1039/c6sc02411d},
file = {:Users/marc/Library/Application Support/Mendeley Desktop/Downloaded/Cacho-Bailo et al. - 2016 - On the molecular mechanisms for the H2CO2separation performance of zeolite imidazolate framework two-layered.pdf:pdf},
issn = {20416539},
journal = {Chemical Science},
number = {1},
pages = {325--333},
title = {{On the molecular mechanisms for the H2/CO2separation performance of zeolite imidazolate framework two-layered membranes}},
volume = {8},
year = {2016}
}
@article{De-enJiangShengDai2009,
author = {{De-en Jiang Sheng Dai}, Valentino R Cooper},
chapter = {4019},
file = {:Users/marc/Library/Application Support/Mendeley Desktop/Downloaded/De-en Jiang Sheng Dai - 2009 - Porous Graphene as the Ultimate Membrane for Gas Separation.pdf:pdf},
journal = {Nano Letters},
number = {12},
pages = {4019--4024},
title = {{Porous Graphene as the Ultimate Membrane for Gas Separation}},
volume = {9},
year = {2009}
}
@article{Wang2016e,
abstract = {Herein, we report the in situ synthesis of ZIF-8 membranes on the external surface of asymmetric ZnO–Al2O3 composite hollow fibers. The prepared ZIF-8 hollow fiber membrane exhibits exceeding performance for hydrogen separation and high reproducibility for an easy scale-up.},
author = {Wang, Xiaobin and Sun, Mingyong and Meng, Bo and Tan, Xiaoyao and Liu, Jian and Wang, Shaobin and Liu, Shaomin},
doi = {10.1039/C6CC06589A},
issn = {1359-7345},
journal = {Chemical Communications},
number = {92},
pages = {13448--13451},
publisher = {The Royal Society of Chemistry},
title = {{Formation of continuous and highly permeable ZIF-8 membranes on porous alumina and zinc oxide hollow fibers}},
url = {http://dx.doi.org/10.1039/C6CC06589A},
volume = {52},
year = {2016}
}
@article{Vannice1979,
author = {Vannice, M Albert and Garten, Robert L},
file = {:Users/marc/Library/Application Support/Mendeley Desktop/Downloaded/Vannice, Garten - 1979 - Supported Palladium Catalysts for Methanation.pdf:pdf},
journal = {Ind. Eng. Chem. Res. Dev},
number = {3},
pages = {186--191},
title = {{Supported Palladium Catalysts for Methanation}},
volume = {18},
year = {1979}
}
@article{Favvas2007,
abstract = {The preparation of carbon hollow fibers by an asymmetric precursor is reported. The effect of the pyrolysis environment on the structure and properties of the produced membranes, both on inert (N2) and reactive (CO2and H2O) conditions, have been examined by a carbonization process up to 900 °C. The surface structure and morphological characteristics were investigated using scanning electron microscopy (SEM). The produced carbon fibers exhibit H2permeance varying from 20 to 52 GPU with a highest H2/CH4permselectivity coefficient of 137 (M3) and very high H2/CO2permselectivity coefficient of 37.82 (M1). Permeation rates of He, H2, Ar, CH4, CO2, CO, O2and N2at variable pressure were measured too. The porosity of the developed membranes was estimated by nitrogen adsorption at 77 K. The results suggest that the formation of the selective layer on the fibers is independent from the initial orientation of the asymmetric polyimide precursor. The size of the pores is influenced by the temperature of the carbonization process whereas the pore volume by the pyrolysis environment conditions. {\textcopyright} 2006 Elsevier B.V. All rights reserved.},
author = {Favvas, E. P. and Kapantaidakis, G. C. and Nolan, J. W. and Mitropoulos, A. Ch and Kanellopoulos, N. K.},
doi = {10.1016/j.jmatprotec.2006.12.024},
file = {:Users/marc/Library/Application Support/Mendeley Desktop/Downloaded/Favvas et al. - 2007 - Preparation, characterization and gas permeation properties of carbon hollow fiber membranes based on Matrimid{\textregistered}5.pdf:pdf},
isbn = {2106503636},
issn = {09240136},
journal = {Journal of Materials Processing Technology},
keywords = {Carbon membranes,Gas separation,Hollow fiber,Nitrogen porosimetry,Pyrolysis},
number = {1-3},
pages = {102--110},
title = {{Preparation, characterization and gas permeation properties of carbon hollow fiber membranes based on Matrimid{\textregistered}5218 precursor}},
volume = {186},
year = {2007}
}
@article{Matthey1993,
author = {Matthey, Johnson},
file = {:Users/marc/Library/Application Support/Mendeley Desktop/Downloaded/Matthey - 1993 - A Selective Review of Metal-Hydrogen Technology in the Former U.S.S.R.pdf:pdf},
pages = {97--101},
title = {{A Selective Review of Metal-Hydrogen Technology in the Former U.S.S.R.}},
year = {1993}
}
@article{Løvvik2014,
author = {L{\o}vvik, O M and Peters, T A and Bredesen, R},
doi = {10.1016/j.memsci.2013.11.035},
issn = {0376-7388},
journal = {Journal of Membrane Science},
keywords = {Atomistic modeling,H2S,Membrane,Pd-alloy},
pages = {525--531},
publisher = {Elsevier},
title = {{First-principles calculations on sulfur interacting with ternary Pd – Ag-transition metal alloy membrane alloys}},
url = {http://dx.doi.org/10.1016/j.memsci.2013.11.035},
volume = {453},
year = {2014}
}
@article{Budd2005,
abstract = {When polymeric membranes are employed to remove selectively one component from a gaseous mixture, there is generally a trade-off between selectivity and permeability. Data are presented for two polymers of intrinsic microporosity, PIM-1 and PIM-7, which show a significant advance across the previous upper bound of performance for commercially important gas pairs, including O2/N2and CO2/CH4. The exceptional properties of PIMs arise from their rigid but contorted molecular structures, which frustrate packing and so create free volume, coupled with chemical functionality giving strong intermolecular interactions. {\textcopyright} 2005 Elsevier B.V. All rights reserved.},
author = {Budd, Peter M. and Msayib, Kadhum J. and Tattershall, Carin E. and Ghanem, Bader S. and Reynolds, Kevin J. and McKeown, Neil B. and Fritsch, Detlev},
doi = {10.1016/j.memsci.2005.01.009},
isbn = {0376-7388},
issn = {03767388},
journal = {Journal of Membrane Science},
keywords = {Gas separation,Permeability,Polymer membranes,Selectivity,Upper bound},
number = {1-2},
pages = {263--269},
title = {{Gas separation membranes from polymers of intrinsic microporosity}},
volume = {251},
year = {2005}
}
@article{Severance2014,
abstract = {Zeolites are microporous, crystalline aluminosilicates with the framework made up of T-O-T (T = Si, Al) bonds and enclosed cages and channels of molecular dimensions. Influencing and manipulating the nucleation and growth characteristics of zeolites can lead to novel frameworks and morphologies, as well as decreased crystallization time. In this study, we show that manipulating the supersaturation during synthesis of zeolite X/Y (FAU) via dehydration led to extensive nucleation. Controlled addition of water to this nucleated state promotes the transport of nutrients, with a 4-fold increase in the rate of crystal growth, as compared to conventional hydrothermal process. Structural signature of the nucleated state was obtained by electron microscopy, NMR, and Raman spectroscopy. This extensively intermediate nucleated state was isolated and used as the starting material for zeolite membrane synthesis on porous polymer supports, with membrane formation occurring within an hour. With this time frame for growth, it becomes practical to fabricate zeolite/polymer membranes using roll-to-roll technology, thus making possible new commercial applications.},
author = {Severance, Michael and Wang, Bo and Ramasubramanian, Kartik and Zhao, Lin and Ho, W. S.Winston and Dutta, Prabir K.},
doi = {10.1021/la5004512},
isbn = {1520-5827 (Electronic)$\backslash$r0743-7463 (Linking)},
issn = {15205827},
journal = {Langmuir},
number = {23},
pages = {6929--6937},
pmid = {24758695},
title = {{Rapid crystallization of faujasitic zeolites: Mechanism and application to zeolite membrane growth on polymer supports}},
volume = {30},
year = {2014}
}
@article{Gustafson2017,
author = {Gustafson, Jenna A. and Wilmer, Christopher E.},
doi = {10.1021/acs.jpcc.6b09740},
file = {:Users/marc/Library/Application Support/Mendeley Desktop/Downloaded/Gustafson, Wilmer - 2017 - Computational Design of Metal–Organic Framework Arrays for Gas Sensing Influence of Array Size and Composit.pdf:pdf},
isbn = {1932-7447
1932-7455},
issn = {1932-7447},
journal = {The Journal of Physical Chemistry C},
pages = {acs.jpcc.6b09740},
title = {{Computational Design of Metal–Organic Framework Arrays for Gas Sensing: Influence of Array Size and Composition on Sensor Performance}},
url = {http://pubs.acs.org/doi/abs/10.1021/acs.jpcc.6b09740},
year = {2017}
}
@article{Pugh2014,
abstract = {The illicit manufacture of drugs in the 21st century presents a danger to first responders, bystanders and the environment, making its detection important. Electronic noses based on metal oxide semiconducting (MOS) sensors present a potential technology to create devices for such purposes. An array of four thick film MOS gas sensors was fabricated, based on zinc oxide inks. Production took place using a commercially available screen printer, a 3 ? 3 mm alumina substrate containing interdigitated electrodes and a platinum heater track. ZnO inks were modified using zeolite b, zeolite Y and mordenite admixtures. The sensors were exposed to four gases commonly found in the clandestine laboratory environment; these were nitrogen dioxide, ethanol, acetone and ammonia. Zeolite modification was found to increase the sensitivity of the sensor, compared to unmodified ZnO sensors, all of which showed strong responses to low ppm concentrations of acetone, ammonia and ethanol and to ppb concentrations of nitrogen dioxide. Machine learning techniques were incorporated to test the selectivity of the sensors. A high level of accuracy was achieved in determining the class of gas observed.},
author = {Pugh, D. C. and Newton, E. J. and Naik, a. J. T. and Hailes, S. M. V. and Parkin, I. P.},
doi = {10.1039/c3ta15049f},
file = {:Users/marc/Library/Application Support/Mendeley Desktop/Downloaded/Pugh et al. - 2014 - The gas sensing properties of zeolite modified zinc oxide.pdf:pdf},
isbn = {2050-7488},
issn = {2050-7488},
journal = {Journal of Materials Chemistry A},
pages = {4758},
title = {{The gas sensing properties of zeolite modified zinc oxide}},
url = {http://xlink.rsc.org/?DOI=c3ta15049f},
volume = {2},
year = {2014}
}
@article{Tang2009,
abstract = {MFI-type zeolite membranes were modified by depositing molecular silica at a small number of active sites in the internal surface by in situ catalytic cracking of silane precursor. The limited silica deposition reduced the effective size of the zeolitic channels that dramatically enhanced the H(2) selectivity without causing a large increase in H(2) transport resistance. The modified zeolite membrane achieved an extraordinary H(2)/CO(2) permselectivity of 141 with a high H(2) permeance of 3.96 x 10(-7) mol/m(2) x s x Pa at 723 K. The effect of pore modification on the gas transport behavior was studied on the basis of single gas permeation data.},
author = {Tang, Zhong and Dong, Junhang and Nenoff, Tina M.},
doi = {10.1021/la900474y},
file = {:Users/marc/Library/Application Support/Mendeley Desktop/Downloaded/Tang, Dong, Nenoff - 2009 - Internal surface modification of MFI-type zeolite membranes for high selectivity and high flux for hydrogen.pdf:pdf},
isbn = {0743-7463 (Print)},
issn = {07437463},
journal = {Langmuir},
number = {9},
pages = {4848--4852},
pmid = {19397346},
title = {{Internal surface modification of MFI-type zeolite membranes for high selectivity and high flux for hydrogen}},
volume = {25},
year = {2009}
}
@article{Huang2011a,
abstract = {MOF constrictor: Covalent modification of a ZIF-90 membrane was achieved by imine condensation of the aldehyde groups of the metal-organic framework (MOF) linker by ethanolamine (see picture). The modification leads to constriction of the pore apertures and prevents unselective permeation through defect pores, thus improving gas separation performance. For a H2/CO2 mixture, selectivity can be increased from 7.3 to 62.5. Copyright {\textcopyright} 2011 WILEY-VCH Verlag GmbH {\&} Co. KGaA, Weinheim.},
author = {Huang, Aisheng and Caro, J{\"{u}}rgen},
doi = {10.1002/anie.201007861},
file = {:Users/marc/Library/Application Support/Mendeley Desktop/Downloaded/Huang, Caro - 2011 - Covalent post-functionalization of Zeolitic imidazolate framework ZIF-90 membrane for enhanced hydrogen selectivity.pdf:pdf},
isbn = {14337851 (ISSN)},
issn = {14337851},
journal = {Angewandte Chemie - International Edition},
keywords = {imidazolate,membranes,metal-organic frameworks,molecular sieves,post-synthetic modification},
number = {21},
pages = {4979--4982},
pmid = {21472930},
title = {{Covalent post-functionalization of Zeolitic imidazolate framework ZIF-90 membrane for enhanced hydrogen selectivity}},
volume = {50},
year = {2011}
}
@article{Grashoff1983,
author = {Grashoff, G J and Pilkington, C E and Corti, C W},
journal = {Platinum Metals Review},
number = {4},
pages = {157--169},
title = {{The Purification of Hydrogen: A Review of the technology emphasising the current status of Palladium Membrane Diffusion}},
volume = {27},
year = {1983}
}
@article{Bosko2011,
author = {Bosko, Mar{\'{i}}a L and Miller, James B and Lombardo, Eduardo A and Gellman, Andrew J and Cornaglia, Laura M},
doi = {10.1016/j.memsci.2010.12.006},
isbn = {03767388},
journal = {Journal of Membrane Science},
number = {1-2},
pages = {267--276},
title = {{Surface characterization of Pd–Ag composite membranes after annealing at various temperatures}},
volume = {369},
year = {2011}
}
@misc{PLUNKEYTY2018a,
abstract = {hgjvhgjkhkjghkjbn},
author = {PLUNKEYTY, MARC and PLUNKEYTY, MARC and PLUNKEYTY, MARC and PLUNKEYTY, MARC},
doi = {sdfdsfsd},
title = {sdfdsfsd},
url = {http://sdfsdfsd},
year = {2018}
}
@inproceedings{Fritsch1996,
author = {Fritsch, D. and Avella, N.},
booktitle = {36th IUPAC International Symposium on Macromolecules},
pages = {411},
title = {{Highly gas permeable poly(amide imide)s}},
year = {1996}
}
@article{Tan2005,
author = {Tan, Xiaoyao and Liu, Yutie and Li, Kang},
file = {:Users/marc/Library/Application Support/Mendeley Desktop/Downloaded/Tan, Liu, Li - 2005 - Preparation of LSCF ceramic Hollow-Fiber Membranes for Oxygen Production by a phase-InversionSintering Technique.pdf:pdf},
journal = {Industrial and Engineering Chemistry Research},
pages = {61--66},
title = {{Preparation of LSCF ceramic Hollow-Fiber Membranes for Oxygen Production by a phase-Inversion/Sintering Technique}},
volume = {44},
year = {2005}
}
@article{Ivanova2016,
abstract = {Hydrogen permeation membranes are a key element in improving the energy conversion efficiency and decreasing the greenhouse gas emissions from energy generation. The scientific community faces the challenge of identifying and optimizing stable and effective ceramic materials for H2 separation membranes at elevated temperature (400-800 degrees C) for industrial separations and intensified catalytic reactors. As such, composite materials with nominal composition BaCe0.8Eu0.2O3-delta:Ce0.8Y0.2O2-delta revealed unprecedented H2 permeation levels of 0.4 to 0.61 mL.min-1.cm-2 at 700 degrees C measured on 500 mum-thick-specimen. A detailed structural and phase study revealed single phase perovskite and fluorite starting materials synthesized via the conventional ceramic route. Strong tendency of Eu to migrate from the perovskite to the fluorite phase was observed at sintering temperature, leading to significant Eu depletion of the proton conducing BaCe0.8Eu0.2O3-delta phase. Composite microstructure was examined prior and after a variety of functional tests, including electrical conductivity, H2-permeation and stability in CO2 containing atmospheres at elevated temperatures, revealing stable material without morphological and structural changes, with segregation-free interfaces and no further diffusive effects between the constituting phases. In this context, dual phase material based on BaCe0.8Eu0.2O3-delta:Ce0.8Y0.2O2-delta represents a very promising candidate for H2 separating membrane in energy- and environmentally-related applications.},
annote = {Ivanova, Mariya E
Escolastico, Sonia
Balaguer, Maria
Palisaitis, Justinas
Sohn, Yoo Jung
Meulenberg, Wilhelm A
Guillon, Olivier
Mayer, Joachim
Serra, Jose M
ENG
England
2016/11/05 06:00
Sci Rep. 2016 Nov 4;6:34773. doi: 10.1038/srep34773.},
author = {Ivanova, M E and Escolastico, S and Balaguer, M and Palisaitis, J and Sohn, Y J and Meulenberg, W A and Guillon, O and Mayer, J and Serra, J M},
doi = {10.1038/srep34773},
isbn = {2045-2322 (Electronic)
2045-2322 (Linking)},
journal = {Sci Rep},
pages = {34773},
pmid = {27812011},
title = {{Hydrogen separation through tailored dual phase membranes with nominal composition BaCe0.8Eu0.2O3-delta:Ce0.8Y0.2O2-delta at intermediate temperatures}},
url = {http://www.ncbi.nlm.nih.gov/pubmed/27812011},
volume = {6},
year = {2016}
}
@unpublished{M.DownerA.Murugan,
author = {{M. Downer  A. Murugan}, A Brown},
file = {:Users/marc/Library/Application Support/Mendeley Desktop/Downloaded/M. Downer A. Murugan - Unknown - Hydrogen Purity Analysis for Fuel Cell Vehicles - Poster.pdf:pdf},
title = {{Hydrogen Purity Analysis for Fuel Cell Vehicles - Poster}}
}
@article{Pomerantz2009,
abstract = {The objective of this work was to fabricate palladium membranes with a Pd/Cu alloy top layer and to investigate the performance and long-term stability of the membranes in small concentrations of H2S. The Pd/Cu alloy membranes with compositions of 8, 18, and 19 wt {\%} copper were fabricated using the electroless deposition method on porous Inconel supports and characterized for several thousand hours in H2, helium, and 45-55 ppm H 2S/H2 mixtures in the temperature range of 350-500°C. Upon exposure to the H2S/H2 mixture at 500°C, the Pd/Cu membranes lost ∼80{\%} of the hydrogen permeance, because of the H 2S forming surface sulfides and blocking H2 adsorption sites, with the amount of permeance lost increasing with decreasing temperature. Reintroducing pure H2 recovered some of the hydrogen permeance, showing that some of the H2S poisoning was irreversible. The amount of irreversible poisoning increased as the H2S exposure time increased and the exposure temperature decreased, because of the exothermic nature of H2S adsorption. The helium leak rate of the Pd/Cu membranes decreased after the poisoning experiments, possibly because of sulfur that segregated to the grain boundaries of the Pd/Cu deposits. {\textcopyright} 2009 American Chemical Society.},
author = {Pomerantz, Natalie and {Yi Hua}, Ma},
doi = {10.1021/ie801947a},
file = {:Users/marc/Library/Application Support/Mendeley Desktop/Downloaded/Pomerantz, Yi Hua - 2009 - Effect of H 2S on the performance and long-term stability of PdCu membranes.pdf:pdf},
isbn = {3589451262},
issn = {08885885},
journal = {Industrial and Engineering Chemistry Research},
number = {8},
pages = {4030--4039},
title = {{Effect of H 2S on the performance and long-term stability of Pd/Cu membranes}},
volume = {48},
year = {2009}
}
@article{Zanjanchi2005,
abstract = {A novel dye incorporated into microporous material is introduced for optical humidity sensing. Methylene blue (MB) was encapsulated into the protonated mordenite zeolite (HMOR) via ion exchange reaction. The dye molecules are strongly retained in the channels of the zeolite and take part in the protonation/deprotonation reactions reversibly. The mechanism of the sensor is based on the protonation or deprotonation of the dye molecules which are associated with the desorption or adsorption of water molecules by the zeolite, respectively. The spectral changes due to different humidity levels are probed by diffuse reflectance spectroscopy. Discs prepared from 200 mg of the dye-loaded zeolite provide a thickness of ???0.8 mm and show good characteristics for an optical humidity sensor. They showed a linear response range from 9 to 92{\%} relative humidity (r2 {\textgreater} 0.99), and good stability and reversibility. The sensor operates at either of the two 650 or 745 nm bands but it exhibits a higher sensitivity for the measurements performed at 650 nm. The sensor demonstrates relatively fast response and recovery times about 2 min in the direction of adsorption and about 4 min in the direction of desorption of water. ?? 2004 Elsevier B.V. All rights reserved.},
author = {Zanjanchi, M. A. and Sohrabnezhad, Sh},
doi = {10.1016/j.snb.2004.07.009},
file = {:Users/marc/Library/Application Support/Mendeley Desktop/Downloaded/Zanjanchi, Sohrabnezhad - 2005 - Evaluation of methylene blue incorporated in zeolite for construction of an optical humidity sensor.pdf:pdf},
isbn = {0925-4005},
issn = {09254005},
journal = {Sensors and Actuators, B: Chemical},
keywords = {Diffuse reflectance spectroscopy,Humidity sensor,Methylene blue,Mordenite zeolite,Optical humidity sensing},
number = {2},
pages = {502--507},
title = {{Evaluation of methylene blue incorporated in zeolite for construction of an optical humidity sensor}},
volume = {105},
year = {2005}
}
@phdthesis{Faculty2014,
author = {Faculty, The Academic and Chandrasekhar, Nita and Fulfillment, In Partial},
file = {:Users/marc/Library/Application Support/Mendeley Desktop/Downloaded/Faculty, Chandrasekhar, Fulfillment - 2014 - Computational Study of Intermetallic and Alloy Membranes for Hydrogen Separation Computatio.pdf:pdf},
number = {May},
title = {{Computational Study of Intermetallic and Alloy Membranes for Hydrogen Separation Computational Study of Intermetallic and Alloy}},
year = {2014}
}
@article{Farjoo2016,
author = {Farjoo, Afrooz and Kuznicki, Steven M.},
doi = {10.1002/cjce.22589},
file = {:Users/marc/Library/Application Support/Mendeley Desktop/Downloaded/Farjoo, Kuznicki - 2016 - H2 separation using tubular stainless steel supported natural clinoptilolite membranes.pdf:pdf},
issn = {1939019X},
journal = {Canadian Journal of Chemical Engineering},
keywords = {clinoptilolite,hydrogen,mixed matrix membranes,stainless steel support},
number = {11},
pages = {2219--2224},
title = {{H2 separation using tubular stainless steel supported natural clinoptilolite membranes}},
volume = {94},
year = {2016}
}
@article{Wang2006,
author = {Wang, W P and Thomas, S and Zhang, X L and Pan, X L and Yang, W S and Xiong, G X},
doi = {10.1016/j.seppur.2006.04.007},
file = {:Users/marc/Library/Application Support/Mendeley Desktop/Downloaded/Wang et al. - 2006 - H2N2 gaseous mixture separation in dense Pd$\alpha$-Al2O3 hollow fiber membranes Experimental and simulation studies.pdf:pdf},
isbn = {13835866},
journal = {Separation and Purification Technology},
number = {1},
pages = {177--185},
title = {{H2/N2 gaseous mixture separation in dense Pd/$\alpha$-Al2O3 hollow fiber membranes: Experimental and simulation studies}},
volume = {52},
year = {2006}
}
@article{Liu2015a,
abstract = {This work investigates the influence of hydrothermal exposure on the separation performance of sol-gel derived cobalt oxide silica membranes for both single gases (He, H{\textless}inf{\textgreater}2{\textless}/inf{\textgreater}, CO{\textless}inf{\textgreater}2{\textless}/inf{\textgreater} and N{\textless}inf{\textgreater}2{\textless}/inf{\textgreater}) and binary gas mixtures (He/CO{\textless}inf{\textgreater}2{\textless}/inf{\textgreater}). The surface area of the materials slightly decreased after exposed to 25mol{\%} water vapour at 550°C for 100h. The membranes complied with activation transport mechanism before and after hydrothermal treatment (HT), and for both single gas and gas mixture permeation. Best values were achieved for He permeance of 3.3×10{\textless}sup{\textgreater}-7{\textless}/sup{\textgreater}molm{\textless}sup{\textgreater}-2{\textless}/sup{\textgreater}s{\textless}sup{\textgreater}-1{\textless}/sup{\textgreater}Pa{\textless}sup{\textgreater}-1{\textless}/sup{\textgreater} at 500°C and permselectivity of 479 for He/CO{\textless}inf{\textgreater}2{\textless}/inf{\textgreater}. After HT, the permeance of He and H{\textless}inf{\textgreater}2{\textless}/inf{\textgreater} decreased by 28{\%} and 22{\%} at 500°C, respectively, while the permeance of CO{\textless}inf{\textgreater}2{\textless}/inf{\textgreater} increased and resulting in a lower He/CO{\textless}inf{\textgreater}2{\textless}/inf{\textgreater} permselectivity of 190. For gas mixtures, the He purity in the permeate side increased from 62{\%} to 97{\%} at 200°C when the He feed molar concentration increased from 10{\%} to 50{\%} before HT. The He permeance remained unchanged with respect to He feed concentrations and was unaffected by the presence of CO{\textless}inf{\textgreater}2{\textless}/inf{\textgreater}, although a reduction of He permeance was observed after HT exposure. The He purity in the permeate side was similar before and after HT exposure as a function of the He concentration in the feed side. Hence, the membrane matrix underwent densification though the overall pore size distribution did not broaden after hydrothermal treatment.},
author = {Liu, Liang and Wang, David K. and Martens, Dana L. and Smart, Simon and da Costa, Jo{\~{a}}o C.Diniz},
doi = {10.1016/j.memsci.2015.06.058},
file = {:Users/marc/Library/Application Support/Mendeley Desktop/Downloaded/Liu et al. - 2015 - Binary gas mixture and hydrothermal stability investigation of cobalt silica membranes.pdf:pdf},
issn = {18733123},
journal = {Journal of Membrane Science},
keywords = {Cobalt silica membranes,Gas mixture,Gas separation,Hydrothermal stability},
pages = {470--477},
publisher = {Elsevier},
title = {{Binary gas mixture and hydrothermal stability investigation of cobalt silica membranes}},
url = {http://dx.doi.org/10.1016/j.memsci.2015.06.058},
volume = {493},
year = {2015}
}
@article{Flanagan1991a,
author = {Flanagan, T.},
doi = {10.1146/annurev.matsci.21.1.269},
file = {:Users/marc/Library/Application Support/Mendeley Desktop/Downloaded/Flanagan, Oates - 1991 - The Palladium-Hydrogen System.pdf:pdf},
issn = {00846600},
journal = {Annual Review of Materials Research},
number = {1},
pages = {269--304},
title = {{The Palladium-Hydrogen System}},
url = {http://matsci.annualreviews.org/cgi/doi/10.1146/annurev.matsci.21.1.269},
volume = {21},
year = {1991}
}
@article{Sakamoto1997,
author = {Sakamoto, F and Kinari, Y and Chen, F L and Sakamoto, Y},
journal = {International Journal of Hydrogen Energy},
number = {4},
pages = {369--375},
title = {{Hydrogen permeation through palladium alloy membranes in mixture gases of 10{\{}{\%}{\}} Nitrogen and Ammonia in the Hydrogen}},
volume = {22},
year = {1997}
}
@article{Huang2012a,
abstract = {Making use of the preferred adsorption affinity and capacity to CO(2) as well as the highly porous structure with huge cavities of 2.4 nm, a highly permeable and selective ZIF-95 molecular sieve membrane was developed for the separation of H(2) from CO(2).},
author = {Huang, Aisheng and Chen, Yifei and Wang, Nanyi and Hu, Zhongqiao and Jiang, Jianwen and Caro, J{\"{u}}rgen},
doi = {10.1039/c2cc35691k},
file = {:Users/marc/Library/Application Support/Mendeley Desktop/Downloaded/Huang et al. - 2012 - A highly permeable and selective zeolitic imidazolate framework ZIF-95 membrane for H2CO2 separation.pdf:pdf},
isbn = {1359-7345},
issn = {1359-7345},
journal = {Chemical Communications},
number = {89},
pages = {10981},
pmid = {23034485},
title = {{A highly permeable and selective zeolitic imidazolate framework ZIF-95 membrane for H2/CO2 separation}},
url = {http://xlink.rsc.org/?DOI=c2cc35691k},
volume = {48},
year = {2012}
}
@article{Dong2015,
author = {Dong, W and Zhu, S and Qu, H and Liu, X},
doi = {10.1179/1743676115Y.0000000013},
file = {:Users/marc/Library/Application Support/Mendeley Desktop/Downloaded/Dong et al. - 2015 - Microstructural evolution and mechanical properties of hot pressed WC – a -Al 2 O 3 with Y 2 O 3 and CeO 2.pdf:pdf},
isbn = {1743676115},
issn = {17436761},
keywords = {al2o3 composites,hot pressing,mechanical property,wc},
number = {0},
pages = {1--6},
title = {{Microstructural evolution and mechanical properties of hot pressed WC – a -Al 2 O 3 with Y 2 O 3 and CeO 2}},
volume = {00},
year = {2015}
}
@article{Huang2012b,
abstract = {A clear separation: A post-synthetic functionalization method is reported to obtain a highly permselective zeolitic imidazolate framework (ZIF-90) membrane. The intercrystalline defects of the ZIF-90 membrane are minimized to enhance the separation selectivity while a high permeance is maintained.},
author = {Huang, Aisheng and Wang, Nanyi and Kong, Chunlong and Caro, J{\"{u}}rgen},
doi = {10.1002/anie.201204621},
file = {:Users/marc/Library/Application Support/Mendeley Desktop/Downloaded/Huang et al. - 2012 - Organosilica-functionalized zeolitic imidazolate framework ZIF-90 membrane with high gas-separation performance.pdf:pdf},
isbn = {1521-3773},
issn = {14337851},
journal = {Angewandte Chemie - International Edition},
keywords = {covalent functionalization,membranes,metal-organic frameworks,molecular sieves,zeolites},
number = {42},
pages = {10551--10555},
pmid = {22987754},
title = {{Organosilica-functionalized zeolitic imidazolate framework ZIF-90 membrane with high gas-separation performance}},
volume = {51},
year = {2012}
}
@article{Qiao2010,
abstract = {Pd-Cu alloy membrane was prepared by electroless plating on porous stainless steel (PSS) support. Sol-gel derived ceria was introduced as the intermediate layer by a sol-dip-coating method to prevent intermetallic diffusion and to enhance the affinity between the support and membrane. Ceria layer moderates the pore size and porosity of the support effectively to meet the needs of the alloy layer deposition. Permeation test was carried out in a high temperature range, i.e. 573-773 K with a pressure difference of 0.1 MPa. The results showed that properly prepared ceria layer was effective as the diffusion barrier in the temperature range examined, which is suitable for water-gas shift reaction. The performance of the membranes in this work is also compared with that in our previous work in which pure Pd membrane and yttrium stabilized zirconia intermediate layer were examined. ?? 2009 Elsevier Ltd. All rights reserved.},
author = {Qiao, Ailing and Zhang, Ke and Tian, Ye and Xie, Lili and Luo, Huajiang and Lin, Y. S. and Li, Yongdan},
doi = {10.1016/j.fuel.2009.12.006},
file = {:Users/marc/Library/Application Support/Mendeley Desktop/Downloaded/Qiao et al. - 2010 - Hydrogen separation through palladium-copper membranes on porous stainless steel with sol-gel derived ceria as diff.pdf:pdf},
isbn = {9780816910526},
issn = {00162361},
journal = {Fuel},
keywords = {Ceria,Electroless plating,Hydrogen permeation,Intermediate layer,Palladium-copper alloy membrane,Sol-gel},
number = {6},
pages = {1274--1279},
publisher = {Elsevier Ltd},
title = {{Hydrogen separation through palladium-copper membranes on porous stainless steel with sol-gel derived ceria as diffusion barrier}},
url = {http://dx.doi.org/10.1016/j.fuel.2009.12.006},
volume = {89},
year = {2010}
}
@article{Wang2014b,
abstract = {Stable inorganic membranes perm-selective to hydrogen will find applications in a number of important industrial processes including water gas shift reaction for hydrogen production. This paper reports a highly stable bilayer MFI zeolite membrane with good hydrogen separation characteristics. The membrane consists of a thin (2$\mu$m) ZSM-5 layer on a thick (8$\mu$m) silicalite base layer supported on macroporous $\alpha$-alumina with a yttria stabilized zirconia intermediate barrier layer. The zeolitic pores of the thin ZSM-5 layer were narrowed by catalytic cracking deposition (CCD). At 500°C, the bilayer zeolite membrane exhibits H2permeance of about 1.2×10-7molm-2s-1Pa-1, with H2to CO2, CO and H2O vapor selectivity respectively of about 23, 28 and 180. The membrane shows slightly improved separation properties (H2permeance, H2/CO and H2/CO2selectivity) as the feed side pressure increases during the test of separation of industrial relevant simulated gas mixture (25{\%} H2: 25{\%} CO: 25{\%} H2O: 25{\%} CO2, with 400ppm H2S) at 500°C for 24 days. Membrane reactor made of the bilayer zeolite membrane shows stable performance under water gas shift reaction conditions (500°C, H2O/CO=3, GHSV=60,000h-1, with a ceria doped iron oxide catalyst) in terms of CO conversion, H2recovery, and H2permeance and selectivity of the zeolite membrane. The unprecedented thermal and chemical stability and good separation properties of the bilayer zeolite membrane are related to its unique bilayer structure and synthesis methods. {\textcopyright} 2013 Elsevier B.V.},
author = {Wang, Haibing and Dong, Xueliang and Lin, Y. S.},
doi = {10.1016/j.memsci.2013.08.030},
file = {:Users/marc/Library/Application Support/Mendeley Desktop/Downloaded/Wang, Dong, Lin - 2014 - Highly stable bilayer MFI zeolite membranes for high temperature hydrogen separation.pdf:pdf},
isbn = {0376-7388},
issn = {03767388},
journal = {Journal of Membrane Science},
keywords = {Hydrogen separation,Long-term stability,Microporous membranes,Stability,Water gas shift,Zeolite},
pages = {425--432},
publisher = {Elsevier},
title = {{Highly stable bilayer MFI zeolite membranes for high temperature hydrogen separation}},
url = {http://dx.doi.org/10.1016/j.memsci.2013.08.030},
volume = {450},
year = {2014}
}
@article{Downey2015,
abstract = {A novel method for the analysis of total sulphur-containing impurities in a hydrogen matrix has been developed. This method has a limit of detection (LoD) significantly lower than that maximum amount fraction for sulphur-containing compounds (4 nmol mol(-1)) specified by the international standard for hydrogen to be used in fuel cell vehicles (ISO 14687-2). To measure the LoD for this method, a novel gas standard containing five different sulphur-containing compounds at low nmol mol(-1) amount fractions has been gravimetrically prepared. Stable primary gas standards that are traceable to the SI were used to successfully validate the amount fractions of the sulphur-containing compounds in this gas standards using gas chromatography with flame ionisation detection (GC-FID) and sulphur chemiluminescence detection (GC-SCD).},
annote = {Downey, Michael L
Murugan, Arul
Bartlett, Sam
Brown, Andrew S
eng
Research Support, Non-U.S. Gov't
Netherlands
2014/12/21 06:00
J Chromatogr A. 2015 Jan 2;1375:140-5. doi: 10.1016/j.chroma.2014.11.076. Epub 2014 Dec 4.},
author = {Downey, M L and Murugan, A and Bartlett, S and Brown, A S},
doi = {10.1016/j.chroma.2014.11.076},
isbn = {1873-3778 (Electronic)
0021-9673 (Linking)},
journal = {J Chromatogr A},
keywords = {*Hydrogen,*Luminescent Measurements,Flame Ionization/*methods/standards,Gas chromatography,Gases/standards,Hydrogen,Purity,Quality,Reference Standards,Sulfur Compounds/*analysis,Total sulfur,Trace analysis},
pages = {140--145},
pmid = {25526978},
title = {{A novel method for measuring trace amounts of total sulphur-containing compounds in hydrogen}},
url = {http://www.ncbi.nlm.nih.gov/pubmed/25526978},
volume = {1375},
year = {2015}
}
@article{Bing2016,
abstract = {The designed preparation of pure and Au-loaded SnO2 hollow multilayered nanosheets for CO detection has been realized via a rational combination of different purposeful multi-step synthetic route. The synthetic step-dependent shape and structure evolution has been investigated and the formation mechanism for the unique hollow nanosheets is also discussed. The characterized results confirm that flowerlike SnO2 nanosheets with a polycrystalline rutile structure are assembled from hexagonal mesoporous and multilayered walls, which can further be subdivided into SnO2 nanoparticles as structural subunits. Furthermore, gas sensors based on the as-prepared pure and Au-loaded SnO2 have also been fabricated. The Au-loaded SnO2 hollow nanosheets show significantly lower working temperature, faster response-recovery and improved selectivity to CO in comparison with that of pure SnO2. Besides structural merits including hollow and easily penetrated multilayered walls, which facilitate the transport rate and augment the adsorption quantity of gas molecule, the significant improvement of the sensing properties is also attributed to the catalytic effect of Au.},
author = {Bing, Yifei and Zeng, Yi and Feng, Shengrao and Qiao, Liang and Wang, Yanzhe and Zheng, Weitao},
doi = {10.1016/j.snb.2015.12.065},
file = {:Users/marc/Library/Application Support/Mendeley Desktop/Downloaded/Bing et al. - 2016 - Multistep assembly of Au-loaded SnO2 hollow multilayered nanosheets for high-performance CO detection.pdf:pdf},
isbn = {09254005},
issn = {09254005},
journal = {Sensors and Actuators, B: Chemical},
keywords = {Au-sensitization,Gas sensor,Nonspherical hollow structures,SnO2},
pages = {362--372},
publisher = {Elsevier B.V.},
title = {{Multistep assembly of Au-loaded SnO2 hollow multilayered nanosheets for high-performance CO detection}},
url = {http://dx.doi.org/10.1016/j.snb.2015.12.065},
volume = {227},
year = {2016}
}
@article{Yamaura2004,
abstract = {The (Ni/sub 0.6/Nb/sub 0.4/)/sub 45/Zr/sub 50/X/sub 5 /(X=Al, Co, Cu, P, Pd, Si, Sn, Ta or Ti) alloy ribbons were produced by the melt-spinning technique. All ribbon specimens were confirmed to have a single amorphous phase by XRD analysis. The crystallization temperature of the melt-spun (Ni/sub 0.6/Nb/sub 0.4 /)/sub 45/Zr/sub 50/X/sub 5/(X=Al, P, Pd, Si or Sn) amorphous alloys are higher than that of the (Ni/sub 0.6/Nb/sub 0.4 /)/sub 50/Zr/sub 50/ amorphous alloy (727 K). Although the hydrogen permeability of the (Ni/sub 0.6/Nb/sub 0.4/)/sub 45/Zr/sub 50/X/sub 5 /(X=Si, Sn, Ta or Ti) amorphous alloys could not be measured due to severe embrittlement during the permeation test, the (Ni/sub 0.6/Nb /sub 0.4/)/sub 45/Zr/sub 50/X/sub 5/(X=Al, Co, Cu, P or Pd) amorphous alloys had high ductility which was enough to measure the permeability. The hydrogen permeabilities of the (Ni/sub 0.6/Nb /sub 0.4/)/sub 45/Zr/sub 50/Co/sub 5/ and the (Ni /sub 0.6/Nb/sub 0.4/)/sub 45/Zr/sub 50/Cu/sub 5 / amorphous alloys were 2.46*10/sup -8/ and 2.34*10/sup -8/[mol middot m/sup -1/ middot s/sup -1 / middot Pa/sup -1/2/] at 673 K, respectively. The (Ni/sub 0.6 /Nb/sub 0.4/)/sub 45/Zr/sub 50/P/sub 5/ amorphous alloy possesses the lowest permeability of 1.36*10/sup -8/(mol middot m/sup -1/ middot s/sup -1 / middot Pa-1/2] at 673 K among the alloys where the permeability was measured. The reduction of the permeability in the (Ni/sub 0.6/Nb/sub 0.4/)/sub 45/Zr/sub 50/P/sub 5 / amorphous alloy is thought to be due to the preferential formation of Zr-P atomic pairs which may suppress the hydrogen solubility and hydrogen diffusivity in the alloy. Since the heat of mixing for Zr-P atomic pairs is negatively larger than that for other pairs such as Ni-P and Nb-P. It is concluded that the Ni-Nb-Zr-X(X=Co or Cu) amorphous alloys have high potential to hydrogen permeable membranes.},
author = {Yamaura, Shin Ichi and Shimpo, Yoichiro and Okouchi, Hitoshi and Nishida, Motonori and Kajita, Osamu and Inoue, Akihisa},
doi = {10.2320/jinstmet.68.1039},
file = {:Users/marc/Library/Application Support/Mendeley Desktop/Downloaded/Yamaura et al. - 2004 - The effect of additional elements on hydrogen permeation properties of melt-spun Ni-Nb-Zr amorphous alloys.pdf:pdf},
isbn = {0021-4876},
issn = {00214876},
journal = {Nippon Kinzoku Gakkaishi/Journal of the Japan Institute of Metals},
keywords = {Amorphous,Hydrogen permeation,Melt-spinning,Membrane,Separation},
number = {12},
pages = {1039--1042},
title = {{The effect of additional elements on hydrogen permeation properties of melt-spun Ni-Nb-Zr amorphous alloys}},
volume = {68},
year = {2004}
}
@article{TanhJeazet2012,
abstract = {Mixed-matrix membranes (MMMs) with metal-organic frameworks (MOFs) as additives (fillers) exhibit enhanced gas permeabilities and possibly also selectivities when compared to the pure polymer. Polyimides (Matrimid[registered sign]) and polysulfones are popular polymer matrices for MOF fillers. Presently investigated MOFs for MMMs include [Cu(SiF6)(4,4[prime or minute]-BIPY)2], [Cu3(BTC)2(H2O)3] (HKUST-1, Cu-BTC), [Cu(BDC)(DMF)], [Zn4O(BDC)3] (MOF-5), [Zn(2-methylimidazolate)2] (ZIF-8), [Zn(purinate)2] (ZIF-20), [Zn(2-carboxyaldehyde imidazolate)2] (ZIF-90), Mn(HCOO)2, [Al(BDC)([small mu ]-OH)] (MIL-53(Al)), [Al(NH2-BDC)([small mu ]-OH)] (NH2-MIL-53(Al)) and [Cr3O(BDC)3(F,OH)(H2O)2] (MIL-101) (4,4[prime or minute]-BIPY = 4,4[prime or minute]-bipyridine, BTC = benzene-1,3,5-tricarboxylate, BDC = benzene-1,4-dicarboxylate, terephthalate). MOF particle adhesion to polyimide and polysulfone organic polymers does not represent a problem. MOF-polymer MMMs are investigated for the permeability of the single gases H2, N2, O2, CH4, CO2 and of the gas mixtures O2/N2, H2/CH4, CO2/CH4, H2/CO2, CH4/N2 and CO2/N2 (preferentially permeating gas named first). Permeability increases can be traced to the MOF porosity. Since the porosity of MOFs can be tuned very precisely, which is not possible with polymeric material, MMMs offer the opportunity of significantly increasing the selectivity compared to the pure polymeric matrix. Additionally in most of the cases the permeability is increased for MMM membranes compared to the pure polymer. Addition of MOFs to polymers in MMMs easily yields performances similar to the best polymer membranes and gives higher selectivities than those reported to date for any pure MOF membrane for the same gas separation. MOF-polymer MMMs allow for easier synthesis and handability compared to pure MOF membranes.},
author = {{Tanh Jeazet}, Harold B. and Staudt, Claudia and Janiak, Christoph},
doi = {10.1039/c2dt31550e},
file = {:Users/marc/Library/Application Support/Mendeley Desktop/Downloaded/Tanh Jeazet, Staudt, Janiak - 2012 - Metal-organic frameworks in mixed-matrix membranes for gas separation.pdf:pdf},
isbn = {1477-9226},
issn = {14779234},
journal = {Dalton Transactions},
number = {46},
pages = {14003--14027},
pmid = {23070078},
title = {{Metal-organic frameworks in mixed-matrix membranes for gas separation}},
volume = {41},
year = {2012}
}
@article{Chu2017,
abstract = {Graphene-Zn2SnO4 composites (G-Zn2SnO4) with different graphene content (0.1, 0.25, 0.5, 0.75, 1, and 3 wt{\%}) were prepared by hydrothermal method. The composites were characterized by Fourier transform infrared spectroscopy (FT-IR), X-ray diffraction (XRD), Scanning electron microscopy (SEM), X-ray photoelectron spectrum (XPS) and Raman spectrum. The gas sensing properties of the composite materials were investigated. The results revealed that the sensor based on 0.5 wt{\%} Graphene-Zn2SnO4 exhibited high response to formaldehyde vapor when the operating temperature of the sensor was 20 °C and the response to 10 ppm formaldehyde vapor attained 1.1, and the response to 1000 ppm formaldehyde vapor increased 18.9-times comparing with that of pure Zn2SnO4, which made it possible to detect formaldehyde vapor at low temperature.},
author = {Chu, Xiangfeng and Hu, Ruxue and Wang, Jiulin and Dong, Yongping and Zhang, Wangbing and Bai, Linshan and Sun, Wenqi},
doi = {10.1016/j.snb.2017.04.086},
file = {:Users/marc/Library/Application Support/Mendeley Desktop/Downloaded/Chu et al. - 2017 - Preparation and gas sensing properties of graphene-Zn2SnO4 composite materials.pdf:pdf},
issn = {09254005},
journal = {Sensors and Actuators, B: Chemical},
keywords = {Formaldehyde,Gas sensor,Graphene,Zn2SnO4},
number = {2},
pages = {120--126},
publisher = {Elsevier B.V.},
title = {{Preparation and gas sensing properties of graphene-Zn2SnO4 composite materials}},
url = {http://dx.doi.org/10.1016/j.snb.2017.04.086},
volume = {251},
year = {2017}
}
@article{Ge2012,
abstract = {ZIF-8 thin layer has been synthesized on the asymmetric porous$\backslash$npolyethersulfone (PES) substrate via secondary seeded growth. Continuous$\backslash$nand dense ZIF-8 layer, containing microcavities, has good affinity with$\backslash$nthe PES support. Single gas permeance was measured for H-2, N-2, CH4,$\backslash$nO-2, and Ar at different pressure gradients and temperatures. Molecular$\backslash$nsieving separation has been achieved for selectively separating hydrogen$\backslash$nfrom larger gases. At 333 K, the H-2 permeance can reach similar to 4 x$\backslash$n10(-7) mol m(-2) s(-1) Pa-1, and the ideal separation factors of H-2$\backslash$nfrom Ar, O-2, N-2, and CH4, are 9.7, 10.8, 9.9, and 10.7, respectively.$\backslash$nLong-term hydrogen permeance and H-2/N-2 separation performance show the$\backslash$nstable permeability of the derived membranes.},
author = {Ge, Lei and Zhou, Wei and Du, Aijun and Zhu, Zhonghua},
doi = {10.1021/jp3035105},
file = {:Users/marc/Library/Application Support/Mendeley Desktop/Downloaded/Ge et al. - 2012 - Porous polyethersulfone-supported zeolitic imidazolate framework membranes for hydrogen separation.pdf:pdf},
isbn = {1932-7447$\backslash$n1932-7455},
issn = {19327447},
journal = {Journal of Physical Chemistry C},
number = {24},
pages = {13264--13270},
title = {{Porous polyethersulfone-supported zeolitic imidazolate framework membranes for hydrogen separation}},
volume = {116},
year = {2012}
}
@article{Cardoso2018,
abstract = {A new AM-3 membrane was prepared on a stainless-steel support for potential application in the separation of light gases, particularly hydrogen containing mixtures. It was dynamically characterized by permeation assays using H2, He, N2, CO2, and O2at fixed and programmed temperatures (between 304 and 394 K), and transmembrane pressure drops from 0.5 to 1.5 bar. The experimental results disclosed high selectivity of the AM-3 membrane towards hydrogen. In terms of transport mechanisms, they evidenced an activated behavior typical of surface diffusion, and a small contribution of macro-defects. The existence of intercrystalline micro-defects was revealed by the permeation of N2, O2, and CO2, whose kinetic diameters are larger than the pore diameter of AM-3. Gas permeation was accurately modeled based on the Maxwell-Stefan approach for surface diffusion in micropores, with additional terms for Knudsen and viscous fluxes through meso- and macro-defects. The global deviation achieved for the five gases was only 3.42{\%}. The calculated results demonstrated that: viscous flow prevailed at low temperature (304 K), surface diffusion dominated when temperature increased, Knudsen transport was residual, the flux through defects predominated at 304 K (53.8–73.5{\%} of total flux) but fell below 15{\%} for temperatures above 370 K, and the influence of the support was negligible.},
author = {Cardoso, Sim{\~{a}}o P. and Lin, Zhi and Portugal, In{\^{e}}s and Rodrigues, Al{\'{i}}rio E. and Silva, Carlos M.},
doi = {10.1016/j.micromeso.2017.11.008},
file = {:Users/marc/Library/Application Support/Mendeley Desktop/Downloaded/Cardoso et al. - 2018 - Synthesis, dynamic characterization, and modeling studies of an AM-3 membrane for light gases separation.pdf:pdf},
issn = {13871811},
journal = {Microporous and Mesoporous Materials},
keywords = {Gas permeation,Maxwell-Stefan approach,Modeling,Permeation mechanism,Titanosilicate membrane},
number = {November 2017},
pages = {170--180},
publisher = {Elsevier},
title = {{Synthesis, dynamic characterization, and modeling studies of an AM-3 membrane for light gases separation}},
url = {https://doi.org/10.1016/j.micromeso.2017.11.008},
volume = {261},
year = {2018}
}
@article{Seshimo2008,
author = {Seshimo, Masahiro and Ozawa, Minoru and Sone, Masato and Sakurai, Makoto and Kameyama, Hideo},
doi = {10.1016/j.memsci.2008.07.007},
isbn = {03767388},
journal = {Journal of Membrane Science},
number = {1-2},
pages = {181--187},
title = {{Fabrication of a novel Pd/$\gamma$-alumina graded membrane by electroless plating on nanoporous $\gamma$-alumina}},
volume = {324},
year = {2008}
}
@article{Kosinov2016,
abstract = {The latest advances in the field of zeolitic membranes for gas separation are critically reviewed with special emphasis on new synthetic protocols. After introducing the most relevant aspects to membrane performance, including adsorption trends, permeation mechanisms and support effects, we review recent achievements in membrane synthesis and discuss in detail the effect of zeolite topology and chemical composition on membrane gas separation. We pay special attention to promising 8MR high-silica structures. As the formation of defects during synthesis remains one of the major challenges for large-scale production of such membranes, we review various approaches to either limit defect formation or decrease their adverse effect by post-synthesis modification. Finally, the current challenges for this field of research are summarized and an outlook is offered on approaches to decrease fabrication costs, improve reproducibility and rational design of zeolite membranes.},
author = {Kosinov, Nikolay and Gascon, Jorge and Kapteijn, Freek and Hensen, Emiel J M},
doi = {10.1016/j.memsci.2015.10.049},
file = {:Users/marc/Library/Application Support/Mendeley Desktop/Downloaded/Kosinov et al. - 2016 - Recent developments in zeolite membranes for gas separation.pdf:pdf},
isbn = {0376-7388},
issn = {18733123},
journal = {Journal of Membrane Science},
keywords = {8MR zeolites,Gas separation,High-silica zeolites,Zeolite membranes},
pages = {65--79},
publisher = {Elsevier},
title = {{Recent developments in zeolite membranes for gas separation}},
url = {http://dx.doi.org/10.1016/j.memsci.2015.10.049},
volume = {499},
year = {2016}
}
@article{Gielens2006,
author = {Gielens, F C and Knibbeler, R J J and Duysinx, P F J and Tong, H D and Vorstman, M A G and Keurentjes, J T F},
doi = {10.1016/j.memsci.2005.12.002},
file = {:Users/marc/Library/Application Support/Mendeley Desktop/Downloaded/Gielens et al. - 2006 - Influence of steam and carbon dioxide on the hydrogen flux through thin PdAg and Pd membranes.pdf:pdf},
isbn = {03767388},
journal = {Journal of Membrane Science},
number = {1-2},
pages = {176--185},
title = {{Influence of steam and carbon dioxide on the hydrogen flux through thin Pd/Ag and Pd membranes}},
volume = {279},
year = {2006}
}
@article{Kniep2010,
abstract = {SrCe(0.95)Tm(0.05)O(3-delta) perovskite-type ceramic membranes offer$\backslash$nhigh hydrogen selectivity, thermal stability, mixed protonic-electronic$\backslash$nconductivity, and mechanical strength at temperatures above 600 degrees$\backslash$nC. However, in order for the SrCeO(3)-based membranes to be used in$\backslash$nindustrial applications, the chemical stability of the membranes in$\backslash$nvarious environments must be improved. The effect of doping zirconium on$\backslash$nthe chemical stability, lattice Structure, protonic and electronic$\backslash$nconductivity, and hydrogen permeation properties of$\backslash$nSrCe(0.95-x)Zr(x)Tm(0.05)O(3-delta) (0 {\textless}= x {\textless}= 0.40) was studied. X-ray$\backslash$ndiffraction analysis verifies that all samples consist of a single$\backslash$nperovskite phase. Doping zirconium in SrCe(0.95)Tm(0.05)O(3-delta)$\backslash$nresults in a decrease in both the protonic and electronic conductivity$\backslash$nof the materials under reducing conditions, and a more significant$\backslash$ndecrease in hydrogen permeability of the membrane in CO(2) free gas$\backslash$nstreams. In a CO(2)-containing environment$\backslash$nSrCe(0.75-x)Zr(0.20)-Tm(0.05)O(3-delta) membranes have a larger$\backslash$nsteady-state H(2) flux and superior chemical stability over$\backslash$nSrCe(0.95)Tm(0.05)O(3-delta) membranes.},
author = {Kniep, Jay and Lin, Y. S.},
doi = {10.1021/ie9015182},
file = {:Users/marc/Library/Application Support/Mendeley Desktop/Downloaded/Kniep, Lin - 2010 - Effect of zirconium doping on hydrogen permeation through strontium cerate membranes.pdf:pdf},
isbn = {0888-5885},
issn = {08885885},
journal = {Industrial and Engineering Chemistry Research},
number = {6},
pages = {2768--2774},
title = {{Effect of zirconium doping on hydrogen permeation through strontium cerate membranes}},
volume = {49},
year = {2010}
}
@article{Tong2005,
abstract = {Submicron thick palladium–silver alloy films with 23 wt.{\%} of silver (Pd–Ag23) have been synthesized by simultaneous sputtering from pure targets of Pd and Ag. Full characterization of the deposited films was performed by using X-ray photoelectron spectroscopy, high-resolution scanning electron microscopy, high-resolution transmission electron microscopy, and X-ray diffraction. The analytical results revealed that the deposited Pd–Ag alloy had a Ag content of about 21–22 wt.{\%}, very close to and within measurement error of the expected Ag content of 23 wt.{\%}. The as-deposited Pd–Ag alloy had a fine microstructure. The characterized Pd–Ag alloy films were then deposited on a supporting microsieve to form Pd–Ag membranes for hydrogen separation. The submicron thick Pd–Ag membranes obtained high separation fluxes up to 4 mol H2/m2 s with a selectivity higher than 1500 for hydrogen over helium.},
author = {Tong, H.D. and vanden Berg, A.H.J. and Gardeniers, J.G.E. and Jansen, H.V. and Gielens, F.C. and Elwenspoek, M.C.},
doi = {10.1016/j.tsf.2004.11.179},
file = {:Users/marc/Library/Application Support/Mendeley Desktop/Downloaded/Tong et al. - 2005 - Preparation of palladium–silver alloy films by a dual-sputtering technique and its application in hydrogen separa.pdf:pdf},
isbn = {00406090},
issn = {00406090},
journal = {Thin Solid Films},
keywords = {Alloys,Hydrogen,Palladium,Sputtering},
number = {1-2},
pages = {89--94},
title = {{Preparation of palladium–silver alloy films by a dual-sputtering technique and its application in hydrogen separation membrane}},
url = {http://www.sciencedirect.com/science/article/pii/S0040609004017730},
volume = {479},
year = {2005}
}
@article{Unemoto2007,
author = {Unemoto, A and Kaimai, A and Sato, K and Otake, T and Yashiro, K and Mizusaki, J and Kawada, T and Tsuneki, T and Shirasaki, Y and Yasuda, I},
doi = {10.1016/j.ijhydene.2007.03.037},
isbn = {03603199},
journal = {International Journal of Hydrogen Energy},
number = {14},
pages = {2881--2887},
title = {{The effect of co-existing gases from the process of steam reforming reaction on hydrogen permeability of palladium alloy membrane at high temperatures}},
volume = {32},
year = {2007}
}
@article{Topologies2012,
author = {Topologies, M O F and Stock, Norbert and Biswas, Shyam},
doi = {10.1021/cr200304e},
file = {:Users/marc/Library/Application Support/Mendeley Desktop/Downloaded/Topologies, Stock, Biswas - 2012 - Synthesis of Metal-Organic Frameworks ( MOFs ) Routes to Various.pdf:pdf},
isbn = {1520-6890 (Electronic) 0009-2665 (Linking)},
issn = {0009-2665},
pages = {933--969},
pmid = {22098087},
title = {{Synthesis of Metal-Organic Frameworks ( MOFs ): Routes to Various}},
year = {2012}
}
@article{Fang2014a,
abstract = {Ni-BaZr0.1Ce0.7Y0.1Yb0.1O3-$\delta$ (Ni-BZCYYb) membrane shows improved and stable performance in dry H2 and CO2 (Fang et al., ACS Appl. Mater. Interfaces 6 (2014) 725-730). However, the stream from steam methane reforming contains high contents of H2O, CO2, and CO, which poses crueler challenges to the chemical stability of Ni-BZCYYb membrane than dry H2 and CO2. In this work, we tested the Ni-BZCYYb membrane in wet H2 and CO2 which generated high content of H2O and CO due to reverse water gas shift (RWGS) reaction at high temperature. High content of H2O improves the proton conductivity of BZCYYb and hydrogen transport through the membrane. On the other hand, H2 content reduction and decomposition of BZCYYb promoted by high content of H2O lead to performance degradation. The steady-state hydrogen flux may increase or decrease depending on the balance among these effects. Besides, CO-induced Ni corrosion was found in both surface and bulk due to metal dusting. In general, Ni-BZCYYb membrane still displayed much better performance stability in wet H2 and CO2 than Ni-BaCe0.8Y0.2O3-$\delta$ and Ni-BaZr0.1Ce0.7Y0.2O3-$\delta$ composites, making it a candidate material system for further studies aimed at membrane processing of hydrocarbons. {\textcopyright} 2014 Elsevier B.V.},
author = {Fang, Shumin and Brinkman, Kyle S. and Chen, Fanglin},
doi = {10.1016/j.memsci.2014.05.008},
file = {:Users/marc/Library/Application Support/Mendeley Desktop/Downloaded/Fang, Brinkman, Chen - 2014 - Hydrogen permeability and chemical stability of Ni-BaZr0.1Ce0.7Y0.1Yb0.1O3-$\delta$ membrane in concentrated H2O.pdf:pdf},
issn = {18733123},
journal = {Journal of Membrane Science},
keywords = {Barium cerate,Chemical stability,High temperature proton conductor,Hydrogen permeation,Reverse water gas shift reaction},
pages = {85--92},
publisher = {Elsevier},
title = {{Hydrogen permeability and chemical stability of Ni-BaZr0.1Ce0.7Y0.1Yb0.1O3-$\delta$ membrane in concentrated H2O and CO2}},
url = {http://dx.doi.org/10.1016/j.memsci.2014.05.008},
volume = {467},
year = {2014}
}
@article{De-enJiangShengDai2009,
author = {{De-en Jiang  Sheng Dai}, Valentino R Cooper},
chapter = {4019},
file = {:Users/marc/Library/Application Support/Mendeley Desktop/Downloaded/De-en Jiang Sheng Dai - 2009 - Porous Graphene as the Ultimate Membrane for Gas Separation.pdf:pdf},
journal = {Nano Letters},
number = {12},
pages = {4019--4024},
title = {{Porous Graphene as the Ultimate Membrane for Gas Separation}},
volume = {9},
year = {2009}
}
@article{Gryaznov2000,
abstract = {AbstractView full textDownload full textRelated$\backslash$n $\backslash$n $\backslash$n$\backslash$n$\backslash$n$\backslash$n$\backslash$n$\backslash$n$\backslash$n$\backslash$n $\backslash$n$\backslash$n$\backslash$n $\backslash$n$\backslash$n$\backslash$n$\backslash$n$\backslash$n$\backslash$n var addthis{\_}config = {\{}$\backslash$n ui{\_}cobrand: "Taylor {\&} Francis Online",$\backslash$n services{\_}compact: "citeulike,netvibes,twitter,technorati,delicious,linkedin,facebook,stumbleupon,digg,google,more",$\backslash$n pubid: "ra-4dff56cd6bb1830b"$\backslash$n {\}};$\backslash$n$\backslash$n Share on facebook$\backslash$n Share on twitter$\backslash$n Share on email$\backslash$n More Sharing Services$\backslash$n $\backslash$n var addthis{\_}config = {\{}"data{\_}track{\_}addressbar":true,"ui{\_}click":true{\}};$\backslash$n $\backslash$n $\backslash$n $\backslash$n Add to shortlist$\backslash$n $\backslash$n $\backslash$n$\backslash$n $\backslash$n$\backslash$n $\backslash$n $\backslash$n $\backslash$n $\backslash$n Link$\backslash$n $\backslash$n$\backslash$n $\backslash$n $\backslash$n $\backslash$n Permalink$\backslash$n $\backslash$n$\backslash$n $\backslash$n $\backslash$n $\backslash$n$\backslash$n $\backslash$n$\backslash$n$\backslash$n$\backslash$n $\backslash$n $\backslash$n $\backslash$n$\backslash$n$\backslash$n$\backslash$n$\backslash$n $\backslash$n $\backslash$n http://dx.doi.org/10.1081/SPM-100100008$\backslash$n $\backslash$n $\backslash$n $\backslash$n $\backslash$n $\backslash$n $\backslash$n $\backslash$n $\backslash$n $\backslash$n$\backslash$n $\backslash$n $\backslash$n $\backslash$n $\backslash$n Download Citation$\backslash$n $\backslash$n $\backslash$n$\backslash$n $\backslash$n $\backslash$n $\backslash$n Recommend to:$\backslash$n $\backslash$n $\backslash$n $\backslash$n $\backslash$n $\backslash$n$\backslash$n $\backslash$n$\backslash$n $\backslash$n $\backslash$n $\backslash$n $\backslash$n $\backslash$n$\backslash$n A friend},
author = {Gryaznov, V.},
doi = {10.1081/SPM-100100008},
file = {:Users/marc/Library/Application Support/Mendeley Desktop/Downloaded/Gryaznov - 2000 - Metal Containing Membranes for the Production of Ultrapure Hydrogen and the Recovery of Hydrogen Isotopes.pdf:pdf},
isbn = {0360-2540},
issn = {1542-2119},
journal = {Separation {\&} Purification Reviews},
number = {2},
pages = {171--187},
title = {{Metal Containing Membranes for the Production of Ultrapure Hydrogen and the Recovery of Hydrogen Isotopes}},
volume = {29},
year = {2000}
}
@article{Pohle2011,
abstract = {Floating gate FET (FGFET) gas sensors based on work function readout allows usage of a wide range of materials to be included as sensing materials. Metal-organic frameworks (MOFs) are a new group of porous materials with extreme high inner surface area that have already shown high potential for applications in gas storage and separation, catalysis and sensing. In this work, MOFs are investigated for the first time with work function readout for gas sensing applications. The results demonstrate the high potential of MOFs for use as gas receptor materials in field-effect based gas sensors. ?? 2011 Published by Elsevier Ltd.},
author = {Pohle, R. and Tawil, A. and Davydovskaya, P. and Fleischer, M.},
doi = {10.1016/j.proeng.2011.12.027},
file = {:Users/marc/Library/Application Support/Mendeley Desktop/Downloaded/Pohle et al. - 2011 - Metal organic frameworks as promising high surface area material for work function gas sensors.pdf:pdf},
issn = {18777058},
journal = {Procedia Engineering},
keywords = {Gas sensor,Metal organic framework,Work function},
pages = {108--111},
publisher = {Elsevier B.V.},
title = {{Metal organic frameworks as promising high surface area material for work function gas sensors}},
url = {http://dx.doi.org/10.1016/j.proeng.2011.12.027},
volume = {25},
year = {2011}
}
@article{Howard2008,
author = {Howard, B. H. and Morreale, B. D.},
doi = {10.1179/174892309X12519750237717},
file = {:Users/marc/Library/Application Support/Mendeley Desktop/Downloaded/Howard, Morreale - 2008 - Effect of H sub2sub S on performance of Pd sub4sub Pt alloy membranes.pdf:pdf},
isbn = {1251975023771},
issn = {1748-9237},
journal = {Energy Materials},
number = {3},
pages = {177--185},
title = {{Effect of H {\textless}sub{\textgreater}2{\textless}/sub{\textgreater} S on performance of Pd {\textless}sub{\textgreater}4{\textless}/sub{\textgreater} Pt alloy membranes}},
volume = {3},
year = {2008}
}
@article{Kirk-Othmer-Vol1840,
author = {{Kirk-Othmer - Vol}, Ammonia},
doi = {10.1002/14356007.a02_143},
isbn = {0824716302},
journal = {World},
pages = {678--710},
title = {{Ammonia 1.}},
volume = {2},
year = {1840}
}
@article{Wei2007,
abstract = {An asymmetric carbon membrane was prepared by coating alcohol solution of novolac phenol-formaldehyde resin containing a little hexamine on a porous resin support from the same material. After drying in air for two days at room temperature, the coated support was heated at 150 °C for 1 h in air (heating rate: 0.5 °C/min) and then carbonized at 800 °C (heating rate: 0.5 °C/min) in Ar atmosphere. The support and the membrane layer were carbonized simultaneously. The coating-pyrolysis cycle only needed one time. SEM photographs showed the carbon membrane had an asymmetric structure formed by a dense skin layer with a thickness of around 35 $\mu$m and a porous substrate. Pure gases of different molecular size (H2, CO2, O2, N2and CH4) were used to test the carbon membrane permeance property. The membrane has a good selectivity for H2/N2and H2/CH4with H2permeance of 4.05 × 10-6cm3cm-2s-1cmHg-1. The permeance is independent of pressure. The results indicate that the gases transport through the membrane according to molecular sieve mechanism. {\textcopyright} 2007 Elsevier B.V. All rights reserved.},
author = {Wei, Wei and Qin, Guotong and Hu, Haoquan and You, Longbo and Chen, Guohua},
doi = {10.1016/j.memsci.2007.06.055},
file = {:Users/marc/Library/Application Support/Mendeley Desktop/Downloaded/Wei et al. - 2007 - Preparation of supported carbon molecular sieve membrane from novolac phenol-formaldehyde resin.pdf:pdf},
isbn = {0376-7388},
issn = {03767388},
journal = {Journal of Membrane Science},
keywords = {Carbon membrane,Gas separation,Molecular sieve,Phenol-formaldehyde resin},
number = {1-2},
pages = {80--85},
title = {{Preparation of supported carbon molecular sieve membrane from novolac phenol-formaldehyde resin}},
volume = {303},
year = {2007}
}
@article{Sun2006,
author = {Sun, G B and Hidajat, K and Kawi, S},
doi = {10.1016/j.memsci.2006.07.015},
isbn = {03767388},
journal = {Journal of Membrane Science},
number = {1-2},
pages = {110--119},
title = {{Ultra thin Pd membrane on {\$}\alpha{\$}-Al2O3 hollow fiber by electroless plating: High permeance and selectivity}},
volume = {284},
year = {2006}
}
@article{Mohammed2017,
author = {Mohammed, Mohammed H. and Ajeel, Fouad N. and Khudhair, Alaa M.},
doi = {10.1016/j.cjph.2017.05.013},
file = {:Users/marc/Library/Application Support/Mendeley Desktop/Downloaded/Mohammed, Ajeel, Khudhair - 2017 - Adsorption of gas molecules on graphene nanoflakes and its implication as a gas nanosensor by DFT inv.pdf:pdf},
issn = {05779073},
journal = {Chinese Journal of Physics},
keywords = {GNFs,DFT,Band gap,DOS,Adsorption energies},
number = {4},
pages = {1576--1582},
publisher = {Elsevier B.V.},
title = {{Adsorption of gas molecules on graphene nanoflakes and its implication as a gas nanosensor by DFT investigations}},
url = {http://linkinghub.elsevier.com/retrieve/pii/S057790731730223X},
volume = {55},
year = {2017}
}
@article{Ariga2013,
abstract = {The Langmuir-Blodgett (LB) technique is known as an elegant method for fabrication of well-defined layered structures with molecular level precision. Since its discovery the LB method has made an indispensable contribution to surface science, physical chemistry, materials chemistry and nanotechnology. However, recent trends in research might suggest the decline of the LB method as alternate methods for film fabrication such as layer-by-layer (LbL) assembly have emerged. Is LB film technology obsolete? This review is presented in order to challenge this preposterous question. In this review, we summarize recent research on LB and related methods including (i) advanced design for LB films, (ii) LB film as a medium for supramolecular chemistry, (iii) LB technique for nanofabrication and (iv) LB involving advanced nanomaterials. Finally, a comparison between LB and LbL techniques is made. The latter reveals the crucial role played by LB techniques in basic surface science, current advanced material sciences and nanotechnologies.},
author = {Ariga, Katsuhiko and Yamauchi, Yusuke and Mori, Taizo and Hill, Jonathan P.},
doi = {10.1002/adma.201302283},
isbn = {1521-4095 (Electronic) 0935-9648 (Linking)},
issn = {09359648},
journal = {Advanced Materials},
keywords = {langmuir-blodgett films,layer-by-layer assembly,molecular machine,nanomaterials,supramolecular chemistry},
number = {45},
pages = {6477--6512},
pmid = {24302266},
title = {{25th Anniversary article: What can be done with the langmuir-blodgett method? Recent developments and its critical role in materials science}},
volume = {25},
year = {2013}
}
@misc{AirProducts2018,
author = {{Air Products}},
title = {{Air Products Engineered Membrane Systems}},
year = {2018}
}
@article{Iwahara1999,
abstract = {By applying a direct current to a proton-conducting ceramic membrane, hydrogen can be electrochemically transported across the membrane. This is a kind of electrochemical hydrogen pump. This pump may be applied to devices for the separation, the removal or the supply of hydrogen. In this paper, the working principles and advantages of the hydrogen pump and its derivatives are described. Some experimental results obtained in the author's laboratory using perovskite-type oxide proton-conducting ceramics as a solid electrolyte are also introduced.},
author = {Iwahara, Hiroyasu},
doi = {10.1016/S0167-2738(99)00185-X},
file = {:Users/marc/Library/Application Support/Mendeley Desktop/Downloaded/Iwahara - 1999 - Hydrogen pumps using proton-conducting ceramics and their applications.pdf:pdf},
isbn = {0167-2738},
issn = {01672738},
journal = {Solid State Ionics},
number = {1},
pages = {271--278},
title = {{Hydrogen pumps using proton-conducting ceramics and their applications}},
volume = {125},
year = {1999}
}
@article{Li2014,
abstract = {An ordered mesoporous precursor synthesized by the soft-templating approach has successfully been used to prepare a carbon interlayer between the thin separation layer and the support of carbon membranes. Morphology and pore structure characteristics of the ordered mesoporous carbon (OMC) interlayer were investigated by HRTEM, XRD, SEM and N2adsorption techniques. Gas separation properties of resultant supported carbon membranes were evaluated by single gas permeation experiments. The results showed that the OMC interlayer can effectively reduce surface defects of the support with large pore sizes, improve the interfacial adhesion of the support to the thin separation layer, and further enhance the gas permeation properties of the supported carbon membranes by ordered and uniform mesoporous channels. The supported carbon membranes were synthesized by one-step coating on the support modified by the OMC interlayer and achieved O2, CO2and H2permeances of 74.5, 88.0 and 545.5molm-2s-1Pa-1×10-10, respectively. These are about 4 times higher than those without the interlayer and are very competitive with respect to other carbon membranes reported in literature. The results clearly indicate that this novel approach using the OMC as an interlayer to fabricate supported carbon membranes with macro-meso-microporous gradient structure have attractive potential for gas separation. {\textcopyright} 2013 Elsevier B.V.},
author = {Li, Lin and Song, Chengwen and Jiang, Huawei and Qiu, Jieshan and Wang, Tonghua},
doi = {10.1016/j.memsci.2013.09.032},
file = {:Users/marc/Library/Application Support/Mendeley Desktop/Downloaded/Li et al. - 2014 - Preparation and gas separation performance of supported carbon membranes with ordered mesoporous carbon interlayer.pdf:pdf},
issn = {03767388},
journal = {Journal of Membrane Science},
keywords = {Carbon membranes,Gas separation,Interlayer,OMC},
pages = {469--477},
publisher = {Elsevier},
title = {{Preparation and gas separation performance of supported carbon membranes with ordered mesoporous carbon interlayer}},
url = {http://dx.doi.org/10.1016/j.memsci.2013.09.032},
volume = {450},
year = {2014}
}
@article{H.YoshidaY.Naruse1983,
author = {{H. Yoshida Y. Naruse}, S Konishi},
chapter = {429},
file = {:Users/marc/Library/Application Support/Mendeley Desktop/Downloaded/H. Yoshida Y. Naruse - 1983 - Effects of impurities on hydrogen permeability through palladium alloy membranes at comparativeley high pr.pdf:pdf},
journal = {Journal of Less Common Metals},
pages = {429--436},
title = {{Effects of impurities on hydrogen permeability through palladium alloy membranes at comparativeley high pressures and temperatures}},
volume = {89},
year = {1983}
}
@misc{InternationalStandardISO14687-2:20122012,
author = {{International Standard ISO 14687-2: 2012}},
booktitle = {Part 2: Proton Exchange Membrane (PEM) Fuel Cell Application for Road Vehicles},
file = {:Users/marc/Library/Application Support/Mendeley Desktop/Downloaded/International Standard ISO 14687-2 2012 - 2012 - Hydrogen Fuel - Product Specification.pdf:pdf},
title = {{Hydrogen Fuel - Product Specification}},
year = {2012}
}
@article{Frontera2017,
abstract = {CO2 methanation is a well-known reaction that is of interest as a capture and storage (CCS) process and as a renewable energy storage system based on a power-to-gas conversion process by substitute or synthetic natural gas (SNG) production. Integrating water electrolysis and CO2 methanation is a highly effective way to store energy produced by renewables sources. The conversion of electricity into methane takes place via two steps: hydrogen is produced by electrolysis and converted to methane by CO2 methanation. The effectiveness and efficiency of power-to-gas plants strongly depend on the CO2 methanation process. For this reason, research on CO2 methanation has intensified over the last 10 years. The rise of active, selective, and stable catalysts is the core of the CO2 methanation process. Novel, heterogeneous catalysts have been tested and tuned such that the CO2 methanation process increases their productivity. The present work aims to give a critical overview of CO2 methanation catalyst production and research carried out in the last 50 years. The fundamentals of reaction mechanism, catalyst deactivation, and catalyst promoters, as well as a discussion of current and future developments in CO2 methanation, are also included.},
author = {Frontera, Patrizia and Macario, Anastasia and Ferraro, Marco and Antonucci, PierLuigi},
doi = {10.3390/catal7020059},
file = {:Users/marc/Library/Application Support/Mendeley Desktop/Downloaded/Frontera et al. - 2017 - Supported Catalysts for CO2 Methanation A Review.pdf:pdf},
issn = {2073-4344},
journal = {Catalysts},
keywords = {carbon dioxide,hydrogenation,metal catalysts,methane,power-to-gas},
number = {2},
pages = {59},
title = {{Supported Catalysts for CO2 Methanation: A Review}},
volume = {7},
year = {2017}
}
@article{Boutilier2014,
abstract = {Gas transport through intrinsic defects and tears is a critical yet poorly understood phenomenon in graphene membranes for gas separation. We report that independent stacking of graphene layers on a porous support exponentially decreases flow through defects. On the basis of experimental results, we develop a gas transport model that elucidates the separate contributions of tears and intrinsic defects on gas leakage through these membranes. The model shows that the pore size of the porous support and its permeance critically affect the separation behavior, and reveals the parameter space where gas separation can be achieved regardless of the presence of nonselective defects, even for single-layer membranes. The results provide a framework for understanding gas transport in graphene membranes and guide the design of practical, selectively permeable graphene membranes for gas separation.},
author = {Boutilier, Michael S H and Sun, Chengzhen and O'Hern, Sean C. and Au, Harold and Hadjiconstantinou, Nicolas G. and Karnik, Rohit},
doi = {10.1021/nn405537u},
file = {:Users/marc/Library/Application Support/Mendeley Desktop/Downloaded/Boutilier et al. - 2014 - Implications of permeation through intrinsic defects in graphene on the design of defect-tolerant membranes fo.pdf:pdf},
isbn = {1936-0851},
issn = {19360851},
journal = {ACS Nano},
keywords = {gas separation,graphene membranes,multilayer graphene,nanofiltration,nanofluidics},
number = {1},
pages = {841--849},
pmid = {24397398},
title = {{Implications of permeation through intrinsic defects in graphene on the design of defect-tolerant membranes for gas separation}},
volume = {8},
year = {2014}
}
@misc{Undefinedhowtoevokesuchemotionsthroughaproduct.Inotherwordsthisisaninvestigationofthemeaningthatcouldbedesignedintoaproductinordertocommunicatewiththeuseratanemotionallevel.Aliteraturesurveyofrecentdesigntr,
author = {in one's mind; which produc {undefined  how to evoke such emotions through a product. In other words, this is an investigation of the "meaning" that could be designed into a product in order to "communicate" with the user at an emotional level. A literature survey of recent design tr}, undefined P Y - This paper explores theoretical issues in ergonomics related to semantics and the emotional content of design. The aim is to find answers to the following questions: how to design products triggering "happiness" and undefined Undefined and response surface methodology {undefined  and the allocation and evaluation of tuning parameters in separation from the evolutionary process. A combination of a genetic algorithm and a gradient-based algorithm is used for tuning of the approximation functions. The problem of the choice}, undefined P Y - This thesis addresses two problems arising in many real-life design optimization applications: the high computational cost of function evaluations and the presence of numerical noise in the function values. The and undefined Undefined and undefined Undefined and particular {undefined  engineering design methods must integrate the many different aspects of designing and the priorities of the end-user. Engineering Design (3rd edition) describes a systematic approach to engineering design. The authors argue that such an approac}, undefined P Y - Engineering design must be carefully planned and systematically executed. In and this paper undefined  we introduce a general framework for product experience that applies to all affective responses that can be experienced in human-product interaction. Three distinct components or levels of product experiences are discussed: aesthetic experience, undefined P Y - In and of products requiring chemical undefined  a new tape that one can remove without peeling off paint, computer chips and an oxygen- enriching device. We find more of our students involved in product design, both in the traditional chemical industry, where the move to high value added che, undefined P Y - We respond to a request to examine the area of product design and its impact on chemical engineering. We focus on how we are educating our undergraduates and deliberately adopt a controversial stand. Examples and study aims to elucidate how industrial designers and engineering designers collaborate {undefined  and how such an alliance reflects in the design process. We conducted in-depth interviews about actual product design projects with 34 industrial and engineering designers from six consumer product manufacturers. We firstly identified individua}, undefined P Y - This and such contemporary design and development issues as identifying customer needs {undefined  design for manufacturing, prototyping, and industrial design, Product Design and Development by Ulrich and Eppinger presents in a clear and detailed way a set of product development techniques aimed at bringing together the marketing, design, a}, undefined P Y - Treating and {undefined  we test the small molecule flexible ligand docking program Glide on a set of 19 non-$\alpha$-helical peptides and systematically improve pose prediction accuracy by enhancing Glide sampling for flexible polypeptides. In addition, scoring of the poses}, undefined P Y - Predicting the binding mode of flexible polypeptides to proteins is an important task that falls outside the domain of applicability of most small molecule and protein−protein docking tools. Here and undefined Undefined and on an extensive literature review and consumer interviews {undefined  the authors define product design and its dimensions. Using data from three samples (6,418 U.S. consumers and 1,083 and 583 European consumers), the authors develop and validate a new scale to measure product design along the dimensions of aest}, undefined P Y - Product design is a source of competitive advantage for companies and is an important driver of company performance. Drawing and fundamental principles of chemical product design and associated systematic tools undefined  within a broad domain of chemical products including molecules, formulations and devices, are still under development. In this paper, we propose a simple and fundamental conceptual model that defines the chemical product design problem as the i, undefined P Y - The and undefined Undefined and fixed product platform upon which derivative products are created through substitution of add-on modules {undefined  the approach here permits the platform itself to be one of several possible options. We first develop function structures for each product. After comparing function structures for common and unique functions, rules are applied to determine poss}, undefined P Y - This paper presents an approach to architecting a product family that shares inter-changeable modules. Rather than a},
title = {{No Title}}
}
@incollection{Li2010,
abstract = {Polymeric hollow fiber membranes are in current use for a number of process applications, including filtration, desalination, gas separation, and biochemical reactions. However, this membrane configuration is seldom found in ceramic membranes, possibly due to the lack of methods available in producing the ceramic membranes in hollow fiber geometry. It would be interesting to introduce a method for preparation of hollow fiber ceramic membranes, as this would unfold their applications to areas where high-temperature operation precludes the use of existing polymeric hollow fiber membranes. This chapter aims to elaborate the production of hollow fiber ceramic membranes using a phase inversion/sintering method, which is a three-step process, including (1) preparation of a spinning suspension; (2) spinning of ceramic hollow fiber precursors; and (3) final sintering. A detailed discussion on this method is provided and examples on fabrication and applications of yittria stabilized zirconia and various perovskite hollow fiber membranes are then given.},
author = {Li, K.},
booktitle = {Comprehensive Membrane Science and Engineering},
doi = {10.1016/B978-0-08-093250-7.00027-X},
isbn = {9780080932507},
pages = {253--273},
title = {{1.12 – Ceramic Hollow Fiber Membranes and Their Applications}},
url = {http://www.sciencedirect.com/science/article/pii/B978008093250700027X},
year = {2010}
}
@article{Ho2016,
author = {Ho, W S Winston and Li, Kang},
doi = {10.1016/j.coche.2016.05.001},
file = {:Users/marc/Library/Application Support/Mendeley Desktop/Downloaded/Ho, Li - 2016 - Editorial overview Separation engineering Recent advances in separation science and technology.pdf:pdf},
isbn = {22113398},
journal = {Current Opinion in Chemical Engineering},
pages = {vii--xi},
title = {{Editorial overview: Separation engineering: Recent advances in separation science and technology}},
volume = {12},
year = {2016}
}
@article{Morreale2007,
author = {Morreale, Bryan D and Howard, Bret H and Iyoha, Osemwengie and Enick, Robert M and Ling, Chen and Sholl, David S},
file = {:Users/marc/Library/Application Support/Mendeley Desktop/Downloaded/Morreale et al. - 2007 - Experimental and Computational Prediction of the Hydrogen Transport Properties of Pd 4 S.pdf:pdf},
journal = {Ind. Eng. Chem. Res.},
pages = {6313--6319},
title = {{Experimental and Computational Prediction of the Hydrogen Transport Properties of Pd 4 S}},
volume = {46},
year = {2007}
}
@article{Arrhenius2016,
abstract = {The traceable and accurate measurement of biogas impurities is essential in order to robustly assess compliance with the specifications for biomethane being developed by CEN/TC408. An essential part of any procedure aiming to determinate the content of impurities is the sampling and the transfer of the sample to the laboratory. Key issues are the suitability of the sample container and minimising the losses of impurities during the sampling and analysis process. In this paper, we review the state-of-the-art in biogas sampling with the focus on trace impurities. Most of the vessel suitability studies reviewed focused on raw biogas. Many parameters need to be studied when assessing the suitability of vessels for sampling and storage, among them, permeation through the walls, leaks through the valves or physical leaks, sorption losses and adsorption effects to the vessel walls, chemical reactions and the expected initial concentration level. The majority of these studies looked at siloxanes, for which sampling bags, canisters, impingers and sorbents have been reported to be fit-for-purpose in most cases, albeit with some limitations. We conclude that the optimum method requires a combination of different vessels to cover the wide range of impurities commonly found in biogas, which have a wide range of boiling points, polarities, water solubilities, and reactivities. The effects from all the parts of the sampling line must be considered and precautions must be undertaken to minimize these effects. More practical suitability tests, preferably using traceable reference gas mixtures, are needed to understand the influence of the containers and the sampling line on sample properties and to reduce the uncertainty of the measurement.},
author = {Arrhenius, Karine and Brown, Andrew S and van der Veen, Adriaan M H},
doi = {https://doi.org/10.1016/j.aca.2015.10.039},
issn = {0003-2670},
journal = {Analytica Chimica Acta},
keywords = {Biogas,Biomethane,Containers,Impurities,Sampling,Siloxanes,Suitability,VOCs},
pages = {22--32},
title = {{Suitability of different containers for the sampling and storage of biogas and biomethane for the determination of the trace-level impurities – A review}},
url = {http://www.sciencedirect.com/science/article/pii/S0003267015013185},
volume = {902},
year = {2016}
}
@article{Xue2017,
abstract = {Mesoporous Ag/In2O3 composite materials with different concentrations of Ag particles have been prepared by the calcination of Ag+-entrapped indium-organic frameworks (InOFs). The structures and components of these MOF-derived Mesoporous Ag/In2O3 composite materials have been characterized thoroughly. Gas sensing measurements indicated that the incorporation of metallic Ag particles into the mesoporous structure significantly improves the gas-sensing properties of In2O3. Specially, the response of 5{\%} Ag-loaded In2O3 sensor to 50 ppm formaldehyde is 5 times higher than that of pure In2O3 particles at 210 °C, which is among the best formaldehyde sensor materials reported to date. Also, Ag/In2O3 composite sensor exhibit short response time ({\~{}}22 s) and excellent recovery. These results indicated that the InOF-derived mesoporous Ag/In2O3 materials maybe can be used to fabricate high performance formaldehyde sensors in practice.},
author = {Xue, Ying Ying and Wang, Jin Lian and Li, Shu Ni and Jiang, Yu Cheng and Hu, Man Cheng and Zhai, Quan Guo},
doi = {10.1016/j.jssc.2017.04.024},
file = {:Users/marc/Library/Application Support/Mendeley Desktop/Downloaded/Xue et al. - 2017 - Mesoporous AgIn2O3 composite derived from indium organic framework as high performance formaldehyde sensor.pdf:pdf},
issn = {1095726X},
journal = {Journal of Solid State Chemistry},
keywords = {Ag-loaded,Formaldehyde sensing,Indium-organic framework,Mesoporous In2O3},
number = {April},
pages = {170--175},
publisher = {Elsevier Inc.},
title = {{Mesoporous Ag/In2O3 composite derived from indium organic framework as high performance formaldehyde sensor}},
url = {http://dx.doi.org/10.1016/j.jssc.2017.04.024},
volume = {251},
year = {2017}
}
@article{Hosseini2016,
abstract = {Metal-organic framework (MOF), a nanoporous compound, has been used as sensing material to fabricate capacitive nanosensor. The proposed nanosensor was fabricated by growing a Cu-BTC (MOF-199) film on a copper substrate using electrochemical method. 1-Methyl,3-octylimidazolium chloride, an ionic liquid (IL), was used as conducting salt in the electrochemical cell. Scanning electron microscopy (SEM), FTIR spectroscopy, X-ray diffraction analysis, and BET techniques were used to characterize the prepared MOF which shows a thin film (about 5 $\mu$m) of Cu-BTC layer with particle size of 2-3 $\mu$m. In order to fabricate the upper electrode of capacitor some interconnected Ag paste dots were patterned on the MOF layer which was coated on the copper surface as back electrode. This fabricated sensor was used for investigation of the capacitance variations in the presence of different amounts of introduced ethanol and methanol vapors. The capacitive sensing parameters were measured by a LCR meter. Relative capacitance variations were measured to verify the potential of the Cu-BTC films for using as dielectric layer in this capacitive sensor. The linear range of the signal vs. concentration is 0-1000 ppm for ethanol and methanol. Limit of detection of the fabricated sensors were 130.0 ppm and 39.1 ppm for ethanol and methanol, respectively. The selectivity of the sensor for polar and nonpolar VOCs were examined by introducing n-hexane in sensing chamber.},
author = {Hosseini, M. S. and Zeinali, S. and Sheikhi, M. H.},
doi = {10.1016/j.snb.2016.02.008},
file = {:Users/marc/Library/Application Support/Mendeley Desktop/Downloaded/Hosseini, Zeinali, Sheikhi - 2016 - Fabrication of capacitive sensor based on Cu-BTC (MOF-199) nanoporous film for detection of ethanol.pdf:pdf},
issn = {09254005},
journal = {Sensors and Actuators, B: Chemical},
keywords = {1-Methyl,3-octylimidazolium chloride,Capacitive sensor,Cu-BTC (MOF-199),Electrochemical synthesis,Ethanol,Ionic liquid,Methanol,Volatile organic compound},
pages = {9--16},
publisher = {Elsevier B.V.},
title = {{Fabrication of capacitive sensor based on Cu-BTC (MOF-199) nanoporous film for detection of ethanol and methanol vapors}},
url = {http://dx.doi.org/10.1016/j.snb.2016.02.008},
volume = {230},
year = {2016}
}
@article{Kitiwan2010,
abstract = {The present work was focused on the preparation of palladium alloy membranes and the effect of properties of ceramic support on the composited membrane morphology. Palladium-base membrane is known to have high selectivity and stability for hydrogen separation. In order to increase hydrogen permeation and separation factor, the membrane must be thinner and defect-free. Palladium membrane supported on a porous alumina prepared by electroless plating is the promising method to provide good hydrogen permeability. The alumina tube substrate was pre-seeded by immersing in the palladium acetate solution and followed by reduction in the alkaline hydrazine solution. After that, the deposition of palladium membrane could be achieved from the plating bath containing ethylenediamine tetraacetic acid (EDTA) stabilized palladium complex and hydrazine. The morphology of palladium film was observed to progress as a function of plating time and a dense layer membrane was available after plating for 3 h. The porosity of ceramic support exhibited an effect on the microstructure of deposited film such that the support with low porosity tended to achieve a defect free palladium membrane. ?? 2010 The Chinese Society for Metals.},
author = {Kitiwan, Mettaya and Atong, Duangduen},
doi = {10.1016/S1005-0302(11)60016-9},
file = {:Users/marc/Library/Application Support/Mendeley Desktop/Downloaded/Kitiwan, Atong - 2010 - Effects of Porous Alumina Support and Plating Time on Electroless Plating of Palladium Membrane.pdf:pdf;:Users/marc/Library/Application Support/Mendeley Desktop/Downloaded/Kitiwan, Atong - 2010 - Effects of Porous Alumina Support and Plating Time on Electroless Plating of Palladium Membrane(2).pdf:pdf},
issn = {10050302},
journal = {Journal of Materials Science and Technology},
keywords = {Electroless plating,Palladium composite membrane,Tubular porous support},
number = {12},
pages = {1148--1152},
publisher = {The Chinese Society for Metals},
title = {{Effects of Porous Alumina Support and Plating Time on Electroless Plating of Palladium Membrane}},
url = {http://dx.doi.org/10.1016/S1005-0302(11)60016-9},
volume = {26},
year = {2010}
}
@article{Phair2006a,
author = {Phair, J W and Badwal, S P S},
doi = {10.1007/s11581-006-0016-4},
isbn = {0947-7047 1862-0760},
journal = {Ionics},
number = {2},
pages = {103--115},
title = {{Review of proton conductors for hydrogen separation}},
volume = {12},
year = {2006}
}
@article{Ryeon2011,
archivePrefix = {arXiv},
arxivId = {arXiv:physics/0201037v1},
author = {Hye Ryeon Lee, Masakoto Kanezashi, Yoshihiro Shimomura, Tomohisa Yoshioka, and Toshinori Tsuru},
doi = {10.1002/aic},
eprint = {0201037v1},
isbn = {9513862380},
issn = {12350621},
journal = {AIChE Journal},
keywords = {Electrostatic droplet actuation,Electrowetting,Lab-on-a-chip,MEMS,Microfluidics,Superhydrophobic surfaces},
number = {10},
pages = {2755--2765},
pmid = {23641116},
primaryClass = {arXiv:physics},
title = {{Evaluation and Fabrication of Pore-Size-Tuned Silica Membranes with Tetraethoxydimethyl Disiloxane for Gas Separation}},
volume = {57},
year = {2011}
}
@article{Chen2014,
abstract = {Traditional semiconducting metal oxide-based gas sensors are always limited on low surface areas and high operating temperatures. Considering the high surface area and high stability of zeolitic imidazolate framework (ZIF), ZIF-67 (surface area of 1832.2 m2 g–1) was first employed as a promising formaldehyde gas sensor at a low operating temperature (150 °C), and the gas sensor could detect formaldehyde as low as 5 ppm. This work develops a new promising application approach for porous metal–organic frameworks.},
author = {Chen, Er-Xia and Yang, Hui and Zhang, Jian},
doi = {10.1021/ic500474j},
file = {:Users/marc/Library/Application Support/Mendeley Desktop/Downloaded/Chen, Yang, Zhang - 2014 - Zeolitic Imidazolate Framework as Formaldehyde Gas Sensor.pdf:pdf},
isbn = {1520-510X (Electronic)$\backslash$r0020-1669 (Linking)},
issn = {0020-1669},
journal = {Inorganic Chemistry},
number = {11},
pages = {5411--5413},
pmid = {24813234},
title = {{Zeolitic Imidazolate Framework as Formaldehyde Gas Sensor}},
url = {http://pubs.acs.org/doi/abs/10.1021/ic500474j},
volume = {53},
year = {2014}
}
@article{Hinchliffe2000,
author = {Hinchliffe, A. B. and Porter, K. E.},
journal = {Trans. Inst.Chem. Eng.},
pages = {255--268},
title = {{A comparison of membrane separation and distillation.}},
volume = {78},
year = {2000}
}
@article{Iordache2017,
abstract = {The aim of this paper is to evaluate the roll-out of a hydrogen refuelling station (HRS) infrastructure in the Member States of Europe Union. The effort contributes to a preliminary, indicative assessment of the commercial viability and profitability of a hydrogen refuelling station network roll-out. In these concrete cases were provided scenarios at both the Member States and Europe Union levels. The business case assessment was realised for the roll-out of HRS network in a 15 years program. The results refer to key metrics to assess the overall profitability of the investment: the annual number evolution of HRSs based on the inputs provided by the user; the annual number evolution of FCEVs; annual total amount of hydrogen sold; annual capital expenditures on HES procurement (CAPEX); cash flow after interest and debt payment representing the cash flow available for equity holders; annual debt service coverage ratio (ADSCR); and net present value (NPV) at the end of the period. The first scenarios were designed to achieve results across the Member States of Europe Union. There were presented three individual detailed examples also. A second series of sensitivity analyses consider that the hydrogen mobility penetration appears more constrained, being limited to a few stakeholders in the EU with well-defined policies and programs or with more favourable conditions. Both categories of scenarios outline the idea that at the beginning of a HRS network development it is already good to have an adequate number of FCEVs for a profitable programme. The net present value (NPV) is positive and the development program is commercially viable if there is a correlation between both the HRS number and FCEV fleet dimension. This correlation leads to good or optimistic results, but for the majority NPV became positive only in the second part of the program. The roll-out's analysis of the hydrogen refuelling station network in the EU indicated that hydrogen is a part of solutions for a decarbonisation of the transport sector. There is a need to carefully develop an adequate hydrogen refuelling infrastructure if the final scope of the program is its commercial viability and profitability.},
author = {Iordache, Mihaela and Schitea, Dorin and Iordache, Ioan},
doi = {https://doi.org/10.1016/j.ijhydene.2017.09.146},
issn = {0360-3199},
journal = {International Journal of Hydrogen Energy},
keywords = {Hydrogen infrastructure,Hydrogen mobility,Hydrogen refuelling station (HRS),UE},
number = {50},
pages = {29629--29647},
title = {{Hydrogen refuelling station infrastructure roll-up, an indicative assessment of the commercial viability and profitability in the Member States of Europe Union}},
url = {http://www.sciencedirect.com/science/article/pii/S036031991733820X},
volume = {42},
year = {2017}
}
@article{Løvvik2005,
abstract = {The surface segregation in Pd based alloys has been investigated by density-functional band-structure calculations. Twelve different metals were substituted in Pd(1 1 1) slabs at a level of 5{\%}: Ag, Au, Cd, Cu, Fe, Mn, Ni, Pb, Pt, Rh, Ru, and Sn. The segregation energy (the difference in calculated total free energy between surface sites and bulk-like sites) was calculated for each alloy, and the results were in very good agreement with experimental data where available. Particularly, we predict an oscillatory depth profile for Cu and Ni, similar to what has been found experimentally for the PdNi(1 0 0) surface. There are more or less pronounced correlations between the segregation energy and the relaxed position at the surface of the substituted atom, the metal radius of the substituted atom, and the experimental surface energy of the metal. It is proposed that the segregation energy is an indirect measure of the stability of Pd based hydrogen permeable membranes. {\textcopyright} 2005 Elsevier B.V. All rights reserved.},
author = {L{\o}vvik, O. M.},
doi = {10.1016/j.susc.2005.03.028},
file = {:Users/marc/Library/Application Support/Mendeley Desktop/Downloaded/L{\o}vvik - 2005 - Surface segregation in palladium based alloys from density-functional calculations.pdf:pdf},
issn = {00396028},
journal = {Surface Science},
keywords = {Alloys,Density functional calculations,Palladium,Surface segregation},
number = {1},
pages = {100--106},
title = {{Surface segregation in palladium based alloys from density-functional calculations}},
volume = {583},
year = {2005}
}
@article{Li2018,
abstract = {Gas storage and separation are closely associated with the alleviation of greenhouse effect, the widespread use of clean energy, the control of toxic gases, and various other aspects in human society. In this review, we highlight the recent advances in gas storage and separation using metal-organic frameworks (MOFs). In addition to summarizing the gas uptakes of some benchmark MOFs, we emphasize on the desired chemical properties of MOFs for different gas storage/separation scenarios. Greenhouse gases (CO2), energy-related gases (H2and CH4), and toxic gases (CO and NH3) are covered in the review.},
author = {Li, Hao and Wang, Kecheng and Sun, Yujia and Lollar, Christina T. and Li, Jialuo and Zhou, Hong Cai},
doi = {10.1016/j.mattod.2017.07.006},
file = {:Users/marc/Library/Application Support/Mendeley Desktop/Downloaded/Li et al. - 2018 - Recent advances in gas storage and separation using metal–organic frameworks.pdf:pdf},
issn = {18734103},
journal = {Materials Today},
number = {2},
pages = {108--121},
title = {{Recent advances in gas storage and separation using metal–organic frameworks}},
url = {https://doi.org/10.1016/j.mattod.2017.07.006},
volume = {21},
year = {2018}
}
@article{Huang2010a,
abstract = {High hydrogen selectivity and thermal stability are displayed by a membrane of zeolitic imidazolate ZIF-22 grown on porous ceramic supports by using 3-aminopropyltriethoxysilane (APTES) as covalent linker (see picture). At 323 K, H2/CO2 selectivity of 7.2 and H2 permeance of 1.6×10−7 mol m−2 s−1 Pa−1 were achieved.},
author = {Huang, Aisheng and Bux, Helge and Steinbach, Frank and Caro, J{\"{u}}rgen},
doi = {10.1002/anie.201001919},
file = {:Users/marc/Library/Application Support/Mendeley Desktop/Downloaded/Huang et al. - 2010 - Molecular-sieve membrane with hydrogen permselectivity ZIF-22 in LTA topology prepared with 3-aminopropyltriethoxy.pdf:pdf},
isbn = {1521-3773},
issn = {14337851},
journal = {Angewandte Chemie - International Edition},
keywords = {Gas separation,Hydrothermal synthesis,Membranes,Metal-organic frameworks,Microporous materials},
number = {29},
pages = {4958--4961},
pmid = {20540135},
title = {{Molecular-sieve membrane with hydrogen permselectivity: ZIF-22 in LTA topology prepared with 3-aminopropyltriethoxysilane as covalent linker}},
volume = {49},
year = {2010}
}
@unpublished{Balachandran2003,
author = {Balachandran, U and Lee, T H and Wang, S and Zuo, C and Dorris, S E},
title = {{Current Status of Dense Cermet Membranes for Hydrogen Separation}},
year = {2003}
}
@article{Uemiya2001,
author = {Uemiya, Shigeyuki and Kato, Wataru and Uyama, Atsuko and Kajiwara, Masataka},
file = {:Users/marc/Library/Application Support/Mendeley Desktop/Downloaded/Uemiya et al. - 2001 - Separation of hydrogen from gas mixtures using supported platinum-group metal membranes.pdf:pdf},
journal = {Separation and Purification Technology},
keywords = {carbon monoxide,cvd,hydrogen permeation,metal membrane,poisoning},
pages = {309--317},
title = {{Separation of hydrogen from gas mixtures using supported platinum-group metal membranes}},
volume = {22-23},
year = {2001}
}
@article{Wei2009,
abstract = {Hydrogen permeation through SrCe1-xTbxO3-?? (x = 0.025, 0.05 and 0.10) membranes using various gas streams as the sweep was investigated. Hydrogen impermeable SrCe1-xTbxO3-?? membranes with air or inert gas in the downstream become hydrogen permeable when there is a reducing gas, such carbon monoxide or hydrogen, existing in the downstream. The membrane remains hydrogen permeable after the downstream sweep gas is changed from the reducing gas to the inert gas. This phenomenon is explained by the electronic conductivity of the materials. These results further confirm that SrCe1-xTbxO3-?? (0.025 {\textless} x {\textless} 0.1) is a mixed proton-electron conducting material in a hydrogen containing atmosphere. The activation energy of hydrogen permeation is close to the activation energy of electronic conduction of the materials, confirming that the hydrogen permeation is determined by the electronic conductivity of the material. For SrCe0.95Tb0.05O3-??, increasing the downstream CO partial pressure from 0.001 to 0.1 atm leads to a small increase in hydrogen flux from 1.4 ?? 10-2 to 1.6 ?? 10-2 ml/cm2 min. The hydrogen flux of SrCe1-xTbxO3-?? increases with upstream hydrogen partial pressure. ?? 2009 Elsevier B.V. All rights reserved.},
author = {Wei, Xiaotong and Kniep, Jay and Lin, Y. S.},
doi = {10.1016/j.memsci.2009.08.041},
file = {:Users/marc/Library/Application Support/Mendeley Desktop/Downloaded/Wei, Kniep, Lin - 2009 - Hydrogen permeation through terbium doped strontium cerate membranes enabled by presence of reducing gas in the.pdf:pdf},
isbn = {0376-7388},
issn = {03767388},
journal = {Journal of Membrane Science},
keywords = {Downstream oxygen partial pressure,Electronic/protonic conductivity,Hydrogen permeation,SrCe1-xTbxO3-?? membranes},
number = {1-2},
pages = {201--206},
title = {{Hydrogen permeation through terbium doped strontium cerate membranes enabled by presence of reducing gas in the downstream}},
volume = {345},
year = {2009}
}
@article{Peters2016a,
author = {Peters, T A and Stange, M and Veenstra, P and Nijmeijer, A and Bredesen, R},
doi = {10.1016/j.memsci.2015.10.031},
file = {:Users/marc/Library/Application Support/Mendeley Desktop/Downloaded/Peters et al. - 2016 - The performance of Pd–Ag alloy membrane films under exposure to trace amounts of H2S.pdf:pdf},
isbn = {03767388},
journal = {Journal of Membrane Science},
pages = {105--115},
title = {{The performance of Pd–Ag alloy membrane films under exposure to trace amounts of H2S}},
volume = {499},
year = {2016}
}
@unpublished{Murugan2017,
author = {Murugan, Arul},
pages = {62},
title = {{16ENG01 MetroHyVe}},
year = {2017}
}
@article{Jang2013,
author = {Jang, Woo Kyung and Yun, Jumi and Kim, Hyung Il and Lee, Young Seak},
doi = {10.1007/s00396-012-2832-6},
file = {:Users/marc/Library/Application Support/Mendeley Desktop/Downloaded/Jang et al. - 2013 - Improvement of ammonia sensing properties of polypyrrole by nanocomposite with graphitic materials.pdf:pdf},
issn = {0303402X},
journal = {Colloid and Polymer Science},
keywords = {Gas sensor,Graphitic material,In situ polymerization,Polypyrrole,Reduced graphene oxide},
number = {5},
pages = {1095--1103},
title = {{Improvement of ammonia sensing properties of polypyrrole by nanocomposite with graphitic materials}},
volume = {291},
year = {2013}
}
@article{Hille1990a,
abstract = {An apparatus for enrichment of trace impurity components in gases employing mass spectrometry for analysis is described. The main component is a zeolite-packed chromatographic column. The analytical procedure is carried out step by step: first the trace components are trapped on the adsorbent employing the frontal analysis technique, then the pressure in the column is decreased to a high vacuum by cryopumping and finally the adsorbed trace components are desorbed directly into a mass spectrometer for analysis. The analytical efficiency of the method is demonstrated. Detection limits are in the low-ng/l range.},
author = {Hille, J{\"{u}}rgen},
doi = {https://doi.org/10.1016/S0021-9673(01)89591-1},
issn = {0021-9673},
journal = {Journal of Chromatography A},
pages = {265--274},
title = {{Enrichment and mass spectrometric analysis of trace impurity concentrations in gases}},
url = {http://www.sciencedirect.com/science/article/pii/S0021967301895911},
volume = {502},
year = {1990}
}
@article{JamesB.MillerBretH.HowardCaseyP.OBrienBryanD.Morreale2009,
author = {{James B. Miller  Bret H. Howard, Casey P. O'Brien, Bryan D. Morreale}, Dominic R Alfonso},
chapter = {18800},
file = {:Users/marc/Library/Application Support/Mendeley Desktop/Downloaded/James B. Miller Bret H. Howard, Casey P. O'Brien, Bryan D. Morreale - 2009 - Hydrogen Dissociation on PdS4 Surfaces.pdf:pdf},
journal = {The Journal of Physical Chemistry C},
pages = {18800--18806},
title = {{Hydrogen Dissociation on PdS4 Surfaces}},
volume = {113},
year = {2009}
}
@article{Muller2013,
author = {M{\"{u}}ller, K and St{\"{a}}dter, M and Rachow, F and Hoffmannbeck, D and Schmei{\ss}er, D},
doi = {10.1007/s12665-013-2609-3},
isbn = {1866-6280
1866-6299},
journal = {Environmental Earth Sciences},
number = {8},
pages = {3771--3778},
title = {{Sabatier-based CO2-methanation by catalytic conversion}},
volume = {70},
year = {2013}
}
@article{Gabitto2009,
abstract = {Many industrial chemical processes attempt to produce inexpensive purified hydrogen for use in industrial applications and in fuel cell power systems. Hydrogen-permeable metal membranes made of palladium and its alloys are the most widely used for hydrogen separation due to their high hydrogen permeability and their ideally infinite hydrogen selectivity. However, sulfur poisoning has been found to have negative effects on the performance of these materials. The goal of this paper is to present experimental and theoretical information that has been published in the open literature on the mechanism of sulfur poisoning of palladium membranes. Studies based upon the reduction mechanism of transition metals have been included when relevant to the subject of this work The collected information has been critically reviewed and conclusions drawn.},
author = {Gabitto, Jorge F and Tsouris, Costas},
doi = {10.1.1.470.5843},
file = {:Users/marc/Library/Application Support/Mendeley Desktop/Downloaded/Gabitto, Tsouris - 2009 - Sulfur Poisoning of Metal Membranes for Hydrogen Separation.pdf:pdf},
journal = {International Review of Chemical Engineering},
keywords = {hydrogen,palladium membranes,sulfur poisoning},
number = {5},
pages = {394--410},
title = {{Sulfur Poisoning of Metal Membranes for Hydrogen Separation}},
volume = {1},
year = {2009}
}
@techreport{UKH2Mobility2013,
author = {{UK H2 Mobility}},
title = {{UK H2 Mobility: Phase 1 Results}},
year = {2013}
}
@article{Yang2008,
author = {Yang, Naitao and Tan, Xiaoyao and Ma, Zifeng},
doi = {10.1016/j.jpowsour.2008.05.006},
file = {:Users/marc/Library/Application Support/Mendeley Desktop/Downloaded/Yang, Tan, Ma - 2008 - A phase inversionsintering process to fabricate nickelyttria-stabilized zirconia hollow fibers as the anode suppo.pdf:pdf},
isbn = {03787753},
journal = {Journal of Power Sources},
number = {1},
pages = {14--19},
title = {{A phase inversion/sintering process to fabricate nickel/yttria-stabilized zirconia hollow fibers as the anode support for micro-tubular solid oxide fuel cells}},
volume = {183},
year = {2008}
}
@article{Cheng2007,
author = {Cheng, Xuan and Shi, Zheng and Glass, Nancy and Zhang, Lu and Zhang, Jiujun and Song, Datong and Liu, Zhong-Sheng and Wang, Haijiang and Shen, Jun},
doi = {10.1016/j.jpowsour.2006.12.012},
file = {:Users/marc/Library/Application Support/Mendeley Desktop/Downloaded/Cheng et al. - 2007 - A review of PEM hydrogen fuel cell contamination Impacts, mechanisms, and mitigation.pdf:pdf},
isbn = {03787753},
journal = {Journal of Power Sources},
number = {2},
pages = {739--756},
title = {{A review of PEM hydrogen fuel cell contamination: Impacts, mechanisms, and mitigation}},
volume = {165},
year = {2007}
}
@article{Cao2014,
abstract = {{\textless}p{\textgreater} SnO {\textless}sub{\textgreater}2{\textless}/sub{\textgreater} –graphene nanocomposite with enhanced photocatalytic activity and gas sensing performance has been synthesized by solid-state chemical reaction in one-pot process. {\textless}/p{\textgreater}},
author = {Cao, Yali and Li, Yizhao and Jia, Dianzeng and Xie, Jing},
doi = {10.1039/C4RA06995A},
file = {:Users/marc/Library/Application Support/Mendeley Desktop/Downloaded/Cao et al. - 2014 - Solid-state synthesis of SnO sub2sub –graphene nanocomposite for photocatalysis and formaldehyde gas sensing.pdf:pdf},
issn = {2046-2069},
journal = {RSC Adv.},
number = {86},
pages = {46179--46186},
publisher = {Royal Society of Chemistry},
title = {{Solid-state synthesis of SnO {\textless}sub{\textgreater}2{\textless}/sub{\textgreater} –graphene nanocomposite for photocatalysis and formaldehyde gas sensing}},
url = {http://xlink.rsc.org/?DOI=C4RA06995A},
volume = {4},
year = {2014}
}
@article{Lubarda2003,
abstract = {A study of the effective lattice parameter in binary alloys is given based on an analysis of the volume change produced by dilute resolution of the solute atoms in the solvent matrix. An apparent size of the solute atom is incorporated in the analysis to approximately account for the electronic interactions between the outermost quantum shells of the solute and solvent atoms. The comparison with experimental data for various alloy systems and deviations from Vegard's law are analyzed. The free energy of binary alloys and the ordering of solute atoms at higher solute concentrations are then discussed. {\textcopyright} 2002 Elsevier Science Ltd. All rights reseved.},
author = {Lubarda, V. A.},
doi = {10.1016/S0167-6636(02)00196-5},
file = {:Users/marc/Library/Application Support/Mendeley Desktop/Downloaded/Lubarda - 2003 - On the effective lattice parameter of binary alloys.pdf:pdf},
isbn = {1858534569},
issn = {01676636},
journal = {Mechanics of Materials},
keywords = {Apparent atomic radius,Binary alloys,Free energy,Lattice parameter,Nonlinear elasticity,Solute atoms,Vegard's law},
number = {1-2},
pages = {53--68},
title = {{On the effective lattice parameter of binary alloys}},
volume = {35},
year = {2003}
}
@article{IrfanHatim2011,
author = {{Irfan Hatim}, M D and Tan, Xiaoyao and Wu, Zhentao and Li, K},
doi = {10.1016/j.ces.2010.12.021},
file = {:Users/marc/Library/Application Support/Mendeley Desktop/Downloaded/Irfan Hatim et al. - 2011 - PdAl2O3 composite hollow fibre membranes Effect of substrate resistances on H2 permeation properties.pdf:pdf},
isbn = {00092509},
journal = {Chemical Engineering Science},
number = {6},
pages = {1150--1158},
title = {{Pd/Al2O3 composite hollow fibre membranes: Effect of substrate resistances on H2 permeation properties}},
volume = {66},
year = {2011}
}
@article{Shi2011,
abstract = {A novel strategy for the preparation of supported PdAu alloy layers allows the facile and fast fabrication of highly permeable and selective H2 separation membranes from refractory metals via electroless plating and low-temperature alloying. Homogenous alloying of multiple, separately deposited Pd and Au layers with thickness in the nm range required less than one week at 773 K under atmospheric H2 as evidenced by X-ray diffraction and H2 permeation measurements. The H2 permeation rate JH2 became stable within a day even, reaching 0.62 mol m-2 s-1 at 773 K and ??PH2 = 100 kPa. The corresponding N2 leak rate remained constant during a 350 h experiment, resulting in an ideal H 2/N2 selectivity of 1400 and demonstrating that such membranes tolerate extended operation at that temperature well. ?? 2010 Professor T. Nejat Veziroglu. Published by Elsevier Ltd. All rights reserved.},
author = {Shi, Lei and Goldbach, Andreas and Xu, Hengyong},
doi = {10.1016/j.ijhydene.2010.11.056},
file = {:Users/marc/Library/Application Support/Mendeley Desktop/Downloaded/Shi, Goldbach, Xu - 2011 - High-flux H2 separation membranes from (PdAu)n nanolayers.pdf:pdf},
isbn = {0360-3199},
issn = {03603199},
journal = {International Journal of Hydrogen Energy},
keywords = {Electroless plating,Hydrogen permeation,Nanolayers,PdAu alloying,PdAu membranes},
number = {3},
pages = {2281--2284},
publisher = {Elsevier Ltd},
title = {{High-flux H2 separation membranes from (Pd/Au)n nanolayers}},
url = {http://dx.doi.org/10.1016/j.ijhydene.2010.11.056},
volume = {36},
year = {2011}
}
@article{Escolastico2012,
abstract = {This work presents the preparation and characterization of the hydrogen permeation of a mixed protonic-electronic conducting membrane based on the doped system La 6WO 12, in this instance, (La 5/6Nd 1/6) 5.5WO 12-$\delta$. The tungstates Ln 6WO 12 present sufficient protonic and electronic mixed conductivity and stability in moist CO 2 environments and are potential candidates for the separation of hydrogen at high temperatures. Hydrogen separation properties of (La 5/6Nd 1/6) 5.5WO 12-$\delta$ were systematically analyzed, specifically, the influence of the H 2 concentration in feed stream, humidification degree and operating temperature on the hydrogen separation were studied. The best permeation results were achieved when both membrane sides were humidified (2.5 vol.{\%} H 2O), i.e., the highest hydrogen flux was 0.046 mL min -1cm - 2 at 1000 °C for membrane thickness of 0.9 mm. {\textcopyright} 2011 Elsevier B.V. All rights reserved.},
author = {Escol{\'{a}}stico, Sonia and Sol{\'{i}}s, Cecilia and Serra, Jos{\'{e}} M.},
doi = {10.1016/j.ssi.2011.11.004},
file = {:Users/marc/Library/Application Support/Mendeley Desktop/Downloaded/Escol{\'{a}}stico, Sol{\'{i}}s, Serra - 2012 - Study of hydrogen permeation in (La 56Nd 16) 5.5WO 12-$\delta$ membranes.pdf:pdf},
issn = {01672738},
journal = {Solid State Ionics},
keywords = {Hydrogen permeable membrane,Hydrogen permeation,Lanthanide tungstate,Ln 6WO 12,Proton conducting oxide},
pages = {31--35},
publisher = {Elsevier B.V.},
title = {{Study of hydrogen permeation in (La 5/6Nd 1/6) 5.5WO 12-$\delta$ membranes}},
url = {http://dx.doi.org/10.1016/j.ssi.2011.11.004},
volume = {216},
year = {2012}
}
@misc{LME2016,
author = {LME},
number = {06/12/2016},
title = {{LME Copper}},
url = {http://www.lme.com/en-gb/metals/non-ferrous/copper/},
volume = {2016},
year = {2016}
}
@article{Chen2015a,
abstract = {Four new lead(ii)-iridium(iii) heterobimetallic coordination frameworks, i.e., [Pb2(L)4(DMF)2][middle dot]ClO4[middle dot]2DMF[middle dot]13H2O (1[middle dot]DMF), [Pb(L)2[middle dot]H2O][middle dot]ClO4[middle dot]3C3H6O[middle dot]3H2O (1[middle dot]Acetone), [Pb(L)2[middle dot]H2O][middle dot]3C3H6O[middle dot]3H2O[middle dot]CH3CN (2[middle dot]Acetone), and [Pb4(L)4I4[middle dot](DMF)2][middle dot]10H2O (3[middle dot]DMF), where is L-H2 = Ir(ppy)2(H2dcbpy)PF6, ppy = 2-phenylpyridine, and H2dcbpy = 4,4[prime or minute]-dicarboxy-2,2[prime or minute]-bipyridine, have been synthesized and structurally characterized using elemental analyses, IR spectroscopy, optical spectroscopy, and single-crystal X-ray diffraction. Heavy atoms, i.e. lead(ii) atoms, were introduced to the frameworks and coordinated with the chromophore to promote an efficient intersystem crossing from the singlet to the triplet 3MLCT excited state and further promote sensitivity to oxygen. The emissions of 1[middle dot]DMF, 1[middle dot]Acetone, 2[middle dot]Acetone and 3[middle dot]DMF (referred to hereafter as 1[middle dot]DMF-3[middle dot]DMF) were ascribed to metal-to-ligand charge transfer transitions (MLCTs). These four compounds, using phosphorescence based detection, were able to detect oxygen sensitively in real gas, and their noteworthy oxygen-sensing properties were also evaluated. The quenching constants, or KSV values, between 1[middle dot]DMF-3[middle dot]DMF and oxygen, can be deduced to be 1.44, 1.60, 2.85, and 5.11, respectively. The minimum detectable concentrations (LODs) of oxygen for 1[middle dot]DMF-3[middle dot]DMF were calculated to be 0.65{\%}, 0.70{\%}, 0.61{\%}, and 0.52{\%}, respectively, from three times the signal to noise ratio. Moreover, taking 3[middle dot]DMF as an example, short response (50 s) and recovery times (24 s) toward oxygen have been measured. It should be noted that the evaluated recovery time for 3[middle dot]DMF is even shorter than those reported for rare-earth MOF films. The gas sensing properties, including sensitivity, sensing linearity, reproducibility, matrix effect, cross-sensitivity effect, temperature effect, and long-term stability, were investigated. Finally, in order to spark a broad spectrum of interest in terms of an application, 3[middle dot]DMF was mixed with a commercial dye, Coumarin 480 (C480), to form a ratiometric oxygen sensor for the analysis of real gas in air.},
author = {Chen, Yi-Ting and Lin, Chun-Yen and Lee, Gene-Hsiang and Ho, Mei-Lin},
doi = {10.1039/C4CE02457E},
file = {:Users/marc/Library/Application Support/Mendeley Desktop/Downloaded/Chen et al. - 2015 - Four new lead(ii)-iridium(iii) heterobimetallic coordination frameworks synthesis, structures, luminescence and oxy.pdf:pdf},
issn = {1466-8033},
journal = {CrystEngComm},
number = {10},
pages = {2129--2140},
publisher = {Royal Society of Chemistry},
title = {{Four new lead(ii)-iridium(iii) heterobimetallic coordination frameworks: synthesis, structures, luminescence and oxygen-sensing properties}},
url = {http://dx.doi.org/10.1039/C4CE02457E{\%}5Cnhttp://pubs.rsc.org/en/Content/ArticleLanding/2015/CE/C4CE02457E},
volume = {17},
year = {2015}
}
@article{Smart2012,
abstract = {In this work high quality cobalt oxide silica membranes were synthesized on alumina supports using a sol-gel, dip coating method. The membranes were subsequently connected into a steel module using a graphite based proprietary sealing method. The sealed membranes were tested for single gas permeance of He, H2, N2and CO2at temperatures up to 600°C and feed pressures up to 600 kPa. Pressure tests confirmed that the sealing system was effective as no gas leaks were observed during testing. A H2permeance of 1.9 × 10-7mol m-2s-1Pa-1was measured in conjunction with a H2/CO2permselectivity of more than 1500, suggesting that the membranes had a very narrow pore size distribution and an average pore diameter of approximately 3 {\AA}. The high temperature testing demonstrated that the incorporation of cobalt oxide into the silica matrix produced a structure with a higher thermal stability, able to resist thermally induced densification up to at least 600°C. Furthermore, the membranes were tested for H2/CO2binary feed mixtures between 400 and 600°C. At these conditions, the reverse of the water gas shift reaction occurred, inadvertently generating CO and water which increased as a function of CO2feed concentration. The purity of H2in the permeate stream significantly decreased for CO2feed concentrations in excess of 50 vol{\%}. However, the gas mixtures (H2, CO2, CO and water) had a more profound effect on the H2permeate flow rates which significantly decreased, almost exponentially as the CO2feed concentration increased. {\textcopyright} 2012 Hydrogen Energy Publications, LLC. Published by Elsevier Ltd. All rights reserved.},
author = {Smart, S. and Vente, J. F. and {Diniz Da Costa}, J. C.},
doi = {10.1016/j.ijhydene.2012.06.031},
file = {:Users/marc/Library/Application Support/Mendeley Desktop/Downloaded/Smart, Vente, Diniz Da Costa - 2012 - High temperature H 2CO 2 separation using cobalt oxide silica membranes.pdf:pdf},
issn = {03603199},
journal = {International Journal of Hydrogen Energy},
keywords = {Cobalt oxide silica membranes,Gas separation,High temperatures},
number = {17},
pages = {12700--12707},
publisher = {Elsevier Ltd},
title = {{High temperature H 2/CO 2 separation using cobalt oxide silica membranes}},
url = {http://dx.doi.org/10.1016/j.ijhydene.2012.06.031},
volume = {37},
year = {2012}
}
@misc{EuropeanCommission2015,
author = {{European Commission}},
doi = {10.2833/105662
10.2833/467316
10.2833/77358},
editor = {{European Commission}},
title = {{EU Energy in Figures - Statistical Pocketbook 2015}},
year = {2015}
}
@article{Tarditi2014,
author = {Tarditi, AM and Imhoff, C and Braun, F},
file = {:Users/marc/Library/Application Support/Mendeley Desktop/Downloaded/Tarditi, Imhoff, Braun - 2014 - PdCuAu ternary alloy membranes Hydrogen permeation properties in the presence of H 2 S.pdf:pdf},
journal = {Journal of Membrane  {\ldots}},
keywords = {pdcuau ternary alloy},
pages = {246--255},
title = {{PdCuAu ternary alloy membranes: Hydrogen permeation properties in the presence of H 2 S}},
url = {http://www.sciencedirect.com/science/article/pii/S0376738814009338},
volume = {479},
year = {2014}
}
@article{Yao2016,
author = {Yao, Ming Shui and Tang, Wen Xiang and Wang, Guan E. and Nath, Bhaskar and Xu, Gang},
doi = {10.1002/adma.201506457},
file = {:Users/marc/Library/Application Support/Mendeley Desktop/Downloaded/Yao et al. - 2016 - MOF Thin Film-Coated Metal Oxide Nanowire Array Significantly Improved Chemiresistor Sensor Performance.pdf:pdf},
isbn = {1521-4095},
issn = {15214095},
journal = {Advanced Materials},
keywords = {VOC sensors,humidity,metal oxides,metal-organic frameworks,selectivity},
number = {26},
pages = {5229--5234},
title = {{MOF Thin Film-Coated Metal Oxide Nanowire Array: Significantly Improved Chemiresistor Sensor Performance}},
volume = {28},
year = {2016}
}
@article{Rodrigues2011,
abstract = {We demonstrate gas sensing in a relatively compact sensor unit in particular for weakly absorbing gases in real time. As a proof-of-concept, we built an oxygen sensor for the A-Band at 760 nm. A VCSEL laser was used as a laser source due to its mode stability and reduced cost compared to DFB lasers and Fabry-Perot lasers. In order to reduce as much as possible the sensor size, a hollow waveguide is used to guide the light and the gas to be analysed in a long path to enhance the sensitivity of the sensor. Two different types of hollow fibres were characterised with respect to their suitability for gas sensing, a photonic crystal fibre, also known as micro-structured optical fibre, and hollow metal-coated capillaries. Characteristics as attenuation, spectral transmission properties and filling time were analysed. At the end, a sensor device with coupling and detection unit was developed. The main advantage of our set-up is the possibility of using the same design for different gases by changing solely the laser, the detector and the coupling lens.},
author = {Rodrigues, A and Lange, V and Kuhlke, D},
doi = {Artn 807311\rDoi 10.1117/12.886176},
file = {:Users/marc/Library/Application Support/Mendeley Desktop/Downloaded/Rodrigues, Lange, Kuhlke - 2011 - Spectroscopy Gas Sensing Based on Hollow Fibres.pdf:pdf},
isbn = {0277-786X},
issn = {0277786X},
journal = {Optical Sensors 2011 and Photonic Crystal Fibers V},
keywords = {fibre gas sensor,gas sensing,hollow fibre,micro-structured optical fibre (mof),optical sensor,oxygen measurement,photonic fibre (pcf),spectroscopy,transmission,wave-guides},
title = {{Spectroscopy Gas Sensing Based on Hollow Fibres}},
volume = {8073},
year = {2011}
}
@article{Huang2012,
abstract = {A facile strategy for the layer-by-layer synthesis of sandwich-like zeolite LTA membranes by using 3-aminopropyltriethoxysilane (APTES) as modifier is reported. APTES reacts with surface hydroxyls and forms a covalently bonded siloxane interlayer. APTES treatment is helpful to promote nucleation and growth of the zeolite LTA layer on solid surfaces. Further, the covalently bonded self-assembled siloxane film can protect the already formed LTA layer from dissolution and transformation during the next hydrothermal LTA synthesis step, thus facilitating the formation of well intergrown and phase-pure multi-layer zeolite LTA membranes. The multi-layer zeolite LTA membranes display molecular sieving performance, and each additional LTA layer improves the selectivity of the sandwich-like LTA layer stack. In mixed gas separation at 100°C, the mixture separation factors for equimolar mixtures of H2/CO2, H2/N2, H2/CH4and H2/C3H8, on the single/multi-layer LTA zeolite membranes in the Na+state are found to be 6.3, 5.6, 4.3 and 6.5 (single-layer), 8.8, 7.2, 5.8 and 12.6 (double-layer), and 12.5, 8.6, 6.5 and 19.3 (triple-layer). {\textcopyright} 2012 Elsevier Inc. All rights reserved.},
author = {Huang, Aisheng and Wang, Nanyi and Caro, J{\"{u}}rgen},
doi = {10.1016/j.micromeso.2012.06.018},
file = {:Users/marc/Library/Application Support/Mendeley Desktop/Downloaded/Huang, Wang, Caro - 2012 - Synthesis of multi-layer zeolite LTA membranes with enhanced gas separation performance by using 3-aminopropy.pdf:pdf},
isbn = {1387-1811},
issn = {13871811},
journal = {Microporous and Mesoporous Materials},
keywords = {Covalent linker,LTA zeolite membrane,Layer-by-layer synthesis,Molecular sieve membrane},
pages = {294--301},
publisher = {Elsevier Inc.},
title = {{Synthesis of multi-layer zeolite LTA membranes with enhanced gas separation performance by using 3-aminopropyltriethoxysilane as interlayer}},
url = {http://dx.doi.org/10.1016/j.micromeso.2012.06.018},
volume = {164},
year = {2012}
}
@article{Bai2015,
abstract = {The hetero-junction hybrids of 1-D molybdenum trioxide nanorods/reduced graphene oxide (MoO{\textless}inf{\textgreater}3{\textless}/inf{\textgreater}/rGO) with different contents of rGO have been successfully synthesized by in-situ one-step microwave hydrothermal method. The morphology, structure and composition of the hybrids were characterized by FESEM, TEM, XRD, FT-IR, XPS, Raman and TG analysis, respectively. The sensing test results show that the novel hybrids not only exhibit enhanced sensitivity, good selectivity, fast response and recovery compared to that of pure MoO{\textless}inf{\textgreater}3{\textless}/inf{\textgreater} but also exhibit stability and reproducibility to ppm-level H{\textless}inf{\textgreater}2{\textless}/inf{\textgreater}S at operating temperature of 110 °C. The sensing mechanism of hybrid to H{\textless}inf{\textgreater}2{\textless}/inf{\textgreater}S is also discussed in detail. The improvement of sensing performance can be attributed to the formation of hetero-junction at the interface in hybrid, which renders the surface of the MoO{\textless}inf{\textgreater}3{\textless}/inf{\textgreater} nanorods more n-type, meantime facilitates the migration of electrons due to the incorporating of rGO. The study will offer wide application in efficient gas sensors.},
author = {Bai, Shouli and Chen, Chao and Luo, Ruixian and Chen, Aifan and Li, Dianqing},
doi = {10.1016/j.snb.2015.04.036},
file = {:Users/marc/Library/Application Support/Mendeley Desktop/Downloaded/Bai et al. - 2015 - Synthesis of MoO3reduced graphene oxide hybrids and mechanism of enhancing H2S sensing performances.pdf:pdf},
isbn = {0925-4005},
issn = {09254005},
journal = {Sensors and Actuators B: Chemical},
pages = {113--120},
publisher = {Elsevier B.V.},
title = {{Synthesis of MoO3/reduced graphene oxide hybrids and mechanism of enhancing H2S sensing performances}},
url = {http://dx.doi.org/10.1016/j.snb.2015.04.036},
volume = {216},
year = {2015}
}
@article{Gao2005,
abstract = {Thin Pd-Cu membranes were prepared by electroless plating technique on porous stainless steel (PSS) disks coated with a mesoporous palladium impregnated zirconia intermediate layer. This intermediate layer provides seeds for electroless plating growth of Pd-Cu film during synthesis and serves as an inter-metallic diffusion barrier that improves membrane stability for practical application. XPS analyses showed that the average surface compositions of the two membranes were respectively Pd84Cu16 and Pd 46Cu54 (at.{\%}). XRD analyses indicated that deposited Pd84Cu16 alloy film contained only ??-fcc (face-centred cubic) phase structure whereas Pd46Cu54 alloy film contained a mixture of ??-fcc and ??-bcc phases. The 10 ??m thick Pd46Cu54/ZrO2-PSS membrane exhibited an infinite separation factor for H2 over N2, with H2 permeance of 1.1 ?? 10-7 mol/m2 s Pa at 753 K. The activation energies for hydrogen permeation and hydrogen pressure exponents are respectively 14.5 kJ/mol and 0.6 for Pd 84Cu16/ZrO2-PSS membrane and 15.4 kJ/mol and 1 for Pd46Cu54/ZrO2-PSS composite membrane. The different permeation properties for the two membranes are discussed in terms of different permeation mechanisms associated with membrane thickness, structure, surface composition and morphology. ?? 2005 Elsevier B.V. All rights reserved.},
author = {Gao, Huiyuan and Lin, Jerry Y S and Li, Yongdan and Zhang, Baoquan},
doi = {10.1016/j.memsci.2005.04.050},
file = {:Users/marc/Library/Application Support/Mendeley Desktop/Downloaded/Gao et al. - 2005 - Electroless plating synthesis, characterization and permeation properties of Pd-Cu membranes supported on ZrO2 modif.pdf:pdf},
isbn = {0376-7388},
issn = {03767388},
journal = {Journal of Membrane Science},
keywords = {Hydrogen permeation,Membrane preparation,Pd-Cu alloy composite membrane,Porous stainless steel,Zirconia},
number = {1-2},
pages = {142--152},
title = {{Electroless plating synthesis, characterization and permeation properties of Pd-Cu membranes supported on ZrO2 modified porous stainless steel}},
volume = {265},
year = {2005}
}
@misc{DavidL.McKinley1964,
author = {{David L. McKinley}},
file = {:Users/marc/Library/Application Support/Mendeley Desktop/Downloaded/David L. McKinley - 1964 - Metal Alloy for Hydrogen Separation and Purification.pdf:pdf},
isbn = {1110110111},
pages = {13--16},
title = {{Metal Alloy for Hydrogen Separation and Purification}},
year = {1964}
}
@article{Yong2013,
abstract = {Polymers of intrinsic microporosity, e.g. PIM-1, are attractive materials for gas sepn. and energy development, which is ascribed mainly to their superior permeability. The H2 and CO2 permeability of PIM-1 is about 1300-4000 Barrer and 3000-8000 Barrer, resp. However, it has a relatively low H2/CO2 selectivity of 0.4-0.8. Different from the previous UV rearrangement approach, for the first time we report here a viable method to tune the intrinsic properties of PIM-1 blend membranes from being CO2-selective to H2-selective via blending with Matrimid and subsequent crosslinking the mixed matrix membrane with diamines at room temp. The ideal H2/CO2 selectivity of the membrane after modification by 2 h triethylenetetramine (TETA) improved dramatically from 0.4-0.8 to 9.6, with a H2 permeability of 395 Barrer. The modified membranes also show exceptional sepn. performance, surpassing the present upper bound for H2/CO2, H2/N2, H2/CH4 and O2/N2 sepns. Positron annihilation lifetime spectroscopy (PALS) and Field emission SEM (FESEM) revealed that the diamine crosslinking successfully alters the membrane morphol. from a dense to a composite structure. The X-ray diffraction (XRD) anal. and sorption data confirmed that the modified membrane has a smaller d-spacing and a decrease in the diffusivity coeff. Our results also affirmed that the spatial structure, rather than the pKa value, of the diamine is the prevailing factor which governs the reactivity of diamines towards the PIM-1/Matrimid membrane due to the low concn. of cross-linkable polyimides distributing randomly in the polymer matrix. The fundamentals and knowledge gained throughout this study may facilitate the development of polymeric membranes for green H2 enrichment processes. [on SciFinder(R)]},
author = {Yong, Wai Fen and Li, Fu Yun and Chung, Tai Shung and Tong, Yen Wah},
doi = {10.1039/c3ta13308g},
file = {:Users/marc/Library/Application Support/Mendeley Desktop/Downloaded/Yong et al. - 2013 - Highly permeable chemically modified PIM-1Matrimid membranes for green hydrogen purification.pdf:pdf},
isbn = {2050-7496},
issn = {20507496},
journal = {Journal of Materials Chemistry A},
number = {44},
pages = {13914--13925},
title = {{Highly permeable chemically modified PIM-1/Matrimid membranes for green hydrogen purification}},
volume = {1},
year = {2013}
}
@article{Zhao2000,
author = {Zhao, HB and Xiong, GX and Baron, GV},
doi = {10.1016/S0920-5861(99)00266-7},
file = {:Users/marc/Library/Application Support/Mendeley Desktop/Downloaded/Zhao, Xiong, Baron - 2000 - Preparation and characterization of palladium-based composite membranes by electroless plating and magnetron.pdf:pdf},
issn = {09205861},
journal = {Catalysis today},
keywords = {electroless plating,h 2 separation,magnetron sputtering,pd-based membrane},
pages = {89--96},
title = {{Preparation and characterization of palladium-based composite membranes by electroless plating and magnetron sputtering}},
url = {http://www.sciencedirect.com/science/article/pii/S0920586199002667},
volume = {56},
year = {2000}
}
@article{Keuler1999,
author = {Keuler, J N and Lorenzen, L and Sanderson, R D and Prozesky, V and Przybylowicz, W J},
journal = {Thin Solid Films},
pages = {91--98},
title = {{Characterization of electroless plated palladium-silver alloy membranes}},
volume = {347},
year = {1999}
}
@misc{Buxbaum2008,
address = {United States},
author = {Buxbaum, Robert E},
editor = {Patent, United States},
file = {:Users/marc/Library/Application Support/Mendeley Desktop/Downloaded/Buxbaum - 2008 - Space Group CP2 Alloys For The Use And Separation Of Hydrogen.pdf:pdf},
title = {{Space Group CP2 Alloys For The Use And Separation Of Hydrogen}},
volume = {US 7,323,0},
year = {2008}
}
@article{Zhu2010,
author = {Zhu, Xuefeng and Wang, Haibing and Lin, Y S},
file = {:Users/marc/Library/Application Support/Mendeley Desktop/Downloaded/Zhu, Wang, Lin - 2010 - Effect of the Membrane Quality on Gas Permeation and Chemical Vapor.pdf:pdf},
journal = {Ind. Eng. Chem. Res},
pages = {10026--10033},
title = {{Effect of the Membrane Quality on Gas Permeation and Chemical Vapor}},
volume = {49},
year = {2010}
}
@incollection{Exter2015,
author = {den Exter, M J},
booktitle = {Palladium Membrane Technology for Hydrogen Production, Carbon Capture and Other Application},
doi = {10.1533/9781782422419.1.43},
editor = {Doukelis, A and Panopoulos, K and Koumanakos, A and Kakaras, E},
file = {:Users/marc/Library/Application Support/Mendeley Desktop/Downloaded/Exter - 2015 - The use of electroless plating as a deposition technology in the fabrication of palladium-based membranes.pdf:pdf},
pages = {43--67},
publisher = {Woodhead Publishing},
title = {{The use of electroless plating as a deposition technology in the fabrication of palladium-based membranes}},
year = {2015}
}
@article{Wang2016,
abstract = {The recent advancement of water stable metal-organic frameworks (MOFs) expands the application of this unique porous material. This review article aims at studying their applications in terms of five major areas: adsorption, membrane separation, sensing, catalysis, and proton conduction. These applications are either conducted in a water-containing environment or directly targeted on water treatment processes. The representative and significant studies in each area were comprehensively reviewed and discussed for perspectives, to serve as a reference for researchers working in related areas. At the end, a summary and future outlook on the applications of water stable MOFs are suggested as concluding remarks.},
annote = {Wang, Chenghong
Liu, Xinlei
Keser Demir, Nilay
Chen, J Paul
Li, Kang
eng
England
2016/07/14 06:00
Chem Soc Rev. 2016 Sep 21;45(18):5107-34. doi: 10.1039/c6cs00362a. Epub 2016 Jul 13.},
author = {Wang, C and Liu, X and {Keser Demir}, N and Chen, J P and Li, K},
doi = {10.1039/c6cs00362a},
isbn = {1460-4744 (Electronic)
0306-0012 (Linking)},
journal = {Chem Soc Rev},
number = {18},
pages = {5107--5134},
pmid = {27406473},
title = {{Applications of water stable metal-organic frameworks}},
url = {https://www.ncbi.nlm.nih.gov/pubmed/27406473},
volume = {45},
year = {2016}
}
@article{Zhou2018,
author = {Zhou, Fanglei and Fathizadeh, Mahdi and Yu, Miao},
file = {:Users/marc/Library/Application Support/Mendeley Desktop/Downloaded/Zhou, Fathizadeh, Yu - 2018 - Single- to Few-Layered , Graphene-Based Separation Membranes.pdf:pdf},
keywords = {gas,graphene,graphene oxide,membrane separation,separation,single layer,water purification},
title = {{Single- to Few-Layered , Graphene-Based Separation Membranes}},
year = {2018}
}
@article{Mukaida2010,
author = {Mukaida, M and Takahashi, N and Hisamatsu, K and Ishitsuka, M and Hara, S and Suda, H and Haraya, K},
doi = {10.1016/j.memsci.2010.09.025},
isbn = {03767388},
journal = {Journal of Membrane Science},
number = {1-2},
pages = {378--381},
title = {{Preparation for defect-free self-supported Pd membranes by an electroless plating method}},
volume = {365},
year = {2010}
}
@article{Zhou2014a,
abstract = {Membrane separation of CO2 from natural gas, biogas, synthesis gas, and flu gas is a simple and energy-efficient alternative to other separation techniques. But results for CO2 -selective permeance have always been achieved by randomly oriented and thick zeolite membranes. Thin, oriented membranes have great potential to realize high-flux and high-selectivity separation of mixtures at low energy cost. We now report a facile method for preparing silica MFI membranes in fluoride media on a graded alumina support. In the resulting membrane straight channels are uniformly vertically aligned and the membrane has a thickness of 0.5 $\mu$m. The membrane showed a separation selectivity of 109 for CO2/H2 mixtures and a CO2 permeance of 51×10(-7) mol m(-2) s(-1) Pa(-1) at -35 °C, making it promising for practical CO2 separation from mixtures.},
author = {Zhou, Ming and Korelskiy, Danil and Ye, Pengcheng and Grahn, Mattias and Hedlund, Jonas},
doi = {10.1002/anie.201311324},
isbn = {1521-3773 (Electronic)$\backslash$r1433-7851 (Linking)},
issn = {15213773},
journal = {Angewandte Chemie - International Edition},
keywords = {carbon dioxide,gas separation,membranes,thin films,zeolites},
number = {13},
pages = {3492--3495},
pmid = {24590761},
title = {{A uniformly oriented MFI membrane for improved CO2 separation}},
volume = {53},
year = {2014}
}

@article{Gao2016,
author = {Gao, Pingyi and Liu, Rong and Huang, Huihan and Jia, Xiao and Pan, Haibo},
doi = {10.1039/C6RA21567J},
file = {:Users/marc/Library/Application Support/Mendeley Desktop/Downloaded/Gao et al. - 2016 - MOF-templated controllable synthesis of $\alpha$-Fe sub2sub O sub3sub porous nanorods and their gas sensing properties.pdf:pdf},
issn = {2046-2069},
journal = {RSC Adv.},
number = {97},
pages = {94699--94705},
publisher = {Royal Society of Chemistry},
title = {{MOF-templated controllable synthesis of $\alpha$-Fe {\textless}sub{\textgreater}2{\textless}/sub{\textgreater} O {\textless}sub{\textgreater}3{\textless}/sub{\textgreater} porous nanorods and their gas sensing properties}},
url = {http://xlink.rsc.org/?DOI=C6RA21567J},
volume = {6},
year = {2016}
}
@article{Ahmed2010,
author = {Ahmed, Shabbir and Lee, Sheldon H D and Papadias, Dionissios D},
doi = {10.1016/j.ijhydene.2010.08.042},
isbn = {03603199},
journal = {International Journal of Hydrogen Energy},
number = {22},
pages = {12480--12490},
title = {{Analysis of trace impurities in hydrogen: Enrichment of impurities using a H2 selective permeation membrane}},
volume = {35},
year = {2010}
}
@article{Xie2010,
abstract = {Phosphorescent cyclometalated iridium tris(2-phenylpyridine) derivatives were designed and incorporated into coordination polymers as tricarboxylate bridging ligands. Three different crystalline coordination polymers were synthesized using a solvothermal technique and were characterized using a variety of methods, including, single-crystal X-ray diffiraction, PXRD, TGA, IR spectroscopy, gas adsorption measurements, and luminescence measurements. The coordination polymer built from 1r-[3-(2-pyridyl)benzoate](3), 1, was found to be highly porous with a nitrogen BET surface area of 764 m(2)/g, whereas the coordination polymers built from Ir-[4-(2-pyridyl)benzoate](3), 2 and were nonporous. The (3)MLCT phosphorescence of each of the three coordination polymers was quenched in the presence of O(2). However, only 1 showed quick and reversible luminescence quenching by oxygen, whereas 2 and 3 exhibited gradual and irreversible luminescence quenching by oxygen. The high permanent porosity of 1 allows for rapid diffusion of oxygen through the open channels, leading to efficient and reversible quenching of the (3)MLCT phosphorescence. This work highlights the opportunity of designing highly porous and luminescent coordination polymers for sensing other important analytes.},
author = {Xie, Zhigang and Ma, Liqing and DeKrafft, Kathryn E. and Jin, Athena and Lin, Wenbin},
doi = {10.1021/ja909629f},
file = {:Users/marc/Library/Application Support/Mendeley Desktop/Downloaded/Xie et al. - 2010 - Porous phosphorescent coordination polymers for oxygen sensing.pdf:pdf},
isbn = {0002-7863},
issn = {00027863},
journal = {Journal of the American Chemical Society},
number = {3},
pages = {922--923},
pmid = {20041656},
title = {{Porous phosphorescent coordination polymers for oxygen sensing}},
volume = {132},
year = {2010}
}
@article{AbuElHawa2015,
author = {{Abu El Hawa}, Hani W and Lundin, Sean-Thomas B and Paglieri, Stephen N and Harale, Aadesh and {Douglas Way}, J},
doi = {10.1016/j.memsci.2015.07.021},
file = {:Users/marc/Library/Application Support/Mendeley Desktop/Downloaded/Abu El Hawa et al. - 2015 - The influence of heat treatment on the thermal stability of Pd composite membranes.pdf:pdf},
isbn = {03767388},
journal = {Journal of Membrane Science},
pages = {113--120},
title = {{The influence of heat treatment on the thermal stability of Pd composite membranes}},
volume = {494},
year = {2015}
}
@article{Nayebossadri2013,
author = {Nayebossadri, Shahrouz and Speight, John and Book, David},
file = {:Users/marc/Library/Application Support/Mendeley Desktop/Downloaded/Nayebossadri, Speight, Book - 2013 - Development of Pd – Cu Membranes for Hydrogen Separation Why Hydrogen Separation.pdf:pdf},
title = {{Development of Pd – Cu Membranes for Hydrogen Separation Why Hydrogen Separation ?}},
year = {2013}
}
@article{Keita2012,
author = {Keita, Namory},
file = {:Users/marc/Library/Application Support/Mendeley Desktop/Downloaded/Keita - 2012 - Lattice model simulation of hydrogen effect on palladium gold alloys used as purification metal membranes.pdf:pdf},
journal = {Vasa},
number = {August},
pages = {60},
title = {{Lattice model simulation of hydrogen effect on palladium gold alloys used as purification metal membranes}},
url = {http://medcontent.metapress.com/index/A65RM03P4874243N.pdf{\%}5Cnhttp://smartech.gatech.edu/handle/1853/44712},
year = {2012}
}
@article{Ohira2015,
abstract = {Industrial gases such as nitrogen, oxygen, argon, and helium are easily contaminated with water during production, transfer and use, because there is a high volume fraction of water in the atmosphere (approximately 1.2{\%} estimated with the average annual atmospheric temperature and relative humidity). Even trace water ({\textless}1 parts per million by volume (ppmv) of H{\textless}inf{\textgreater}2{\textless}/inf{\textgreater}O, dew point {\textless} -76 ??C) in the industrial gases can cause quality problems in the process such as production of semiconductors. Therefore, it is important to monitor and to control trace water levels in industrial gases at each supplying step, and especially during their use. In the present study, a fiber optic gas sensor was investigated for monitoring trace water levels in industrial gases. The sensor consists of a film containing a metal organic framework (MOF). MOFs are made of metals coordinated to organic ligands, and have mesoscale pores that adsorb gas molecules. When the MOF, copper benzene-1,3,5-tricarboxylate (Cu-BTC), was used as a sensing material, we investigated the color of Cu-BTC with water adsorption changed both in depth and tone. Cu-BTC crystals appeared deep blue in dry gases, and then changed to light blue in wet gases. An optical gas sensor with the Cu-BTC film was developed using a light emitting diode as the light source and a photodiode as the light intensity detector. The sensor showed a reversible response to trace water, did not require heating to remove the adsorbed water molecules. The sample gas flow rate did not affect the sensitivity. The obtained limit of detection was 40 parts per billion by volume (ppbv). The response time for sample gas containing 2.5 ppmvH{\textless}inf{\textgreater}2{\textless}/inf{\textgreater}O was 23 s. The standard deviation obtained for daily analysis of 1.0 ppmvH{\textless}inf{\textgreater}2{\textless}/inf{\textgreater}O standard gas over 20 days was 9{\%}. Furthermore, the type of industrial gas did not affect the sensitivity. These properties mean the sensor will be applicable to trace water detection in various industrial gases.},
author = {Ohira, Shin Ichi and Miki, Yusuke and Matsuzaki, Toru and Nakamura, Nao and ki Sato, Yu and Hirose, Yasuo and Toda, Kei},
doi = {10.1016/j.aca.2015.05.045},
file = {:Users/marc/Library/Application Support/Mendeley Desktop/Downloaded/Ohira et al. - 2015 - A fiber optic sensor with a metal organic framework as a sensing material for trace levels of water in industrial.pdf:pdf},
issn = {18734324},
journal = {Analytica Chimica Acta},
keywords = {Fiber optic gas sensor,Industrial gas,Metal organic frameworks,Trace water},
pages = {188--193},
pmid = {26320652},
publisher = {Elsevier Ltd},
title = {{A fiber optic sensor with a metal organic framework as a sensing material for trace levels of water in industrial gases}},
url = {http://dx.doi.org/10.1016/j.aca.2015.05.045},
volume = {886},
year = {2015}
}
@article{Jeon2011,
author = {Jeon, S Y and Lim, D K and Choi, M B and Wachsman, E D and Song, S J},
doi = {10.1016/j.seppur.2011.03.018},
isbn = {13835866},
journal = {Separation and Purification Technology},
number = {3},
pages = {337--341},
title = {{Hydrogen separation by Pd–CaZr0.9Y0.1O3−{\$}\delta{\$} cermet composite membranes}},
volume = {79},
year = {2011}
}
@article{Lin2001,
abstract = {The paper attempts to summarize recent significant progress in synthesis of microporous and dense inorganic membranes with the help of the results obtained in the author's laboratory or reported in the literature. The paper first reviews several methods for synthesis of microporous amorphous and polycrystalline (zeolite) inorganic membranes and their major characteristics. These microporous membranes exhibit fascinating gas permeation and separation properties. The paper also reviews equally impressive progress on oxygen semipermeable dense oxygen ionic-conducting ceramic membranes and hydrogen semipermeable metal membranes. Searching for better membrane materials, developing effective membrane synthesis methods, and improving chemical and structural stability of the current membrane materials will continue to be the focus of active research in these areas. Specific examples of the future research include synthesis of micro and mesoporous polycrystalline inorganic membranes with oriented pores and development of more effective dense proton-conducting ceramic membranes for high temperature hydrogen separation and membrane reactor applications. ?? 2001 Elsevier Science B.V. All rights reserved.},
author = {Lin, Y. S.},
doi = {10.1016/S1383-5866(01)00089-2},
file = {:Users/marc/Library/Application Support/Mendeley Desktop/Downloaded/Lin - 2001 - Microporous and dense inorganic membranes Current status and prospective.pdf:pdf},
isbn = {1513556347},
issn = {13835866},
journal = {Separation and Purification Technology},
keywords = {Dense membrane,Ionic-conducting membranes,Membrane synthesis,Metal membranes,Microporous membrane,Separation,Zeolite membranes},
number = {1-3},
pages = {39--55},
title = {{Microporous and dense inorganic membranes: Current status and prospective}},
volume = {25},
year = {2001}
}
@article{Murugan2014,
author = {Murugan, Arul and Brown, Andrew S},
doi = {10.1039/c3ay42174k},
isbn = {1759-9660
1759-9679},
journal = {Analytical Methods},
number = {15},
pages = {5472},
title = {{Advancing the analysis of impurities in hydrogen by use of a novel tracer enrichment method}},
volume = {6},
year = {2014}
}
@article{Scholes2010,
abstract = {Rubbery polymeric membranes demonstrate potential in gas separation as they can be selective for carbon dioxide over less condensable gases such as hydrogen. In many industrial processes where these membranes are applicable, minor gas components such as carbon monoxide, hydrogen sulfide and water are present. Here, the effect of these minor components on the performance of a polydimethyl siloxane rubbery membrane in CO2/N2 separation is experimentally determined. The permeability of CO2 through PDMS is reduced upon exposure to CO, H2S and water, due to competitive sorption of these gases into the polymeric matrix. Similar behavior is observed for N2; however the presence of H2S gives rise to an observed increase in N2 permeability. Extension of the Flory-Huggins theory to a quaternary system allows the relevant interaction parameters for this system to be determined. Upon exposure to a wet feed, CO2 and N2 permeability decrease due to water occupying free volume within PDMS, with the Flory-Huggins interaction parameters indicating mixing of CO2 or N2 with water within PDMS is highly unfavorable. Quantitative analysis of the change in permeability of CO2 and N2 in PDMS in the presence of these minor components enables more accurate prediction of membrane performance in industrial applications to be undertaken. {\textcopyright} 2010 Elsevier B.V. All rights reserved.},
author = {Scholes, Colin A. and Stevens, Geoff W. and Kentish, Sandra E.},
doi = {10.1016/j.memsci.2009.12.027},
isbn = {0376-7388},
issn = {03767388},
journal = {Journal of Membrane Science},
keywords = {Carbon dioxide,Carbon monoxide,Flory-Huggins interaction parameter,Hydrogen sulfide,Polydimethyl siloxane,Water},
number = {1-2},
pages = {189--199},
title = {{The effect of hydrogen sulfide, carbon monoxide and water on the performance of a PDMS membrane in carbon dioxide/nitrogen separation}},
volume = {350},
year = {2010}
}
@article{Tao2002,
abstract = {Nonstoichiometric mixed perovskites A3B???1+xB???2-xO9-??, e.g. Ba3Ca1.18Nb1.82O9-??, exhibit high proton and oxygen-ion conductivity. It is expected that mixed ionic and electronic conductors may be found in these compounds if the B-sites are partially substituted by a first row transition element. These mixed conductors may be potential anode materials for fuel cell applications. The structure of single phase SrCu0.4Nb0.6O2.9 was studied by both X-ray and neutron diffraction. It is tetragonal with space group P4/mmm (123), a=3.9608(4) ??, c=3.9757(2) ??, V=62.37(2) ??3 according to neutron diffraction. Rietveld refinement indicates that the oxygen vacancy tends to stay at O1 (1c) site with O2 (2e) fully occupied. AC impedance measurements indicate that electronic conduction is probably dominant in air. The DC conductivity of SrCu0.4Nb0.6O2.9 at pO2 in the range of 10-22-10-12 atm exhibits a p(O2) -1/4 dependence consistent with n-type electronic conduction. The material was unstable in 5{\%} H2 at elevated temperatures but stable in argon at 900 ??C. Using manganese instead of copper, a phase that is redox stable was prepared. SrMn0.4Nb0.6O3-?? exhibits an orthorhombic structure with space group Pbnm (62), a=5.6451(3) ??, b=5.6589(2) ??, c=7.9729(2) ??, V=254.69(7) ??3 according to X-ray diffraction. Such a unit cell indicates that it is a double perovskite and therefore the formula is better written as Sr2Mn0.8Nb1.2O6-??. The material maintains perovskite structure in 5{\%} H2 although thermal expansion was observed on reduction. The conductivity of Sr2Mn0.8Nb1.2O6 is 0.36 S/cm in air at 900 ??C. Conductivity decreases in 5{\%} H2 indicates p-type conduction at low pO2. ?? 2002 Elsevier Science B.V. All rights reserved.},
author = {Tao, Shanwen and Irvine, John T S},
doi = {10.1016/S0167-2738(02)00516-7},
file = {:Users/marc/Library/Application Support/Mendeley Desktop/Downloaded/Tao, Irvine - 2002 - Structure and properties of nonstoichiometric mixed perovskites A3B1xB2-xO9-.pdf:pdf},
isbn = {0167-2738},
issn = {01672738},
journal = {Solid State Ionics},
keywords = {Defect,Electrical conductivity,Fuel cell,Neutron diffraction,Perovskite structure,Stability},
pages = {659--667},
title = {{Structure and properties of nonstoichiometric mixed perovskites A3B???1+xB???2-xO9-??}},
volume = {154-155},
year = {2002}
}
@article{Sircar2000,
author = {Sircar, S. and Golden, T. C.},
journal = {Sep. Sci. Technol.},
pages = {667 -- 687},
title = {{Purification of hydrogen by pressure swing adsorption.}},
volume = {35},
year = {2000}
}
@article{Lee2007,
abstract = {The carbon molecular sieving membranes were prepared from the polymer blend of polyphenylene oxide (PPO) and polypyrrolidone (PVP) as thermally stable and labile polymer, respectively. The permeation results for the carbon membranes derived from the polymer blends of PPO/PVP showed that the transport of gas species was affected by the molecular sieving effect and that the permeation performances had a strong dependency upon the pyrolysis temperature and PVP molecular weight. The PPO/PVP derived carbon membranes with lower PVP molecular weight than 40 K showed decreased gas permeances and increased permselectivities due to decrease in the pore structure. Meanwhile, it was observed that gas permeance for the carbon membranes of higher molecular weights than 40 K increased due to the enhanced diffusional pathways in the thermally labile polymer region. It is considered that the introduction of the thermally labile polymer leads to control the pore structure through the permeation results for the carbon membrane derived from the PPO/PVP and that the permeation performance is affected by the molecular weight and pyrolysis temperature. {\textcopyright} 2007 Elsevier B.V. All rights reserved.},
author = {Lee, Hong Joo and Suda, Hiroyuki and Haraya, Kenji and Moon, Seung Hyeon},
doi = {10.1016/j.memsci.2007.03.025},
file = {:Users/marc/Library/Application Support/Mendeley Desktop/Downloaded/Lee et al. - 2007 - Gas permeation properties of carbon molecular sieving membranes derived from the polymer blend of polyphenylene oxid.pdf:pdf},
isbn = {03767388 (ISSN)},
issn = {03767388},
journal = {Journal of Membrane Science},
keywords = {Carbon membrane,Gas permeation,Permeability,Polyphenylene oxide,Polypyrrolidone},
number = {1-2},
pages = {139--146},
title = {{Gas permeation properties of carbon molecular sieving membranes derived from the polymer blend of polyphenylene oxide (PPO)/polyvinylpyrrolidone (PVP)}},
volume = {296},
year = {2007}
}
@article{ShushilAdhikari2006,
author = {{Shushil Adhikari}, Sandum Fernando},
chapter = {875},
journal = {Industrial and Engineering Chemistry Research},
pages = {875--881},
title = {{Hydrogen Membrane Separation Techniques}},
volume = {45},
year = {2006}
}
@article{Zhu2011,
abstract = {Ni–Ba(Zr0.1Ce0.7Y0.2)O3−$\delta$ (BZCY) metal–ceramic asymmetric membranes consisted of Ni–BZCY top membrane and porous substrate were successfully prepared and developed as hydrogen permeation membrane for the first time via a method to combine co-pressing technique and two-step sintering process. The uniform fine NiO–BZCY composite powders as the precursor of top membranes were co-synthesized through the citrate–nitrate combustion route (co-synthesis method), which was the key to fabricating Ni–BZCY thin membrane. The homogeneity and phase structure of two phases in powders were characterized using element-map technique and X-ray diffraction analysis, respectively. The fluxes through a metal–ceramic membrane of about 30-$\mu$m-thickness were measured as a function of temperature under different feed gas hydrogen partial pressures. The results indicated the asymmetric membrane displayed high hydrogen permeation flux and using 80{\%}H2/N2 (with 3{\%} of H2O) as feed gas and dry high purity argon as sweep gas, a maximum flux of 2.4×10−7molcm−2s−1 was achieved at 900°C, exhibiting the predominance of asymmetric structures.},
author = {Zhu, Zhiwen and Sun, Wenping and Yan, Litao and Liu, Weifeng and Liu, Wei},
doi = {10.1016/j.ijhydene.2011.02.029},
file = {:Users/marc/Library/Application Support/Mendeley Desktop/Downloaded/Zhu et al. - 2011 - Synthesis and hydrogen permeation of Ni–Ba(Zr0.1Ce0.7Y0.2)O3−$\delta$ metal–ceramic asymmetric membranes.pdf:pdf},
isbn = {03603199},
issn = {03603199},
journal = {International Journal of Hydrogen Energy},
keywords = {1 ce 0,2,7 y 0,ni e ba,o 3 {\`{a}} d,zr 0},
number = {10},
pages = {6337--6342},
publisher = {Elsevier Ltd},
title = {{Synthesis and hydrogen permeation of Ni–Ba(Zr0.1Ce0.7Y0.2)O3−$\delta$ metal–ceramic asymmetric membranes}},
url = {http://linkinghub.elsevier.com/retrieve/pii/S0360319911003466},
volume = {36},
year = {2011}
}
@article{K.NagaiA.Higuchi1994,
author = {{K. Nagai, A. Higuchi}, T. Nakagawa},
journal = {J. Appl. Polym. Sci.},
pages = {1207},
title = {{Bromination and gas permeability of poly(1-trimethylsilyl-1-propyne) membrane}},
volume = {54},
year = {1994}
}
@article{Gallucci2007,
author = {Gallucci, F and Chiaravalloti, F and Tosti, S and Drioli, E and Basile, A},
doi = {10.1016/j.ijhydene.2006.09.034},
file = {:Users/marc/Library/Application Support/Mendeley Desktop/Downloaded/Gallucci et al. - 2007 - The effect of mixture gas on hydrogen permeation through a palladium membrane Experimental study and theoretica.pdf:pdf},
isbn = {03603199},
journal = {International Journal of Hydrogen Energy},
number = {12},
pages = {1837--1845},
title = {{The effect of mixture gas on hydrogen permeation through a palladium membrane: Experimental study and theoretical approach}},
volume = {32},
year = {2007}
}
@article{Braun2012,
abstract = {PdAgAu alloy films were prepared on porous stainless steel supports by sequential electroless deposition. Two specific compositions, Pd83Ag2Au15 and Pd74Ag14Au12, were studied for their sulfur tolerance. The alloys and a reference Pd foil were exposed to 1000H2S/H2 at 623 K for periods of 3 and 30 h. The microstructure, morphology and bulk composition of both non-exposed and H2S-exposed samples were characterized by X-ray diffraction (XRD), scanning electron microscopy (SEM) and energy-dispersive X-ray spectroscopy (EDS). XRD and SEM analysis revealed time-dependent growth of a bulk Pd4S phase on the Pd foil during H2S exposure. In contrast, the PdAgAu ternary alloys displayed the same FCC structure before and after H2S exposure. In agreement with the XRD and SEM results, sulfur was not detected in the bulk of either ternary alloy samples by EDS, even after 30 h of H2S exposure. X-ray photoelectron spectroscopy (XPS) depth profiles were acquired for both PdAgAu alloys after 3 and 30 h of exposure to characterize sulfur contamination near their surfaces. Very low S 2p and S 2s XPS signals were observed at the top-surfaces of the PdAgAu alloys, and those signals disappeared before the etch depth reached ∼10 nm, even for samples exposed to H2S for 30 h. The depth profile analyses also revealed silver and gold segregation to the surface of the alloys; preferential location of Au on the alloys surface may be related to their resistance to bulk sulfide formation. In preliminary tests, a PdAgAu alloy membrane displayed higher initial H2 permeability than a similarly prepared pure Pd sample and, consistent with resistance to bulk sulfide formation, lower permeability loss in H2S than pure Pd.},
annote = {NULL},
author = {Braun, Fernando and Miller, James B. and Gellman, Andrew J. and Tarditi, Ana M. and Fleutot, Benoit and Kondratyuk, Petro and Cornaglia, Laura M.},
doi = {10.1016/j.ijhydene.2012.09.040},
file = {:Users/marc/Library/Application Support/Mendeley Desktop/Downloaded/Braun et al. - 2012 - PdAgAu alloy with high resistance to corrosion by H2S.pdf:pdf;:Users/marc/Library/Application Support/Mendeley Desktop/Downloaded/Braun et al. - 2012 - PdAgAu alloy with high resistance to corrosion by H2S(2).pdf:pdf},
issn = {03603199},
journal = {International Journal of Hydrogen Energy},
number = {23},
pages = {18547--18555},
publisher = {Elsevier Ltd},
title = {{PdAgAu alloy with high resistance to corrosion by H2S}},
url = {http://dx.doi.org/10.1016/j.ijhydene.2012.09.040},
volume = {37},
year = {2012}
}
@article{Tereschenko2007,
author = {Tereschenko, G and Ermilova, M and Mordovin, V and Orekhova, N and Gryaznov, V and Iulianelli, A and Gallucci, F and Basile, A},
doi = {10.1016/j.ijhydene.2007.03.044},
file = {:Users/marc/Library/Application Support/Mendeley Desktop/Downloaded/Tereschenko et al. - 2007 - New Ti–Ni dense membranes with low palladium content.pdf:pdf},
isbn = {03603199},
journal = {International Journal of Hydrogen Energy},
number = {16},
pages = {4016--4022},
title = {{New Ti–Ni dense membranes with low palladium content}},
volume = {32},
year = {2007}
}
@article{Liu2009,
abstract = {The first continuous and well-intergrown MOF-5 membrane, evidenced from SEM imaging and X-ray diffraction (XRD), was successfully prepared on porous $\alpha$-alumina substrate by in situ solvothermal synthesis. The BET measurements on crystals taken from the same mother liquor that was used for membrane synthesis yield a Langmuir surface area of 2259 m2/g and a narrow pore size distribution centered at 1.56 nm. The permeation data for simple gases of H2, CH4, N2, CO2 and SF6 show that the diffusion of simple gases through a MOF-5 membrane follows the Knudsen diffusion behavior. {\textcopyright} 2008 Elsevier Inc.},
author = {Liu, Yunyang and Ng, Zhenfu and Khan, Easir A. and Jeong, Hae Kwon and bun Ching, Chi and Lai, Zhiping},
doi = {10.1016/j.micromeso.2008.08.054},
file = {:Users/marc/Library/Application Support/Mendeley Desktop/Downloaded/Liu et al. - 2009 - Synthesis of continuous MOF-5 membranes on porous $\alpha$-alumina substrates.pdf:pdf},
isbn = {1387-1811},
issn = {13871811},
journal = {Microporous and Mesoporous Materials},
keywords = {Gas permeation,MOF-5 membranes,Metal-organic frameworks,Solvothermal synthesis},
number = {1-3},
pages = {296--301},
publisher = {Elsevier Inc.},
title = {{Synthesis of continuous MOF-5 membranes on porous $\alpha$-alumina substrates}},
url = {http://dx.doi.org/10.1016/j.micromeso.2008.08.054},
volume = {118},
year = {2009}
}
@article{Conde2016,
author = {Conde, Julio J and Maro{\~{n}}o, Marta and S{\'{a}}nchez-Herv{\'{a}}s, Jos{\'{e}} Mar{\'{i}}a},
doi = {10.1080/15422119.2016.1212379},
file = {:Users/marc/Library/Application Support/Mendeley Desktop/Downloaded/Conde, Maro{\~{n}}o, S{\'{a}}nchez-Herv{\'{a}}s - 2016 - Pd-Based Membranes for Hydrogen Separation Review of Alloying Elements and Their Influence on.pdf:pdf},
isbn = {1542-2119
1542-2127},
journal = {Separation {\&} Purification Reviews},
number = {2},
pages = {152--177},
title = {{Pd-Based Membranes for Hydrogen Separation: Review of Alloying Elements and Their Influence on Membrane Properties}},
volume = {46},
year = {2016}
}
@article{Kulprathipanja2004,
author = {Kulprathipanja, Ames and Alptekin, O and Falconer, John L and Way, J Douglas},
file = {:Users/marc/Library/Application Support/Mendeley Desktop/Downloaded/Kulprathipanja et al. - 2004 - Effects of Water Gas Shift Gases on Pd-Cu Alloy Membrane Surface.pdf:pdf},
journal = {Ind. Eng. Chem. Res},
pages = {4188--4198},
title = {{Effects of Water Gas Shift Gases on Pd-Cu Alloy Membrane Surface}},
year = {2004}
}
@article{Zhou2001a,
abstract = {A thermosetting phenolic resin with a pendant sulfonic acid group was prepared by reacting a resol-type phenolic resin (PF) with a Novalak-type sulfonated phenolic resin (SPF). Large amounts of gaseous molecules with similar and small size such as H2O and SO2 evolved in the range of 110 and 350 oC during the pyrolysis of this thermosetting phenolic resin (PF/SPF). Highly permeable carbon molecular sieve (CMS) membranes were obtained by pyrolysis of PF/SPF(45/55) precursor membranes which were dip-coated on porous alumina tubes. For example, the membrane pyrolyzed at 500 oC for 1.5 h displayed H2, CO2, and O2 permeances of 1950, 800, and 240 [GPU (gas permeation units) = 10-6 cm3(STP).s-1.cm-2.cmHg-1], respectively, and ideal H2/CH4, CO2/CH4, and O2/N2 separation factors of 65, 27, and 5.2 at 35 oC and 1 atm, respectively. Sulfonic acid groups linked to thermostable polymer chains might act as bonded templates and showed attractive potential in the preparation of CMS membranes.},
author = {Zhou, Weiliang and Yoshino, Makoto and Kita, Hidetoshi and Okamoto, Ken Ichi},
doi = {10.1021/ie010402v},
file = {:Users/marc/Library/Application Support/Mendeley Desktop/Downloaded/Zhou et al. - 2001 - Carbon molecular sieve membranes derived from phenolic resin with a pendant sulfonic acid group.pdf:pdf},
isbn = {0888-5885},
issn = {08885885},
journal = {Industrial and Engineering Chemistry Research},
number = {22},
pages = {4801--4807},
title = {{Carbon molecular sieve membranes derived from phenolic resin with a pendant sulfonic acid group}},
volume = {40},
year = {2001}
}
@article{Yassine2016,
abstract = {Herein we report the fabrication of an advanced sensor for the detection of hydrogen sulfide (H2S) at room temperature, using thin films of rare-earth metal (RE)-based metal–organic framework (MOF) with underlying fcu topology. This unique MOF-based sensor is made via the in situ growth of fumarate-based fcu-MOF (fum-fcu-MOF) thin film on a capacitive interdigitated electrode. The sensor showed a remarkable detection sensitivity for H2S at concentrations down to 100 ppb, with the lower detection limit around 5 ppb. The fum-fcu-MOF sensor exhibits a highly desirable detection selectivity towards H2S vs. CH4, NO2, H2, and C7H8 as well as an outstanding H2S sensing stability as compared to other reported MOFs.},
author = {Yassine, Omar and Shekhah, Osama and Assen, Ayalew H. and Belmabkhout, Youssef and Salama, Khaled N. and Eddaoudi, Mohamed},
doi = {10.1002/anie.201608780},
file = {:Users/marc/Library/Application Support/Mendeley Desktop/Downloaded/Yassine et al. - 2016 - H2S Sensors Fumarate-Based fcu-MOF Thin Film Grown on a Capacitive Interdigitated Electrode.pdf:pdf},
issn = {15213773},
journal = {Angewandte Chemie - International Edition},
keywords = {H2S,interdigitated electrodes,metal???organic frameworks (MOFs),rare-earth metals,sensors},
number = {51},
pages = {15879--15883},
title = {{H2S Sensors: Fumarate-Based fcu-MOF Thin Film Grown on a Capacitive Interdigitated Electrode}},
volume = {55},
year = {2016}
}
@article{Mejdell2008,
author = {Mejdell, A L and Klette, H and Ramachandran, A and Borg, A and Bredesen, R},
doi = {10.1016/j.memsci.2007.09.024},
isbn = {03767388},
journal = {Journal of Membrane Science},
number = {1},
pages = {96--104},
title = {{Hydrogen permeation of thin, free-standing Pd/Ag23{\{}{\%}{\}} membranes before and after heat treatment in air}},
volume = {307},
year = {2008}
}
@misc{TheBritishStandardsInstitute2013,
author = {{The British Standards Institute}},
booktitle = {Part 2},
title = {{Hydrogen fuel - Product Specification: Proton exchange membrane (PEM) fuel cell applications for road vehicles}},
year = {2013}
}
@article{Takano2004,
author = {Takano, T. and Ishikawa, K. and Matsuda, T. and Aoki, K.},
doi = {10.2320/matertrans.45.3360},
file = {:Users/marc/Library/Application Support/Mendeley Desktop/Downloaded/Takano et al. - 2004 - Hydrogen Permeation of Eutectic Nb-Zr-Ni Alloy Membranes Containing Primary Phases.pdf:pdf},
issn = {13459678},
journal = {Materials Transactions},
keywords = {alloy design,eutectic structure,hydrogen permeation,intermetallic compounds},
number = {12},
pages = {3360--3362},
title = {{Hydrogen Permeation of Eutectic Nb-Zr-Ni Alloy Membranes Containing Primary Phases}},
volume = {45},
year = {2004}
}
@article{Kang2018,
abstract = {Membranes with well-defined pore structure which have thin active layers may be promising materials for efficient gas separation. Graphene oxide (GO) materials have potential applications in the field of membrane separation. Here we describe a strategy for the construction of ultra-thin and flexible HKUST-1@GO intercalated membranes, where HKUST-1 is a copper-based metal–organic framework with coordinatively unsaturated metal sites, with simultaneous and synergistic modulation of permeance and selectivity to achieve high H2/CO2 separation. CuO nanosheets@GO membranes are fabricated layer-by-layer via repeated filtration cycles, then transformed to HKUST-1@GO membranes upon in situ reaction with linkers. The HKUST-1@GO membranes show enhanced performance for gas separation of H2/CO2 mixture. The number of filtration cycles is optimized to obtain H2 permeance of 5.77 × 10−7 mol m−2 s−1 Pa−1 and H2/CO2 selectivity of 73.2. Our work provides a facile strategy for the construction of membranes based on metal–organic frameworks and GO, which may be applied in the preparation of flexible membranes for gas separation applications.},
author = {Kang, Zixi and Wang, Sasa and Fan, Lili and Zhang, Minghui and Kang, Wenpei and Pang, Jia and Du, Xinxin and Guo, Hailing and Wang, Rongming and Sun, Daofeng},
doi = {10.1038/s42004-017-0002-y},
file = {:Users/marc/Library/Application Support/Mendeley Desktop/Downloaded/Kang et al. - 2018 - In situ generation of intercalated membranes for efficient gas separation.pdf:pdf},
issn = {2399-3669},
journal = {Communications Chemistry},
number = {1},
pages = {3},
publisher = {Springer US},
title = {{In situ generation of intercalated membranes for efficient gas separation}},
url = {http://www.nature.com/articles/s42004-017-0002-y},
volume = {1},
year = {2018}
}
@article{Kim2016,
abstract = {{\textcopyright} 2016 The Electrochemical Society.For the high performance and long-term stability of Pt/C catalyst for polymer electrolyte fuel cell, the deactivation of the catalysts by agglomeration, dissolution or detachment of Pt particles from carbon support have to be improved. For this purpose, we adopted a new shape of nano-carbon material for the surface modification of Pt catalyst. Our Pt nano particles encapsulated with porous graphene shells showed similar initial activity compared with the commercial catalysts showing more than 150{\%} higher long-term stability.},
author = {Kim, H. and Robertson, A.W. and Warner, J.H. and Kim, S.O.},
doi = {10.1149/07514.0837ecst},
file = {:Users/marc/Library/Application Support/Mendeley Desktop/Downloaded/Kim et al. - 2016 - Porous graphene layers on Pt catalyst for long-term stability of fuel cell electrode.pdf:pdf},
isbn = {9781607685395},
issn = {19386737 19385862},
journal = {ECS Transactions},
number = {14},
pages = {837--840},
title = {{Porous graphene layers on Pt catalyst for long-term stability of fuel cell electrode}},
volume = {75},
year = {2016}
}
@article{Kozhakhmetov2015,
author = {Kozhakhmetov, S and Sidorov, N and Piven, V and Sipatov, I and Gabis, I and Arinov, B},
doi = {10.1016/j.jallcom.2015.01.242},
file = {:Users/marc/Library/Application Support/Mendeley Desktop/Downloaded/Kozhakhmetov et al. - 2015 - Alloys based on Group 5 metals for hydrogen purification membranes.pdf:pdf},
isbn = {09258388},
journal = {Journal of Alloys and Compounds},
pages = {S36--S40},
title = {{Alloys based on Group 5 metals for hydrogen purification membranes}},
volume = {645},
year = {2015}
}
@article{Zhu2014,
abstract = {Abstract In order to obtain chemically stable hydrogen-permeable cermet membranes against CO2 and H2O, the composite membranes consisting of Ni and Ba(Zr0.7Pr0.1Y0.2)O3−$\delta$ (BZPY) are fabricated by the dry-press technique and reducing atmosphere sintering process. SEM results show that the cermet membrane is extremely dense and metal nickel is randomly distributed in BZPY oxide matrix. Hydrogen permeation properties of the Ni-BZPY membranes are systemically studied including the influence of the operating temperature, H2 concentration in feed stream, humidification degree and membrane thickness. The Ni-BZPY membrane presents good chemical stability in humid condition or CO2-containing environments and is potential candidates for hydrogen separation.},
author = {Zhu, Zhiwen and Sun, Wenping and Dong, Yingchao and Wang, Zhongtao and Shi, Zhen and Zhang, Qingping and Liu, Wei},
doi = {10.1016/j.ijhydene.2014.05.163},
file = {:Users/marc/Library/Application Support/Mendeley Desktop/Downloaded/Zhu et al. - 2014 - Evaluation of hydrogen permeation properties of Ni–Ba(Zr0.7Pr0.1Y0.2)O3−$\delta$ cermet membranes.pdf:pdf},
issn = {03603199},
journal = {International Journal of Hydrogen Energy},
number = {22},
pages = {11683--11689},
publisher = {Elsevier Ltd},
title = {{Evaluation of hydrogen permeation properties of Ni–Ba(Zr0.7Pr0.1Y0.2)O3−$\delta$ cermet membranes}},
url = {http://linkinghub.elsevier.com/retrieve/pii/S0360319914015754},
volume = {39},
year = {2014}
}
@misc{Tabouret2012,
abstract = {Dissolved barium and molybdenum incorporation in the calcite shell was investigated in the Great Scallop Pecten maximus. Sixty six individuals were exposed for 16 days to two successive dissolved Ba and Mo concentrations accurately differentiated by two different isotopic enrichments ( 97Mo, 95Mo; 135Ba, 137Ba). Soft tissue and shell isotopic composition were determined respectively by quantitative ICP-MS (Inductively Coupled Plasma Mass Spectrometer) and laser ablation - ICP-MS. Results from Ba enrichment indicate the direct incorporation of dissolved Ba into the shell in proportion to the levels in the water in which they grew with a 6-8 day delay. The low spike contributions and the low partition coefficient (D Mo = 0.0049 ± 0.0013), show that neither the soft tissue nor the shell were significantly sensitive to Mo enrichment. These results eliminate direct Mo shell enrichment by the dissolved phase, and favour a trophic uptake that will be investigated using the successive isotopic enrichment approach developed in this study. {\textcopyright} 2012 Elsevier Ltd.},
author = {Tabouret, H{\'{e}}l{\`{e}}ne and Pomerleau, S{\'{e}}bastien and Jolivet, Aur{\'{e}}lie and P{\'{e}}cheyran, Christophe and Riso, Ricardo and Th{\'{e}}bault, Julien and Chauvaud, Laurent and Amouroux, David},
booktitle = {Marine Environmental Research},
doi = {10.1016/j.marenvres.2012.03.006},
number = {undefined},
title = {{Specific pathways for the incorporation of dissolved barium and molybdenum into the bivalve shell: An isotopic tracer approach in the juvenile Great Scallop (Pecten maximus)}},
volume = {78},
year = {2012}
}
@article{Yasuda2009,
abstract = {In the face of tight product tolerances and stringent environmental regulation, the impetus for developing fast and portable chemical sensors has grown significantly in recent years. Transducers have improved greatly in sensitivity and compactness, but profitable matches with materials able to imbue them with suitable sensitivities have not always progressed at a similar pace. Pellistor-type sensors in particular have enjoyed dramatic degrees of miniaturization over the years, yet suffer nevertheless from the drawback of being broadly active to all combustible species - a legacy of the nonselective nature of combustion catalysis in general. This paper describes the effects of combustion catalyst-containing zeolites on the output signals of micromachined thin-film calorimeters, and demonstrates that sensitivity to different species can be notably altered depending on the nature (e.g. pore size) of the zeolite used. {\textcopyright} 2008 Elsevier Inc. All rights reserved.},
author = {Yasuda, Ken E. and Visser, Jacobus H. and Bein, Thomas},
doi = {10.1016/j.micromeso.2008.11.009},
file = {:Users/marc/Library/Application Support/Mendeley Desktop/Downloaded/Yasuda, Visser, Bein - 2009 - Molecular sieve catalysts on microcalorimeter chips for selective chemical sensing.pdf:pdf},
isbn = {1387-1811},
issn = {13871811},
journal = {Microporous and Mesoporous Materials},
keywords = {Chemical sensor,Combustion catalysis,Micromachined pellistor,Molecular sieve,Zeolite},
number = {1-3},
pages = {356--359},
publisher = {Elsevier Inc.},
title = {{Molecular sieve catalysts on microcalorimeter chips for selective chemical sensing}},
url = {http://dx.doi.org/10.1016/j.micromeso.2008.11.009},
volume = {119},
year = {2009}
}
@article{Huang2015,
abstract = {In comparison with traditional chemical separation processes, membrane separation is much simpler and more efficient. An ideal membrane for molecular separation should be as thin as possible to maximize its solvent flux, be mechanically robust to prevent it from fracture, and have well-defined pore sizes to guarantee its selectivity. Graphene is an excellent platform for developing size-selective membranes because of its atomic thickness, high mechanical strength, and chemical inertness. In this Perspective, we review the recent advancements on the fabrication of nanoporous graphene membranes and graphene oxide membranes (GOMs) for molecular separation. The methods of fabricating these membranes are summarized, and the mechanisms of molecular separation based on these two types of graphene membranes are compared. The challenges of synthesizing and transferring large-area nanoporous graphene membranes and engineering the performances of GOMs are discussed.},
annote = {Huang, Liang
Zhang, Miao
Li, Chun
Shi, Gaoquan
ENG
Research Support, Non-U.S. Gov't
Review
2015/08/13 06:00
J Phys Chem Lett. 2015 Jul 16;6(14):2806-15. doi: 10.1021/acs.jpclett.5b00914. Epub 2015 Jul 8.},
author = {Huang, L and Zhang, M and Li, C and Shi, G},
doi = {10.1021/acs.jpclett.5b00914},
file = {:Users/marc/Library/Application Support/Mendeley Desktop/Downloaded/Huang et al. - 2015 - Graphene-Based Membranes for Molecular Separation.pdf:pdf},
isbn = {1948-7185 (Electronic)
1948-7185 (Linking)},
journal = {J Phys Chem Lett},
keywords = {*Membranes, Artificial,Graphite/*chemistry,Oxides/chemistry},
number = {14},
pages = {2806--2815},
pmid = {26266866},
title = {{Graphene-Based Membranes for Molecular Separation}},
url = {http://www.ncbi.nlm.nih.gov/pubmed/26266866},
volume = {6},
year = {2015}
}
@article{Huang2014,
author = {Huang, Hubiao and Ying, Yulong and Peng, Xinsheng},
doi = {10.1039/c4ta02359e},
isbn = {2050-7488
2050-7496},
journal = {Journal of Materials Chemistry A},
number = {34},
pages = {13772},
title = {{Graphene oxide nanosheet: an emerging star material for novel separation membranes}},
volume = {2},
year = {2014}
}
@article{Ratnasamy2009,
author = {Ratnasamy, Chandra and Wagner, Jon P},
doi = {10.1080/01614940903048661},
isbn = {0161-4940
1520-5703},
journal = {Catalysis Reviews},
number = {3},
pages = {325--440},
title = {{Water Gas Shift Catalysis}},
volume = {51},
year = {2009}
}
@article{Jayaraman1995,
abstract = {The present work focuses on the synthesis and gas permeation properties of ceramic supported ultrathin palladium-silver alloy membranes. PdAg films with a thickness ranging from 250 to 500 nm are coated on the surface of 3 nm pore sol-gel derived $\gamma$-alumina support using an RF magnetron sputtering equipment. The coated PdAg membranes exhibit the same composition and phase structure as those of the PdAg foil used as the target in sputter deposition. The hydrogen to nitrogen separation factor of the ultrathin PdAg membrane is 5.7 at 250°C and increases with increasing temperature. Under proper preparation conditions, use of a pinhole-free $\gamma$-alumina support is the key to ensure the gas-tightness and high-selectivity of the coated PdAg membranes. A method is demonstrated for studying hydrogen permeation through ultrathin metallic films. Hydrogen permeation data at different hydrogen pressures and temperatures (100–250°C) are reported to examine the mechanism of hydrogen permeation through the ultrathin metallic membranes. The experimental results clearly indicate the dominant role of surface reactions for hydrogen permeation through ultrathin metallic films at low temperatures.},
author = {Jayaraman, V. and Lin, Y.S. S},
doi = {10.1016/0376-7388(95)00040-J},
file = {:Users/marc/Library/Application Support/Mendeley Desktop/Downloaded/Jayaraman, Lin - 1995 - Synthesis and hydrogen permeation properties of ultrathin palladium-silver alloy membranes.pdf:pdf},
isbn = {0376-7388},
issn = {03767388},
journal = {Journal of Membrane Science},
keywords = {ceramic membrane,hydrogen permeation,metallic membranes,sputter deposition},
number = {3},
pages = {251--262},
title = {{Synthesis and hydrogen permeation properties of ultrathin palladium-silver alloy membranes}},
volume = {104},
year = {1995}
}
@article{Braun2012c,
abstract = {PdAgAu alloy films were prepared on porous stainless steel supports by sequential electroless deposition. Two specific compositions, Pd83Ag2Au15 and Pd74Ag14Au12, were studied for their sulfur tolerance. The alloys and a reference Pd foil were exposed to 1000H2S/H2 at 623 K for periods of 3 and 30 h. The microstructure, morphology and bulk composition of both non-exposed and H2S-exposed samples were characterized by X-ray diffraction (XRD), scanning electron microscopy (SEM) and energy-dispersive X-ray spectroscopy (EDS). XRD and SEM analysis revealed time-dependent growth of a bulk Pd4S phase on the Pd foil during H2S exposure. In contrast, the PdAgAu ternary alloys displayed the same FCC structure before and after H2S exposure. In agreement with the XRD and SEM results, sulfur was not detected in the bulk of either ternary alloy samples by EDS, even after 30 h of H2S exposure. X-ray photoelectron spectroscopy (XPS) depth profiles were acquired for both PdAgAu alloys after 3 and 30 h of exposure to characterize sulfur contamination near their surfaces. Very low S 2p and S 2s XPS signals were observed at the top-surfaces of the PdAgAu alloys, and those signals disappeared before the etch depth reached ∼10 nm, even for samples exposed to H2S for 30 h. The depth profile analyses also revealed silver and gold segregation to the surface of the alloys; preferential location of Au on the alloys surface may be related to their resistance to bulk sulfide formation. In preliminary tests, a PdAgAu alloy membrane displayed higher initial H2 permeability than a similarly prepared pure Pd sample and, consistent with resistance to bulk sulfide formation, lower permeability loss in H2S than pure Pd.},
author = {Braun, Fernando and Miller, James B. and Gellman, Andrew J. and Tarditi, Ana M. and Fleutot, Benoit and Kondratyuk, Petro and Cornaglia, Laura M.},
doi = {10.1016/j.ijhydene.2012.09.040},
file = {:Users/marc/Library/Application Support/Mendeley Desktop/Downloaded/Braun et al. - 2012 - PdAgAu alloy with high resistance to corrosion by H2S(2).pdf:pdf},
issn = {03603199},
journal = {International Journal of Hydrogen Energy},
number = {23},
pages = {18547--18555},
title = {{PdAgAu alloy with high resistance to corrosion by H2S}},
volume = {37},
year = {2012}
}
@article{An2012,
abstract = {We present a detailed investigation of the nucleation sites, growth, and morphology of large-area graphene samples synthesized via chemical vapor deposition (CVD) on bulk palladium substrates. The CVD chamber was systematically controlled over a large range of growth temperatures and durations, and the nature of graphene growth under these conditions was thoroughly investigated using a combination of scanning electron microscopy and a statistical analysis of {\textgreater}500 Raman spectra. Graphene growth was found to initiate at ?825 °C, above which the growth rate increased rapidly. At T = 1000 °C, defect-free high-quality graphene was found to grow at an unprecedented rate of tens of micrometers per second, orders of magnitude faster than past reports on Cu- or Ni-based growth, thus leading to macroscopic coverage of the substrate within seconds of growth initiation. By arresting the growth at lower temperatures, we found that graphene nanoislands preferred to nucleate at very specific positions close to terrace edges and step inner edges. Evidence of both epitaxial and self-limiting growth was found. Along with monolayer graphene, both Bernal and turbostratic multilayer graphene could be obtained. A detailed evolution of the different types of graphene, as a function of both growth temperature and duration, has been presented. From these, optimal growth conditions for any chosen type of graphene sample can be inferred.},
author = {An, Xiaohong and Liu, Fangze and Jung, Yung Joon and Kar, Swastik},
doi = {10.1021/jp301196u},
file = {:Users/marc/Library/Application Support/Mendeley Desktop/Downloaded/An et al. - 2012 - Large-area synthesis of graphene on palladium and their Raman spectroscopy.pdf:pdf},
isbn = {1932-7447},
issn = {19327447},
journal = {Journal of Physical Chemistry C},
number = {31},
pages = {16412--16420},
title = {{Large-area synthesis of graphene on palladium and their Raman spectroscopy}},
volume = {116},
year = {2012}
}
@article{Boon2015,
author = {Boon, Jurriaan and Pieterse, J A Z and van Berkel, F P F and van Delft, Y C and {van Sint Annaland}, M},
doi = {10.1016/j.memsci.2015.08.061},
file = {:Users/marc/Library/Application Support/Mendeley Desktop/Downloaded/Boon et al. - 2015 - Hydrogen permeation through palladium membranes and inhibition by carbon monoxide, carbon dioxide, and steam.pdf:pdf},
isbn = {03767388},
journal = {Journal of Membrane Science},
pages = {344--358},
title = {{Hydrogen permeation through palladium membranes and inhibition by carbon monoxide, carbon dioxide, and steam}},
volume = {496},
year = {2015}
}
@article{Escolastico2011,
abstract = {This contribution presents the preparation, permeation and stability study of mixed protonic-electronic conducting membranes based on the system Nd 5LnWO12. The tungstates Ln6WO12 are proton conducting crystalline materials, which show sufficient protonic and electronic mixed conductivity and stability in moist CO2 environments to consider them as potential candidates for the separation of hydrogen at high temperatures. Hydrogen separation properties of A-substoichiometric Nd 6WO12 and Nd5LaWO12 were systematically analyzed, i.e. the influence of the H2 concentration in feed stream, humidification degree and operating temperature on the hydrogen separation was studied. Finally, the stability of these materials at different temperatures and CO2-rich and sulfur-containing environments was evaluated. {\textcopyright} 2011, Hydrogen Energy Publications, LLC. Published by Elsevier Ltd. All rights reserved.},
author = {Escol{\'{a}}stico, Sonia and Sol{\'{i}}s, Cecilia and Serra, Jos{\'{e}} M.},
doi = {10.1016/j.ijhydene.2011.06.026},
file = {:Users/marc/Library/Application Support/Mendeley Desktop/Downloaded/Escol{\'{a}}stico, Sol{\'{i}}s, Serra - 2011 - Hydrogen separation and stability study of ceramic membranes based on the system Nd5LnWO12.pdf:pdf},
isbn = {03603199},
issn = {03603199},
journal = {International Journal of Hydrogen Energy},
keywords = {Hydrogen permeation,Hydrogen-permeable membrane,Lanthanide tungstate,Ln6WO12,Proton conducting oxide},
number = {18},
pages = {11946--11954},
title = {{Hydrogen separation and stability study of ceramic membranes based on the system Nd5LnWO12}},
volume = {36},
year = {2011}
}
@article{Salleh2011,
abstract = {Carbon membranes prepared by pyrolysis/carbonization of polymeric precursors have been studied in the last few years as a promising candidate for gas separation process. As the aim of this paper, a review on polymer precursor selection and effect of pyrolysis conditions on carbon membrane characteristics and performances were discussed in detail. A number of different polymer precursors have been surveyed for their utility as materials in carbon membrane fabrication. The gas transport properties of various types of carbon membrane that produced by different researchers was summarized. Furthermore, the potential applications and future directions of carbon membrane in gas separation processes were also briefly identified.},
author = {Salleh, W. N W and Ismail, A. F. and Matsuura, T. and Abdullah, M. S.},
doi = {10.1080/15422119.2011.555648},
file = {:Users/marc/Library/Application Support/Mendeley Desktop/Downloaded/Salleh et al. - 2011 - Precursor selection and process conditions in the preparation of carbon membrane for gas separation A review.pdf:pdf},
isbn = {1542-2119},
issn = {15422119},
journal = {Separation and Purification Reviews},
keywords = {Precursor,carbon membrane,pyrolysis,separation,stabilization},
number = {4},
pages = {261--311},
title = {{Precursor selection and process conditions in the preparation of carbon membrane for gas separation: A review}},
volume = {40},
year = {2011}
}
@article{Nagai1995,
author = {Nagai, K. and Higuchi, A. and Nakagawa, T.},
journal = {J. Polym. Sci.: Part B: Polym. Phys.},
pages = {289},
title = {{Gas permeability and stability of poly(1-trimethylsilyl-1-propyne-co-1-phenyl-1-propyne) membranes}},
volume = {33},
year = {1995}
}
@article{DiFelice2016,
abstract = {Oxygen permeation through BSCF perovskite hollow fiber membranes has been investigated in a laboratory scale reactor in a temperature range of 850???1000 ??C at atmospheric pressure. An experimental study was conducted with particular focus on a new sealing technique which was found to provide fully gas-tight conditions with a 100{\%} success rate for the 6 membranes reported in this work (and for 15 membranes in total). An overview of experimental tests on other sealants proposed in the literature is given. A new method for checking the sealing is suggested that allows detecting the possible leakages in all the parts of the membrane module. A loose-end reactor with a shell-and-tube layout is presented which allows accurate measurements of both gas leakages (if any) and product gas composition in a wide range of operating conditions (flow rates on both the permeate and feed side, O2 feed partial pressure, temperature). Permeation fluxes were studied for BSCF membranes with different thicknesses (0.2 and 0.5 mm) using He as sweep gas. The O2 flux increased with both increasing the O2 driving force (i.e., the O2 partial pressure difference through the dense ceramic layer) and the reactor temperature. Long term tests (600 h) have been carried out proving the high stability of both the membranes used and the newly proposed sealing technique. The presence of both O2 bulk diffusion and surface exchange resistance is suggested to limit the permeation rate of the ceramic hollow fibers for the range of wall thicknesses investigated. Finally, LSCF capillaries have also been successfully sealed allowing to extend the use of the proposed gas tight sealing system to different compositions of hollow fibers perovskite membranes.},
author = {{Di Felice}, L. and Middelkoop, V. and Anzoletti, V. and Snijkers, F. and {van Sint Annaland}, M. and Gallucci, F.},
doi = {10.1016/j.cep.2014.12.004},
file = {:Users/marc/Library/Application Support/Mendeley Desktop/Downloaded/Di Felice et al. - 2016 - New high temperature sealing technique and permeability data for hollow fiber BSCF perovskite membranes.pdf:pdf},
isbn = {0255-2701},
issn = {02552701},
journal = {Chemical Engineering and Processing: Process Intensification},
keywords = {Membrane module design,O2 flux measurement,O2 membrane,Sealing procedure},
pages = {206--219},
title = {{New high temperature sealing technique and permeability data for hollow fiber BSCF perovskite membranes}},
volume = {107},
year = {2016}
}
@article{Lober1997,
abstract = {The interaction of atomic hydrogen with the fcc(111) surfaces of Pd and Rh was investigated theoretically with an ab initio method, to find out the differences and similarities between these neighboring metals. At the Rh surface the hcp site of the threefold-coordinated adsorption sites is preferred, while at Pd almost no difference between the hcp and fee sites was found. For Pd, the occupation of subsurface positions was calculated to be more stable than bulklike positions. The energy gain caused by hydrogen absorption in subsurface positions is only about 100 meV lower than for hydrogen adsorption at the surface. In contrast, for Rh, significant differences between adsorption and absorption were calculated. The diffusion barrier for hydrogen diffusion from surface to subsurface positions was calculated and compared to the diffusion barrier in bulk. The hydrogen-induced work-function changes for the considered 4d transition-metal surfaces were positive for coverage theta=1.},
author = {L{\"{o}}ber, R and Hennig, D},
doi = {10.1103/PhysRevB.55.4761},
file = {:Users/marc/Library/Application Support/Mendeley Desktop/Downloaded/L{\"{o}}ber, Hennig - 1997 - Interaction of hydrogen with transition metal fcc(111) surfaces.pdf:pdf},
issn = {0163-1829},
journal = {Physical Review B: Condensed Matter and Materials Physics},
number = {7},
pages = {4761--4765},
title = {{Interaction of hydrogen with transition metal fcc(111) surfaces}},
url = {http://link.aps.org/doi/10.1103/PhysRevB.55.4761},
volume = {55},
year = {1997}
}
@article{GuptaChatterjee2015,
abstract = {Sensing of gas molecules is critical to environmental monitoring, control of chemical processes, agricultural, and medical applications. In particular, the detection of industrial toxic gases such as CO, NO{\textless}inf{\textgreater}x{\textless}/inf{\textgreater}, and NH{\textless}inf{\textgreater}3{\textless}/inf{\textgreater} is very important for many industries. Metal oxides have been widely studied for the sensitivity of their properties to gases even though they do have some limitations. Recently, graphene has been considered as a promising material for gas sensing since its electronic properties are strongly affected by the adsorption of foreign molecules. Intrinsic graphene has high sensitivity at low gas concentrations; but the sensor selectivity is poor which limits its use in many practical applications. Hence, hybrid architectures formed by blending of nanoparticles of metal-oxides with graphene or its derivatives have been explored by several researchers which showed improved gas sensing ability, especially the sensitivity and selectivity at room temperature. Here we review the state of the art of gas sensors based on graphene and metal oxide hybrid nanostructures for detection of various common toxic gases.},
author = {{Gupta Chatterjee}, Shyamasree and Chatterjee, Somenath and Ray, Ajoy K. and Chakraborty, Amit K.},
doi = {10.1016/j.snb.2015.07.070},
file = {:Users/marc/Library/Application Support/Mendeley Desktop/Downloaded/Gupta Chatterjee et al. - 2015 - Graphene-metal oxide nanohybrids for toxic gas sensor A review.pdf:pdf},
isbn = {4518806034},
issn = {09254005},
journal = {Sensors and Actuators, B: Chemical},
keywords = {Carbon nanotube,Functionalization,Gas sensor,Graphene,Metal-oxide,Nanocomposite,Nanohybrid,Reduced graphene oxide},
number = {2},
pages = {1170--1181},
publisher = {Elsevier B.V.},
title = {{Graphene-metal oxide nanohybrids for toxic gas sensor: A review}},
url = {http://dx.doi.org/10.1016/j.snb.2015.07.070},
volume = {221},
year = {2015}
}
@article{Zubkov1998,
author = {Zubkov, A and Fujino, T and Sato, N and Yamada, K},
file = {:Users/marc/Library/Application Support/Mendeley Desktop/Downloaded/Zubkov et al. - 1998 - Enthalpies of formation of the palladium sulphides.pdf:pdf},
journal = {J. Chem. Thermodynamics},
keywords = {enthalpy of formation,high-temperature mixing calorimetry,palladium sulphides},
pages = {571--581},
title = {{Enthalpies of formation of the palladium sulphides}},
volume = {30},
year = {1998}
}
@article{V.JayaramanM.PakalaR.Y.Lin1995,
author = {{V. Jayaraman M. Pakala, R.Y. Lin}, Y S Lin},
file = {:Users/marc/Library/Application Support/Mendeley Desktop/Downloaded/V. Jayaraman M. Pakala, R.Y. Lin - 1995 - Fabrication of ultrathin metallic membranes on ceramic supports by sputter deposition.pdf:pdf},
journal = {Journal of Membrane Science},
pages = {89--100},
title = {{Fabrication of ultrathin metallic membranes on ceramic supports by sputter deposition}},
volume = {99},
year = {1995}
}
@misc{HyGrid2018,
author = {HyGrid},
title = {{Hydrogen Recovery from natural gas grids}},
year = {2018}
}
@article{Nejad2016,
abstract = {Mixed matrix membranes (MMM) consist of a polymeric base with additive fillers. Zeolites, carbon molecular sieves, graphene and carbon nanotubes (CNTs) are most commonly used fillers in the development of MMMs. Among these materials, CNTs have been proposed recently for gas separation application due to their attractive properties. Although CNTs have excellent separation properties, preparation of CNT-MMMs is more complicated. To employ CNTs as effective reinforcement in the polymeric matrix, proper dispersion and suitable interfacial adhesion between the CNTs and the polymer matrix have to be guaranteed. In this paper, recent advances and developments on CNTs dispersion and alignment in the matrix were reviewed. Also, a critical comparison of various CNT functionalization methods and different functional groups are given. Applications of CNT-MMMs in gas separation are also reviewed.},
author = {Nejad, Mahsa Nahavandi and Asghari, Morteza and Afsari, Morteza},
doi = {10.1002/cben.201600012},
file = {:Users/marc/Library/Application Support/Mendeley Desktop/Downloaded/Nejad, Asghari, Afsari - 2016 - Investigation of Carbon Nanotubes in Mixed Matrix Membranes for Gas Separation A Review.pdf:pdf},
isbn = {2196-9744},
issn = {21969744},
journal = {ChemBioEng Reviews},
keywords = {10,1002,2016,201600012,accepted,carbon nanotube,cben,doi,gas separation,july 14,mixed matrix membrane,november 09,november 11,polymeric membrane,received,revised,selectivity},
number = {6},
pages = {276--298},
title = {{Investigation of Carbon Nanotubes in Mixed Matrix Membranes for Gas Separation: A Review}},
url = {http://doi.wiley.com/10.1002/cben.201600012},
volume = {3},
year = {2016}
}
@article{Zong2012,
abstract = {A new type of nanoscale coordination particles (NCPs) are successfully synthesized on a large scale through a coordination-induced self-assembling process. The as-prepared NCPs exhibit fascinating fluorescence properties including large stokes shifts, strong photoluminescence (PL) intensity, high photochemical stability, as well as tunable emission spectra. Excitingly, this new type of NCPs enable rapid, sensitive and modification-free detection of H2S as its fluorescence can be selectively quenched in the presence of H2S. With this NCPs-based detection system, the lowest concentration to quantify H2S can be down to 2 ppm, which is five times lower than the permissible exposure limit value set by the US National Institute for Occupational Safety and Health (NIOSH). Importantly, NCPs can serve as ‘inks' for writeable detection of H2S. The ability to directly write the H2S-sensitive NCPs under ambient conditions is really convenient and offers promising perspectives for real-time monitoring H2S.},
author = {Zong, Chenghua and Liu, Xiaojuan and Sun, Hongmei and Zhang, Guo and Lu, Lehui},
doi = {10.1039/c2jm32802j},
file = {:Users/marc/Library/Application Support/Mendeley Desktop/Downloaded/Zong et al. - 2012 - A new type of nanoscale coordination particles toward modification-free detection of hydrogen sulfide gas.pdf:pdf},
issn = {0959-9428},
journal = {Journal of Materials Chemistry},
number = {35},
pages = {18418},
title = {{A new type of nanoscale coordination particles: toward modification-free detection of hydrogen sulfide gas}},
url = {http://xlink.rsc.org/?DOI=c2jm32802j},
volume = {22},
year = {2012}
}
@article{Okazaki2009,
abstract = {Hydrogen permeation performance of palladium membranes supported on porous alpha-alumina and yttria-stabilized zirconia (YSZ) was studied at 300-850 degrees C. The hydrogen permeation flux across the palladium-alpha-alumina membrane decreased markedly during permeation tests conducted at {\textgreater}600 degrees C. The SEM and XPS studies of the post-test membrane revealed the presence of aluminium in the palladium layer. Such migration of aluminium was not observed by heating the palladium-alpha-alumina membrane under an argon atmosphere, indicating that hydrogen is responsible for this phenomenon. Hydrogen-induced strong metal-support interaction might be related to this considerable loss of the hydrogen flux. Reduction of alumina to Al(0) by active hydrogen at the membrane-support interface and subsequent migration of Al(0) into the palladium layer represents the most plausible mechanism for the aluminium diffusion. Actually, Al(0) that migrated into the palladium membrane layer generated less hydrogen-permeable palladium-aluminium alloy or inter-metallic compound phase. In contrast, no such strong interaction was found between the YSZ support and the palladium membrane. This composite membrane exhibited a steady permeation of hydrogen at 650 degrees C for 336 h. Having a remarkably high reduction potential, Y(III) is unlikely to be reduced to Y(0), although Zr(IV) has a comparable reduction potential to that of Al(III). A binary phase diagram shows a liquid alloy phase present for the Pd/Al couple at temperatures greater than 615 degrees C (eutectic point), while an inter-metallic compound or liquid alloy phase in the Pd-Zr binary system is not apparent at temperatures less than 750 degrees C. Consequently, inter-diffusion of zirconium with palladium did not occur during operations at 650 degrees C.},
annote = {Okazaki, Junya
Ikeda, Takuji
Pacheco Tanaka, David A
Llosa Tanco, Margot A
Wakui, Yoshito
Sato, Koich
Mizukami, Fujio
Suzuki, Toshishige M
eng
Research Support, Non-U.S. Gov't
England
2009/09/24 06:00
Phys Chem Chem Phys. 2009 Oct 14;11(38):8632-8. doi: 10.1039/b909401f. Epub 2009 Jul 23.},
author = {Okazaki, J and Ikeda, T and {Pacheco Tanaka}, D A and {Llosa Tanco}, M A and Wakui, Y and Sato, K and Mizukami, F and Suzuki, T M},
doi = {10.1039/b909401f},
isbn = {1463-9084 (Electronic)
1463-9076 (Linking)},
journal = {Phys Chem Chem Phys},
keywords = {*Hot Temperature,*Membranes, Artificial,Aluminum Oxide/*chemistry,Hydrogen/*chemistry,Palladium/*chemistry,Porosity,Surface Properties,Zirconium/chemistry},
number = {38},
pages = {8632--8638},
pmid = {19774298},
title = {{Importance of the support material in thin palladium composite membranes for steady hydrogen permeation at elevated temperatures}},
url = {https://www.ncbi.nlm.nih.gov/pubmed/19774298},
volume = {11},
year = {2009}
}
@article{Darmawan2015,
abstract = {This work investigates the preparation, characterisation and performance of binary iron/cobalt oxide silica membranes by sol-gel synthesis using tetraethyl orthosilicate as the silica precursor, and cobalt and iron nitrates. It was found that cobalt and iron oxides were generally dispersed homogeneously in the silica structure, with the exception of a few minor patches rich in cobalt oxide. The sol-gel synthesis affected the micro-structural formation of binary metal oxide silica matrices. Increasing the iron content favoured condensation reactions and the formation of siloxane bridges, and consequently larger average pore sizes which lead to low He/N2permselectivity values below 20. In the case of high cobalt content, a higher silanol to siloxane ratio was observed with tighter pore size tailoring, as evidenced by higher He/N2permselectivities reaching 170. The binary metal oxide and silica interfaces proved to follow a molecular sieving mechanism characterised by activated transport where the permeance of the smaller gas molecules (He and H2) increased with temperature up to 500°C, whilst the permeance of larger gas molecules (CO2and N2) decreased.},
author = {Darmawan, Adi and Motuzas, Julius and Smart, Simon and Julbe, Anne and {Diniz da Costa}, Jo{\~{a}}o C.},
doi = {10.1016/j.memsci.2014.09.033},
file = {:Users/marc/Library/Application Support/Mendeley Desktop/Downloaded/Darmawan et al. - 2015 - Binary iron cobalt oxide silica membrane for gas separation.pdf:pdf},
issn = {18733123},
journal = {Journal of Membrane Science},
keywords = {Binary metal oxide doping,Cobalt oxide,Iron oxide,Silica membranes},
pages = {32--38},
publisher = {Elsevier},
title = {{Binary iron cobalt oxide silica membrane for gas separation}},
url = {http://dx.doi.org/10.1016/j.memsci.2014.09.033},
volume = {474},
year = {2015}
}
@article{Liu2015,
abstract = {Graphene is a well-known two-dimensional material that exhibits preeminent electrical, mechanical and thermal properties owing to its unique one-atom-thick structure. Graphene and its derivatives (e.g., graphene oxide) have become emerging nano-building blocks for separation membranes featuring distinct laminar structures and tunable physicochemical properties. Extraordinary molecular separation properties for purifying water and gases have been demonstrated by graphene-based membranes, which have attracted a huge surge of interest during the past few years. This tutorial review aims to present the latest groundbreaking advances in both the theoretical and experimental chemical science and engineering of graphene-based membranes, including their design, fabrication and application. Special attention will be given to the progresses in processing graphene and its derivatives into separation membranes with three distinct forms: a porous graphene layer, assembled graphene laminates and graphene-based composites. Moreover, critical views on separation mechanisms within graphene-based membranes will be provided based on discussing the effect of inter-layer nanochannels, defects/pores and functional groups on molecular transport. Furthermore, the separation performance of graphene-based membranes applied in pressure filtration, pervaporation and gas separation will be summarized. This article is expected to provide a compact source of relevant and timely information and will be of great interest to all chemists, physicists, materials scientists, engineers and students entering or already working in the field of graphene-based membranes and functional films.},
annote = {Liu, Gongping
Jin, Wanqin
Xu, Nanping
ENG
England
2015/05/20 06:00
Chem Soc Rev. 2015 Aug 7;44(15):5016-30. doi: 10.1039/c4cs00423j. Epub 2015 May 18.},
author = {Liu, G and Jin, W and Xu, N},
doi = {10.1039/c4cs00423j},
isbn = {1460-4744 (Electronic)
0306-0012 (Linking)},
journal = {Chem Soc Rev},
number = {15},
pages = {5016--5030},
pmid = {25980986},
title = {{Graphene-based membranes}},
url = {http://www.ncbi.nlm.nih.gov/pubmed/25980986},
volume = {44},
year = {2015}
}
@article{Ayturk2006,
author = {Ayturk, M Engin and Mardilovich, Ivan P and Engwall, Erik E and Hua, Yi},
doi = {10.1016/j.memsci.2006.09.008},
file = {:Users/marc/Library/Application Support/Mendeley Desktop/Downloaded/Ayturk et al. - 2006 - Synthesis of composite Pd-porous stainless steel ( PSS ) membranes with a Pd Ag intermetallic diffusion barrier.pdf:pdf},
journal = {Journal of Membrane Science},
keywords = {ag,ag barrier,bi-metal multi-layer,bmml,deposition,hydrogen separation,intermetallic diffusion,pd,pss composite membrane},
pages = {385--394},
title = {{Synthesis of composite Pd-porous stainless steel ( PSS ) membranes with a Pd / Ag intermetallic diffusion barrier}},
volume = {285},
year = {2006}
}
@article{Kato1979,
author = {Kato, M.},
file = {:Users/marc/Library/Application Support/Mendeley Desktop/Downloaded/Kato - 1979 - Electroless Gold Plating Bath Using Ascorbic Acid as Reducing Agent - Recent Improvements.pdf:pdf},
issn = {0140-6736 (Print)},
journal = {Lancet (London, England)},
keywords = {Ascorbic Acid,Biological Availability,Common Cold,Dose-Response Relationship,Drug,Humans,administration {\&} dosage,metabolism,prevention {\&} control,therapeutic use},
number = {8116},
pages = {615},
pmid = {85208},
title = {{Electroless Gold Plating Bath Using Ascorbic Acid as Reducing Agent - Recent Improvements}},
volume = {1},
year = {1979}
}
@article{Huang2010,
abstract = {A novel synthesis strategy was developed for the seeding-free preparation of dense LTA zeolite membranes by using 3-aminopropyltriethoxysilane as covalent linker between zeolite membrane and porous Al2O3 support. In a first step, the ethoxy groups of the 3-aminopropyltriethoxysilane react with surface hydroxy groups of the support. In a second step, the 3-aminopropylsilyl groups react with silanols of the zeolite crystals, leading to a "bridge" between the growing LTA zeolite membrane and the support being built to anchor the LTA zeolite layer onto the support. The SEM image and XRD indicate that a dense and pure LTA zeolite membrane with a thickness of about 3.5 $\mu$m can be formed on the $\alpha$-Al2O3 support, and no cracks, pinholes or other defects are visible. The LTA zeolite membrane displays good molecular sieving performance. For binary mixtures at 20 °C, the separation factors of H2/CH4, H2/N2, H2/O2 and H2/CO2 are 3.6, 4.2, 4.4 and 5.5, respectively, which are higher than the corresponding value of Knudsen diffusion. Moreover, relative high H2 permeances of about 3.0 × 10-7 mol m-2 s-1 Pa-1 can be obtained. {\textcopyright} 2009 Elsevier B.V. All rights reserved.},
author = {Huang, Aisheng and Liang, Fangyi and Steinbach, Frank and Caro, J{\"{u}}rgen},
doi = {10.1016/j.memsci.2009.12.029},
file = {:Users/marc/Library/Application Support/Mendeley Desktop/Downloaded/Huang et al. - 2010 - Preparation and separation properties of LTA membranes by using 3-aminopropyltriethoxysilane as covalent linker.pdf:pdf},
isbn = {0376-7388},
issn = {03767388},
journal = {Journal of Membrane Science},
keywords = {Covalent linker,Hydrothermal synthesis,LTA zeolite membrane,Molecular sieve membrane},
number = {1-2},
pages = {5--9},
title = {{Preparation and separation properties of LTA membranes by using 3-aminopropyltriethoxysilane as covalent linker}},
volume = {350},
year = {2010}
}
@misc{Lu2015,
abstract = {The seashell has been studied as a proxy for the marine researches since it is the biomineralization product recording the growth development and the ocean ecosystem evolution. In this work a hybrid of Laser Induced Breakdown Spectroscopy (LIBS) and Raman spectroscopy was introduced to the composition analysis of seashell (scallop, bivalve, Zhikong). Without any sample treatment, the compositional distribution of the shell was obtained using LIBS for the element detection and Raman for the molecule recognition respectively. The elements Ca, K, Li, Mg, Mn and Sr were recognized by LIBS; the molecule carotene and carbonate were identified with Raman. It was found that the LIBS detection result was more related to the shell growth than the detection result of Raman. The obtained result suggested the shell growth might be developing in both horizontal and vertical directions. It was indicated that the LIBS-Raman combination could be an alternative way for the shell researches.},
author = {Lu, Yuan and Li, Yuandong and Li, Ying and Wang, Yangfan and Wang, Shi and Bao, Zhenmin and Zheng, Ronger},
booktitle = {Spectrochimica Acta - Part B Atomic Spectroscopy},
doi = {10.1016/j.sab.2015.05.012},
number = {undefined},
title = {{Micro spatial analysis of seashell surface using laser-induced breakdown spectroscopy and Raman spectroscopy}},
volume = {110},
year = {2015}
}
@article{Li2018a,
author = {Li, Wanbin and Shi, Jiali and Li, Zhanjun and Wu, Wufeng and Xia, Yan and Yu, Yang and Zhang, Guoliang},
doi = {10.1002/admi.201800032},
file = {:Users/marc/Library/Application Support/Mendeley Desktop/Downloaded/Li et al. - 2018 - Hydrothermally Reduced Graphene Oxide Interfaces for Synthesizing High-Performance Metal–Organic Framework Hollow F.pdf:pdf},
issn = {21967350},
journal = {Advanced Materials Interfaces},
keywords = {graphene oxides,hydrogen purification,interfacial synthesis,metal–organic frameworks,ultrathin membranes},
number = {14},
pages = {1--7},
title = {{Hydrothermally Reduced Graphene Oxide Interfaces for Synthesizing High-Performance Metal–Organic Framework Hollow Fiber Membranes}},
volume = {5},
year = {2018}
}
@article{Fu2017,
author = {Fu, Haitao and Yang, Xiaohong and An, Xizhong and Fan, Weiren and Jiang, Xuchuan and Yu, Aibing},
doi = {10.1016/j.snb.2017.05.027},
file = {:Users/marc/Library/Application Support/Mendeley Desktop/Downloaded/Fu et al. - 2017 - Experimental and Theoretical Studies of V 2 O 5 @TiO 2 Core-shell Hybrid Composites with High Gas Sensing Performance.pdf:pdf},
issn = {09254005},
journal = {Sensors and Actuators B: Chemical},
keywords = {tio 2 core-shell composites,v 2 o 5},
pages = {103--115},
publisher = {Elsevier B.V.},
title = {{Experimental and Theoretical Studies of V 2 O 5 @TiO 2 Core-shell Hybrid Composites with High Gas Sensing Performance towards Ammonia}},
url = {http://linkinghub.elsevier.com/retrieve/pii/S0925400517308407},
volume = {252},
year = {2017}
}
@article{A.C.SavocaA.D.Surnamer1993,
author = {{A.C. Savoca, A.D. Surnamer}, C.-F. Tien},
journal = {Macromolecules},
pages = {6211},
title = {{Gas transport in poly(silylpropynes): the chemical structure point of view}},
volume = {26},
year = {1993}
}
@article{Lang1985,
abstract = {The electrostatic interaction between two adsorbates, and, in particular, between an adsorbed atom and an adsorbed or adsorbing molecule is studied. Based on self-consistent calculations of the electrostatic potential around a series of atoms outside a jellium surface, it is shown that a simple electrostatic interaction can explain a large number of experimental observations concerning the influence of pre-adsorbed atoms on the adsorption rate, stability and adsorption configuration of simple molecules on metal surfaces. The role of pre-adsorbed alkalis as promoters and of electronegative atoms like P, S, Cl and O as poisons for the adsorption of electron acceptor molecules like H2, O2, N2 and CO is discussed, as well as the relative magnitude of the influence of the alkalis and the electronegative atoms. The peculiar effects that pre-adsorbed atoms have on molecules like H2O and NH3 are ascribed to the large intra-molecular electron transfer in these molecules. {\{}{\textcopyright}{\}} 1985.},
author = {Lang, N D and Holloway, S and N{\o}rskov, J K},
doi = {10.1016/0039-6028(85)90208-0},
isbn = {00396028 (ISSN)},
issn = {00396028},
journal = {Surface Science},
number = {1},
pages = {24--38},
title = {{Electrostatic adsorbate-adsorbate interactions: The poisoning and promotion of the molecular adsorption reaction}},
volume = {150},
year = {1985}
}
@article{Wang2004,
author = {Wang, D and Flanagan, Ted B and Shanahan, Kirk L},
doi = {10.1016/j.jallcom.2003.09.150},
file = {:Users/marc/Library/Application Support/Mendeley Desktop/Downloaded/Wang, Flanagan, Shanahan - 2004 - Permeation of hydrogen through pre-oxidized Pd membranes in the presence and absence of CO.pdf:pdf},
isbn = {09258388},
journal = {Journal of Alloys and Compounds},
number = {1-2},
pages = {158--164},
title = {{Permeation of hydrogen through pre-oxidized Pd membranes in the presence and absence of CO}},
volume = {372},
year = {2004}
}
@article{Yang2015,
abstract = {A copper-based metal-org. framework-graphene nanocomposite (Cu-MOF-GN) (i.e. Cu3(BTC)2, BTC = 1,3,5-benzene-tricarboxylate) was prepd. by a facile one-step method for the first time.  Unlike the conventional strategies, in this procedure graphene oxide was reduced to GN by an endogenous reducing agent produced by DMF, which was used as solvent in the synthesis of Cu-MOF.  The nanocomposite exhibited high stability due to the hydrogen bonding, $\pi$-$\pi$ stacking and Cu-O coordination between Cu-MOF and GN.  Owing to the synergetic effect of Cu-MOF and GN, the Cu-MOF-GN nanocomposite showed high electrocatalytic activity.  When it was used for constructing H2O2 and ascorbic acid sensors, it presented good performance.  Thus, the Cu-MOF-GN nanocomposite has potential applications in the electrochem. field. [on SciFinder(R)]},
author = {Yang, Juan and Zhao, Faqiong and Zeng, Baizhao},
doi = {10.1039/C4RA16950F},
file = {:Users/marc/Library/Application Support/Mendeley Desktop/Downloaded/Yang, Zhao, Zeng - 2015 - One-step synthesis of a copper-based metal–organic framework–graphene nanocomposite with enhanced electroc.pdf:pdf},
issn = {2046-2069},
journal = {RSC Adv.},
number = {28},
pages = {22060--22065},
publisher = {Royal Society of Chemistry},
title = {{One-step synthesis of a copper-based metal–organic framework–graphene nanocomposite with enhanced electrocatalytic activity}},
url = {http://xlink.rsc.org/?DOI=C4RA16950F},
volume = {5},
year = {2015}
}
@article{Mundschau2006,
author = {Mundschau, M V and Xie, X and Iv, C R Evenson and Sammells, A F},
doi = {10.1016/j.cattod.2006.01.042},
file = {:Users/marc/Library/Application Support/Mendeley Desktop/Downloaded/Mundschau et al. - 2006 - Dense inorganic membranes for production of hydrogen from methane and coal with carbon dioxide sequestration.pdf:pdf},
journal = {Catalysis today},
keywords = {oxygen transport membranes,sequestration,water-gas shift reactor},
pages = {12--23},
title = {{Dense inorganic membranes for production of hydrogen from methane and coal with carbon dioxide sequestration}},
volume = {118},
year = {2006}
}
@article{Stassen2016,
abstract = {We present a highly sensitive gas detection approach for the infamous 'nerve agent' group of alkyl phosphonate compds. Signal transduction is achieved by monitoring the work function shift of metal-org. framework UiO-66-NH2 coated electrodes upon exposure to ppb-level concns. of a target simulant. Using the Kelvin probe technique, we demonstrate the potential of elec. insulating MOFs for integration in field effect devices such as ChemFETs: a three orders of magnitude improvement over previous work function-based detection of nerve agent simulants. Moreover, the signal is fully reversible both in dry and humid conditions, down to low ppb concns. Comprehensive investigation of the interactions that lead towards this high sensitivity points towards a series of confined interactions between the analyte and the pore interior of UiO-66-NH2. [on SciFinder(R)]},
author = {Stassen, I. and Bueken, B. and Reinsch, H. and Oudenhoven, J. F. M. and Wouters, D. and Hajek, J. and {Van Speybroeck}, V. and Stock, N. and Vereecken, P. M. and {Van Schaijk}, R. and {De Vos}, D. and Ameloot, R.},
doi = {10.1039/C6SC00987E},
file = {:Users/marc/Library/Application Support/Mendeley Desktop/Downloaded/Stassen et al. - 2016 - Towards metal–organic framework based field effect chemical sensors UiO-66-NH 2 for nerve agent detection.pdf:pdf},
issn = {2041-6520},
journal = {Chem. Sci.},
keywords = {UiO66NH2 chem sensor nerve agent metal org framewo},
number = {9},
pages = {5827--5832},
publisher = {Royal Society of Chemistry},
title = {{Towards metal–organic framework based field effect chemical sensors: UiO-66-NH 2 for nerve agent detection}},
url = {http://xlink.rsc.org/?DOI=C6SC00987E},
volume = {7},
year = {2016}
}
@article{Chen2012a,
abstract = {Interaction of carbon monoxide (CO) with transition metal surfaces is an essential part of CO oxidation catalysis. In this report, we investigate and compare CO adsorption behavior on Pt (1. 1. 1) and Pd (1. 1. 1) surfaces combining first-principles (FP) calculations and lattice gas Monte-Carlo (LG-MC) simulations. Our results indicate that despite stronger CO binding on Pd (1. 1. 1) at low coverage, more repulsive lateral interactions on Pd surface lead to a more rapid adsorption energy decrease with respect to coverage. This results in lower saturation coverage and weaker CO desorption energies on Pd (1. 1. 1), which could contribute to its excellent reactivity observed under high pressure reaction conditions. ?? 2011 Elsevier B.V.},
author = {Chen, R. and Chen, Z. and Ma, B. and Hao, X. and Kapur, N. and Hyun, J. and Cho, K. and Shan, B.},
doi = {10.1016/j.comptc.2011.07.015},
file = {:Users/marc/Library/Application Support/Mendeley Desktop/Downloaded/Chen et al. - 2012 - CO adsorption on Pt (111) and Pd (111) surfaces A first-principles based lattice gas Monte-Carlo study.pdf:pdf},
isbn = {2210-271X},
issn = {2210271X},
journal = {Computational and Theoretical Chemistry},
keywords = {CO adsorption,First-principles,Lattice gas,Monte-Carlo},
pages = {77--83},
title = {{CO adsorption on Pt (111) and Pd (111) surfaces: A first-principles based lattice gas Monte-Carlo study}},
volume = {987},
year = {2012}
}
@article{Yi2016,
abstract = {Metal–organic frameworks (MOFs) as chemical sensors have developed rapidly in recent years. There have been many papers concerning this field and interest is still growing. The reason is that the specific merits of MOFs can be utilized to enhance sensitivity and selectivity by various energy/charge transfers occurring among different ligands, ligand, and metal centers, such as from ligands to metal centers or metal centers to ligands, as well as from MOF skeletons to guest species. This review intends to provide an update on recent progress in various applications of different MOF-based sensors on the basis of their luminescent and electrochemical responses towards small molecules, gas molecules, ions (cations and anions), pH, humidity, temperature, and biomolecules. MOF-based sensors function by utilizing different mechanisms, including luminescent responses of “turn-on” and “turn-off”, as well as electrochemical responses.},
author = {Yi, Fei-Yan and Chen, Dongxiao and Wu, Meng-Ke and Han, Lei and Jiang, Hai-Long},
doi = {10.1002/cplu.201600137},
file = {:Users/marc/Library/Application Support/Mendeley Desktop/Downloaded/Yi et al. - 2016 - Chemical Sensors Based on Metal-Organic Frameworks.pdf:pdf},
isbn = {8148631867},
issn = {21926506},
journal = {ChemPlusChem},
number = {8},
pages = {675--690},
title = {{Chemical Sensors Based on Metal-Organic Frameworks}},
url = {http://doi.wiley.com/10.1002/cplu.201600137},
volume = {81},
year = {2016}
}
@article{Hille1990,
author = {Hille, Jurgen},
journal = {Journal of Chromatography},
pages = {265--274},
title = {{Enrichment and mass spectrometric analysis of trace impurity concentrations in gases}},
volume = {502},
year = {1990}
}
@article{Kingsbury2009,
author = {Kingsbury, Benjamin F K and Li, K},
doi = {10.1016/j.memsci.2008.11.050},
isbn = {03767388},
journal = {Journal of Membrane Science},
number = {1-2},
pages = {134--140},
title = {{A morphological study of ceramic hollow fibre membranes}},
volume = {328},
year = {2009}
}
@article{Roa2003,
abstract = {Several Pd-Cu composite membranes with different alloy compositions were made by electroless deposition on ceramic supports, and permeation rates for pure H-2 and N-2 were measured. The alloy composition was found to have a strong influence on the hydrogen permeance for these membranes. The effect of alloy composition on the H2 permeance was found to be in qualitative agreement with literature reports. A Pd60Cu40 composite membrane was prepared, and it was found to exhibit a permeance of 0.32 cm(3) (STP)/cm(2.)s(.)cmHg(1/2) at 350 degreesC, the highest among all of the membranes studied. Also, we present experimental data from the testing of another Pd-Cu composite membrane that showed an ideal selectivity for H-2 over N-2 of well over 7000 and discuss how this outcome might be related to changes in the membrane fabrication method. The influence of temperature and alloy composition on the exponential dependence (n value) of hydrogen diffusion through palladium-copper composite membranes is also discussed. Values for n oscillated between 0.5 and slightly over 1, indicating the influence of different transport mechanisms. X-ray diffraction and atomic force microscopy were used to study the surface morphology and to understand the differences in transport properties between the palladium-copper composite membranes prepared in this work and Pd60Cu40 foils commonly used.},
author = {Roa, Fernando and Way, J Douglas},
doi = {10.1021/ie030426x},
file = {:Users/marc/Library/Application Support/Mendeley Desktop/Downloaded/Roa, Way - 2003 - Influence of Alloy Composition and Membrane Fabrication on the Pressure Dependence of the Hydrogen Flux of Palladium-C.pdf:pdf},
isbn = {0888-5885},
issn = {08885885},
journal = {Ind. Eng. Chem. Res},
number = {23},
pages = {5827--5835},
title = {{Influence of Alloy Composition and Membrane Fabrication on the Pressure Dependence of the Hydrogen Flux of Palladium-Copper Membranes}},
volume = {42},
year = {2003}
}
@article{Hashi2005,
abstract = {Microstructures and hydrogen permeability (phi) of as-cast Nb-Ti-Ni alloys have been investigated by scanning, electron microscopy (SEM) and by the gas permeation technique, respectively. The phi value increases with increasing temperature and the amount of the Nb content for every alloy. In these alloys, the Nb39Ti31Ni30 alloy, consisting of the primary bcc-(Nb, Ti) solid solution and the eutectic {\{}(Nb, Ti)+TiNi{\}} phase, shows the highest phi, which is equivalent to that of Pd. On the other hand, the Nb10Ti50Ni40 alloy, consisting of the primary B2-TiNi compound and the eutectic {\{}(Nb, Ti)+TiNi{\}} phase, shows the lowest phi value among the alloys for which phi is measurable. Eutectic microstructures suppress the hydrogen embrittlement, while the primary (Nb, Ti) phase contributes to the hydrogen permeation in these alloys. The present work demonstrates that duplex alloys containing eutectic microstructures are promising for hydrogen permeation membranes with high resistance to the hydrogen embrittlement.},
author = {Hashi, Kunihiko and Ishikawa, Kazuhiro and Matsuda, Takeshi and Aoki, Kiyoshi},
doi = {10.2320/matertrans.46.1026},
file = {:Users/marc/Library/Application Support/Mendeley Desktop/Downloaded/Hashi et al. - 2005 - Microstructures and Hydrogen Permeability of Nb-Ti-Ni Alloys with High Resistance to Hydrogen Embrittlement.pdf:pdf},
isbn = {1345-9678},
issn = {1345-9678},
journal = {Materials Transactions},
keywords = {eutectic structure,hydrogen embrittlement,hydrogen permeation,tini intermetallic},
number = {5},
pages = {1026--1031},
title = {{Microstructures and Hydrogen Permeability of Nb-Ti-Ni Alloys with High Resistance to Hydrogen Embrittlement}},
url = {https://www.jstage.jst.go.jp/article/matertrans/46/5/46{\_}5{\_}1026/{\_}article},
volume = {46},
year = {2005}
}
@article{OchoaBique2018,
abstract = {This work provides a comprehensive investigation of the feasibility of hydrogen as transportation fuel from a supply chain point of view. It introduces an approach for the identification the best hydrogen infrastructure pathways making decision of primary energy source, production, storage and distribution networks to aid the target of greenhouse gas emissions reduction in Germany. The minimization of the total hydrogen supply chain (HSC) network cost for Germany in 2030 and 2050 years is the objective of this study. The model presented in this paper is expanded to take into account water electrolysis technology driven by solar and wind energy. Two scenarios are evaluated, including a full range of technologies and “green” technologies using only renewable resources. The resulting model is a mixed integer linear program (MILP) that is solved with the Advanced Integrated Multidimensional Modeling System (AIMMS). The results show that renewable energy as a power source has the potential to replace common used fossil fuel in the near future even though currently coal gasification technology is the still the dominant technology.},
author = {{Ochoa Bique}, Anton and Zondervan, Edwin},
doi = {https://doi.org/10.1016/j.cherd.2018.03.037},
issn = {0263-8762},
journal = {Chemical Engineering Research and Design},
keywords = {AIMMS,Fuel infrastructures,Germany,Hydrogen supply chain design,Mixed integer linear programming},
pages = {90--103},
title = {{An outlook towards hydrogen supply chain networks in 2050 — Design of novel fuel infrastructures in Germany}},
url = {http://www.sciencedirect.com/science/article/pii/S0263876218301643},
volume = {134},
year = {2018}
}
@article{Escolastico2012a,
abstract = {The structural and transport properties as well as the chemical stability of a series of proton-conducting oxides based on yttrium-doped barium zirconate were investigated. Specifically, Pr-, Fe- and Mn-doped BaZr1-x-yYxMyO3-delta compounds were prepared by solid state reaction. The compound exhibiting the highest total and protonic conductivity at elevated temperatures under reducing atmospheres was BaZr0.8Y0.15Mn0.05O3-delta. Temperature-programmed reduction experiments revealed a particular redox behavior related to the Mn-species under selected conditions. The hydrogen permeation was thoroughly studied as a function of the temperature, hydrogen concentration and the humidification degree in the sweep gas. Moreover, the transient processes induced by alternate step changes in the humidification degree of the sweep gas were analysed. The highest steady hydrogen evolution flow exceeded 0.03 ml min(-1) cm(-2) (0.9 mm-thick membrane) at 1000 degrees C for the humidified sweep gas. The stability of BaZr0.8Y0.15M0.05O3-delta under operation-relevant atmospheres (CO2-rich reducing atmosphere at high temperature) was tested using different techniques ( X-ray diffraction (XRD), Raman, SEM, TEM and TG) and the results showed that this material is stable even when exposed to 115 ppm H2S.},
author = {Escol{\'{a}}stico, Sonia and Ivanova, Mariya and Sol{\'{i}}s, Cecilia and Roitsch, Stefan and Meulenberg, Wilhelm A. and Serra, Jos{\'{e}} M.},
doi = {10.1039/c2ra20214j},
file = {:Users/marc/Library/Application Support/Mendeley Desktop/Downloaded/Escol{\'{a}}stico et al. - 2012 - Improvement of transport properties and hydrogen permeation of chemically-stable proton-conducting oxides.pdf:pdf},
isbn = {10.1039/C2RA20214J},
issn = {2046-2069},
journal = {RSC Advances},
number = {11},
pages = {4932},
title = {{Improvement of transport properties and hydrogen permeation of chemically-stable proton-conducting oxides based on the system BaZr1-x-yYxMyO3-$\delta$}},
url = {http://xlink.rsc.org/?DOI=c2ra20214j},
volume = {2},
year = {2012}
}
@article{Peters2008,
author = {Peters, T A and Stange, M and Klette, H and Bredesen, R},
doi = {10.1016/j.memsci.2007.08.056},
file = {:Users/marc/Library/Application Support/Mendeley Desktop/Downloaded/Peters et al. - 2008 - High pressure performance of thin Pd–23{\%}Agstainless steel composite membranes in water gas shift gas mixtures i.pdf:pdf},
isbn = {03767388},
journal = {Journal of Membrane Science},
number = {1-2},
pages = {119--127},
title = {{High pressure performance of thin Pd–23{\%}Ag/stainless steel composite membranes in water gas shift gas mixtures; influence of dilution, mass transfer and surface effects on the hydrogen flux}},
volume = {316},
year = {2008}
}
@incollection{Ohs2012,
author = {Ohs, Jan Hendrik and Sauter, Ulrich S and Maass, Sebastian},
booktitle = {Fuel Cell Science and Engineering: Materials, Processes, Systems and Technology},
chapter = {20},
editor = {Stolten, Detlef and Emonts, Bernd},
publisher = {Wiley-VCH},
title = {{Degradation caused by Dynamic Operation and Starvation Conditions}},
volume = {1},
year = {2012}
}
@article{Sabina2010,
abstract = {This work highlights new research into the fundamental properties of palladium-gold alloy membranes. Two types of self-supported palladium-gold foils were studied; membranes produced by magnetron sputtering and membranes produced by cold-working. The cold-worked membranes had thicknesses of 25 microns and gold contents from 0-40 wt{\%} Au, while the sputtered films ranged from 10-31 microns in thickness and 5-10 wt{\%} Au. These films were characterized by single-gas permeation testing in the temperature range of 473-773K and at pressures of up to 772 kPa. Membranes were studied before and after testing by XRD, XPS, XRF, and SEM/EDS. Hydrogen permeability in the 0-20 wt Au{\%} range was found to be a function of synthesis technique as much as alloy content, with no single alloy having superior permeability at all temperatures. Sputtered materials had generally higher permeability than cold-worked materials of equivalent composition, although the thicker sputtered membrane had reduced hydrogen permeability compared to its thinner counterparts. In this composition range, the addition of gold generally acted to reduce activation energy of hydrogen permeation. The differences in membrane permeability by fabrication technique are primarily attributed to preferential orientation effects. These effects also appear to contribute to other permeation phenomena, such as low-temperature hydrogen embrittlement, the dependence of flux on feed pressure, and the formation of long range ordered surface phases},
author = {Sabina, K. Gade and Kent, E. Coulter and {Douglas Way}, J.},
doi = {10.1007/BF03214998},
file = {:Users/marc/Library/Application Support/Mendeley Desktop/Downloaded/Sabina, Kent, Douglas Way - 2010 - Effects of fabrication technique upon material properties and permeation characteristics of palladium.pdf:pdf},
issn = {0017-1557},
journal = {Gold Bulletin},
number = {4},
pages = {287--297},
title = {{Effects of fabrication technique upon material properties and permeation characteristics of palladium-gold alloy membranes for hydrogen separations}},
volume = {43},
year = {2010}
}
@article{Castricum2015,
abstract = {Hybrid silica membranes are of great interest for molecular separation owing to their outstanding hydrothermal stability. Despite good separation properties in liquid applications, the selectivity for gas separations has yet been too low. Here, we report membranes from 1,2-bis(triethoxysilyl)ethane (BTESE) with H{\textless}inf{\textgreater}2{\textless}/inf{\textgreater}/N{\textless}inf{\textgreater}2{\textless}/inf{\textgreater} permselectivity between 50 and over 400. The membranes are fabricated from a dip-sol with a H{\textless}sup{\textgreater}+{\textless}/sup{\textgreater}:Si ratio of 0.01 that is applied onto a support system with a controlled low water content (pre-treated at RH{\textless}0.5{\%}). For support systems pre-treated at 90{\%} RH, H{\textless}inf{\textgreater}2{\textless}/inf{\textgreater}/N{\textless}inf{\textgreater}2{\textless}/inf{\textgreater} permselectivities≤10 are obtained, indicating larger pores. The pore formation process is studied in situ by Small-Angle X-ray Scattering in a dedicated setup. The formation of larger pores can be understood by a higher condensation rate and longer drying times when more water is present. This results in a stronger network that better withstands the compressive forces during drying. By limiting both the water and acid contents in the dipped sol, a dense pore structure is obtained that gives the highest H{\textless}inf{\textgreater}2{\textless}/inf{\textgreater}/N{\textless}inf{\textgreater}2{\textless}/inf{\textgreater} and CO{\textless}inf{\textgreater}2{\textless}/inf{\textgreater}/CH{\textless}inf{\textgreater}4{\textless}/inf{\textgreater} permselectivities found to date for hybrid silica membranes. Further variation of the water and acid concentration will allow for additional tuning of the separation properties for both gas and liquid separation.},
author = {Castricum, Hessel L. and Qureshi, Hammad F. and Nijmeijer, Arian and Winnubst, Louis},
doi = {10.1016/j.memsci.2015.03.084},
file = {:Users/marc/Library/Application Support/Mendeley Desktop/Downloaded/Castricum et al. - 2015 - Hybrid silica membranes with enhanced hydrogen and CO2 separation properties.pdf:pdf},
issn = {18733123},
journal = {Journal of Membrane Science},
keywords = {Gas separation,Hybrid silica,In-situ SAXS,Microporous membrane,Sol-gel processing},
pages = {121--128},
publisher = {Elsevier},
title = {{Hybrid silica membranes with enhanced hydrogen and CO2 separation properties}},
url = {http://dx.doi.org/10.1016/j.memsci.2015.03.084},
volume = {488},
year = {2015}
}
@article{Zhu2012,
abstract = {In this work, highly doped ceria with lanthanum, La 0.5Ce 0.5O 2-$\delta$ (LDC), are developed as hydrogen separation membrane material. LDC presents a mixed electronic and protonic conductivity in reducing atmosphere and good stability in moist CO 2 environment. LDC separation membranes with asymmetrical structure are fabricated by a cost-saving co-pressing method, using NiO + LDC + corn starch mixture as substrate and LDC as top membrane layer. Hydrogen permeation properties are systemically studied, including the influence of operating temperature, hydrogen partial pressure in feed stream and water vapor in both sides of the membrane on hydrogen permeating fluxes. Hydrogen permeability increases as the increasing of temperature and hydrogen partial pressure in feed gas. Using 20{\%} H 2/N 2 (with 3{\%} of H 2O) as feed gas and dry high purity argon as sweep gas, an acceptable flux of 2.6 × 10 -8 mol cm -2 s -1 is achieved at 900°C. The existing of water in both sides of membrane has significant effect on hydrogen permeation and the corresponding reasons are analyzed and discussed. {\textcopyright} 2012 Hydrogen Energy Publications, LLC. Published by Elsevier Ltd. All rights reserved.},
author = {Zhu, Zhiwen and Yan, Litao and Liu, Haowei and Sun, Wenping and Zhang, Qingping and Liu, Wei},
doi = {10.1016/j.ijhydene.2012.06.033},
file = {:Users/marc/Library/Application Support/Mendeley Desktop/Downloaded/Zhu et al. - 2012 - A mixed electronic and protonic conducting hydrogen separation membrane with asymmetric structure.pdf:pdf},
issn = {03603199},
journal = {International Journal of Hydrogen Energy},
keywords = {Fluorite structure,Hydrogen separation,La 0.5Ce 0.5O 2-$\delta$,Mixed electronic and protonic conductor},
number = {17},
pages = {12708--12713},
publisher = {Elsevier Ltd},
title = {{A mixed electronic and protonic conducting hydrogen separation membrane with asymmetric structure}},
url = {http://dx.doi.org/10.1016/j.ijhydene.2012.06.033},
volume = {37},
year = {2012}
}
@article{Barrett2012,
abstract = {Three phosphorescent Ir and Ru complexes containing dicarboxylate functional groups have been doped into the framework of Zr6O4(OH)(4)(bpdc)(6) (UiO-67, bpdc = para-biphenyldicarboxylate) to yield stable metal-organic frameworks (MOFs 1-3) which are highly porous with BET surface areas of 2568, 2292, and 1277 m(2) g(-1), respectively. The (MLCT)-M-3 phosphorescence of 1-3 can be effectively quenched by O-2 to provide an efficient method for oxygen detection.},
author = {Barrett, Seth M. and Wang, Cheng and Lin, Wenbin},
doi = {10.1039/c2jm15549d},
file = {:Users/marc/Library/Application Support/Mendeley Desktop/Downloaded/Barrett, Wang, Lin - 2012 - Oxygen sensing via phosphorescence quenching of doped metal–organic frameworks.pdf:pdf},
isbn = {0959-9428},
issn = {0959-9428},
journal = {Journal of Materials Chemistry},
number = {20},
pages = {10329},
title = {{Oxygen sensing via phosphorescence quenching of doped metal–organic frameworks}},
url = {http://xlink.rsc.org/?DOI=c2jm15549d},
volume = {22},
year = {2012}
}
@article{Xu2013,
author = {Xu, Lianqiang and Cheng, Li},
doi = {10.1155/2013/731875},
file = {:Users/marc/Library/Application Support/Mendeley Desktop/Downloaded/Xu, Cheng - 2013 - Graphite Oxide under High Pressure A Raman Spectroscopic Study.pdf:pdf},
issn = {1687-4110},
pages = {1--5},
title = {{Graphite Oxide under High Pressure : A Raman Spectroscopic Study}},
volume = {2013},
year = {2013}
}
@article{Chen2016,
author = {Chen, Che-Hsuan and Huang, Yu-Rewi and Liu, Chen-Wei and Wang, Kuan-Wen},
doi = {10.1016/j.tsf.2016.04.049},
file = {:Users/marc/Library/Application Support/Mendeley Desktop/Downloaded/Chen et al. - 2016 - Preparation and modification of PdAg membranes by electroless and electroplating process for hydrogen separation.pdf:pdf},
isbn = {00406090},
journal = {Thin Solid Films},
pages = {189--194},
title = {{Preparation and modification of PdAg membranes by electroless and electroplating process for hydrogen separation}},
volume = {618},
year = {2016}
}
@article{Braun2014,
annote = {NULL},
author = {Braun, Fernando and Tarditi, Ana M and Miller, James B and Cornaglia, Laura M},
doi = {10.1016/j.memsci.2013.09.026},
file = {:Users/marc/Library/Application Support/Mendeley Desktop/Downloaded/Braun et al. - 2014 - Pd-based binary and ternary alloy membranes Morphological and perm-selective characterization in the presence of H.pdf:pdf},
isbn = {03767388},
journal = {Journal of Membrane Science},
pages = {299--307},
title = {{Pd-based binary and ternary alloy membranes: Morphological and perm-selective characterization in the presence of H2S}},
volume = {450},
year = {2014}
}
@article{Howard2008a,
annote = {NULL},
author = {Howard, B. H. and Morreale, B. D.},
doi = {10.1179/174892309X12519750237717},
file = {:Users/marc/Library/Application Support/Mendeley Desktop/Downloaded/Howard, Morreale - 2008 - Effect of H sub2sub S on performance of Pd sub4sub Pt alloy membranes.pdf:pdf},
isbn = {1251975023771},
issn = {1748-9237},
journal = {Energy Materials},
keywords = {corrosion,hydrogen,hydrogen separation membranes,hydrogen sulphide,membrane,palladium,platinum alloy},
number = {3},
pages = {177--185},
title = {{Effect of H {\textless}sub{\textgreater}2{\textless}/sub{\textgreater} S on performance of Pd {\textless}sub{\textgreater}4{\textless}/sub{\textgreater} Pt alloy membranes}},
url = {http://www.tandfonline.com/doi/full/10.1179/174892309X12519750237717},
volume = {3},
year = {2008}
}
@article{Heras1997,
abstract = {The adsorption–thermodesorption behavior of H2O molecules on clean, thin polycrystalline Pd-films has been studied by means of work function (WF) and mass resolved temperature programmed desorption (TPD). Film morphology was varied and stabilized by annealing in the range 77–473 K. As deduced from WF changes during adsorption and TPD, unannealed films formed on glass at 77 K are highly porous. On the films at 77 K, H2O adsorbs molecularly and binds to the surface through the O-atom as suggested by the decrease in work function. Data can be rationalized assuming that during TPD, the adsorbed H2O molecules undergo decomposition. H2 never evolved from the films, not even at 473 K, suggesting that the observed H2O desorption peak at T{\{}{\textgreater}{\}}220 K should involve molecule rebuilding, as found in Pd single crystals.},
author = {Heras, J M and Esti{\'{u}}, G and Viscido, L},
doi = {10.1016/S0169-4332(96)00686-1},
issn = {01694332},
journal = {Applied Surface Science},
number = {4},
pages = {455--464},
title = {{The interaction of water with clean palladium films: A thermal desorption and work function study}},
url = {http://www.sciencedirect.com/science/article/pii/S0169433296006861},
volume = {108},
year = {1997}
}
@article{Song2016,
abstract = {SnO2 quantum wire/reduced graphene oxide nanocomposites (SnO2 QW/rGO) were synthesized by a facile one-step hydrothermal method with rGO and SnCl4.5H2O as the precursors. The SnO2 QW/rGO nanocomposites well-dispersed in ethanol were spin-coated onto ceramics substrates to construct chemiresistive gas sensors. The H2S-sensing isotherm curves were obtained based on the real-time response curves of the SnO2 QW/rGO gas sensors when operated at different temperatures ranging from 30??C to 70??C, from which the adsorbing/sensing performance of the specific materials were extracted and the kinetic parameters (such as response rate constant k and activation energy Ea ) of the sensing-materials were quantitatively modeled. The H2S-sensing mechanism was found to follow Langmuir isotherm and pseudo-first-order model. Compared to pure SnO2 QW sensors, the SnO2 QW/rGO gas sensors exhibited higher sensitivity and faster response rate toward H2S, which was attributed to its lower activation energy. The SnO2 QW/rGO gas sensors can even detect H2S at room temperature, highly attractive for the detection of H2S detection with lower power consumption.},
author = {Song, Zhilong and Liu, Jingyao and Liu, Qian and Yu, Haoxiong and Zhang, Wenkai and Wang, Yang and Huang, Zhao and Zang, Jianfeng and Liu, Huan},
doi = {10.1016/j.snb.2017.04.023},
file = {:Users/marc/Library/Application Support/Mendeley Desktop/Downloaded/Song et al. - 2016 - Enhanced H2S gas sensing properties based on SnO2 quantum wirereduced graphene oxide nanocomposites Equilibrium and.pdf:pdf},
issn = {09254005},
journal = {Sensors and Actuators, B: Chemical},
keywords = {Gas sensor,Kinetic modeling,Reduced graphene oxide,Sensing isotherm,SnO2 quantum wire},
pages = {632--638},
publisher = {Elsevier B.V.},
title = {{Enhanced H2S gas sensing properties based on SnO2 quantum wire/reduced graphene oxide nanocomposites: Equilibrium and kinetics modeling}},
url = {http://dx.doi.org/10.1016/j.snb.2017.04.023},
volume = {249},
year = {2016}
}
@article{Wang2006,
author = {Wang, W P and Thomas, S and Zhang, X L and Pan, X L and Yang, W S and Xiong, G X},
doi = {10.1016/j.seppur.2006.04.007},
isbn = {13835866},
journal = {Separation and Purification Technology},
number = {1},
pages = {177--185},
title = {{H2/N2 gaseous mixture separation in dense Pd/{\$}\alpha{\$}-Al2O3 hollow fiber membranes: Experimental and simulation studies}},
volume = {52},
year = {2006}
}
@phdthesis{Callaghan2006,
author = {Callaghan, Caitlin A},
isbn = {9780542638268; 0542638266},
title = {{Kinetics and Catalysis of the Water-Gas-Shift Reaction: A Microkinetic and Graph Theoretic Approach}},
year = {2006}
}
@article{Si2011,
abstract = {The amine-decorated microporous metal-organic framework CAU-1 was readily synthesized and activated using a home-made efficient protocol. It exhibited a high heat of adsorption for CO2, high CO2 uptake capacity, and an impressive selectivity for CO2 over N2. At 273 K and up to 1 atm, CO2 uptake capacity can reach as much as 7.2 mmol g-1. Comparatively, the CH4 and N2 uptakes at 273 K and 1 atm were only 1.34 mmol g-1 and 0.37 mmol g-1, respectively. The CO2/N2 selectivity was 101 : 1 at 273 K. The isosteric heat of adsorption (Qst) for CO2 was [similar]48 kJ mol-1 at the onset of adsorption, and it decreases to [similar]28 kJ mol-1 at higher CO2 pressures. Furthermore, CAU-1 can adsorb 2.0 wt{\%} and 4.0 wt{\%} hydrogen at 77 K under 1 atm and 30 atm, respectively. The adsorption characteristics of CAU-1 for methanol investigated in situ with a quartz crystal microbalance (QCM), indicated that this particular MOF structure can be used as a highly sensitive sensor for methanol detection such as direct methanol fuel cells.},
author = {Si, Xiaoliang and Jiao, Chengli and Li, Fen and Zhang, Jian and Wang, Shuang and Liu, Shuang and Li, Zhibao and Sun, Lixian and Xu, Fen and Gabelica, Zelimir and Schick, Christoph},
doi = {10.1039/C1EE01380G},
file = {:Users/marc/Library/Application Support/Mendeley Desktop/Downloaded/Si et al. - 2011 - High and selective CO2 uptake, H2storage and methanol sensing on the amine-decorated 12-connected MOF CAU-1.pdf:pdf},
isbn = {1754-5692},
issn = {1754-5692},
journal = {Energy {\&} Environmental Science},
number = {11},
pages = {4522--4527},
title = {{High and selective CO2 uptake, H2storage and methanol sensing on the amine-decorated 12-connected MOF CAU-1}},
url = {http://dx.doi.org/10.1039/C1EE01380G},
volume = {4},
year = {2011}
}
@article{He2009,
abstract = {An effort has been made to develop a new kind of SnO2-CuO gas sensor which could detect an extremely small amount of H2S gas at relatively low working temperature. The sensor nanomaterials were prepared from SnO2 hollow spheres (synthesized by employing carbon microspheres as temples) and Cu precursor by dipping method. The composition and structural characteristics of the as-prepared CuO-doped SnO2 hollow spheres were studied by X-ray photoelectron spectroscopy, X-ray powder diffraction, scanning electron microscopy, and transmission electron microscopy. Gas-sensing properties of CuO-doped SnO2 hollow sphere were also investigated. It was found that the sensor showed good selectivity and high sensitivity to H2S gas. A ppb level detection limit was obtained with the sensor at the relatively low temperature of 35 A degrees C. Such good performances are probably attributed to the hollow sphere nanostructures. Our results imply that materials with hollow sphere nanostructures are promising candidates for high-performance gas sensors.},
author = {He, Lifang and Jia, Yong and Meng, Fanli and Li, Minqiang and Liu, Jinhuai},
doi = {10.1007/s10853-009-3645-y},
file = {:Users/marc/Library/Application Support/Mendeley Desktop/Downloaded/He et al. - 2009 - Development of sensors based on CuO-doped SnO2 hollow spheres for ppb level H2S gas sensing.pdf:pdf},
isbn = {0022-2461},
issn = {00222461},
journal = {Journal of Materials Science},
number = {16},
pages = {4326--4333},
title = {{Development of sensors based on CuO-doped SnO2 hollow spheres for ppb level H2S gas sensing}},
volume = {44},
year = {2009}
}
@article{Weil2005,
abstract = {Thermal cycle and exposure tests were conducted on ceramic-to-metal joints prepared by a new sealing technique. Known as reactive air brazing, this joining method is currently being considered for use in sealing various high-temperature solid-state electrochemical devices, including planar solid oxide fuel cells (pSOFC). In order to simulate a typical pSOFC application, test specimens were prepared by joining ceramic anode/electrolyte bilayers to metal washers, of the same composition as the common frame materials employed in pSOFC stacks, using a filler metal composed of 4 mol{\%} CuO in silver. The brazed samples were exposure tested at 750 ??C for 200, 400, and 800 h in both simulated fuel and air environments and thermally cycled at rapid rate (75 ??C min-1) between room temperature and 750 ??C for as many as 50 cycles. Subsequent joint strength testing and microstructural analysis indicated that the samples exposure tested in air displayed little degradation with respect to strength, hermeticity, or microstructure out to 800 h of exposure. Those tested in fuel showed no change in rupture strength or loss in hermeticity after 800 h of high-temperature exposure, but did undergo microstructural change due to the dissolution of hydrogen into the silver-based braze material. Air-brazed specimens subjected to rapid thermal cycling exhibited no loss in joint strength or hermeticity, but displayed initial signs of seal delamination along the braze-electrolyte interface after 50 cycles.},
author = {Weil, K. Scott and Coyle, Christopher A. and Darsell, Jens T. and Xia, Gordon G. and Hardy, John S.},
doi = {10.1016/j.jpowsour.2005.01.053},
file = {:Users/marc/Library/Application Support/Mendeley Desktop/Downloaded/Weil et al. - 2005 - Effects of thermal cycling and thermal aging on the hermeticity and strength of silver-copper oxide air-brazed seal.pdf:pdf},
isbn = {0378-7753},
issn = {03787753},
journal = {Journal of Power Sources},
keywords = {Hermeticity,Silver-copper oxide,Thermal cycling},
number = {1-2},
pages = {97--104},
title = {{Effects of thermal cycling and thermal aging on the hermeticity and strength of silver-copper oxide air-brazed seals}},
volume = {152},
year = {2005}
}
@article{Ye2014,
author = {Ye, Zongbiao and Jiang, Yadong and Tai, Huiling and Guo, Ningjie and Xie, Guangzhong and Yuan, Zhen},
doi = {10.1007/s10854-014-2472-3},
file = {:Users/marc/Library/Application Support/Mendeley Desktop/Downloaded/Ye et al. - 2014 - The investigation of reduced graphene oxide@ SnO2polyaniline composite thin films for ammonia detection at room tempe.pdf:pdf},
issn = {1573482X},
journal = {Journal of Materials Science: Materials in Electronics},
number = {2},
pages = {833--841},
title = {{The investigation of reduced graphene oxide@ SnO2???polyaniline composite thin films for ammonia detection at room temperature}},
volume = {26},
year = {2014}
}
@incollection{Basile2011,
author = {Basile, A and Iulianelli, A and Longo, T and Liguori, S and {De Falco}, Marcello},
booktitle = {Membrane Reactors for Hydrogen Production Processes},
doi = {10.1007/978-0-85729-151-6_2},
editor = {{De Falco  L.; laquaniello, G.}, Marcello; Marrelli},
file = {:Users/marc/Library/Application Support/Mendeley Desktop/Downloaded/Basile et al. - 2011 - Pd-based Selective Membrane State-of-the-Art.pdf:pdf},
pages = {21--55},
title = {{Pd-based Selective Membrane State-of-the-Art}},
year = {2011}
}
@article{Zhou2001,
abstract = {The compressibility factor (z) of hydrogen was evaluated from the experimental p-V-T data and calculated from equations of state, such as the SRK equation and BWR equation as well. The effect of z on the adsorbed amount calculated was demonstrated by the adsorption of hydrogen on 5A-zeolite. The compressibility factor and the fugacity coefficient of hydrogen were formulated based on the experimental p-V-T data for 60-333 K and up to 25 MPa. {\textcopyright} 2001 International Association for Hydrogen Energy.},
annote = {NULL},
author = {Zhou, Li and Zhou, Yaping},
doi = {10.1016/S0360-3199(00)00123-3},
file = {:Users/marc/Library/Application Support/Mendeley Desktop/Downloaded/Zhou, Zhou - 2001 - Determination of compressibility factor and fugacity coefficient of hydrogen in studies of adsorptive storage.pdf:pdf},
isbn = {0360-3199},
issn = {03603199},
journal = {International Journal of Hydrogen Energy},
number = {6},
pages = {597--601},
title = {{Determination of compressibility factor and fugacity coefficient of hydrogen in studies of adsorptive storage}},
volume = {26},
year = {2001}
}
@article{Zhu2015,
abstract = {In this work, hydrogen permeation properties of Ni-Lainf0.5/infCeinf0.5/infOinf2-$\delta$/inf (LDC) asymmetrical cermet membrane are investigated, including hydrogen fluxes (JHinf2/inf) under different hydrogen partial pressures, the influence of water vapor on JHinf2/inf and the long-term stability of the membrane operating under the containing-COinf2/inf atmosphere. Ni-LDC asymmetrical membrane shows the best hydrogen permeability among LDC-based hydrogen separation membranes, inferior to Ni-BaZrinf0.1/infCeinf0.7/infYinf0.2/infOinf3-$\delta$/inf asymmetrical membrane. The water vapor in feed gas is beneficial to hydrogen transport process, which promote an increase of JHinf2/inf from 5.64 × 10sup-8/sup to 6.83 × 10sup-8/sup mol cmsup-2/sup ssup-1/sup at 900 °C. Stability testing of hydrogen permeation suggests that Ni-LDC membrane remains stable against COinf2/inf. A dual function of combining hydrogen separation and generation can be realized by humidifying the sweep gas and enhance the hydrogen output by 1.0-1.5 times. Ni-LDC membrane exhibits desirable performance and durability in dual-function mode. Morphologies and phase structures of the membrane after tests are also characterized by SEM and XRD.},
author = {Zhu, Zhiwen and Sun, Wenping and Wang, Zhongtao and Cao, Jiafeng and Dong, Yingchao and Liu, Wei},
doi = {10.1016/j.jpowsour.2015.02.005},
file = {:Users/marc/Library/Application Support/Mendeley Desktop/Downloaded/Zhu et al. - 2015 - A high stability Ni-Lainf0.5infCeinf0.5infOinf2-$\delta$inf asymmetrical metal-ceramic membrane for hydrogen separation an.pdf:pdf},
issn = {03787753},
journal = {Journal of Power Sources},
keywords = {Asymmetrical membrane,Dual function,Hydrogen permeation,Stability},
pages = {417--424},
publisher = {Elsevier B.V},
title = {{A high stability Ni-Lainf0.5/infCeinf0.5/infOinf2-$\delta$/inf asymmetrical metal-ceramic membrane for hydrogen separation and generation}},
url = {http://dx.doi.org/10.1016/j.jpowsour.2015.02.005},
volume = {281},
year = {2015}
}
@article{Kang2017,
abstract = {Gas separation is one of the critical and challenging steps for industrial processes, and Metal-organic Framework (MOF) membranes are potential candidates for this application. This review mainly focuses on the recent advances in improving the performance of MOF membranes, the facing issue of MOF designation and growth in a practical application. First, we discuss three strategies for permeability and selectivity enhancement of MOF membranes, in terms of obtaining ultra-thin two-dimensional (2D) MOF nanosheets, fine-tuning of pore size of MOF framework and integrating with other species. Secondly, we review the recent potential resolutions to problems of MOF membranes in future practical applications, including scale-up preparation and stability improvement. Finally, we summarize our work by providing some general conclusions on the state of the art and an outlook on some development directions of molecule sieving membranes.},
author = {Kang, Zixi and Fan, Lili and Sun, Daofeng},
doi = {10.1039/C7TA01142C},
file = {:Users/marc/Library/Application Support/Mendeley Desktop/Downloaded/Kang, Fan, Sun - 2017 - Recent advances and challenges of metal–organic framework membranes for gas separation.pdf:pdf},
isbn = {0002-7863},
issn = {2050-7488},
journal = {J. Mater. Chem. A},
number = {21},
pages = {10073--10091},
pmid = {25290574},
publisher = {Royal Society of Chemistry},
title = {{Recent advances and challenges of metal–organic framework membranes for gas separation}},
url = {http://xlink.rsc.org/?DOI=C7TA01142C},
volume = {5},
year = {2017}
}
@article{Bizon2015,
author = {Bizon, Nicu and Oproescu, Mihai and Raceanu, Mircea},
doi = {10.1016/j.enconman.2014.11.002},
isbn = {01968904},
journal = {Energy Conversion and Management},
pages = {93--110},
title = {{Efficient energy control strategies for a Standalone Renewable/Fuel Cell Hybrid Power Source}},
volume = {90},
year = {2015}
}
@article{AirProducts2016,
author = {{Air Products}},
file = {:Users/marc/Library/Application Support/Mendeley Desktop/Downloaded/Air Products - 2016 - PRISM{\textregistered} Membrane Systems For Oil Refinery Applications.pdf:pdf},
title = {{PRISM{\textregistered} Membrane Systems For Oil Refinery Applications}},
url = {http://www.airproducts.com/{~}/media/Files/PDF/products/supply-options/prism-membrane/en-prism-process-gas-brochure.pdf?la=en},
year = {2016}
}
@article{Nagarkar2014,
abstract = {Hydrogen sulphide (H2S) is known to play a vital role in human physiology and pathology which stimulated interest in understanding complex behaviour of H2S. Discerning the pathways of H2S production and its mode of action is still a challenge owing to its volatile and reactive nature. Herein we report azide functionalized metal-organic framework (MOF) as a selective turn-on fluorescent probe for H2S detection. The MOF shows highly selective and fast response towards H2S even in presence of other relevant biomolecules. Low cytotoxicity and H2S detection in live cells, demonstrate the potential of MOF towards monitoring H2S chemistry in biological system. To the best of our knowledge this is the first example of MOF that exhibit fast and highly selective fluorescence turn-on response towards H2S under physiological conditions.},
annote = {Nagarkar, Sanjog S
Saha, Tanmoy
Desai, Aamod V
Talukdar, Pinaki
Ghosh, Sujit K
ENG
Research Support, Non-U.S. Gov't
England
2014/11/15 06:00
Sci Rep. 2014 Nov 14;4:7053. doi: 10.1038/srep07053.},
author = {Nagarkar, S S and Saha, T and Desai, A V and Talukdar, P and Ghosh, S K},
doi = {10.1038/srep07053},
isbn = {2045-2322 (Electronic)
2045-2322 (Linking)},
journal = {Sci Rep},
keywords = {Cell Line,Cell Survival,Fluorescent Dyes/*chemistry,Humans,Hydrogen Sulfide/*chemistry,Metals/*chemistry,Organic Chemicals/*chemistry},
pages = {7053},
pmid = {25394493},
title = {{Metal-organic framework based highly selective fluorescence turn-on probe for hydrogen sulphide}},
url = {http://www.ncbi.nlm.nih.gov/pubmed/25394493},
volume = {4},
year = {2014}
}
@article{Okazaki2006,
abstract = {A series of Pd-Ag membranes with different atomic ratio (Ag = 0, 5, 10, 15, 20 and 23{\{}{\%}{\}}) was fabricated by controlling the chemical components in the electroless plating bath followed by thermal annealing of the deposited metals. Gas permeation of the membranes was examined at the temperature ranges at 100-300 °C. The hydrogen flux for the membranes of low Ag content gave a marked change giving a peak in the range of 170-200 °C, which was attributed to the {\$}\alpha{\$}-{\$}\beta{\$} crystal phase transition of the hydride. The peak in hydrogen flux became less significant and appears as shoulder along with the increase of silver content. This observation coincides with the decrease of lattice size difference between {\$}\alpha{\$} and {\$}\beta{\$} phase by the increase of Ag content in the Pd-Ag alloy. Improved durability of the alloy membrane (Ag {\{}{\textgreater}{\}} 20{\{}{\%}{\}}) was demonstrated by cyclic change of gas and temperature and was attributed to the suppression of lattice expansion by alloying with more than 20{\{}{\%}{\}} of silver. {\{}{\textcopyright}{\}} 2006 Elsevier B.V. All rights reserved.},
author = {Okazaki, Junya and Tanaka, David A Pacheco and Tanco, Margot A Llosa and Wakui, Yoshito and Mizukami, Fujio and Suzuki, Toshishige M},
doi = {10.1016/j.memsci.2006.05.042},
issn = {03767388},
journal = {Journal of Membrane Science},
keywords = {{\$}\alpha{\$}-{\$}\beta{\$} phase transition,Durability,Hydrogen permeation,Pd-Ag alloy membrane},
number = {1-2},
pages = {370--374},
title = {{Hydrogen permeability study of the thin Pd-Ag alloy membranes in the temperature range across the {\$}\alpha{\$}-{\$}\beta{\$} phase transition}},
volume = {282},
year = {2006}
}
@article{Hartley1972,
author = {Hartley, F. R.},
file = {:Users/marc/Library/Application Support/Mendeley Desktop/Downloaded/Hartley - 1972 - Mean Palladium and Platinum-Halogen Bond Energies in Complexes of Divalent and Tetravalent Metals.pdf:pdf},
journal = {Nature physical science},
pages = {75--77},
title = {{Mean Palladium and Platinum-Halogen Bond Energies in Complexes of Divalent and Tetravalent Metals}},
volume = {236},
year = {1972}
}
@article{Subramanian1991,
author = {Subramanian, P R and Laughlin, D E},
doi = {10.1007/BF02645723},
issn = {1054-9714},
journal = {Journal of Phase Equilibria},
number = {2},
pages = {231--243},
title = {{Cu-Pd (Copper-Palladium)}},
url = {https://doi.org/10.1007/BF02645723},
volume = {12},
year = {1991}
}
@article{Bastani2013,
abstract = {Polymeric membrane technology has received extensive attention in the field of gas separation, recently. However, the tradeoff between permeability and selectivity is one of the biggest problems faced by pure polymer membranes, which greatly limits their further application in the chemical and petrochemical industries. To enhance gas separation performances, recent works have focused on improving polymeric membranes selectivity and permeability by fabricating mixed matrix membranes (MMMs). Inorganic zeolite materials distributed in the organic polymer matrix enhance the separation performance of the membranes well beyond the intrinsic properties of the polymer matrix. This concept combines the advantages of both components: high selectivity of zeolite molecular sieve, and mechanical integrity as well as economical processability of the polymeric materials. In this paper gas permeation mechanism through polymeric and zeolitic membranes, material selection for MMMs and their interaction with each other were reviewed. Also, interfacial morphology between zeolite and polymer in MMMs and modification methods of this interfacial region were discussed. In addition, the effect of different parameters such as zeolite loading, zeolite pore size, zeolite particle size, etc. on gas permeation tests through MMMs was critically reviewed. {\textcopyright} 2012 The Korean Society of Industrial and Engineering Chemistry.},
archivePrefix = {arXiv},
arxivId = {arXiv:1011.1669v3},
author = {Bastani, Dariush and Esmaeili, Nazila and Asadollahi, Mahdieh},
doi = {10.1016/j.jiec.2012.09.019},
eprint = {arXiv:1011.1669v3},
file = {:Users/marc/Library/Application Support/Mendeley Desktop/Downloaded/Bastani, Esmaeili, Asadollahi - 2013 - Polymeric mixed matrix membranes containing zeolites as a filler for gas separation applications.pdf:pdf},
isbn = {1226-086X},
issn = {1226086X},
journal = {Journal of Industrial and Engineering Chemistry},
keywords = {Gas separation,Mixed matrix membranes (MMMs),Polymer,Review,Zeolite},
number = {2},
pages = {375--393},
pmid = {19499895},
publisher = {The Korean Society of Industrial and Engineering Chemistry},
title = {{Polymeric mixed matrix membranes containing zeolites as a filler for gas separation applications: A review}},
url = {http://dx.doi.org/10.1016/j.jiec.2012.09.019},
volume = {19},
year = {2013}
}
@incollection{Atsonios2015,
author = {Atsonios, K and Panopoulos, K D and Doukelis, A and Koumanakos, A K and Kakaras, E and Peters, T A and van Delft, Y C},
booktitle = {Palladium Membrane Technology for Hydrogen Production, Carbon Capture and Other Application},
doi = {10.1533/9781782422419.1},
editor = {Doukelis, A and Panopoulos, K and Koumanakos, A and Kakaras, E},
pages = {1--21},
publisher = {Woodhead Publishing},
title = {{Introduction to palladium membrane technology}},
year = {2015}
}
@article{Flanagan1991,
author = {Flanagan, Ted B and Oates, W A},
file = {:Users/marc/Library/Application Support/Mendeley Desktop/Downloaded/Flanagan, Oates - 1991 - The Palladium-Hydrogen System.pdf:pdf},
journal = {Annu. Rev. Mater. Sci},
pages = {269--304},
title = {{The Palladium-Hydrogen System}},
volume = {21},
year = {1991}
}
@techreport{Gouwen2010,
author = {Gouwen, R J and de Kok, C J G M and Pieterse, J A Z},
booktitle = {Hydrogen Membrane Technologies},
editor = {Brouwer, Jan},
publisher = {Cato2},
title = {{Noble Metal Membrane Preparation: Industrial Film Formation Techniques}},
year = {2010}
}
@article{Peters2011,
abstract = {The manufacturing, characterisation, and H
                        2 permeation properties of thin membranes of a variety of binary Pd-Cu (???5-53at.{\%} Cu), Pd-TM (???5at.{\%} TM=Au, Ru, Mo, Ta, Nb, Y), and ternary Pd-Cu-TM alloys (base alloy???Pd
                        70Cu
                        30 where Cu is replaced by various amounts of TM) prepared by magnetron sputtering are reported. From the XRD characterisation it can be concluded that all the membranes prepared by magnetron sputtering exhibit a strong preferential orientation along the ???111??? direction. Among the components investigated, Pd-Au and Pd-Y prove to result in binary Pd-alloys with the highest H
                        2 permeability. While Pd
                        95Au
                        5 shows a similar H
                        2 permeability compared to pure Pd, Pd
                        95Y
                        5 gives a H
                        2 permeability around 15{\%} larger than pure Pd. Binary Pd-alloy with the addition of 5at.{\%} Cu, Ru, and Nb gives rise to similar H
                        2 permeabilities in the range of 80{\%} of pure Pd. The addition of 5at.{\%} Mo and Ta does not result in binary Pd-alloy materials with a large H
                        2 permeability. Among the ternary Pd-Cu-TM alloys explored the addition of ???1at.{\%} TM always result in an increase in permeability. For the investigated alloys a trend is found showing that the permeability increases and its activation energy decreases with increasing fcc lattice constant of the alloy in the temperature range 300-400??C. The addition of Ta or Y, resulting in the Pd
                        73Cu
                        26Ta
                        1 and Pd
                        73Cu
                        26Y
                        1 alloys, gives rise to an increase in H
                        2 permeability of roughly 10 and 45{\%}, respectively. For the Pd
                        65Cu
                        21Ag
                        14 alloy, a H
                        2 permeability gain of almost 65{\%} is obtained compared to the binary Pd-Cu counterpart at the same Pd content. ?? 2011 Elsevier B.V.},
author = {Peters, T. A. and Kaleta, T. and Stange, M. and Bredesen, R.},
doi = {10.1016/j.memsci.2011.08.050},
file = {:Users/marc/Library/Application Support/Mendeley Desktop/Downloaded/Peters et al. - 2011 - Development of thin binary and ternary Pd-based alloy membranes for use in hydrogen production.pdf:pdf},
issn = {03767388},
journal = {Journal of Membrane Science},
keywords = {Hydrogen flux,Hydrogen selective membrane,Magnetron sputtering,Palladium membrane,Pd-alloy,Ternary and binary},
number = {1-2},
pages = {124--134},
publisher = {Elsevier B.V.},
title = {{Development of thin binary and ternary Pd-based alloy membranes for use in hydrogen production}},
url = {http://dx.doi.org/10.1016/j.memsci.2011.08.050},
volume = {383},
year = {2011}
}
@article{Liu2008,
abstract = {Carbon molecular sieve membranes (CMSMs) were prepared from a novel polymeric precursor of poly(phthalazinone ether sulfone ketone) by thermostabilization in air at 400?500 °C and carbonization at 650?850 °C. Gas permeation properties of the resultant CMSMs were investigated with four pure gases, H2, CO2, O2, and N2, and a gas mixture of O2 and N2. The obtained CMSMs exhibited excellent gas separation performance. The temperatures of thermostabilization and carbonization had a large effect on the gas permeability and selectivity of the CMSMs. The CMSMs thermostabilized at 460 °C and carbonized at 650 °C showed the highest permeabilities for pure gases [PH2 = 1016.4 barrer (1 barrer = 10-10 cm3 (STP){\textperiodcentered}cm{\textperiodcentered}cm-2{\textperiodcentered}s-1{\textperiodcentered}cmHg-1), PCO2 = 710.3 barrer, PO2 = 187.8 barrer, PN2 = 13.9 barrer] and the O2/N2 mixture (PO2 = 165 barrer, PN2 = 10.8 barrer). The CMSMs thermostabilized at 460 °C and carbonized at 850 °C presented the highest selectivities for pure gases (PH2/PN2 = 129.2, PCO2/PN2 = 93.1, PO2/PN2 = 21.9) and O2/N2 mixture (PO2/PN2 = 23.6).$\backslash$nCarbon molecular sieve membranes (CMSMs) were prepared from a novel polymeric precursor of poly(phthalazinone ether sulfone ketone) by thermostabilization in air at 400?500 °C and carbonization at 650?850 °C. Gas permeation properties of the resultant CMSMs were investigated with four pure gases, H2, CO2, O2, and N2, and a gas mixture of O2 and N2. The obtained CMSMs exhibited excellent gas separation performance. The temperatures of thermostabilization and carbonization had a large effect on the gas permeability and selectivity of the CMSMs. The CMSMs thermostabilized at 460 °C and carbonized at 650 °C showed the highest permeabilities for pure gases [PH2 = 1016.4 barrer (1 barrer = 10-10 cm3 (STP){\textperiodcentered}cm{\textperiodcentered}cm-2{\textperiodcentered}s-1{\textperiodcentered}cmHg-1), PCO2 = 710.3 barrer, PO2 = 187.8 barrer, PN2 = 13.9 barrer] and the O2/N2 mixture (PO2 = 165 barrer, PN2 = 10.8 barrer). The CMSMs thermostabilized at 460 °C and carbonized at 850 °C presented the highest selectivities for pure gases (PH2/PN2 = 129.2, PCO2/PN2 = 93.1, PO2/PN2 = 21.9) and O2/N2 mixture (PO2/PN2 = 23.6).},
author = {Liu, Shili and Wang, Tonghua and Liu, Qingling and Zhang, Shouhai and Zhao, Zongchang and Liang, Changhai},
doi = {10.1021/ie070734l},
file = {:Users/marc/Library/Application Support/Mendeley Desktop/Downloaded/Liu et al. - 2008 - Gas permeation properties of carbon molecular sieve membranes derived from novel poly(phthalazinone ether sulfone ke.pdf:pdf},
issn = {08885885},
journal = {Industrial and Engineering Chemistry Research},
number = {3},
pages = {876--880},
title = {{Gas permeation properties of carbon molecular sieve membranes derived from novel poly(phthalazinone ether sulfone ketone)}},
volume = {47},
year = {2008}
}
@article{Huang2013,
abstract = {A seeding-free synthesis strategy was developed for the preparation of dense and phase-pure zeolite LTA membranes on 3-chloropropyltrimethoxysilane (CPTMS) functionalized supports. Through the functionalization of supports by using CPTMS, the nucleation and growth of a thin, well intergrown zeolite LTA membrane could be promoted on various porous supports such as Al2O3disks and tubes, and TiO2disks, as well as on dense supports such as stainless steel, PTFE, and glass disks. The SEM and XRD characterizations indicate that a relative thin but dense and phase-pure zeolite LTA membrane with a thickness of about 3.0$\mu$m can be formed on the porous Al2O3and TiO2disks by crystallization at 60°C for 24h, and no cracks, pinholes or other defects were observed in the membrane layer. The zeolite LTA membranes prepared on CPTMS-functionalized Al2O3disks were evaluated in single gas permeation and mixed gas separation. It is found that the zeolite LTA membranes prepared on CPTMS-modified Al2O3disks show molecular sieving in gas separation. For binary mixtures at 20°C and 1bar, the mixed gas separation factors of H2/CO2, H2/N2, H2/CH4, H2/C3H8and H2/C3H6, are found to be 7.4, 6.8, 5.2, 15.3 and 36.8, which are higher than the corresponding Knudsen coefficients. {\textcopyright} 2013 Elsevier B.V.},
author = {Huang, Aisheng and Liu, Qian and Wang, Nanyi and Tong, Xin and Huang, Bingxin and Wang, Meng and Caro, J{\"{u}}rgen},
doi = {10.1016/j.memsci.2013.02.058},
file = {:Users/marc/Library/Application Support/Mendeley Desktop/Downloaded/Huang et al. - 2013 - Covalent synthesis of dense zeolite LTA membranes on various 3-chloropropyltrimethoxysilane functionalized support.pdf:pdf},
isbn = {8657476286685},
issn = {03767388},
journal = {Journal of Membrane Science},
keywords = {3-Chloropropyltrimethoxysilane,Covalent linker,Gas separation,Molecular sieve membrane,Zeolite LTA membrane},
pages = {57--64},
title = {{Covalent synthesis of dense zeolite LTA membranes on various 3-chloropropyltrimethoxysilane functionalized supports}},
volume = {437},
year = {2013}
}
@article{H.YoshidaY.Naruse1983,
author = {{H. Yoshida  Y. Naruse}, S Konishi},
chapter = {429},
file = {:Users/marc/Library/Application Support/Mendeley Desktop/Downloaded/H. Yoshida Y. Naruse - 1983 - Effects of impurities on hydrogen permeability through palladium alloy membranes at comparativeley high pr.pdf:pdf},
journal = {Journal of Less Common Metals},
pages = {429--436},
title = {{Effects of impurities on hydrogen permeability through palladium alloy membranes at comparativeley high pressures and temperatures}},
volume = {89},
year = {1983}
}
@article{Brown2014,
abstract = {Molecular sieving metal-organic framework (MOF) membranes have great potential for energy-efficient chemical separations, but a major hurdle is the lack of a scalable and inexpensive membrane fabrication mechanism. We describe a route for processing MOF membranes in polymeric hollow fibers, combining a two-solvent interfacial approach for positional control over membrane formation (at inner and outer surfaces, or in the bulk, of the fibers), a microfluidic approach to replenishment or recycling of reactants, and an in situ module for membrane fabrication and permeation. We fabricated continuous molecular sieving ZIF-8 membranes in single and multiple poly(amide-imide) hollow fibers, with H2/C3H8 and C3H6/C3H8 separation factors as high as 370 and 12, respectively. We also demonstrate positional control of the ZIF-8 films and characterize the contributions of membrane defects and lumen bypass.},
author = {Brown, Andrew J and Brunelli, Nicholas A and Eum, Kiwon and Rashidi, Fereshteh and Johnson, J R and Koros, William J and Jones, Christopher W and Nair, Sankar and Gascon, J. and Kapteijn, F. and Zornoza, B. and Sebasti{\'{a}}n, V. and Casado, C. and Coronas, J. and Varoon, K. and Zhang, X. and Elyassi, B. and Brewer, D. D. and Gettel, M. and Kumar, S. and Lee, J. A. and Maheshwari, S. and Mittal, A. and Sung, C. Y. and Cococcioni, M. and Francis, L. F. and McCormick, A. V. and Mkhoyan, K. A. and Tsapatsis, M. and Shah, M. and McCarthy, M. C. and Sachdeva, S. and Lee, A. K. and Jeong, H. K. and Buonomenna, M. G. and Tsapatsis, M. and Pham, T. C. and Kim, H. S. and Yoon, K. B. and Choi, J. and Jeong, H. K. and Snyder, M. A. and Stoeger, J. A. and Masel, R. I. and Tsapatsis, M. and Pan, Y. and Wang, B. and Lai, Z. and Thompson, J. A. and Blad, C. R. and Brunelli, N. A. and Lydon, M. E. and Lively, R. P. and Jones, C. W. and Nair, S. and Park, K. S. and Ni, Z. and C{\^{o}}t{\'{e}}, A. P. and Choi, J. Y. and Huang, R. and Uribe-Romo, F. J. and Chae, H. K. and O'Keeffe, M. and Yaghi, O. M. and Huang, A. and Dou, W. and Caro, J. and Brown, A. J. and Johnson, J. R. and Lydon, M. E. and Koros, W. J. and Jones, C. W. and Nair, S. and Ameloot, R. and Vermoortele, F. and Vanhove, W. and Roeffaers, M. B. and Sels, B. F. and Vos, D. E. De and Pera-Titus, M. and Mallada, R. and Llorens, J. and Cunill, F. and Santamaria, J. and Li, K. and Olson, D. H. and Seidel, J. and Emge, T. J. and Gong, H. and Zeng, H. and Li, J. and Jang, K. S. and Kim, H.-J. and Johnson, J. R. and Kim, W.-. and Koros, W. J. and Jones, C. W. and Nair, S. and Gummalla, M. and Tsapatsis, M. and Watkins, J. J. and Vlachos, D. G. and Pan, Y. and Li, T. and Lestari, G. and Lai, Z. and Bux, H. and Feldhoff, A. and Cravillon, J. and Wiebcke, M. and Li, Y.-S. and Caro, J. and Pan, Y. and Lai, Z. and Kwon, H. T. and Jeong, H. K. and Lively, R. P. and Mysona, J. A. and Chance, R. R. and Koros, W. J. and Pinnau, I. and He, Z. and Shi, Y. and Burns, C. M. and Feng, X. and Zhang, C. and Lively, R. P. and Zhang, K. and Johnson, J. R. and Karvan, O. and Koros, W. J. and Chiu, W. V. and Park, I.-S. and Shqau, K. and White, J. C. and Schillo, M. C. and Ho, W. S. W. and Dutta, P. K. and Verweij, H. and Henis, J. M. S. and Tripodi, M. K.},
doi = {10.1126/science.1251181},
file = {:Users/marc/Library/Application Support/Mendeley Desktop/Downloaded/Brown et al. - 2014 - Separation membranes. Interfacial microfluidic processing of metal-organic framework hollow fiber membranes.pdf:pdf},
isbn = {0036-8075, 1095-9203},
issn = {1095-9203},
journal = {Science (New York, N.Y.)},
number = {6192},
pages = {72--5},
pmid = {24994649},
title = {{Separation membranes. Interfacial microfluidic processing of metal-organic framework hollow fiber membranes.}},
url = {http://www.ncbi.nlm.nih.gov/pubmed/24994649},
volume = {345},
year = {2014}
}
@article{Qiu2014,
abstract = {{\textless}p{\textgreater}This review provides current techniques for the fabrication of MOF membranes and separation applications of diverse MOF membranes.{\textless}/p{\textgreater}},
author = {Qiu, Shilun and Xue, Ming and Zhu, Guangshan},
doi = {10.1039/C4CS00159A},
file = {:Users/marc/Library/Application Support/Mendeley Desktop/Downloaded/Qiu, Xue, Zhu - 2014 - Metal–organic framework membranes from synthesis to separation application.pdf:pdf},
isbn = {0306-0012},
issn = {0306-0012},
journal = {Chem. Soc. Rev.},
number = {16},
pages = {6116--6140},
pmid = {24967810},
title = {{Metal–organic framework membranes: from synthesis to separation application}},
url = {http://xlink.rsc.org/?DOI=C4CS00159A},
volume = {43},
year = {2014}
}
@article{Lee2004,
abstract = {A highly hydrogen permselective composite silica membrane was obtained by depositing a thin silica layer on a porous alumina support by the chemical vapor deposition (CVD) of tetraethylorthosilicate (TEOS) at 873K in inert gas at atmospheric pressure. The silica/alumina membrane showed a high hydrogen permeance (???10-7molm-2s-1Pa-1) with selectivity over CH4, CO, and CO2 in excess of 1000 at 873K. Cross-sectional and surface images of the membranes obtained from scanning electron microscopy (SEM) and atomic force microscopy (AFM) showed that the silica film deposited on the ??-Al2O3 support was uniform with a thickness of 20-30nm. This indicated that the high temperature thermal decomposition of tetraethylorthosilicate (TEOS) used in this work was excellent in controlling the uniformity and thickness of the silica film formed on the porous alumina support. On the fresh alumina support the permeance of gases (He, H2, CH4, CO, and CO 2) decreased with temperature and molecular weight in agreement with a Knudsen transport mechanism. However, on the silica/alumina membrane the permeation of H2 and He was activated and increased with temperature. The transport mechanism for the small gas molecules (H2 and He) through the silica membrane was analyzed using a permeation mechanism which involves the jumping of the diffusing molecules between adjacent solubility sites. The model analysis indicated that the structure of the silica layer is more open than that of vitreous silica glass, with larger interconnecting passageways and low activation energies for permeation, allowing for easier gas diffusion. ?? 2003 Elsevier B.V. All rights reserved.},
author = {Lee, D. and Zhang, L. and Oyama, S. T. and Niu, S. and Saraf, R. F.},
doi = {10.1016/j.memsci.2003.10.044},
file = {:Users/marc/Library/Application Support/Mendeley Desktop/Downloaded/Lee et al. - 2004 - Synthesis, characterization, and gas permeation properties of a hydrogen permeable silica membrane supported on poro.pdf:pdf},
issn = {03767388},
journal = {Journal of Membrane Science},
keywords = {AFM,Gas and vapor permeation,Gas separation,Inorganic membrane,SEM},
number = {1-2},
pages = {117--126},
title = {{Synthesis, characterization, and gas permeation properties of a hydrogen permeable silica membrane supported on porous alumina}},
volume = {231},
year = {2004}
}
@article{Oh2009,
abstract = {The europium dopant concentration in strontium cerate was studied to achieve maximum hydrogen permeation. In order to determine high ambipolar conductivity, total conductivity and open circuit potential measurements were performed. Among the three different compositions of Eu-doped SrCe1 - xEuxO3 - ?? (x = 0.1, 0.15 and 0.2) studied, SrCe0.9Eu0.1O3 - ?? showed highest total conductivity between 600????C and 900????C. However, transference number measurements showed increasing electronic conductivity with increasing dopant concentration and a stronger temperature dependence for electronic conduction. Therefore, the highest ambipolar conductivity was obtained over the compositional range from SrCe0.85Eu0.15O3 - ?? to SrCe0.8Eu0.2O3 - ?? depending on temperature. Finally, the hydrogen permeation flux was calculated based on the ambipolar conductivity and compared with experimental results. ?? 2009 Elsevier B.V. All rights reserved.},
author = {keun Oh, Tak and Yoon, Heesung and Wachsman, E. D.},
doi = {10.1016/j.ssi.2009.07.001},
file = {:Users/marc/Library/Application Support/Mendeley Desktop/Downloaded/Oh, Yoon, Wachsman - 2009 - Effect of Eu dopant concentration in SrCe1 - xEuxO3 - on ambipolar conductivity.pdf:pdf},
isbn = {0167-2738},
issn = {01672738},
journal = {Solid State Ionics},
keywords = {Ambipolar,Conductivity,Membrane,Proton,SrCeO3},
number = {23-25},
pages = {1233--1239},
title = {{Effect of Eu dopant concentration in SrCe1 - xEuxO3 - ?? on ambipolar conductivity}},
volume = {180},
year = {2009}
}
@article{OBrien2011,
author = {O'Brien, Casey P and Gellman, Andrew J and Morreale, Bryan D and Miller, James B},
doi = {10.1016/j.memsci.2011.01.044},
isbn = {03767388},
journal = {Journal of Membrane Science},
number = {1-2},
pages = {263--267},
title = {{The hydrogen permeability of Pd4S}},
volume = {371},
year = {2011}
}
@article{LlosaTanco2016,
abstract = {Carbon molecular sieve membranes (CMSMs) are an important alternative for gas separation because of their ease of manufacture, high selectivity due to molecular sieve separation, and high permeance. The integration of separation by membranes and reaction in only one unit lead to a high degree of process integration/intensification, with associated benefits of increased energy, production efficiencies and reduced reactor or catalyst volume. This review focuses on recent advances in carbon molecular sieve membranes and their applications in membrane reactors.},
author = {{Llosa Tanco}, Margot and {Pacheco Tanaka}, David},
doi = {10.3390/pr4030029},
file = {:Users/marc/Library/Application Support/Mendeley Desktop/Downloaded/Llosa Tanco, Pacheco Tanaka - 2016 - Recent Advances on Carbon Molecular Sieve Membranes (CMSMs) and Reactors.pdf:pdf},
issn = {2227-9717},
journal = {Processes},
keywords = {carbon membrane reactor,carbon molecular sieve membrane,gas separation},
number = {3},
pages = {29},
title = {{Recent Advances on Carbon Molecular Sieve Membranes (CMSMs) and Reactors}},
url = {http://www.mdpi.com/2227-9717/4/3/29},
volume = {4},
year = {2016}
}
@article{Li2016,
abstract = {Network nano-sheet arrays of Co3O4 for high precision NH3 sensing application were prepared on alumina tube using a facile hydrothermal process without template or surfactant, and their morphology, nanostructures and NH3 gas sensing performance were investigated. The prepared nano-sheet Co3O4 arrays showed a network structure with an average sheet thickness of 39.5 nm. Detailed structural analysis confirmed that the synthesized Co3O4 nano-sheets were consisted of nanoparticles with an average diameter of 20.0 nm. NH3 gas sensor based on these network Co3O4 nano-sheet arrays showed a low detection limit (0.2 ppm), rapid response/recovery time (9 s/134 s for 0.2 ppm NH3), good reproducibility and long-term stability for NH3 detection at room temperature.},
author = {Li, Zhijie and Lin, Zhijie and Wang, Ningning and Wang, Junqiang and Liu, Wei and Sun, Kai and Fu, Yong Qing and Wang, Zhiguo},
doi = {10.1016/j.snb.2016.05.063},
file = {:Users/marc/Library/Application Support/Mendeley Desktop/Downloaded/Li et al. - 2016 - High precision NH3 sensing using network nano-sheet Co3O4 arrays based sensor at room temperature.pdf:pdf},
issn = {09254005},
journal = {Sensors and Actuators, B: Chemical},
keywords = {Co3O4,Gas sensor,Hydrothermal,NH3 gas,Nano-sheet array},
pages = {222--231},
publisher = {Elsevier B.V.},
title = {{High precision NH3 sensing using network nano-sheet Co3O4 arrays based sensor at room temperature}},
url = {http://dx.doi.org/10.1016/j.snb.2016.05.063},
volume = {235},
year = {2016}
}
@article{Das2013,
abstract = {In this paper we investigated a path towards preparation of a highly$\backslash$noriented improved SAPO 34 zeolite membrane on a silica modified low cost$\backslash$nclay-Al2O3 support by selective deposition of oriented seed crystals,$\backslash$nfollowed by an epitaxial secondary growth hydrothermal technique. The$\backslash$nmembrane thickness was found to be similar to 26 mm. The silica layer$\backslash$nthat have an abundance of reactive hydroxyl groups while its interior is$\backslash$nconnected by the siloxane group Si-O-Si. These Si-OH groups on the$\backslash$nsilica-coated substrate could provide a hydrogen bonding interaction$\backslash$nwith surface hydroxyl groups of SAPO 34 seed crystals. This outcome is a$\backslash$npossible result of the formation of a uniform seed monolayer with the$\backslash$nsame orientation. The stronger intensity of the single peak in XRD$\backslash$npattern confirms the formation of an oriented membrane layer. For$\backslash$ncomparison, a seed layer was deposited on an unmodified support surface.$\backslash$nA discrete and randomly oriented membrane layer was obtained on that$\backslash$nsupport. Highly oriented SAPO 34 membranes on the silica modified$\backslash$nsupport are better for hydrogen gas separation and attained higher$\backslash$nselectivity values for the gas mixture. A selectivity of 16.66 and 20.91$\backslash$nwas achieved for H-2-CO2 and H-2-N-2, respectively, at room temperature.$\backslash$nThe obtained values were improved compared to the reported literature$\backslash$nvalues. Herein, we report for the first time, that this work is an$\backslash$nimprovement towards a highly crystallographic orientation of the SAPO 34$\backslash$nmembrane layer with reduced defects and higher gas separation$\backslash$nefficiency. The synthesized membrane enhanced the reproducibility and$\backslash$nlong term durability for hydrogen gas separation with good results.},
author = {Das, Jugal Kishore and Das, Nandini and Bandyopadhyay, Sibdas},
doi = {10.1039/c3ta01095c},
file = {:Users/marc/Library/Application Support/Mendeley Desktop/Downloaded/Das, Das, Bandyopadhyay - 2013 - Highly oriented improved SAPO 34 membrane on low cost support for hydrogen gas separation.pdf:pdf},
issn = {2050-7488},
journal = {Journal of Materials Chemistry A},
number = {16},
pages = {4966},
title = {{Highly oriented improved SAPO 34 membrane on low cost support for hydrogen gas separation}},
url = {http://xlink.rsc.org/?DOI=c3ta01095c},
volume = {1},
year = {2013}
}
@article{Chen2010a,
abstract = {Composite palladium membranes are particularly useful in hydrogen separation because of their perfect permeability and permselectivity toward hydrogen, and porous ceramics are their most common substrate materials. High working temperatures favor membrane output, but create difficulties with membrane sealing and assembling. This work suggests a kind of facile and effective connector with graphite as the sealing material. The connector is resistant to temperature cycling, and the leakage kinetics was discussed. The possible graphite hydrogenation and the consequent membrane contamination at high temperature were also investigated. ?? 2010 Elsevier B.V. All rights reserved.},
author = {Chen, Weidong and Hu, Xiaojuan and Wang, Rongxia and Huang, Yan},
doi = {10.1016/j.seppur.2010.01.010},
file = {:Users/marc/Library/Application Support/Mendeley Desktop/Downloaded/Chen et al. - 2010 - On the assembling of Pdceramic composite membranes for hydrogen separation.pdf:pdf},
issn = {13835866},
journal = {Separation and Purification Technology},
keywords = {Composite palladium membrane,Graphite,Hydrogen separation,Porous ceramics,Sealing},
number = {1},
pages = {92--97},
publisher = {Elsevier B.V.},
title = {{On the assembling of Pd/ceramic composite membranes for hydrogen separation}},
url = {http://dx.doi.org/10.1016/j.seppur.2010.01.010},
volume = {72},
year = {2010}
}
@article{Hara2000,
author = {Hara, S and Sakaki, K and Itoh, N and Kimura, H.-M. and Asami, K and Inoue, A},
file = {:Users/marc/Library/Application Support/Mendeley Desktop/Downloaded/Hara et al. - 2000 - An amorphous alloy membrane without noble metals for gaseous hydrogen separation.pdf:pdf},
journal = {Journal of Membrane Science},
pages = {289--294},
title = {{An amorphous alloy membrane without noble metals for gaseous hydrogen separation}},
volume = {164},
year = {2000}
}
@article{Nagai1995a,
author = {Nagai, K. and Higuchi, A. and Nakagawa, T.},
journal = {J. Polym. Sci.: Part B: Polym. Phys.},
pages = {289},
title = {{Gas permeability and stability of poly(1-trimethylsilyl-1-propyne-co-1-phenyl-1-propyne) membranes}},
volume = {33},
year = {1995}
}
@article{McCool1999,
abstract = {Submicron thick continuous palladium-silver films were deposited by sputtering on mesoporous $\gamma$-alumina substrates from a target composed of 75{\%} Pd and 25{\%} Ag. These membranes were tested in a multigas permeation system for hydrogen permeance and hydrogen selectivity over helium. Hydrogen permeance for the membrane used in the study was found to be in the range of 3x10-8 to 1x10-7mol/m2sPa and H2/He selectivities were in the range of 4-4000, depending mainly on Ag concentration and microstructure of the Pd-Ag films. A simple model based on the material balance concept has been developed to describe the transient composition behavior of the multicomponent sputter deposition process. The model is compared with experimental data over a range of deposition powers and target equilibration times. The results of the model agree well with experimental data. Both theoretical and experimental data show that deposition power and target equilibration are the most important variables affecting membrane composition. Copyright (C) 1999 Elsevier Science B.V.},
author = {McCool, B. and Xomeritakis, G. and Lin, Y. S.},
doi = {10.1016/S0376-7388(99)00087-3},
file = {:Users/marc/Library/Application Support/Mendeley Desktop/Downloaded/McCool, Xomeritakis, Lin - 1999 - Composition control and hydrogen permeation characteristics of sputter deposited palladium-silver memb.pdf:pdf},
isbn = {5135562761},
issn = {03767388},
journal = {Journal of Membrane Science},
number = {1-2},
pages = {67--76},
title = {{Composition control and hydrogen permeation characteristics of sputter deposited palladium-silver membranes}},
volume = {161},
year = {1999}
}
@article{Huang2010b,
author = {Huang, Aisheng and Dou, Wei},
file = {:Users/marc/Library/Application Support/Mendeley Desktop/Downloaded/Huang, Dou - 2010 - Steam-Stable Zeolitic Imidazolate Framework ZIF-90 Membrane with Hydrogen Selectivity through Covalent Functionaliza.pdf:pdf},
pages = {15562--15564},
title = {{Steam-Stable Zeolitic Imidazolate Framework ZIF-90 Membrane with Hydrogen Selectivity through Covalent Functionalization}},
year = {2010}
}
@article{Dong2012,
abstract = {A compact two-gas sensor based on quartz- enhanced photoacoustic spectroscopy (QEPAS) was de- veloped for trace methane and ammonia quantification in impure hydrogen. The sensor is equipped with a micro- resonator to confine the sound wave and enhance QEPAS signal. The normalized noise-equivalent absorption coef- ficients (1$\sigma$) of 2.45 × 10−8 cm−1 W/√Hz and 9.1 × 10−9 cm−1 W/√Hz for CH4 detection at 200 Torr and NH3 detection at 50 Torrwere demonstrated with theQEPAS sen- sor configuration, respectively. The influence of water vapor on the CH4 channel was also investigated.},
author = {Dong, L. and Wright, J. and Peters, B. and Ferguson, B. A. and Tittel, F. K. and McWhorter, S.},
doi = {10.1007/s00340-012-4908-x},
file = {:Users/marc/Library/Application Support/Mendeley Desktop/Downloaded/Dong et al. - 2012 - Compact QEPAS sensor for trace methane and ammonia detection in impure hydrogen.pdf:pdf},
issn = {09462171},
journal = {Applied Physics B: Lasers and Optics},
number = {2},
pages = {459--467},
title = {{Compact QEPAS sensor for trace methane and ammonia detection in impure hydrogen}},
volume = {107},
year = {2012}
}
@article{Chen2009,
author = {Chen, Chengmeng and Yang, Quan-Hong and Yang, Yonggang and Lv, Wei and Wen, Yuefang and Hou, Peng-Xiang and Wang, Maozhang and Cheng, Hui-Ming},
doi = {10.1002/adma.200803726},
isbn = {09359648
15214095},
journal = {Advanced Materials},
number = {29},
pages = {3007--3011},
title = {{Self-Assembled Free-Standing Graphite Oxide Membrane}},
volume = {21},
year = {2009}
}
@article{Ishitsuka2008,
author = {Ishitsuka, M and Hara, S and Mukaida, M and Haraya, K and Kita, K and Kato, K},
doi = {10.1016/j.desal.2007.09.097},
isbn = {00119164},
journal = {Desalination},
number = {1-3},
pages = {293--299},
title = {{Hydrogen separation from dry gas mixtures using a membrane module consisting of palladium-coated amorphous-alloy}},
volume = {234},
year = {2008}
}
@phdthesis{Lee2016,
address = {London},
author = {Lee, Melanie Wei Ting},
booktitle = {Department of Chemical Engineering},
publisher = {Imperial College London},
title = {{Micro-Channel Enhanced Alumina Membranes - Designing and Tailoring Their Porperties For Wdiened Applications}},
volume = {PhD in Che},
year = {2016}
}
@article{LI2000,
author = {LI, Anwu and Liang, Weiqiang and Hughes, Ronald},
file = {:Users/marc/Library/Application Support/Mendeley Desktop/Downloaded/LI, Liang, Hughes - 2000 - Fabrication of dense palladium composite membranes for hydrogen separation.pdf:pdf},
journal = {Catalysis today},
pages = {45--51},
title = {{Fabrication of dense palladium composite membranes for hydrogen separation}},
volume = {56},
year = {2000}
}
@article{Li2018b,
abstract = {{\textless}p{\textgreater}A highly oriented 2D nanosheet metal–organic framework membrane is fabricated by a direct growth strategy.{\textless}/p{\textgreater}},
author = {Li, Yujia and Liu, Haiou and Wang, Huanting and Qiu, Jieshan and Zhang, Xiongfu},
doi = {10.1039/C7SC04815G},
file = {:Users/marc/Library/Application Support/Mendeley Desktop/Downloaded/Li et al. - 2018 - GO-guided direct growth of highly oriented metal–organic framework nanosheet membranes for H sub2sub CO sub2sub sep.pdf:pdf},
issn = {2041-6520},
journal = {Chemical Science},
number = {17},
pages = {4132--4141},
publisher = {Royal Society of Chemistry},
title = {{GO-guided direct growth of highly oriented metal–organic framework nanosheet membranes for H {\textless}sub{\textgreater}2{\textless}/sub{\textgreater} /CO {\textless}sub{\textgreater}2{\textless}/sub{\textgreater} separation}},
url = {http://xlink.rsc.org/?DOI=C7SC04815G},
volume = {9},
year = {2018}
}
@article{Huang2014a,
abstract = {Through layer-by-layer (LBL) deposition of a graphene oxide (GO) suspension on a semicontinuous ZIF-8 layer, we have developed a novel bicontinuous ZIF-8@GO membrane. Since only the gaps between the ZIF-8 crystals are sealed by the GO layer due to capillary forces and covalent bonds, the gas molecules can only permeate through the ZIF-8 micropore system (0.34 nm). Therefore, the ZIF-8@GO membranes show high hydrogen selectivity. At 250 °C and 1 bar, the mixture separation factors of H2/CO2, H2/N2, H2/CH4, and H2/C3H8 are 14.9, 90.5, 139.1, and 3816.6, with H2 permeances of about 1.3 × 10(-7) mol{\textperiodcentered}m(-2){\textperiodcentered}s(-1){\textperiodcentered}Pa(-1), which is promising for hydrogen separation and purification by molecular sieving.},
author = {Huang, Aisheng and Liu, Qian and Wang, Nanyi and Zhu, Yaqiong and Caro, J{\"{u}}rgen},
doi = {10.1021/ja5083602},
file = {:Users/marc/Library/Application Support/Mendeley Desktop/Downloaded/Huang et al. - 2014 - Bicontinuous zeolitic imidazolate framework zif-8@go membrane with enhanced hydrogen selectivity.pdf:pdf},
isbn = {0002-7863},
issn = {15205126},
journal = {Journal of the American Chemical Society},
number = {42},
pages = {14686--14689},
pmid = {25290574},
title = {{Bicontinuous zeolitic imidazolate framework zif-8@go membrane with enhanced hydrogen selectivity}},
volume = {136},
year = {2014}
}
@misc{TheMckeowngroup,
author = {{The Mckeown group}},
title = {{PIMs Patents}}
}
@article{Qi2012,
abstract = {Silica-based microporous membranes for the separation of gases with relatively small kinetic diameters, like hydrogen, carbon dioxide, nitrogen and oxygen under harsh industrial processes, will offer great potential for integration in CO2capture technologies. Development of membranes with integrated performances of permeability, selectivity and stability in the presence of hot vapor, is one of the prerequisites for their successful implementation. Herein, we reported a novel microporous hybrid silica membrane, fabricated through sol-gel deposition of an ethylene-bridged silsesquioxane layer on a multilayer porous support, by adjusting the amount of niobium alkoxide precursor. When the Nb content was less than 50{\%} (in mole), both hybrid siliceous microporous networks and generated Lewis acid sites imparted very low CO2permeance to the membrane while retaining its comparatively high H2permeance. Dominant densification shall take effect when Nb content was higher than 50{\%}, which leads to both low H2permeance and H2/CO2permselectivity. Hybrid silica membranes with niobium loading amount of 17{\%} and 33{\%} respectively, showed excellent stabilities in the presence of 150kPa steam under 200°C, as evidenced by steady H2permeances and exceptionally high H2/CO2permselectivities ({\textgreater}700) during long-term stability test up to 300h, which demonstrating a promising CO2separation membrane. {\textcopyright} 2012 Elsevier B.V.},
author = {Qi, Hong and Chen, Huiru and Li, Li and Zhu, Guizhi and Xu, Nanping},
doi = {10.1016/j.memsci.2012.07.010},
file = {:Users/marc/Library/Application Support/Mendeley Desktop/Downloaded/Qi et al. - 2012 - Effect of Nb content on hydrothermal stability of a novel ethylene-bridged silsesquioxane molecular sieving membrane.pdf:pdf},
isbn = {0376-7388},
issn = {03767388},
journal = {Journal of Membrane Science},
keywords = {Carbon dioxide capture,Hydrothermal stability,Microporous hybrid silica membranes,Niobium,Sol-gel processes},
pages = {190--200},
title = {{Effect of Nb content on hydrothermal stability of a novel ethylene-bridged silsesquioxane molecular sieving membrane for H2/CO2separation}},
volume = {421-422},
year = {2012}
}
@article{Fu2016,
abstract = {The search for new types of membrane materials has been of continuous interest from both academia and industry given their importance in a plethora of applications particularly for energy-efficient separation technology. In this contribution, we demonstrate for the first time that metal-organic framework (MOF) can be grown on the covalent-organic framework (COF) membrane to fabricate COF-MOF composite membranes. The resultant COF-MOF composite membranes demonstrate higher separation selectivity of H2/CO2 gas mixtures than the individual COF and MOF membranes. A sound proof for the synergy between two porous materials is the fact that the COF-MOF composite membranes surpass the Robeson upper bound of polymer membranes for mixture separation of H2/CO2 gas pair and among the best gas-separation MOF membranes reported thus far.},
author = {Fu, Jingru and Das, Saikat and Xing, Guolong and Ben, Teng and Valtchev, Valentin and Qiu, Shilun},
doi = {10.1021/jacs.6b03348},
file = {:Users/marc/Library/Application Support/Mendeley Desktop/Downloaded/Fu et al. - 2016 - Fabrication of COF-MOF Composite Membranes and Their Highly Selective Separation of H2CO2.pdf:pdf},
isbn = {1520-5126 (Electronic)$\backslash$r0002-7863 (Linking)},
issn = {15205126},
journal = {Journal of the American Chemical Society},
number = {24},
pages = {7673--7680},
pmid = {27225027},
title = {{Fabrication of COF-MOF Composite Membranes and Their Highly Selective Separation of H2/CO2}},
volume = {138},
year = {2016}
}
@article{PachecoTanaka2005,
author = {{Pacheco Tanaka}, David A and {Llosa Tanco}, Margot A and Niwa, Shu-ichi and Wakui, Yoshito and Mizukami, Fujio and Namba, Takemi and Suzuki, Toshishige M},
doi = {10.1016/j.memsci.2004.06.002},
file = {:Users/marc/Library/Application Support/Mendeley Desktop/Downloaded/Pacheco Tanaka et al. - 2005 - Preparation of palladium and silver alloy membrane on a porous $\alpha$-alumina tube via simultaneous electrole.pdf:pdf},
isbn = {03767388},
journal = {Journal of Membrane Science},
number = {1-2},
pages = {21--27},
title = {{Preparation of palladium and silver alloy membrane on a porous $\alpha$-alumina tube via simultaneous electroless plating}},
volume = {247},
year = {2005}
}
@article{Li1998,
author = {Li, Anwu and Liang, Weiqiang and Hughes, Ronald},
journal = {Journal of Membrane Science},
pages = {259--268},
title = {{Characterisation and permeation of palladium/stainles steel composite membranes}},
volume = {149},
year = {1998}
}
@article{Nguyen2009,
author = {Nguyen, Thu Hoai and Mori, Shinsuke and Suzuki, Masaaki},
doi = {10.1016/j.cej.2009.06.024},
isbn = {13858947},
journal = {Chemical Engineering Journal},
number = {1-2},
pages = {55--61},
title = {{Hydrogen permeance and the effect of H2O and CO on the permeability of Pd0.75Ag0.25 membranes under gas-driven permeation and plasma-driven permeation}},
volume = {155},
year = {2009}
}
@article{Oh2009a,
abstract = {SrZr0.2Ce0.8-xEuxO3-?? (x = 0.1 and 0.15) membranes were investigated for their hydrogen permeation properties as a function of membrane thickness, from 17 to 50 ??m. Membrane permeation flux was proportional to [ PH2 ]1 / 4 matching Norby and Larring's model when protons and electrons are the dominating defects. The observed inverse linear dependence of flux with membrane thickness indicates bulk diffusion is the rate-limiting factor for hydrogen permeation. After correcting for water formation, higher Eu doping in the membrane was confirmed to result in greater hydrogen flux, consistent with our previous conductivity studies. A H2 flux of 0.35 cc/min cm2 was achieved for 17 ??m thickness SrZr0.2Ce0.7Eu0.1O3-?? membrane at 900 ??C and 100{\%} H2 in the feed gas. However, taking observed water vapor formation into consideration, a maximum H2 flux of 0.50 cc/min cm2 was achieved under the same conditions. ?? 2009 Elsevier B.V. All rights reserved.},
author = {Oh, Takkeun and Yoon, Heesung and Li, Jianlin and Wachsman, E. D.},
doi = {10.1016/j.memsci.2009.08.031},
file = {:Users/marc/Library/Application Support/Mendeley Desktop/Downloaded/Oh et al. - 2009 - Hydrogen permeation through thin supported SrZr0.2Ce0.8-xEuxO3- membranes.pdf:pdf},
issn = {03767388},
journal = {Journal of Membrane Science},
keywords = {Hydrogen,Permeation,Proton conducting membrane,SrZr1-xCe1-x-yEuyO3-??},
number = {1-2},
pages = {1--4},
title = {{Hydrogen permeation through thin supported SrZr0.2Ce0.8-xEuxO3-?? membranes}},
volume = {345},
year = {2009}
}
@article{Thomas2011,
abstract = {In situ IR detection of carbon monoxide in the presence of hydrocarbons (methanol and pentane) using Pd-containing zeolite thin films is reported. The thin films are prepared by spin coating deposition of nanosized LTL and BEA type zeolites suspensions the palladium clusters are introduced in the nanosized zeolites by ion exchange followed by $\gamma$ radiolysis of the coating suspensions. The Pd-containing zeolite films with a thickness of 200 nm are exposed to a single gas (either CO or hydrocarbons) or gas mixtures in the presence of water (100 ppm), and the IR spectra are collected continuously at 25, 75, and 100 °C. The fast recognition of very low concentrations of CO (2-100 ppm) in the presence of highly concentrated vapors of methanol or pentane (400-4000 ppm) with the Pd-containing zeolite films is demonstrated. The detection of CO and hydrocarbons is instant, which is a function of the low thickness of the films, small size of the individual zeolite crystals, and regular size and high stability of the Pd clusters in the zeolite films. The heat of adsorption for all experiments is similar (15 kJ.mol(-1)), which is explained with weak interactions between the carbon monoxide and palladium clusters in the zeolite films at temperatures below 100 °C. The nanosized zeolites with homogeneously distributed Pd clusters deposited in thin films demonstrate high molecular recognition capacity toward low concentrations of carbon monoxide under real environmental conditions, i.e., in the presence of water and hydrocarbons.},
author = {Thomas, S{\'{e}}bastien and Bazin, Philippe and Lakiss, Louwanda and {De Waele}, Vincent and Mintova, Svetlana},
doi = {10.1021/la203075m},
file = {:Users/marc/Library/Application Support/Mendeley Desktop/Downloaded/Thomas et al. - 2011 - In situ infrared molecular detection using palladium-containing zeolite films.pdf:pdf},
isbn = {1520-5827 (Electronic)$\backslash$r0743-7463 (Linking)},
issn = {07437463},
journal = {Langmuir},
number = {23},
pages = {14689--14695},
pmid = {21981338},
title = {{In situ infrared molecular detection using palladium-containing zeolite films}},
volume = {27},
year = {2011}
}
@article{Zhang2011,
abstract = {Highly aligned SnO2 nanorods on graphene 3-D array structures were synthesized by a straightforward nanocrystal-seeds-directing hydrothermal method. The diameter and density of the nanorods grown on the graphene can be easily tuned as required by varying the seeding concentration and temperature. The array structures were used as gas sensors and exhibit improved sensing performances to a series of gases in comparison to that of SnO2 nanorod flowers. For nanorod arrays of optimal diameter and distribution, these structures were proved to exert an enhanced sensitivity to reductive gases (especially H2S), which was twice as high as that obtained using SnO2 nanorod flowers. The improved sensing properties are attributed to the synergism of the large surface area of SnO2 nanorod arrays and the superior electronic characteristics of graphene.},
author = {Zhang, Zhenyu and Zou, Rujia and Song, Guosheng and Yu, Li and Chen, Zhigang and Hu, Junqing},
doi = {10.1039/c1jm12987b},
file = {:Users/marc/Library/Application Support/Mendeley Desktop/Downloaded/Zhang et al. - 2011 - Highly aligned SnO2 nanorods on graphene sheets for gas sensors.pdf:pdf},
isbn = {10.1039/C1JM12987B},
issn = {0959-9428},
journal = {Journal of Materials Chemistry},
number = {43},
pages = {17360},
title = {{Highly aligned SnO2 nanorods on graphene sheets for gas sensors}},
url = {http://xlink.rsc.org/?DOI=c1jm12987b},
volume = {21},
year = {2011}
}
@article{Gong2014,
abstract = {A luminescent microporous metal-organic framework based on a $\pi$-electron-rich tricarboxylate ligand and an In(3+) ion has been solvothermally obtained and characterized and exhibits highly selective CO2 adsorption over CH4 and N2 gases and selective sensing of the nitro explosive 2,4,6-trinitrophenol.},
author = {Gong, Yun Nan and Huang, Yong Liang and Jiang, Long and Lu, Tong Bu},
doi = {10.1021/ic501413r},
file = {:Users/marc/Library/Application Support/Mendeley Desktop/Downloaded/Gong et al. - 2014 - A luminescent microporous metal-organic framework with highly selective CO2 adsorption and sensing of nitro explosi.pdf:pdf},
isbn = {0020-1669},
issn = {1520510X},
journal = {Inorganic Chemistry},
number = {18},
pages = {9457--9459},
pmid = {25170531},
title = {{A luminescent microporous metal-organic framework with highly selective CO2 adsorption and sensing of nitro explosives}},
volume = {53},
year = {2014}
}
@article{Illing2005,
abstract = {A novel process has been developed to produce defect-free polyaniline-based composite membranes. These membranes consist of a thin polyaniline film (0.8-10.5 $\mu$m) and polyvinylidene difluoride (PVDF) as support material. The composite structure enhances flexibility and ease of handling of the material. This is demonstrated through tensile testing of the membranes. The polyaniline composite membranes were subjected to gas permeability studies. Despite being prepared with a thickness of less than 1 $\mu$m the dense supported polyaniline maintained its previously reported intrinsic selectivity for different gases. The novel fabrication process has the potential of producing nano scale membranes with much enhanced permeation rates. {\textcopyright} 2005 Elsevier B.V. All rights reserved.},
author = {Illing, G. and Hellgardt, K. and Schonert, M. and Wakeman, R. J. and Jungbauer, A.},
doi = {10.1016/j.memsci.2004.12.031},
isbn = {0376-7388},
issn = {03767388},
journal = {Journal of Membrane Science},
keywords = {Composite membranes,Gas separation,Membrane preparation,Polyaniline,Submicron},
number = {1-2},
pages = {199--208},
title = {{Towards ultrathin polyaniline films for gas separation}},
volume = {253},
year = {2005}
}
@article{Rei2009,
author = {Rei, M H},
doi = {10.1016/j.jtice.2008.12.011},
file = {:Users/marc/Library/Application Support/Mendeley Desktop/Downloaded/Rei - 2009 - A decade's study and developments of palladium membrane in Taiwan.pdf:pdf},
isbn = {18761070},
journal = {Journal of the Taiwan Institute of Chemical Engineers},
number = {3},
pages = {238--245},
title = {{A decade's study and developments of palladium membrane in Taiwan}},
volume = {40},
year = {2009}
}
@article{Pinnau1996,
abstract = {Teflon AF 2400 (Du Pont) is an amorphous, glassy perfluorinated copolymer containing 87 mol{\%} 2,2-bistrifluoromethyl-4,5-difluoro-1,3-dioxole and 13 mol{\%} tetrafluoroethylene. The polymer has an extremely high fractional free volume of 0.327. Permeability coefficients for helium, hydrogen, carbon dioxide, oxygen, nitrogen, methane, ethane, propane, and chlorodifluoromethane (Freon 22) were determined at temperatures from 25 to 60°C and pressures from 20 to 120 psig. Permeation properties were also determined at a feed pressure of 200 psig at 25°C with a 2 mol{\%} n-butane/98 mol{\%} methane mixture. Permeabilities of permanent gases in Teflon AF 2400 are among the highest of all known polymers; the oxygen permeability coefficient at 25°C is 1600 x 10-10cm3(STP) cm/cm2s cmHg and the nitrogen permeability coefficient is 780 x 10-10cm3(STP) cm/cm2s cmHg. The permeabilities of organic vapors increase up to 20-fold as the vapor activity increases from 0.1 to unity, indicating that Teflon AF 2400 is easily plasticized. Although Teflon AF 2400 is an ultrahigh-free-volume polymer like poly(1-trimethylsilyl-1-propyne) [PTMSP], their gas permeation properties differ significantly. Teflon AF 2400 shows gas transport behavior similar to that of conventional, low-free-volume glassy polymers. PTMSP, on the other hand, acts more like a nanoporous carbon than a conventional glassy polymer.},
author = {Pinnau, Ingo and Toy, Lora G.},
doi = {10.1016/0376-7388(95)00193-X},
isbn = {0376-7388},
issn = {03767388},
journal = {Journal of Membrane Science},
keywords = {2,2-bistrifluoromethyl-4,5-difluoro-1,3-dioxole,Activation energy of permeation,Copolymer,Gas transport,PTMSP,Teflon AF 2400,Tetrafluoroethylene,Vapor separation},
number = {1},
pages = {125--133},
title = {{Gas and vapor transport properties of amorphous perfluorinated copolymer membranes based on 2,2-bistrifluoromethyl-4,5-difluoro-1,3-dioxole/tetrafluoroethylene}},
volume = {109},
year = {1996}
}
@article{Kulprathipanja2005,
annote = {NULL},
author = {Kulprathipanja, A and Alptekin, G and Falconer, J and Way, J},
doi = {10.1016/j.memsci.2004.11.031},
file = {:Users/marc/Library/Application Support/Mendeley Desktop/Downloaded/Kulprathipanja et al. - 2005 - Pd and Pd–Cu membranes inhibition of H2 permeation by H2S.pdf:pdf},
isbn = {03767388},
journal = {Journal of Membrane Science},
number = {1-2},
pages = {49--62},
title = {{Pd and Pd–Cu membranes: inhibition of H2 permeation by H2S}},
volume = {254},
year = {2005}
}
@article{Sanz2012,
abstract = {A new synthesis method to prepare Pd membranes by novelty modified electroless plating over tubular porous stainless steel supports (PSS) has been developed. This new pore plating method basically consists on feeding both plating solution and reducing agent from opposite sides of support, allowing the preparation of totally hydrogen selective membranes with a significantly lower Pd consumption than the corresponding to the conventional electroless plating procedure. In the latter, both reducing agent and plating solution are added simultaneously in one side of the PSS support. This new plating method has been applied over raw commercial PSS supports and air calcined supports in order to generate a Fe-Cr oxide intermediate layer. A completely dense Pd membrane with a thickness in the range 11-20 ??m directly over tubular porous stainless steel tubes with a high roughness has been achieved. The permeation properties of the membranes have been tested at different operating conditions for pure feed gases: retentate pressure (1-4 bar) and temperature (350-450 ??C). All membranes present good permeance reproducibility after several thermal cycles and a complete hydrogen ideal selectivity, since complete retention of nitrogen is maintained for all tested experiment conditions, ensuring 100{\%} purity in the hydrogen permeate flux. The permeance of both membranes is maintained in the range of 1-3??10 -4 mol m -2 s -1 Pa -0.5. Copyright ?? 2012, Hydrogen Energy Publications, LLC. Published by Elsevier Ltd. All rights reserved.},
author = {Sanz, R. and Calles, J. A. and Alique, D. and Furones, L.},
doi = {10.1016/j.ijhydene.2012.09.084},
file = {:Users/marc/Library/Application Support/Mendeley Desktop/Downloaded/Sanz et al. - 2012 - New synthesis method of Pd membranes over tubular PSS supports via pore-plating for hydrogen separation processes.pdf:pdf},
isbn = {03603199},
issn = {03603199},
journal = {International Journal of Hydrogen Energy},
keywords = {Electroless plating,Hydrogen separation,PSS,Palladium,Pore-plating},
number = {23},
pages = {18476--18485},
publisher = {Elsevier Ltd},
title = {{New synthesis method of Pd membranes over tubular PSS supports via "pore-plating" for hydrogen separation processes}},
url = {http://dx.doi.org/10.1016/j.ijhydene.2012.09.084},
volume = {37},
year = {2012}
}
@article{Qin2012,
abstract = {The solubility and diffusivity of hydrogen in disordered fcc Pd 1-xCu x alloys are investigated using a combination of first-principles calculations, a composition-dependent local cluster expansion (CDLCE) technique, and kinetic Monte Carlo simulations. We demonstrate that a linear CDCLE model can accurately describe interstitial H in fcc Pd 1-xCu x alloys over the entire composition range (0 ??? x ??? 1) with accuracy comparable to that of direct first-principles calculations. Our predicted H solubility and permeability results are in reasonable agreement with experimental measurements. The proposed model is quite general and can be employed to rapidly and accurately screen a large number of alloy compositions for potential membrane applications. Extension to ternary or higher-order alloy systems should be straightforward. Our study also highlights the significant effect of local lattice relaxations on H energetics in size-mismatched disordered alloys, which has been largely overlooked in the literature. ?? 2012 Hydrogen Energy Publications, LLC. Published by Elsevier Ltd. All rights reserved.},
author = {Qin, Lin and Jiang, Chao},
doi = {10.1016/j.ijhydene.2012.06.029},
file = {:Users/marc/Library/Application Support/Mendeley Desktop/Downloaded/Qin, Jiang - 2012 - First-principles based modeling of hydrogen permeation through Pd-Cu alloys.pdf:pdf},
isbn = {1806890461},
issn = {03603199},
journal = {International Journal of Hydrogen Energy},
keywords = {Cluster expansion,First-principles calculation,Hydrogen permeability,Kinetic Monte Carlo,Membrane},
number = {17},
pages = {12760--12764},
title = {{First-principles based modeling of hydrogen permeation through Pd-Cu alloys}},
volume = {37},
year = {2012}
}
@article{Sato2012,
author = {Sato, Koichi and Nishioka, Masateru and Higashi, Hideo and Inoue, Tomoya and Hasegawa, Yasuhisa and Wakui, Yoshito and Suzuki, Toshishige M and Hamakawa, Satoshi},
doi = {10.1016/j.memsci.2012.04.053},
isbn = {03767388},
journal = {Journal of Membrane Science},
pages = {85--92},
title = {{Influence of CO2 and H2O on the separation of hydrogen over two types of Pd membranes: Thin metal membrane and pore-filling-type membrane}},
volume = {415-416},
year = {2012}
}
@article{Tanabe2011,
abstract = {Metal-organic frameworks (MOFs) are an important class of hybrid inorganic-organic materials. In this tutorial review, a progress report on the postsynthetic modification (PSM) of MOFs is provided. PSM refers to the chemical modification of the MOF lattice in a heterogeneous fashion. This powerful synthetic approach has grown in popularity and resulted in a number of advances in the functionalization and application of MOFs. The use of PSM to develop MOFs with improved gas sorption, catalytic activity, bioactivity, and more robust physical properties is discussed. The results reported to date clearly show that PSM is an important approach for the development and advancement of these hybrid solids.},
author = {Tanabe, Kristine K. and Cohen, Seth M.},
doi = {10.1039/C0CS00031K},
file = {:Users/marc/Library/Application Support/Mendeley Desktop/Downloaded/Tanabe, Cohen - 2011 - Postsynthetic modification of metal–organic frameworks—a progress report.pdf:pdf},
isbn = {0306-0012},
issn = {0306-0012},
journal = {Chem. Soc. Rev.},
number = {2},
pages = {498--519},
pmid = {21103601},
title = {{Postsynthetic modification of metal–organic frameworks—a progress report}},
url = {http://xlink.rsc.org/?DOI=C0CS00031K},
volume = {40},
year = {2011}
}
@article{Bai2016,
abstract = {SnO2-CuO heterostructures have been synthesized via electrospinning method, which overcomes the defect of SnO2 generally prepared at high temperature. The structure, morphology, size, specific surface, thermal stability and surface composition of nanocomposite were characterized by XRD, SEM, TEM, BET, TG and XPS. The results testify that the change in property and structure of composite depends on the CuO content in composite. The SnO2 composite with 30 wt{\%} CuO not only exhibits excellent selectivity and high response that is 16 and 2.5 times higher than that of pure CuO and SnO2, respectively, but also reduces the operating temperature from 295 ??C to 235 ??C. The mechanism enhanced sensing properties was discussed in detail, except for enhancing adsorption of gas on the material surface; the enhancement can be attributed to the formation of p-n heterojunction at the interface between the SnO2 and the CuO.},
author = {Bai, Shouli and Guo, Wentao and Sun, Jianhua and Li, Jiao and Tian, Ye and Chen, Aifan and Luo, Ruixian and Li, Dianqing},
doi = {10.1016/j.snb.2015.11.028},
file = {:Users/marc/Library/Application Support/Mendeley Desktop/Downloaded/Bai et al. - 2016 - Synthesis of SnO2-CuO heterojunction using electrospinning and application in detecting of CO.pdf:pdf},
issn = {09254005},
journal = {Sensors and Actuators, B: Chemical},
keywords = {CO sensor,CuO,Electrospinning,SnO2,p-n heterojunction},
pages = {96--103},
publisher = {Elsevier B.V.},
title = {{Synthesis of SnO2-CuO heterojunction using electrospinning and application in detecting of CO}},
url = {http://dx.doi.org/10.1016/j.snb.2015.11.028},
volume = {226},
year = {2016}
}
@article{K.L.YeungA.Varma1999,
author = {{K. L. Yeung A. Varma}, S C Christiansen},
file = {:Users/marc/Library/Application Support/Mendeley Desktop/Downloaded/K. L. Yeung A. Varma - 1999 - Palladium composite membranes by electroless plating technique Relationships between plating kinetics, fil.pdf:pdf},
journal = {Journal of Membrane Science},
pages = {107--122},
title = {{Palladium composite membranes by electroless plating technique: Relationships between plating kinetics, film microstructure and membrane performance}},
volume = {159},
year = {1999}
}
@article{Meng2015a,
abstract = {In this work, Cu2O nanorods modified by reduced graphene oxide (rGO) were produced via a two-step synthesis method. CuO rods were firstly prepared in graphene oxide (GO) solution using cetyltrimethyl ammonium bromide (CTAB) as a soft template by the microwave-assisted hydrothermal method, accompanied with the reduction of GO. The complexes were subsequently annealed and Cu2O nanorods/rGO composites were obtained. The as-prepared composites were evaluated using various characterization methods, and were utilized as sensing materials. The room-temperature NH3 sensing properties of a sensor based on the Cu2O nanorods/rGO composites were systematically investigated. The sensor exhibited an excellent sensitivity and linear response toward NH3 at room temperature. Furthermore, the sensor could be easily recovered to its initial state in a short time after exposure to fresh air. The sensor also showed excellent repeatability and selectivity to NH3. The remarkably enhanced NH3-sensing performances could be attributed to the improved conductivity, catalytic activity for the oxygen reduction reaction and increased gas adsorption in the unique hybrid composites. Such composites showed great potential for manufacturing a new generation of low-power and portable ammonia sensors.},
author = {Meng, Hu and Yang, Wei and Ding, Kun and Feng, Liang and Guan, Yafeng},
doi = {10.1039/C4TA06024E},
file = {:Users/marc/Library/Application Support/Mendeley Desktop/Downloaded/Meng et al. - 2015 - Cu sub2sub O nanorods modified by reduced graphene oxide for NH sub3sub sensing at room temperature.pdf:pdf},
isbn = {2050-7488},
issn = {2050-7488},
journal = {J. Mater. Chem. A},
number = {3},
pages = {1174--1181},
title = {{Cu {\textless}sub{\textgreater}2{\textless}/sub{\textgreater} O nanorods modified by reduced graphene oxide for NH {\textless}sub{\textgreater}3{\textless}/sub{\textgreater} sensing at room temperature}},
url = {http://xlink.rsc.org/?DOI=C4TA06024E},
volume = {3},
year = {2015}
}
@article{DEUTSCHMANN,
author = {DEUTSCHMANN, OLAF and KNOZINGER, HELMUT and KOCHLOEFL, KARL and TUREK, THOMAS},
doi = {10.1002/14356007.o05},
isbn = {9783527306732},
journal = {ULLMANNS ENCYCLOPEDIA OF INDUSTRIAL CHEMISTRY},
title = {{Heterogeneous Catalysis and Solid Catalysts, 3. Industrial Applications OLAF}}
}
@article{GouveiaGil2015,
abstract = {In this study, a highly permeable ceramic hollow fibre substrate has been fabricated for developing Pd composite membranes. The substrate consists of one thin outer sponge-like layer for depositing Pd membranes by electroless plating and a plurality of self-organized micro-channels for reducing gas permeation resistance. A dense defect-free Pd membrane with approximately 1 $\mu$m was formed on the outer surface of the sponge-like layer, which suggests a great uniformity of the substrate. As a result, a hydrogen permeation flux of 0.87 mol s−1 m−2 can be achieved at 450 °C and 165 KPa. Hydrogen permeation of the several composite membranes with different Pd thicknesses (1.0 and 3.3 $\mu$m) and different substrates sintered at different temperatures (1300 and 1400 °C) was also investigated. It was found that an intermediate layer, which is normally formed due to the Pd penetration during the electroless plating, shows an adverse effect in hydrogen permeation, especially when the Pd membrane is very thin.},
author = {{Gouveia Gil}, Ana and Reis, Miria Hespanhol M and Chadwick, David and Wu, Zhentao and Li, K},
doi = {https://doi.org/10.1016/j.ijhydene.2015.01.021},
issn = {0360-3199},
journal = {International Journal of Hydrogen Energy},
keywords = {Alumina hollow fibre,Asymmetric structure,Hydrogen separation,Palladium membrane},
number = {8},
pages = {3249--3258},
title = {{A highly permeable hollow fibre substrate for Pd/Al2O3 composite membranes in hydrogen permeation}},
url = {http://www.sciencedirect.com/science/article/pii/S0360319915000580},
volume = {40},
year = {2015}
}
@article{Cheng2005,
abstract = {The hydrogen permeation behavior in a Fe3Al-based alloy, with three types of heat treatments, was investigated by an ultrahigh vacuum gaseous permeation technique at the temperature range of 240-320 ??C. A well-defined second stage is observed in hydrogen permeation curves of the Fe3Al alloy superimposed on the basic curves. On the basis of optical microstructures, determined permeabilities and diffusivities, corresponding activation energies, it is suggested that the two sigmoidal curves result from the ordered B 2 phase and disordered ??-Fe phase in the Fe3Al alloy; the first stage of permeation reflects the hydrogen transport in matrix disordered ??-Fe phase, while the second stage reflects the hydrogen transport in ordered B2 phase. ?? 2004 Elsevier B.V. All rights reserved.},
author = {Cheng, X. Y. and Wu, Q. Y. and Sun, Y. K.},
doi = {10.1016/j.jallcom.2004.05.080},
file = {:Users/marc/Library/Application Support/Mendeley Desktop/Downloaded/Cheng, Wu, Sun - 2005 - Hydrogen permeation behavior in a Fe3Al-based alloy at high temperature.pdf:pdf},
issn = {09258388},
journal = {Journal of Alloys and Compounds},
keywords = {Activation energy,Gaseous permeation technique,Hydrogen diffusivity,Iron aluminides},
number = {1-2},
pages = {198--203},
title = {{Hydrogen permeation behavior in a Fe3Al-based alloy at high temperature}},
volume = {389},
year = {2005}
}
@inproceedings{Darling1963,
author = {Darling, A. S.},
booktitle = {Symposium on the less common means of separation. Institution of Chemical Engineers},
title = {{Hydrogen Separation by diffusion through palladium alloy membranes}},
year = {1963}
}
@article{Robinson2013,
abstract = {A series of experiments were performed on tubular BaCe0.2Zr0.7Y0.1O3 -?? (BCZY27) protonic ceramic hydrogen diffusion membranes in which hydrogen flux was obtained using a tube-in-shell test apparatus with a 22-cm2 membrane active area and hermetic glass seals. This was accomplished by measuring permeate composition using a mass spectrometer and bubble-meter measurement of outlet flow rates over a broad range of current densities and furnace temperatures. A maximum current efficiency of 98.6{\%} was achieved at 775??C at 800mAcm-2, producing over 6nmLcm-2min-1 of hydrogen at virtually 100{\%} selectivity. The data generated through these experiments has been valuable in the evaluation of fabrication methods and hermetic sealing techniques required for large area membranes that are suitable for commercialization. Applications include hydrogen separation, ammonia synthesis and gas-to-liquids fuel processing via methane dehydrogenation. Additionally, the experiments provided valuable insight into the charge-transport processes in BCZY27 membranes where multiple charge carrier species are involved. ?? 2013 Elsevier B.V.},
author = {Robinson, Shay and Manerbino, Anthony and {Grover Coors}, W.},
doi = {10.1016/j.memsci.2013.06.026},
file = {:Users/marc/Library/Application Support/Mendeley Desktop/Downloaded/Robinson, Manerbino, Grover Coors - 2013 - Galvanic hydrogen pumping in the protonic ceramic perovskite BaCe0.2Zr0.7Y0.1O3 -.pdf:pdf},
issn = {03767388},
journal = {Journal of Membrane Science},
keywords = {Hydrogen membranes,Hydrogen pumping,Membrane reactor,Protonic ceramics},
pages = {99--105},
publisher = {Elsevier},
title = {{Galvanic hydrogen pumping in the protonic ceramic perovskite BaCe0.2Zr0.7Y0.1O3 -??}},
url = {http://dx.doi.org/10.1016/j.memsci.2013.06.026},
volume = {446},
year = {2013}
}
@article{Nagaraju2013,
abstract = {We demonstrate the synthesis of CuBTC and ZIF-8 on a polysulfone based porous asymmetric ultrafiltration (UF) membrane by in situ growth followed by the LBL deposition of crystals without any need for pre-seeding or surface modification of the membrane. In this way, the top surface of the UF membrane pores is completely covered by MOFs; while the remaining part of the membrane offers a flexible support to the MOFs. The pore apertures of the MOF nanoparticles located at the pore opening of the UF membrane act as channels for the entry of penetrants. The remaining porous sublayer of the membrane carries penetrants on the permeate side without significant resistance. These composite membranes were characterized by PXRD and SEM. The gas permeation study was performed using pure gases of industrial significance (H2, C3H6 and CO2). The performance of CuBTC@PSF showed enhanced selectivity, of 7.2 and 5.7 for H2/CO2 and H2/C3H6 respectively, to that of the pristine PSF membrane.},
author = {Nagaraju, Divya and Bhagat, Deepti G. and Banerjee, Rahul and Kharul, Ulhas K.},
doi = {10.1039/c3ta10438a},
file = {:Users/marc/Library/Application Support/Mendeley Desktop/Downloaded/Nagaraju et al. - 2013 - In situ growth of metal-organic frameworks on a porous ultrafiltration membrane for gas separation.pdf:pdf},
isbn = {2050-7488},
issn = {2050-7488},
journal = {Journal of Materials Chemistry A},
number = {31},
pages = {8828},
title = {{In situ growth of metal-organic frameworks on a porous ultrafiltration membrane for gas separation}},
url = {http://xlink.rsc.org/?DOI=c3ta10438a},
volume = {1},
year = {2013}
}
@article{Haaf1964,
author = {Haaf, H. Koch and W.},
doi = {10.15227/orgsyn.044.0001},
issn = {00786209},
journal = {Organic Syntheses},
number = {September},
pages = {1},
title = {{1-Adamantanecarboxylic Acid}},
url = {http://orgsyn.org/demo.aspx?prep=CV5P0020},
volume = {44},
year = {1964}
}
@article{Hu2010,
abstract = {Increasing hydrogen energy utilization has greatly stimulated the development of the hydrogen-permeable palladium membrane, which is comprised of a thin layer of palladium or palladium alloy on a porous substrate. This work chose the low-cost macroporous Al2O3 as the substrate material, and the surface modification was carried out with a conventional 2B pencil, the lead of which is composed of graphite and clay. Based on the modified substrate, a highly permeable and selective Pd/pencil/Al 2O3 composite membrane was successfully fabricated via electroless plating. The membrane was characterized by SEM (scanning electron microscopy), field-emission SEM and metallographic microscopy. The hydrogen flux and H2/N2 selectivity of the membrane (with a palladium thickness of 5 ??m) under 1 bar at 723 K were 25 m3/(m2 h) and 3700, respectively; the membrane was found to be stable during a time-on-stream of 330 h at 723 K. ?? 2010 Professor T. Nejat Veziroglu. Published by Elsevier Ltd. All rights reserved.},
author = {Hu, Xiaojuan and Chen, Weidong and Huang, Yan},
doi = {10.1016/j.ijhydene.2010.05.102},
file = {:Users/marc/Library/Application Support/Mendeley Desktop/Downloaded/Hu, Chen, Huang - 2010 - Fabrication of Pdceramic membranes for hydrogen separation based on low-cost macroporous ceramics with pencil c.pdf:pdf},
issn = {03603199},
journal = {International Journal of Hydrogen Energy},
keywords = {Electroless plating,Hydrogen separation,Palladium membrane,Pencil coating,Porous ceramics,Substrate modification},
number = {15},
pages = {7803--7808},
publisher = {Elsevier Ltd},
title = {{Fabrication of Pd/ceramic membranes for hydrogen separation based on low-cost macroporous ceramics with pencil coating}},
url = {http://dx.doi.org/10.1016/j.ijhydene.2010.05.102},
volume = {35},
year = {2010}
}
@incollection{Nehir2009,
author = {Nehir, M H and Wang, C},
booktitle = {Modeling and Control of Fuel Cells: Distributed Generation Applications},
publisher = {John Wiley {\&} Sons},
title = {{Dynamic Modeling and Simulation of PEM Fuel Cells}},
year = {2009}
}
@article{Meng2015,
abstract = {Gas sensors can detect combustible, explosive and toxic gases, and have been widely used in safety monitoring and process control in residential buildings, industries and mines. Recently, graphene-based hybrids were widely investigated as chemiresistive gas sensors with high sensitivity and selectivity. This systematic review is therefore timely and necessary to evaluate the success of graphene-based hybrids in gas detection and to identify their challenges. We review the sensing principles and the synthesis process of the graphene-based hybrids with noble metals, metal oxides and conducting polymers to achieve better understanding and design of novel gas sensors. Our review will assist researchers to understand the evolution and the challenges of graphene-based hybrids, and create interest in development of gas-sensing techniques.},
author = {Meng, Fan Li and Guo, Zheng and Huang, Xing Jiu},
doi = {10.1016/j.trac.2015.02.008},
file = {:Users/marc/Library/Application Support/Mendeley Desktop/Downloaded/Meng, Guo, Huang - 2015 - Graphene-based hybrids for chemiresistive gas sensors.pdf:pdf},
isbn = {0165-9936},
issn = {18793142},
journal = {TrAC - Trends in Analytical Chemistry},
keywords = {Chemiresistive gas sensor,Gas detection,Gas sensor,Graphene,Graphene-based hybrid,Graphene-metal hybrid,Graphene-metal-oxide hybrid,Graphene-polymer hybrid,Nanocomposite,Sensing},
pages = {37--47},
publisher = {Elsevier B.V.},
title = {{Graphene-based hybrids for chemiresistive gas sensors}},
url = {http://dx.doi.org/10.1016/j.trac.2015.02.008},
volume = {68},
year = {2015}
}
@misc{Thebault2007,
abstract = {We investigated the oxygen isotope composition ($\delta$18O) of shell striae from juvenile Comptopallium radula (Mollusca; Pectinidae) specimens collected live in New Caledonia. Bottom-water temperature and salinity were monitored in-situ throughout the study period. External shell striae form with a 2-day periodicity in this scallop, making it possible to estimate the date of precipitation for each calcite sample collected along a growth transect. The oxygen isotope composition of shell calcite ($\delta$18Oshell calcite) measured at almost weekly resolution on calcite accreted between August 2002 and July 2003 accurately tracks bottom-water temperatures. A new empirical paleotemperature equation for this scallop species relates temperature and $\delta$18Oshell calcite:t (° C) = 20.00 (± 0.61) - 3.66 (± 0.39) × ($\delta$18 Oshell calcite VPDB - $\delta$18 Owater VSMOW)The mean absolute accuracy of temperature estimated using this equation is 1.0 °C at temperatures between 20 and 30 °C. Uncertainties regarding the precise timing of CaCO3 deposition and the actual variations in $\delta$18Owater at our study sites probably contribute to this error. Comparison with a previously published empirical paleotemperature equation indicates that C. radula calcite is enriched in 18O by ∼0.7‰ relative to equilibrium. Given the direction of this offset and the lack of correlation between shell growth rate and $\delta$18Oshell calcite, this disequilibrium is unlikely to be related to kinetic isotope effects. We suggest that this enrichment reflects (1) a relatively low pH in the scallop's marginal extrapallial fluid (EPF), (2) an isotopic signature of the EPF different from that of seawater, or (3) Rayleigh fractionation during the biocalcification process. Relative changes in $\delta$18Oshell calcite reflect seawater temperature variability at this location and we suggest that the shell of C. radula may be useful as an archive of past seawater temperatures. {\textcopyright} 2006 Elsevier Inc. All rights reserved.},
author = {Th{\'{e}}bault, Julien and Chauvaud, Laurent and Clavier, Jacques and Guarini, Jennifer and Dunbar, Robert B and Fichez, Renaud and Mucciarone, David A and Morize, Eric},
booktitle = {Geochimica et Cosmochimica Acta},
doi = {10.1016/j.gca.2006.10.017},
number = {4},
title = {{Reconstruction of seasonal temperature variability in the tropical Pacific Ocean from the shell of the scallop, Comptopallium radula}},
volume = {71},
year = {2007}
}
@article{JamesB.MillerBretH.HowardCaseyP.OBrienBryanD.Morreale2009,
author = {{James B. Miller Bret H. Howard, Casey P. O'Brien, Bryan D. Morreale}, Dominic R Alfonso},
chapter = {18800},
file = {:Users/marc/Library/Application Support/Mendeley Desktop/Downloaded/James B. Miller Bret H. Howard, Casey P. O'Brien, Bryan D. Morreale - 2009 - Hydrogen Dissociation on PdS4 Surfaces.pdf:pdf},
journal = {The Journal of Physical Chemistry C},
pages = {18800--18806},
title = {{Hydrogen Dissociation on PdS4 Surfaces}},
volume = {113},
year = {2009}
}
@article{Liang1994,
abstract = {Two different nonstoichiometric mixed perovskites: Sr2(Sc1+)xNb1-xO6-?? (with x=0.05 and 0.1) and Ba3(Ca1.18Nb1.82)O9-??, are shown to become good high-temperature protonic conductors upon exposure to H2O (or D2O) atmospheres. Conductivity is studied in the temperature range of 300 ???550 K where the proton concentration is frozen in. The activation energies EH for proton migration in these compounds are 0.62 and 0.54 eV, respectively. The conductivities fall in the same range as those for M3+-doped SrCeO3 and BaCeO3, but the present materials do not show electronic conduction after treatment in highly reducing atmospheres as do the cerates. It is therefore concluded that these compounds deserve serious consideration as fast high-temperature protonic conductors. ?? 1994.},
author = {Liang, K. C. and Du, Yang and Nowick, A. S.},
doi = {10.1016/0167-2738(94)90399-9},
file = {:Users/marc/Library/Application Support/Mendeley Desktop/Downloaded/Liang, Du, Nowick - 1994 - Fast high-temperature proton transport in nonstoichiometric mixed perovskites.pdf:pdf},
isbn = {0167-2738},
issn = {01672738},
journal = {Solid State Ionics},
number = {2},
pages = {117--120},
title = {{Fast high-temperature proton transport in nonstoichiometric mixed perovskites}},
volume = {69},
year = {1994}
}
@article{Chong2016,
author = {Chong, J Y and Wang, B and Li, K},
doi = {10.1016/j.coche.2016.04.002},
isbn = {22113398},
journal = {Current Opinion in Chemical Engineering},
pages = {98--105},
title = {{Graphene oxide membranes in fluid separations}},
volume = {12},
year = {2016}
}
@article{Ozaki2003,
author = {Ozaki, Tetsuya and Zhang, Yi and Komaki, Masao and Nishimura, Chikashi},
file = {:Users/marc/Library/Application Support/Mendeley Desktop/Downloaded/Ozaki et al. - 2003 - Preparation of Palladium-coated V and V-15Ni membranes for hydrogen purification by electroless plating technique.pdf:pdf},
journal = {International Journal of Hydrogen Energy},
pages = {297--302},
title = {{Preparation of Palladium-coated V and V-15Ni membranes for hydrogen purification by electroless plating technique}},
volume = {28},
year = {2003}
}
@article{Kim2014,
abstract = {Ultrathin graphene oxide (GO) ({\textless}5 nm) membranes were prepared by spin-casting onto microporous polymeric support membranes. GO membranes exhibited a highly CO2 permeable character, which is suitable for CO2 separation. In the presence of water vapour, high CO2 selectivity (e.g., CO2/H2, CO2/N2, and CO2/CH4) was achieved by enhanced CO2 sorption.},
author = {Kim, Hyo Won and Yoon, Hee Wook and Yoo, Byung Min and Park, Jae Sung and Gleason, Kristofer L. and Freeman, Benny D. and Park, Ho Bum},
doi = {10.1039/C4CC06207H},
file = {:Users/marc/Library/Application Support/Mendeley Desktop/Downloaded/Kim et al. - 2014 - High-performance CO sub2sub -philic graphene oxide membranes under wet-conditions.pdf:pdf},
isbn = {1364-548X (Electronic)$\backslash$r1359-7345 (Linking)},
issn = {1359-7345},
journal = {Chem. Commun.},
number = {88},
pages = {13563--13566},
pmid = {25243726},
publisher = {Royal Society of Chemistry},
title = {{High-performance CO {\textless}sub{\textgreater}2{\textless}/sub{\textgreater} -philic graphene oxide membranes under wet-conditions}},
url = {http://xlink.rsc.org/?DOI=C4CC06207H},
volume = {50},
year = {2014}
}
@article{TECNALIA2016,
author = {TECNALIA},
file = {:Users/marc/Library/Application Support/Mendeley Desktop/Downloaded/TECNALIA - 2016 - WP3 – Membrane Development and Scale-up D3.5 Report on development of new membranes for WP4.pdf:pdf},
pages = {3--6},
title = {{WP3 – Membrane Development and Scale-up| D3.5 Report on development of new membranes for WP4}},
year = {2016}
}
@article{Mejdell2008,
author = {Mejdell, A L and Klette, H and Ramachandran, A and Borg, A and Bredesen, R},
doi = {10.1016/j.memsci.2007.09.024},
file = {:Users/marc/Library/Application Support/Mendeley Desktop/Downloaded/Mejdell et al. - 2008 - Hydrogen permeation of thin, free-standing PdAg23{\%} membranes before and after heat treatment in air.pdf:pdf},
isbn = {03767388},
journal = {Journal of Membrane Science},
number = {1},
pages = {96--104},
title = {{Hydrogen permeation of thin, free-standing Pd/Ag23{\%} membranes before and after heat treatment in air}},
volume = {307},
year = {2008}
}
@article{Tarditi2011a,
abstract = {Dense PdAgCu ternary alloy composite membranes were synthesized by the sequential electroless plating of Pd, Ag and Cu on top of both disk and tubular porous stainless steel substrates. X-ray diffraction and scanning electron microscopy were employed to study the structure and morphology of the tested samples. The hydrogen permeation performance of these membranes was investigated over a 350-450 ??C temperature range and a trans-membrane pressure up to 100 kPa. After annealing at 500 ??C in hydrogen stream followed by permeation experiments, the alloy layer presented a FCC crystalline phase with a bulk concentration of 68{\%} Pd, 7{\%} Ag and 25{\%} Cu as revealed by EDS. The PdAgCu tubular membrane was found to be stable during more than 300 h on hydrogen stream. The permeabilities of the PdAgCu ternary alloy samples were higher than the permeabilities of the PdCu alloy membranes with a FCC phase. The co-segregation of silver and copper to the membrane surface was observed after hydrogen permeation experiments at high temperature as determined by XPS. ?? 2011 Elsevier B.V. All rights reserved.},
annote = {NULL},
author = {Tarditi, Ana M. and Braun, Fernando and Cornaglia, Laura M.},
doi = {10.1016/j.apsusc.2011.02.089},
file = {:Users/marc/Library/Application Support/Mendeley Desktop/Downloaded/Tarditi, Braun, Cornaglia - 2011 - Novel PdAgCu ternary alloy Hydrogen permeation and surface properties.pdf:pdf},
issn = {01694332},
journal = {Applied Surface Science},
keywords = {Hydrogen separation membranes,Palladium ternary alloys,PdAgCu},
number = {15},
pages = {6626--6635},
publisher = {Elsevier B.V.},
title = {{Novel PdAgCu ternary alloy: Hydrogen permeation and surface properties}},
url = {http://dx.doi.org/10.1016/j.apsusc.2011.02.089},
volume = {257},
year = {2011}
}
@article{Seshimo2009,
author = {Seshimo, Masahiro and Hirai, Takayuki and Rahman, Md Mizanur and Ozawa, Minoru and Sone, Masato and Sakurai, Makoto and Higo, Yakichi and Kameyama, Hideo},
doi = {10.1016/j.memsci.2009.07.007},
file = {:Users/marc/Library/Application Support/Mendeley Desktop/Downloaded/Seshimo et al. - 2009 - Functionally graded Pd$\gamma$-alumina composite membrane fabricated by electroless plating with emulsion of supercrit.pdf:pdf},
isbn = {03767388},
journal = {Journal of Membrane Science},
number = {1-2},
pages = {321--326},
title = {{Functionally graded Pd/$\gamma$-alumina composite membrane fabricated by electroless plating with emulsion of supercritical CO2}},
volume = {342},
year = {2009}
}
@article{Fine2010,
abstract = {: Metal oxide semiconductor gas sensors are utilised in a variety of different roles and industries. They are relatively inexpensive compared to other sensing technologies, robust, lightweight, long lasting and benefit from high material sensitivity and quick response times. They have been used extensively to measure and monitor trace amounts of environmentally important gases such as carbon monoxide and nitrogen dioxide. In this review the nature of the gas response and how it is fundamentally linked to surface structure is explored. Synthetic routes to metal oxide semiconductor gas sensors are also discussed and related to their affect on surface structure. An overview of important contributions and recent advances are discussed for the use of metal oxide semiconductor sensors for the detection of a variety of gases—CO, NOx, NH3 and the particularly challenging case of CO2. Finally a description of recent advances in work completed at University College London is presented including the use of selective zeolites layers, new perovskite type materials and an innovative chemical vapour deposition approach to 
film deposition.},
author = {Fine, George F and Cavanagh, Leon M and Afonja, Ayo and Binions, Russell},
doi = {10.3390/s100605469},
file = {:Users/marc/Library/Application Support/Mendeley Desktop/Downloaded/Fine et al. - 2010 - Metal Oxide Semi-Conductor Gas Sensors in Environmental Monitoring.pdf:pdf},
keywords = {environmental monitoring,metal oxides,semiconductor,zeolites},
number = {x},
pages = {5469--5502},
title = {{Metal Oxide Semi-Conductor Gas Sensors in Environmental Monitoring}},
year = {2010}
}
@article{Pham2011,
abstract = {Applications of zeolite films benefit from alignment of the integrated channels, but methods for film growth have nearly always introduced orientational randomization in the direction normal to the substrate. We now report facile methods to grow silicalite-1 films and pure silica beta zeolite films on substrates with straight or sinusoidal channels positioned uniformly upright at a thickness of up to 8 micrometers. Precise gel compositions and processing temperatures are critical to promote secondary growth on pre-formed oriented crystal monolayers while suppressing self-crystallization in the bulk medium. Preliminary results highlight the potential of these uniformly oriented films in the nonlinear optical response and separation of xylene isomers.},
author = {Pham, T. C. T. and Kim, H. S. and Yoon, K. B. and Nozik, a J and Vest, R E and Korde, R and Schmidtke, H and Desor, R and Sykora, M and Joo, J and Pietryga, J M and Klimov, V I and Allan, G and Pijpers, J J H and Bonn, M and Chang, L Y and Geyer, S M and Bawendi, M G and Schaller, R D and Hillhouse, H W and Huges, B K and Beard, M C and Novet, T and Parkinson, B a and Hinds, S and Brzozowski, L and Sargent, E H and Lee, J S and Kovalenko, M V and Shevchenko, E V and Talapin, D V and Murray, C B and Leschkies, K S and Beatty, T J and Kang, M S and Norris, D J and Aydil, E S},
doi = {10.1126/science.1212472},
file = {:Users/marc/Library/Application Support/Mendeley Desktop/Downloaded/Pham et al. - 2011 - Growth of Uniformly Oriented Silica MFI and BEA Zeolite Films on Substrates.pdf:pdf},
issn = {0036-8075},
journal = {Science},
number = {December},
pages = {1533--1538},
title = {{Growth of Uniformly Oriented Silica MFI and BEA Zeolite Films on Substrates}},
url = {http://www.sciencemag.org.ezproxy2.library.drexel.edu/content/334/6062/1533.full?sid=86ecd4e0-a7e1-442e-a05f-acd99ada94dc},
volume = {334},
year = {2011}
}
@article{Darmawan2016,
abstract = {This work investigates the redox cycling effect on the physicochemical characteristics and gas permeation behaviour of binary metal oxide (iron/cobalt) silica membranes prepared by sol–gel method from tetraethyl orthosilicate, cobalt and iron nitrates, peroxide and water. A Fe/Co ratio of 10/90 conferred more stable silica microstructure under redox cycling, contrary to Fe/Co ratios ⩾25/75 which led to structural densification after the first redox cycle. The gas permeance redox effect of Fe/Co = 10/90 silica membranes resulted in a switchable permeance increase and decrease as the membrane was reduced and oxidised, respectively. However, the extent of the switchable changes were not constant and tended to decrease at each cycle, as evidenced by the He/CO2permselectivity shift from a higher value of 130 (first oxidation cycle) down to 80 (third oxidation cycle). It was found that a shift in temperature corresponding to the exothermic oxidation peak, attributed to the progressive changes in the formation of FexCo1−xOymixed oxide embedded in the silica matrix at each redox cycle. This was further supported by the sequential disappearance of the infrared absorbance peak at 565 cm−1upon redox cycling, thus demonstrating structural re-arrangement of the membrane. The permselectivity of He and H2over CO2sequentially decreased at a higher ratio for after the oxidation cycle, as a result of the silica matrix densification at high temperatures (400 °C).},
author = {Darmawan, Adi and Motuzas, Julius and Smart, Simon and Julbe, Anne and {Diniz da Costa}, Jo{\~{a}}o C.},
doi = {10.1016/j.seppur.2016.07.030},
file = {:Users/marc/Library/Application Support/Mendeley Desktop/Downloaded/Darmawan et al. - 2016 - Gas permeation redox effect of binary iron oxidecobalt oxide silica membranes.pdf:pdf},
issn = {18733794},
journal = {Separation and Purification Technology},
keywords = {Cobalt oxide,Gas permeance,Iron oxide,Redox,Silica membranes},
pages = {248--255},
publisher = {Elsevier B.V.},
title = {{Gas permeation redox effect of binary iron oxide/cobalt oxide silica membranes}},
url = {http://dx.doi.org/10.1016/j.seppur.2016.07.030},
volume = {171},
year = {2016}
}
@article{Qi2011,
abstract = {A novel microporous hybrid silica membrane for the separation of carbon dioxide, fabricated through sol-gel deposition of a microporous Nb-doped ethylene-bridged silsesquioxane layer on a multilayer porous support, was reported. Effect of the calcination temperature on H2/CO2separation properties of Nb-BTESE membrane was investigated. Low CO2permeance was imparted by doping acidic niobium centers into the hybrid silica networks. Denser hybrid silica networks as well as more Lewis acid sites were generated as the calcination temperature elevated, which imparted very low CO2permeance to the novel hybrid membrane while retaining its relative high H2flux in the order of ∼10-7molm-2s-1Pa-1. Dominant densification occurred in the Nb-doped hybrid silica networks when the calcination temperature was lower than 400°C. Meanwhile, the Nb-BTESE membrane showed relatively weak acidity which was induced by niobium doping. Dual effects are working when the heat-treated temperature was higher than 400°C. On the one hand, the increased surface acidity reduced the number of sites and/or affinity for adsorption of CO2as the calcination temperature elevated. On the other hand, membrane densification occurred during the calcination process. Therefore, the permselectivity of H2/CO2for Nb-BTESE membrane could be tuned by altering the calcination temperature. The Nb-BTESE membrane calcined at 450°C showed both relative high hydrogen permeance (∼9.7×10-8molm-2s-1Pa-1) and excellent H2/CO2permselectivity (220), as compared with Nb-BTESE membranes calcined at other temperatures. {\textcopyright} 2011 Elsevier B.V.},
author = {Qi, Hong and Han, Jing and Xu, Nanping},
doi = {10.1016/j.memsci.2011.08.013},
file = {:Users/marc/Library/Application Support/Mendeley Desktop/Downloaded/Qi, Han, Xu - 2011 - Effect of calcination temperature on carbon dioxide separation properties of a novel microporous hybrid silica memb.pdf:pdf},
isbn = {0376-7388},
issn = {03767388},
journal = {Journal of Membrane Science},
keywords = {1,2-Bis(triethoxysilyl)ethane,CO2 separation,Calcination temperature,Hybrid silica membranes,Niobium penta(n)butoxide},
number = {1-2},
pages = {231--237},
publisher = {Elsevier B.V.},
title = {{Effect of calcination temperature on carbon dioxide separation properties of a novel microporous hybrid silica membrane}},
url = {http://dx.doi.org/10.1016/j.memsci.2011.08.013},
volume = {382},
year = {2011}
}
@article{Zhou2015,
abstract = {A novel electrochemical sensor based on Cu-MOF-199 [Cu-MOF-199 = Cu3(BTC)2 (BTC = 1,3,5-benzenetricarboxylicacid)] and SWCNTs (single-walled carbon nanotubes) was fabricated for the simultaneous determination of hydroquinone (HQ) and catechol (CT). The modification procedure was carried out through casting SWCNTs on the bare glassy carbon electrode (GCE) and followed by the electrodeposition of Cu-MOF-199 on the SWCNTs modified electrode. Cyclic voltammetry (CV), electrochemical impedance spectroscopy (EIS) and scanning electron microscopy (SEM) were performed to characterize the electrochemical performance and surface characteristics of the as-prepared sensor. The composite electrode exhibited an excellent electrocatalytic activity with increased electrochemical signals towards the oxidation of HQ and CT, owing to the synergistic effect of SWCNTs and Cu-MOF-199. Under the optimized condition, the linear response range were from 0.1 to 1453 $\mu$mol L-1 (RHQ = 0.9999) for HQ and 0.1-1150 $\mu$mol L-1 (RCT = 0.9990) for CT. The detection limits for HQ and CT were as low as 0.08 and 0.1 $\mu$mol L-1, respectively. Moreover, the modified electrode presented the good reproducibility and the excellent anti-interference performance. The analytical performance of the developed sensor for the simultaneous detection of HQ and CT had been evaluated in practical samples with satisfying results.},
author = {Zhou, Jian and Li, Xi and Yang, Linlin and Yan, Songlin and Wang, Mengmeng and Cheng, Dan and Chen, Qi and Dong, Yulin and Liu, Peng and Cai, Weiquan and Zhang, Chaocan},
doi = {10.1016/j.aca.2015.09.054},
file = {:Users/marc/Library/Application Support/Mendeley Desktop/Downloaded/Zhou et al. - 2015 - The Cu-MOF-199single-walled carbon nanotubes modified electrode for simultaneous determination of hydroquinone and.pdf:pdf},
issn = {18734324},
journal = {Analytica Chimica Acta},
keywords = {Catechol,Cu-MOF-199,Electrocatalysis,Hydroquinone,Simultaneous determination,Single-walled carbon nanotubes},
pages = {57--65},
pmid = {26547493},
title = {{The Cu-MOF-199/single-walled carbon nanotubes modified electrode for simultaneous determination of hydroquinone and catechol with extended linear ranges and lower detection limits}},
volume = {899},
year = {2015}
}
@article{Muller2013,
author = {M{\"{u}}ller, K and St{\"{a}}dter, M and Rachow, F and Hoffmannbeck, D and Schmei{\ss}er, D},
doi = {10.1007/s12665-013-2609-3},
isbn = {1866-6280 1866-6299},
journal = {Environmental Earth Sciences},
number = {8},
pages = {3771--3778},
title = {{Sabatier-based CO2-methanation by catalytic conversion}},
volume = {70},
year = {2013}
}
@article{Vigen2014,
abstract = {Acceptor doped LaCrO3 is a promising material for dense, ceramic hydrogen permeable membranes, displaying hydrogen flux in the order of 10-4mlmin-1cm-1 in a 10{\%} H2+2.5{\%} H2O/dry Ar gradient at 1000°C. In this work we have characterized the ambipolar proton electron hole conductivity in La0.87Sr0.13CrO3-$\delta$ by means of hydrogen flux measurements. Proton transport parameters were extracted, yielding a pre-exponential factor of 3cm2KV-1s-1 and an enthalpy of mobility of 65kJmol-1. Hydrogen flux measurements showed that applying a layer of Pt on both feed and sweep side surfaces significantly altered the temperature dependency and increased the hydrogen flux in a 550$\mu$m thick membrane. This indicates that surface kinetics will limit the hydrogen flux in uncoated membranes. From hydrogen surface exchange measurements, a surface exchange coefficient ranging from 10-10 to 10-8molcm-2s-1 at 325-600°C was obtained. {\textcopyright} 2014 Elsevier B.V.},
author = {Vigen, Camilla K. and Haugsrud, Reidar},
doi = {10.1016/j.memsci.2014.06.012},
file = {:Users/marc/Library/Application Support/Mendeley Desktop/Downloaded/Vigen, Haugsrud - 2014 - Hydrogen flux in La0.87Sr0.13CrO3-$\delta$.pdf:pdf},
issn = {18733123},
journal = {Journal of Membrane Science},
keywords = {Hydrogen permeation,Proton conductivity,Surface kinetics},
pages = {317--323},
title = {{Hydrogen flux in La0.87Sr0.13CrO3-$\delta$}},
volume = {468},
year = {2014}
}
@article{Athayde2016,
author = {Athayde, Daniel D and Souza, Douglas F and Silva, Alysson M A and Vasconcelos, Daniela and Nunes, Eduardo H M and {Diniz da Costa}, Jo{\~{a}}o C and Vasconcelos, Wander L},
doi = {10.1016/j.ceramint.2016.01.130},
isbn = {02728842},
journal = {Ceramics International},
number = {6},
pages = {6555--6571},
title = {{Review of perovskite ceramic synthesis and membrane preparation methods}},
volume = {42},
year = {2016}
}
@book{Li2007,
author = {Li, K},
booktitle = {Chemical Engineering},
isbn = {9780470014400},
title = {{Ceramic Membranes for separation and reaction}},
year = {2007}
}
@article{Choi2014,
abstract = {Diagnostic sensing device using exhaled breath of human have critical advantages due to the noninvasive diagnosis and high potential for portable device with simple analysis process. Here, we report ultrafast as well as highly sensitive bumpy WO3 hemitube nanostructure assisted by O2 plasma surface modification with functionalization of graphene-based material for the detection of acetone (CH3COCH3) and hydrogen sulfide (H2S) which are biomarkers for the diagnosis of diabetes and halitosis, respectively. 0.1 wt {\%} graphene oxide (GO)- and 0.1 wt {\%} thin layered graphite (GR)- WO3 hemitube composites showed response times of 11.5 ± 2.5 s and 13.5 ± 3.4 s to 1 ppm acetone as well as 12.5 ± 1.9 s and 10.0 ± 1.6 s to 1 ppm of H2S, respectively. In addition, low limits of detection (LOD) of 100 ppb (Rair/Rgas = 1.7 for acetone and Rair/Rgas = 3.3 for H2S at 300 °C) were achieved. The superior sensing properties were ascribed to the electronic sensitization of graphene based materials by modulating space charged layers at the interfaces between n-type WO3 hemitubes and p-type graphene based materials, as identified by Kelvin Probe Force Microscopy (KPFM). Rapid response and superior sensitivity of the proposed sensing materials following cyclic thermal aging demonstrates good potential for real-time exhaled breath diagnosis of diseases.},
author = {Choi, Seon Jin and Fuchs, Franz and Demadrille, Renaud and Gr{\'{e}}vin, Benjamin and Jang, Bong Hoon and Lee, Seo Jin and Lee, Jong Heun and Tuller, Harry L. and Kim, Il Doo},
doi = {10.1021/am501394r},
file = {:Users/marc/Library/Application Support/Mendeley Desktop/Downloaded/Choi et al. - 2014 - Fast responding exhaled-breath sensors using WO3 hemitubes functionalized by graphene-based electronic sensitizers.pdf:pdf},
isbn = {1944-8244},
issn = {19448252},
journal = {ACS Applied Materials and Interfaces},
keywords = {diagnosis of diseases,electrospinning,exhaled breath sensor,graphene},
number = {12},
pages = {9061--9070},
pmid = {24844154},
title = {{Fast responding exhaled-breath sensors using WO3 hemitubes functionalized by graphene-based electronic sensitizers for diagnosis of diseases}},
volume = {6},
year = {2014}
}
@article{Li1999a,
author = {Li, Anwu and Liang, Weiqiang and Hughes, Ronald},
journal = {Separation and Purification Technology},
pages = {113--119},
title = {{Repair of Pd/Al2O3 composite membranes containing defects}},
volume = {15},
year = {1999}
}
@article{OBrien2010,
author = {O'Brien, Casey P and Howard, Bret H and Miller, James B and Morreale, Bryan D and Gellman, Andrew J},
doi = {10.1016/j.memsci.2009.11.070},
isbn = {03767388},
journal = {Journal of Membrane Science},
number = {1-2},
pages = {380--384},
title = {{Inhibition of hydrogen transport through Pd and Pd47Cu53 membranes by H2S at 350°C}},
volume = {349},
year = {2010}
}
@article{Brown2011a,
author = {Brown, Andrew S. and Vargha, Gergely M. and Downey, Michael L. and Hart, Nick J and Ferrier, Gordon F and Hall, Karen L},
file = {:Users/marc/Library/Application Support/Mendeley Desktop/Downloaded/Brown et al. - 2011 - NPL report AS 64 - Methods for the analysis of trace-level impurities in hydrogen for fuel cell applications.pdf:pdf},
number = {August},
pages = {33},
title = {{NPL report AS 64 - Methods for the analysis of trace-level impurities in hydrogen for fuel cell applications}},
url = {http://www.tigeroptics.com/TA/photo/view.php?gal=users;site,cms,files{\&}s=orig{\&}f=Published+NPL+Report+AS+64+(2011)+-+hydrogen+purity+methods+(2).pdf},
year = {2011}
}
@article{Paillard2012a,
abstract = {Two methods of oxidant-fuel gas mixture preparation at high pressure have been analyzed and fit with the objective to avoid explosion damage during filling of cylinder. First method is called "safe method" as the flammability range is not crossed during cylinder filling due to the introduction of major component first in cylinder. The second method is called "accurate method" as minor component is introduced first in cylinder leading to a risk of explosion during the introduction of major component at the end while crossing the flammability domain. The diagrams representing the maximum filling pressure of filling to prevent any accident are given as a function of fuel concentration for lean and rich mixtures. These diagrams are deduced from flammability and detonation limits versus pressure at ambient temperature. Both methods have been applied to hydrogen-air mixtures. Copyright {\textcopyright} 2012, Hydrogen Energy Publications, LLC. Published by Elsevier Ltd. All rights.},
author = {Paillard, Claude Etienne and Naudet, Val{\'{e}}rie},
doi = {10.1016/j.ijhydene.2012.08.026},
file = {:Users/marc/Library/Application Support/Mendeley Desktop/Downloaded/Paillard, Naudet - 2012 - Risk assessment for the preparation of compressed oxidant-fuel gas mixtures. Application to the H 2air mixture.pdf:pdf},
issn = {03603199},
journal = {International Journal of Hydrogen Energy},
keywords = {Flammability limits,Gas explosion,High pressure gas manufacturing,Hydrogen risk,Oxidant-fuel gas mixture manufacturing},
number = {22},
pages = {17336--17349},
title = {{Risk assessment for the preparation of compressed oxidant-fuel gas mixtures. Application to the H 2/air mixture manufacturing}},
volume = {37},
year = {2012}
}
@article{Kargari2014,
abstract = {In the present study, the composite polyetherimide (PEI) membrane coated with poly dimethyl siloxane (PDMS) was synthesized and optimum conditions of coating were obtained for separation of hydrogen from methane. Three coating techniques "pouring solution inclined by 45°", "film casting" and "dip-coating" were used. The effect of sequential coating for different methods on permselectivity of the membranes was investigated. In addition, the influences of coating conditions including coating solution concentration, coating and curing temperatures were examined. The results showed that when the concentration of PDMS coating solution was increased; the permeance of H2was initially declined rapidly and was then gradually leveled off. The optimum concentration of coating solution was 15 wt.{\%}. The examination of the curing and coating temperatures showed no significant effect on H2permeance and selectivity. In the "dip coating" method, two times coating showed superior permeance and selectivity and in "film casting", the performance of triple coating was promising. Higher selectivities for the composite membrane prepared by "dip-coating" introduced this method as the best method. The sequential dip-coating with different PDMS concentrations was applied and the selectivity was enhanced significantly from 26 to 96 for pure gases and from 22 to 70 for the binary gas mixture. Finally, the influence of pressure on the separation performance of the fabricated membrane was investigated. Copyright {\textcopyright} 2014, Hydrogen Energy Publications, LLC. Published by Elsevier Ltd. All rights reserved.},
author = {Kargari, Ali and {Arabi Shamsabadi}, Ahmad and {Bahrami Babaheidari}, Masoud},
doi = {10.1016/j.ijhydene.2014.02.009},
file = {:Users/marc/Library/Application Support/Mendeley Desktop/Downloaded/Kargari, Arabi Shamsabadi, Bahrami Babaheidari - 2014 - Influence of coating conditions on the H2 separation performance from H2CH4 gas.pdf:pdf},
isbn = {0360-3199},
issn = {03603199},
journal = {International Journal of Hydrogen Energy},
keywords = {Coating method,PDMS/PEI composite membrane,Permselectivity},
number = {12},
pages = {6588--6597},
publisher = {Elsevier Ltd},
title = {{Influence of coating conditions on the H2 separation performance from H2/CH4 gas mixtures by the PDMS/PEI composite membrane}},
url = {http://dx.doi.org/10.1016/j.ijhydene.2014.02.009},
volume = {39},
year = {2014}
}
@article{Vashist2011,
abstract = {{\textless}p{\textgreater}Quartz crystal microbalance (QCM) has gained exceptional importance in the fields of (bio)sensors, material science, environmental monitoring, and electrochemistry based on the phenomenal development in QCM-based sensing during the last two decades. This review provides an overview of recent advances made in QCM-based sensors, which have been widely employed in a plethora of applications for the detection of chemicals, biomolecules and microorganisms.{\textless}/p{\textgreater}},
author = {Vashist, Sandeep Kumar and Vashist, Priya},
doi = {10.1155/2011/571405},
file = {:Users/marc/Library/Application Support/Mendeley Desktop/Downloaded/Vashist, Vashist - 2011 - Recent advances in quartz crystal microbalance-based sensors.pdf:pdf},
issn = {1687725X},
journal = {Journal of Sensors},
title = {{Recent advances in quartz crystal microbalance-based sensors}},
volume = {2011},
year = {2011}
}
@article{Tan2014,
abstract = {In this work, BaCe0.8Y0.2O3-?? (BCY) perovskite hollow fibre membranes were fabricated by a phase inversion and sintering method. BCY powder was prepared by the sol-gel technique using ethylenediaminetetraacetic acid (EDTA) and citric acid as the complexing agents. Gel calcination was carried out at high temperature to form the desired crystal structure. The qualified BCY hollow fibre membranes could not be achieved even the sintering was carried out at temperatures up to 1550oC due to the poor densification behavior of the BCY material. The addition of sintering aid (1 wt{\%} Co2O3) inside BCY powder as the membrane starting material significantly improved the densification process, leading to the formation of gas-tight BCY hollow fibres. The optimum sintering temperature of BCY hollow fibre membrane was 1400 C to achieve the best mechanical strength. H2 permeation through the BCY hollow fibre membranes was carried out between 700 and 1050 C using 25{\%} H2-He mixture as feed gas and N 2 as sweep gas, respectively. For comparison purpose, the disk-shaped BCY membrane with a thickness of 1 mm was also prepared. The measured H 2 permeation flux through the BCY hollow fibres reached up to 0.38 mL cm-2 min-1 at 1050 C strikingly contrasting to the low values of less than 0.01 mL cm-2 min-1 from the disk-shaped membrane. After the permeation test, the microstructure of BCY hollow fibre membrane was still maintained well without signals of membrane disintegration or peeling off. ?? 2013 Elsevier Ltd and Techna Group S.r.l.},
author = {Tan, Xihan and Tan, Xiaoyao and Yang, Naitao and Meng, Bo and Zhang, Kun and Liu, Shaomin},
doi = {10.1016/j.ceramint.2013.09.132},
file = {:Users/marc/Library/Application Support/Mendeley Desktop/Downloaded/Tan et al. - 2014 - High performance BaCe0.8Y0.2O3-A (BCY) hollow fibre membranes for hydrogen permeation.pdf:pdf},
isbn = {0272-8842},
issn = {02728842},
journal = {Ceramics International},
keywords = {A. Extrusion,A. Sintering,D. Perovskites,E. Membranes,Hydrogen permeation},
number = {2},
pages = {3131--3138},
title = {{High performance BaCe0.8Y0.2O3-A (BCY) hollow fibre membranes for hydrogen permeation}},
volume = {40},
year = {2014}
}
@article{Tanaka1992,
abstract = {Permeability and solubility coefficients for H2, CO2, O2, CO, N2, and CH4 in polyimides prepared from 6FDA and methyl-substituted phenylenediamines were measured to investigate effects of the substituents on gas permeability and permselectivity. The methyl substituents restrict internal rotation around the bonds between the phenyl rings and the imide rings. The rigidity and nonplanar structure of the polymer chain, and the bulkiness of methyl groups make chain packing inefficient, resulting in increases in both diffusion and solubility coefficients of the gases. Polyimides from tetramethyl-p-phenylenediamine and trimethyl-m-phenylenediamine display very high permeability coefficients and very low permselectivity due to very high diffusion coefficients and very low diffusivity selectivity, as compared with the other polyimides having a similar fraction of free space. This suggests that these polyimides have high fractions of large-size free spaces.},
author = {Tanaka, Kazuhiro and Okano, Masaaki and Toshino, Hiroyuki and Kita, Hidetoshi and Okamoto, Ken‐Ichi ‐I},
doi = {10.1002/polb.1992.090300813},
isbn = {1099-0488},
issn = {10990488},
journal = {Journal of Polymer Science Part B: Polymer Physics},
keywords = {diffusion of gases in polyimides,permeation and solubility of gases in polyimides p,polyimides from methyl‐substituted phenylenediamin},
number = {8},
pages = {907--914},
title = {{Effect of methyl substituents on permeability and permselectivity of gases in polyimides prepared from methyl‐substituted phenylenediamines}},
volume = {30},
year = {1992}
}
@article{Deng2014,
abstract = {A new type of quasi-1D nanofiber architecture with a heterostructure was prepared via a combination of electrospinning and hydrothermal strategies in the case of CuO-TiO2, which required a low operating temperature, and showed a high response and excellent selectivity to formaldehyde and ethanol gases.},
author = {Deng, Jianan and Wang, Lili and Lou, Zheng and Zhang, Tong},
doi = {10.1039/C4TA00160E},
file = {:Users/marc/Library/Application Support/Mendeley Desktop/Downloaded/Deng et al. - 2014 - Design of CuO-TiO2 heterostructure nanofibers and their sensing performance.pdf:pdf},
isbn = {2050-7488},
issn = {2050-7488},
journal = {Journal of Materials Chemistry A},
number = {24},
pages = {9030--9034},
title = {{Design of CuO-TiO2 heterostructure nanofibers and their sensing performance}},
url = {http://dx.doi.org/10.1039/C4TA00160E},
volume = {2},
year = {2014}
}
@article{Li2011,
abstract = {Hydrogen permeation through SrCe0.7Zr0.2Eu0.1O3-?? membranes was investigated as a function of temperature, feed H2O partial pressure, feed H2 partial pressure and flow rate. Hydrogen permeation flux was proportional to the transmembrane H2 partial pressure gradient with a 1/4 dependence. A maximum H2 permeation flux of 0.23 and 0.21cm3/cm2min was obtained at 900??C for 100{\%} H2 and 97vol{\%} H2/3vol{\%} H2O conditions, respectively. The activation energy decreased with increasing H2 partial pressure and/or decreasing steam partial pressure. Permeation flux through the SrCe0.7Zr0.2Eu0.1O3-?? membrane was stable under wet H2, water gas shift reaction and steam reforming of methane conditions. ?? 2011 Elsevier B.V.},
author = {Li, Jianlin and Yoon, Heesung and Wachsman, E. D.},
doi = {10.1016/j.memsci.2011.07.032},
file = {:Users/marc/Library/Application Support/Mendeley Desktop/Downloaded/Li, Yoon, Wachsman - 2011 - Hydrogen permeation through thin supported SrCe0.7Zr0.2Eu0.1O3- membranes dependence of flux on defect equil.pdf:pdf},
issn = {03767388},
journal = {Journal of Membrane Science},
keywords = {Hydrogen,Permeation,Proton conducting membrane,SrCe0.7Zr0.2Eu0.1O3-??},
number = {1-2},
pages = {126--131},
publisher = {Elsevier B.V.},
title = {{Hydrogen permeation through thin supported SrCe0.7Zr0.2Eu0.1O3-?? membranes; dependence of flux on defect equilibria and operating conditions}},
url = {http://dx.doi.org/10.1016/j.memsci.2011.07.032},
volume = {381},
year = {2011}
}
@article{Ertl1977,
author = {Ertl, G. and Tornau, J.},
journal = {International journal of research in physical chemistry and chemical physics},
number = {8},
pages = {301--308},
title = {{The Catalytic Decomposition of Formaldehyde on Palladium}},
volume = {104},
year = {1977}
}
@article{Yoon2007,
annote = {doi: 10.1021/ar000119c},
author = {Yoon, Kyung Byung},
doi = {10.1021/ar000119c},
issn = {0001-4842},
journal = {Accounts of Chemical Research},
month = {jan},
number = {1},
pages = {29--40},
publisher = {American Chemical Society},
title = {{Organization of Zeolite Microcrystals for Production of Functional Materials}},
url = {https://doi.org/10.1021/ar000119c},
volume = {40},
year = {2007}
}
@article{Tarditi2014a,
author = {Tarditi, A M and Imhoff, C and Braun, F},
journal = {Journal of Membrane {\{}{\ldots}{\}}},
pages = {246--255},
title = {{PdCuAu ternary alloy membranes: Hydrogen permeation properties in the presence of H 2 S}},
volume = {479},
year = {2014}
}
@article{Opalka2011,
abstract = {Atomic modeling was conducted to investigate the origin of S interactions with Pd alloy H selective membrane candidates selected from the Pd-Cu, Pd-Ag, and Pd-Au binary systems, as well as their constitutive metals. The electronic characteristics of these alloy/metal systems played a more predominant role in controlling S bonding behavior than surface site geometries. The electronic coupling of S p orbitals bonding with alloy/metal d-bands in the adsorbate/slab density of states split the lower energy p bonding state and the d-band center further apart with increasing S bonding strength. A universal linear correlation was established for increasing adsorption strength (decreasing adsorption enthalpy) of 0.25 monolayer S with the increasing density of states energy difference: [d-band center - S p bonding peak]. The S interactions predicted at higher coverage provided indications of alloy susceptibility to irreversible S corrosion. The reversible adsorption of 1.0 monolayer S was only the most stable configuration on the more open Pd0.5Cu0.5 Im3??m and P4mmm (110) surfaces. The most competitive configuration for the interaction of a full S monolayer with the Pd0.75Cu0.25 Pm3??m and Pd0.875Au0.125 Fm3??m surfaces was the partial desorption and coupling of S. Partial incorporation of S to form a mixed absorbed/adsorbed S monolayer was more favorable for the Pd Fm3??m (111) surface, and also on the Pd0.5Cu0.5 P4mmm (101) and Pd0.75Ag0.25 Pm3??m (111) surfaces when accompanied by Pd segregation. The combination of S incorporation and Pd segregation was interpreted to be the first step towards nucleation of irreversible Pd4S formation. ?? 2011 Susanne M. Opalka.},
author = {Opalka, Susanne M. and L??vvik, Ole M. and Emerson, Sean C. and She, Ying and Vanderspurt, Thomas H.},
doi = {10.1016/j.memsci.2011.03.018},
file = {:Users/marc/Library/Application Support/Mendeley Desktop/Downloaded/Opalka et al. - 2011 - Electronic origins for sulfur interactions with palladium alloys for hydrogen-selective membranes.pdf:pdf},
isbn = {0376-7388},
issn = {03767388},
journal = {Journal of Membrane Science},
keywords = {DFT,Hydrogen membrane,Pd alloys,Sulfur tolerance},
number = {1-2},
pages = {96--103},
title = {{Electronic origins for sulfur interactions with palladium alloys for hydrogen-selective membranes}},
volume = {375},
year = {2011}
}
@article{Friebe2017,
abstract = {The 3D metal–organic framework (MOF) structure UiO-66 [Zr6O4(OH)4(bdc)6], featuring triangular pores of approximately 6 {\AA}, has been successfully prepared as a thin supported membrane layer with high crystallographic orientation on ceramic $\alpha$-Al2O3 supports. The adhesion of the MOF layer to the ceramic support was investigated in different taxing conditions. Furthermore, by coating this UiO-66 membrane with a thin polyimide (Matrimid) top layer, we prepared a multilayer composite. Said membranes have been evaluated in the separation of hydrogen (H2) from different binary mixtures at room temperature. H2 as the smallest molecule (2.9 {\AA}) should pass the UiO-66 membrane preferably since the kinetic diameters of all the other gases under study are larger. The gas mixture separation factors for the neat UiO-66 membrane were indeed found to be H2/CO2 = 5.1, H2/N2 = 4.7, H2/CH4 = 12.9, H2/C2H6 = 22.4, and H2/C3H8 = 28.5. The coating with Matrimid led to a sharp cutoff for gases with kinetic diameters greater than ...},
author = {Friebe, Sebastian and Geppert, Benjamin and Steinbach, Frank and Caro, J{\"{u}}rgen},
doi = {10.1021/acsami.7b02105},
file = {:Users/marc/Library/Application Support/Mendeley Desktop/Downloaded/Friebe et al. - 2017 - Metal-Organic Framework UiO-66 Layer A Highly Oriented Membrane with Good Selectivity and Hydrogen Permeance.pdf:pdf},
isbn = {1944-8244},
issn = {19448252},
journal = {ACS Applied Materials and Interfaces},
keywords = {MOF membrane,UiO-66,crystallographic orientation,hydrogen separation,polymer coating,stability},
number = {14},
pages = {12878--12885},
title = {{Metal-Organic Framework UiO-66 Layer: A Highly Oriented Membrane with Good Selectivity and Hydrogen Permeance}},
volume = {9},
year = {2017}
}
@article{Fennell2016,
abstract = {Chemiresistive sensors are becoming increasingly important as they offer an inexpensive option to conventional analytical instrumentation, they can be readily integrated into electronic devices, and they have low power requirements. Nanowires (NWs) are a major theme in chemosensor development. High surface area, interwire junctions, and restricted conduction pathways give intrinsically high sensitivity and new mechanisms to transduce the binding or action of analytes. This Review details the status of NW chemosensors with selected examples from the literature. We begin by proposing a principle for understanding electrical transport and transduction mechanisms in NW sensors. Next, we offer the reader a review of device performance parameters. Then, we consider the different NW types followed by a summary of NW assembly and different device platform architectures. Subsequently, we discuss NW functionalization strategies. Finally, we propose future developments in NW sensing to address selectivity, sensor drift, sensitivity, response analysis, and emerging applications.},
author = {Fennell, John F. and Liu, Sophie F. and Azzarelli, Joseph M. and Weis, Jonathan G. and Rochat, S{\'{e}}bastien and Mirica, Katherine A. and Ravnsb{\ae}k, Jens B. and Swager, Timothy M.},
doi = {10.1002/anie.201505308},
file = {:Users/marc/Library/Application Support/Mendeley Desktop/Downloaded/Fennell et al. - 2016 - Nanowire ChemicalBiological Sensors Status and a Roadmap for the Future.pdf:pdf},
isbn = {1521-3773 (Electronic)$\backslash$r1433-7851 (Linking)},
issn = {15213773},
journal = {Angewandte Chemie - International Edition},
keywords = {metal oxides,nanocarbons,nanowires,sensors,transduction mechanism},
number = {4},
pages = {1266--1281},
pmid = {26661299},
title = {{Nanowire Chemical/Biological Sensors: Status and a Roadmap for the Future}},
volume = {55},
year = {2016}
}
@article{Yun2011,
author = {Yun, Samhun and {Ted Oyama}, S},
doi = {10.1016/j.memsci.2011.03.057},
file = {:Users/marc/Library/Application Support/Mendeley Desktop/Downloaded/Yun, Ted Oyama - 2011 - Correlations in palladium membranes for hydrogen separation A review.pdf:pdf},
isbn = {03767388},
journal = {Journal of Membrane Science},
number = {1-2},
pages = {28--45},
title = {{Correlations in palladium membranes for hydrogen separation: A review}},
volume = {375},
year = {2011}
}
@article{Wagner1992,
author = {Wagner, E. S and Froment, G. F},
journal = {Hydrocarbon Processing},
number = {7},
title = {{Steam Reforming Analyzed}},
volume = {71},
year = {1992}
}
@article{Wang2016a,
abstract = {We report the intergrowth of ZIF-8 crystals on ultrathin graphene oxide (GO) membranes, which helps to reduce the non-selective pores of pristine GO membranes leading to gas selectivities as high as 406, 155, and 335 for H2/CO2, H2/N2, and H2/CH4 mixtures, respectively.},
annote = {Wang, Xuerui
Chi, Chenglong
Tao, Jifang
Peng, Yongwu
Ying, Shaoming
Qian, Yuhong
Dong, Jinqiao
Hu, Zhigang
Gu, Yuandong
Zhao, Dan
ENG
England
2016/05/18 06:00
Chem Commun (Camb). 2016 Jun 21;52(52):8087-90. doi: 10.1039/c6cc02013e.},
author = {Wang, X and Chi, C and Tao, J and Peng, Y and Ying, S and Qian, Y and Dong, J and Hu, Z and Gu, Y and Zhao, D},
doi = {10.1039/c6cc02013e},
isbn = {1364-548X (Electronic)
1359-7345 (Linking)},
journal = {Chem Commun (Camb)},
number = {52},
pages = {8087--8090},
pmid = {27181340},
title = {{Improving the hydrogen selectivity of graphene oxide membranes by reducing non-selective pores with intergrown ZIF-8 crystals}},
url = {http://www.ncbi.nlm.nih.gov/pubmed/27181340},
volume = {52},
year = {2016}
}
@article{Bosko2011a,
abstract = {The effect of plating times and metal deposition sequence on the morphology of Pd-Ag films on porous supports and also on alloy formation has been investigated. Pd-Ag composite membranes were prepared by electroless plating; the substrates were modified using dip coated alumina and in situ hydrothermal synthesis of NaA zeolite with secondary growth. The chemical and physical phenomena involved in the atomic diffusion during the thermal treatment as well as the crystalline structure were tracked by scanning electron microscopy, energy-dispersive X-ray analysis, X-ray photoelectron spectroscopy, and X-ray diffraction. In order to isolate the main component of the hydrogen transport (solution-diffusion mechanism), the composite membranes were considered as layers in series, with porous and dense parts; the porous areas were modeled by the dusty gas model. This allowed a more accurate estimation of Sievert's transport parameters and a critical comparison with data reported in the literature. ?? 2010, Hydrogen Energy Publications, LLC. Published by Elsevier Ltd. All rights reserved.},
author = {Bosko, M. L. and Lombardo, E. A. and Cornaglia, L. M.},
doi = {10.1016/j.ijhydene.2010.12.056},
file = {:Users/marc/Library/Application Support/Mendeley Desktop/Downloaded/Bosko, Lombardo, Cornaglia - 2011 - The effect of electroless plating time on the morphology, alloy formation and H2 transport propertie.pdf:pdf},
issn = {03603199},
journal = {International Journal of Hydrogen Energy},
keywords = {Alloy formation,Diffusion-solution mechanism,Knudsen contribution},
number = {6},
pages = {4068--4078},
publisher = {Elsevier Ltd},
title = {{The effect of electroless plating time on the morphology, alloy formation and H2 transport properties of Pd-Ag composite membranes}},
url = {http://dx.doi.org/10.1016/j.ijhydene.2010.12.056},
volume = {36},
year = {2011}
}
@article{Escolastico2013a,
abstract = {La(5.5) WO11.25-$\delta$ is a proton-conducting oxide that shows high protonic conductivity, sufficient electronic conductivity, and stability in moist CO2 environments. However, the H2 flows achieved to date when using La(5.5) WO11.25-$\delta$ membranes are still below the threshold for practical application in industrial processes. With the aim of improving the H2 flow obtained with this material, La(5.5) WO11.25-$\delta$ was doped in the W position by using Re and Mo; the chosen stoichiometry was La(5.5) W0.8 M0.2 O11.25-$\delta$ . This work presents the electrochemical characterization of these two compounds under reducing conditions, the H2 separation properties, as well as the influence of the H2 concentration in the feed stream, degree of humidification, and operating temperature. Doping with both Re and Mo enabled the magnitude of H2 permeation to be enhanced, reaching unrivaled values of up to 0.095 mL min(-1) cm(-2) at 700 °C for a La(5.5) W0.8 Re0.2 O11.25-$\delta$ membrane (760 $\mu$m thick). The spent membranes were investigated by using XRD, SEM, and TEM on focused-ion beam lamellas. Furthermore, the stability in CO2 -rich and H2 S-containing atmospheres was evaluated, and the compounds were shown to be stable in the atmospheres studied.},
author = {Escolastico, Sonia and Seeger, Janka and Roitsch, Stefan and Ivanova, Mariya and Meulenberg, Wilhelm A. and Serra, Jos{\'{e}} M.},
doi = {10.1002/cssc.201300091},
file = {:Users/marc/Library/Application Support/Mendeley Desktop/Downloaded/Escolastico et al. - 2013 - Enhanced H2 separation through mixed proton-electron conducting membranes based on La5.5W0.8M0.2O 11.25-$\delta$.pdf:pdf},
issn = {18645631},
journal = {ChemSusChem},
keywords = {conducting materials,hydrogen,lanthanum,membranes,proton transport},
number = {8},
pages = {1523--1532},
pmid = {23828818},
title = {{Enhanced H2 separation through mixed proton-electron conducting membranes based on La5.5W0.8M0.2O 11.25-$\delta$}},
volume = {6},
year = {2013}
}
@article{Kim2017,
abstract = {A linear humidity sensor based on bi-layered sensing area using exfoliated chromium carbide (Cr3C2) and polyacrylamide (PAM) thin films has been fabricated through all printed methods. Transition metal carbides have unique properties that can be applied in various gas sensing applications. In this work, different combinations of Cr3C2 with two polymers have been used to fabricate high performance linear humidity sensors. Cr3C2 was processed using wet grinding exfoliation to achieve flakes and small particles of the material. Based on the working principle and sensing results of the single layered sensors, a novel bi-layered structure was proposed and fabricated with stable linear response and good sensitivity. The sensors are capable of measuring percentage relative humidity accurately in range of 0{\%} RH to 90{\%} RH with an absolute change of 660 $\Omega$/{\%}RH and fast response and recovery times of 1 s and 1.9 s respectively. The sensors can also work with simple frequency based read-out circuit with a highly linear and stable response. The fabricated sensors are cheap, easy to fabricate, and are ideal to be used in high end environmental and health monitoring applications.},
author = {Kim, Hyun Bum and Sajid, Memoon and Kim, Kwang Tae and Na, Kyoung Hoan and Choi, Kyung Hyun},
doi = {10.1016/j.snb.2017.06.052},
file = {:Users/marc/Library/Application Support/Mendeley Desktop/Downloaded/Kim et al. - 2017 - Linear humidity sensor fabrication using bi-layered active region of transition metal carbide and polymer thin films.pdf:pdf},
issn = {09254005},
journal = {Sensors and Actuators, B: Chemical},
keywords = {Bi-layer,Composite,Humidity sensor,Linear,Polymer,Transition metal carbides},
pages = {725--734},
publisher = {Elsevier B.V.},
title = {{Linear humidity sensor fabrication using bi-layered active region of transition metal carbide and polymer thin films}},
url = {http://dx.doi.org/10.1016/j.snb.2017.06.052},
volume = {252},
year = {2017}
}
@article{Hong2013,
abstract = {MFI zeolite membranes were synthesized on porous $\alpha$-alumina hollow fibers by in-situ hydrothermal synthesis. The membranes were further modified for H2separation by on-stream catalytic cracking deposition of methyldiethoxysilane (MDES) in the zeolitic pores. The separation performance of the modified membranes was characterized by separation of H2/CO2gas mixture at 500 C. Activation of MFI zeolite membranes by air at 500 C was found to promote catalytic cracking deposition of silane in the zeolitic pores effectively, which resulted in significant improvement of H2-separating performance. The H2/CO2separation factor of 45.6 with H2permeance of 1.0 × 10-8mol m-2s-1Pa-1was obtained at 500 C for a modified hollow fiber MFI zeolite membrane. The as-made membranes showed good thermochemical stability for the separation of H2/CO2gas mixture containing H2O and H2S, respectively. {\textcopyright} Copyright {\textcopyright} 2012, Hydrogen Energy Publications, LLC. Published by Elsevier Ltd. All rights reserved.},
author = {Hong, Zhou and Sun, Feng and Chen, Dongdong and Zhang, Chun and Gu, Xuehong and Xu, Nanping},
doi = {10.1016/j.ijhydene.2013.04.154},
file = {:Users/marc/Library/Application Support/Mendeley Desktop/Downloaded/Hong et al. - 2013 - Improvement of hydrogen-separating performance by on-stream catalytic cracking of silane over hollow fiber MFI zeol.pdf:pdf},
isbn = {0360-3199},
issn = {03603199},
journal = {International Journal of Hydrogen Energy},
keywords = {Hollow fiber,Hydrogen separation,MFI zeolite membrane,Modification,Stability},
number = {20},
pages = {8409--8414},
publisher = {Elsevier Ltd},
title = {{Improvement of hydrogen-separating performance by on-stream catalytic cracking of silane over hollow fiber MFI zeolite membrane}},
url = {http://dx.doi.org/10.1016/j.ijhydene.2013.04.154},
volume = {38},
year = {2013}
}
@article{Wang2016f,
abstract = {We report the intergrowth of ZIF-8 crystals on ultrathin graphene oxide (GO) membranes, which helps to reduce the non-selective pores of pristine GO membranes leading to gas selectivities as high as 406, 155, and 335 for H2/CO2, H2/N2, and H2/CH4 mixtures, respectively.},
author = {Wang, Xuerui and Chi, Chenglong and Tao, Jifang and Peng, Yongwu and Ying, Shaoming and Qian, Yuhong and Dong, Jinqiao and Hu, Zhigang and Gu, Yuandong and Zhao, Dan},
doi = {10.1039/c6cc02013e},
file = {:Users/marc/Library/Application Support/Mendeley Desktop/Downloaded/Wang et al. - 2016 - Improving the hydrogen selectivity of graphene oxide membranes by reducing non-selective pores with intergrown ZIF-.pdf:pdf},
isbn = {1359-7345},
issn = {1364548X},
journal = {Chemical Communications},
number = {52},
pages = {8087--8090},
pmid = {27181340},
publisher = {Royal Society of Chemistry},
title = {{Improving the hydrogen selectivity of graphene oxide membranes by reducing non-selective pores with intergrown ZIF-8 crystals}},
url = {http://dx.doi.org/10.1039/c6cc02013e},
volume = {52},
year = {2016}
}
@article{Wang2009a,
abstract = {A series of copolymer, poly(phthalazinone ether sulfone ketone)s (PPESKs) with the sulfone over ketone unit (S/K) ratio varying from 20/80, 50/50 to 80/20, were used as precursors to prepare carbon membranes. The effects of chemical structure as S/K ratio of PPESKs on the microstructure and gas separation performance of their derived carbon membranes were mainly investigated. The properties of PPESKs were detected in terms of density, fractional free volume, char yield, interlayer distance and glass transition temperature. During the formation process of carbon membranes (i.e., stabilization and pyrolysis), the changes in functional groups, microstructural parameters and gas permeation were monitored by FTIR, X-ray diffraction, TEM and single gas permeation techniques. The results have shown that the microstructure and gas permeation of obtained carbon membranes are significantly affected by the S/K ratio in precursor PPESKs. Carbon membranes exhibit higher selectivity and lower permeability when prepared at low pyrolytic temperature (i.e., 650 °C and 800 °C) and from PPESKs with S/K ratio equaling 50/50, followed with 20/80 and 80/20. As for carbon membranes prepared at high pyrolytic temperature (i.e., 950 °C), the selectivity order of them is well in accordance with S/K mole ratio in precursor PPESKs: 20/80 {\textgreater} 50/50 {\textgreater} 80/20, and vice versa for permeability. {\textcopyright} 2009 Elsevier B.V. All rights reserved.},
author = {Wang, Tonghua and Zhang, Bing and Qiu, Jieshan and Wu, Yonghong and Zhang, Shouhai and Cao, Yiming},
doi = {10.1016/j.memsci.2009.01.006},
file = {:Users/marc/Library/Application Support/Mendeley Desktop/Downloaded/Wang et al. - 2009 - Effects of sulfoneketone in poly(phthalazinone ether sulfone ketone) on the gas permeation of their derived carbon.pdf:pdf},
issn = {03767388},
journal = {Journal of Membrane Science},
keywords = {Gas separation,Microporous carbon,Molecular sieving,Pyrolysis,Stabilization},
number = {1-2},
pages = {319--325},
title = {{Effects of sulfone/ketone in poly(phthalazinone ether sulfone ketone) on the gas permeation of their derived carbon membranes}},
volume = {330},
year = {2009}
}
@article{Mejdell2009,
author = {Mejdell, A L and J{\o}ndahl, M and Peters, T A and Bredesen, R and Venvik, H J},
doi = {10.1016/j.seppur.2009.04.025},
isbn = {13835866},
journal = {Separation and Purification Technology},
number = {2},
pages = {178--184},
title = {{Effects of CO and CO2 on hydrogen permeation through a ∼3$\mu$m Pd/Ag 23wt.{\%} membrane employed in a microchannel membrane configuration}},
volume = {68},
year = {2009}
}
@article{Borg2013,
author = {Borg, Anne and Venvik, Hilde},
file = {:Users/marc/Library/Application Support/Mendeley Desktop/Downloaded/Borg, Venvik - 2013 - Pd-based Membranes for Hydrogen Separation - Membrane Structure and Hydrogen Sorption and Permeation Behavior Liv.pdf:pdf},
title = {{Pd-based Membranes for Hydrogen Separation - Membrane Structure and Hydrogen Sorption and Permeation Behavior Live Nova N{\ae}ss}},
year = {2013}
}
@article{Li2008a,
author = {Li, Hui and Tang, Yanghua and Wang, Zhenwei and Shi, Zheng and Wu, Shaohong and Song, Datong and Zhang, Jianlu and Fatih, Khalid and Zhang, Jiujun and Wang, Haijiang and Liu, Zhongsheng and Abouatallah, Rami and Mazza, Antonio},
doi = {10.1016/j.jpowsour.2007.12.068},
isbn = {03787753},
journal = {Journal of Power Sources},
number = {1},
pages = {103--117},
title = {{A review of water flooding issues in the proton exchange membrane fuel cell}},
volume = {178},
year = {2008}
}
@article{Darmawan2015a,
abstract = {Abstract This work investigates the performance of iron cobalt oxide silica membranes for the separation of binary gas mixtures containing H{\textless}inf{\textgreater}2{\textless}/inf{\textgreater} and Ar up to 500 °C. A series of membranes were prepared by fixing the iron/cobalt molar ratio at 10/90, 25/75 and 50/50. H{\textless}inf{\textgreater}2{\textless}/inf{\textgreater} preferentially permeated though the membranes, and H{\textless}inf{\textgreater}2{\textless}/inf{\textgreater} purity in the permeate stream increased with temperature for all H{\textless}inf{\textgreater}2{\textless}/inf{\textgreater}/Ar binary gas mixtures. The fluxes of H{\textless}inf{\textgreater}2{\textless}/inf{\textgreater} from binary gas mixtures complied, for the most part, with a temperature dependent transport mechanism, similar to that delivered by single gas permeation. However, it was notable to observe a "transition point" where H{\textless}inf{\textgreater}2{\textless}/inf{\textgreater} purity versus H{\textless}inf{\textgreater}2{\textless}/inf{\textgreater} flux clearly changed from temperature independent to temperature dependent. This gas separation transition point was also found to be a function of the quality of the membrane. Indeed the best performing membrane (Fe/Co = 10/90) also had the highest gas separation transition point at ∼70{\%} H{\textless}inf{\textgreater}2{\textless}/inf{\textgreater} purity. This reduced to ∼60{\%} for the medium quality membrane (Fe/Co = 25/75) and was at its lowest ∼43{\%} for the lower quality membrane (Fe/Co = 50/50). The binary gas fractions therefore affect the H{\textless}inf{\textgreater}2{\textless}/inf{\textgreater} fluxes and H{\textless}inf{\textgreater}2{\textless}/inf{\textgreater} purity more significantly than that expected in single gas permeation. Therefore, the relationship between membrane quality and gas separation transition point is established for the first time in this work.},
author = {Darmawan, Adi and Motuzas, Julius and Smart, Simon and Julbe, Anne and {Diniz Da Costa}, Jo{\~{a}}o C.},
doi = {10.1016/j.seppur.2015.07.055},
file = {:Users/marc/Library/Application Support/Mendeley Desktop/Downloaded/Darmawan et al. - 2015 - Temperature dependent transition point of purity versus flux for gas separation in FeCo-silica membranes.pdf:pdf},
issn = {18733794},
journal = {Separation and Purification Technology},
keywords = {Binary gas,H{\textless}inf{\textgreater}2{\textless}/inf{\textgreater} fluxes,H{\textless}inf{\textgreater}2{\textless}/inf{\textgreater} purity,Temperature dependent,Transition point},
pages = {284--291},
publisher = {Elsevier B.V.},
title = {{Temperature dependent transition point of purity versus flux for gas separation in Fe/Co-silica membranes}},
url = {http://dx.doi.org/10.1016/j.seppur.2015.07.055},
volume = {151},
year = {2015}
}
@article{Startsev2013,
abstract = {A new catalytic reaction of hydrogen sulfide decomposition is discovered, the reaction occurs on metal catalysts in gas phase according to equation 2H(2)S {\textless}-{\textgreater} 2H(2) + S-2((gas)) to produce hydrogen and gaseous diatomic sulfur, conversion of hydrogen sulfide at room temperature is close to 15 {\%}. The thermodynamic driving force of the reaction is the formation of the chemical sulfur-sulfur bond between two hydrogen sulfide molecules adsorbed on two adjacent metal atoms in the key surface intermediate and elimination of hydrogen into gas phase. "Fingerprints" of diatomic sulfur adsorbed on the solid surfaces and dissolved in different solvents are studied. In closed vessels in adsorbed or dissolved states, this molecule is stable for a long period of time (weeks). A possible electronic structure of diatomic gaseous sulfur in the singlet state is considered. According to DFT/CASSCF calculations, energy of the singlet state of S-2 molecule is over the triplet ground state energy for 10.4/14.4 kcal/mol. Some properties of gaseous diatomic sulfur are also investigated. Catalytic solid systems, both bulk and supported on porous carriers, are developed. When hydrogen sulfide is passing through the solid catalyst immersed in liquid solvent which is capable of dissolving sulfur generated, conversion of hydrogen sulfide at room temperature achieves 100 {\%}, producing hydrogen in gas phase. This gives grounds to consider hydrogen sulfide as inexhaustible potential source of hydrogen-a very valuable chemical reagent and environmentally friendly energy product.},
author = {Startsev, A. N. and Kruglyakova, O. V. and Chesalov, Yu A. and Ruzankin, S. Ph and Kravtsov, E. A. and Larina, T. V. and Paukshtis, E. A.},
doi = {10.1007/s11244-013-0061-y},
file = {:Users/marc/Library/Application Support/Mendeley Desktop/Downloaded/Startsev et al. - 2013 - Low temperature catalytic decomposition of hydrogen sulfide into hydrogen and diatomic gaseous sulfur.pdf:pdf},
isbn = {1022-5528},
issn = {10225528},
journal = {Topics in Catalysis},
keywords = {Diatomic gaseous sulfur,Hydrogen production,Hydrogen sulfide decomposition,Reaction thermodynamics,Sulfur electronic structure},
number = {11},
pages = {969--980},
title = {{Low temperature catalytic decomposition of hydrogen sulfide into hydrogen and diatomic gaseous sulfur}},
volume = {56},
year = {2013}
}
@article{Okazaki2008,
author = {Okazaki, Junya and Tanaka, David Alfredo Pacheco and Tanco, Margot Anabel Llosa and Wakui, Yoshito and Ikeda, Takuji and Mizukami, Fujio and Suzuki, Toshishige M},
doi = {10.2320/matertrans.MBW200720},
isbn = {1347-5320 1345-9678},
journal = {Materials Transactions},
number = {3},
pages = {449--452},
title = {{Preparation and Hydrogen Permeation Properties of Thin Pd-Au Alloy Membranes Supported on Porous {\$}\alpha{\$}-Alumina Tube}},
volume = {49},
year = {2008}
}
@article{Sun2014,
abstract = {We present an investigation of molecular permeation of gases through nanoporous graphene membranes via molecular dynamics simulations; four different gases are investigated, namely helium, hydrogen, nitrogen, and methane. We show that in addition to the direct (gas-kinetic) flux of molecules crossing from the bulk phase on one side of the graphene to the bulk phase on the other side, for gases that adsorb onto the graphene, significant contribution to the flux across the membrane comes from a surface mechanism by which molecules cross after being adsorbed onto the graphene surface. Our results quantify the relative contribution of the bulk and surface mechanisms and show that the direct flux can be described reasonably accurately using kinetic theory, provided the latter is appropriately modified assuming steric molecule-pore interactions, with gas molecules behaving as hard spheres of known kinetic diameters. The surface flux is negligible for gases that do not adsorb onto graphene (e.g., He and H2), while for gases that adsorb (e.g., CH4 and N2) it can be on the order of the direct flux or larger. Our results identify a nanopore geometry that is permeable to hydrogen and helium, is significantly less permeable to nitrogen, and is essentially impermeable to methane, thus validating previous suggestions that nanoporous graphene membranes can be used for gas separation. We also show that molecular permeation is strongly affected by pore functionalization; this observation may be sufficient to explain the large discrepancy between simulated and experimentally measured transport rates through nanoporous graphene membranes.},
author = {Sun, Chengzhen and Boutilier, Michael S H and Au, Harold and Poesio, Pietro and Bai, Bofeng and Karnik, Rohit and Hadjiconstantinou, Nicolas G.},
doi = {10.1021/la403969g},
file = {:Users/marc/Library/Application Support/Mendeley Desktop/Downloaded/Sun et al. - 2014 - Mechanisms of molecular permeation through nanoporous graphene membranes.pdf:pdf},
isbn = {0743-7463},
issn = {07437463},
journal = {Langmuir},
number = {2},
pages = {675--682},
pmid = {24364726},
title = {{Mechanisms of molecular permeation through nanoporous graphene membranes}},
volume = {30},
year = {2014}
}
@article{Song2015,
abstract = {BaCe0.85Tb0.05Co0.10O3-?? (BCTCo) perovskite hollow fibre membranes were fabricated by a combined phase-inversion and sintering technique. The hollow fibre surfaces were modified by coating Ni or Pd particles. Hydrogen permeation fluxes at 700-1000??C can be improved due to the surface modification from the original 0.009-0.164mL(STP) cm-2 min-1 to 0.018-0.269mLcm-2 min-1 for the Ni-coated membranes with maximum improvement by 64{\%}, and to 0.1-0.42mLcm-2 min-1 for the Pd-loaded membranes with maximum enhancement by 155{\%}, respectively. Loading of the catalyst on the hollow fibre outer surface is better than on the inner surface, but coating on both sides may enhance the hydrogen permeation most effectively. The permeation enhancement depends on both the catalyst loading amount and its structure, which can be controlled by the plating conditions. The optimal Pd loading and coverage should be around 0.667mgcm-2 and 82{\%}, respectively for maximizing the permeation improvement.},
author = {Song, Jian and Kang, Jian and Tan, Xiaoyao and Meng, Bo and Liu, Shaomin},
doi = {10.1016/j.jeurceramsoc.2016.01.006},
file = {:Users/marc/Library/Application Support/Mendeley Desktop/Downloaded/Song et al. - 2015 - Proton conducting perovskite hollow fibre membranes with surface catalytic modification for enhanced hydrogen separ.pdf:pdf},
issn = {1873619X},
journal = {Journal of the European Ceramic Society},
keywords = {Hollow fibre,Hydrogen separation,Perovskite membrane,Surface modification},
number = {7},
pages = {1669--1677},
publisher = {Elsevier Ltd},
title = {{Proton conducting perovskite hollow fibre membranes with surface catalytic modification for enhanced hydrogen separation}},
url = {http://dx.doi.org/10.1016/j.jeurceramsoc.2016.01.006},
volume = {36},
year = {2015}
}
@article{Gorgojo2008,
author = {Gorgojo, Patricia and de la Iglesia, {\'{O}}scar and Coronas, Joaqu{\'{i}}n},
doi = {10.1016/s0927-5193(07)13005-9},
isbn = {09275193},
pages = {135--175},
title = {{Preparation and Characterization of Zeolite Membranes}},
volume = {13},
year = {2008}
}
@article{Zhao1998a,
abstract = {A thin palladium composite membrane was produced by modified electroless plating procedure. Compared with the conventional electroless plating procedure, the modified electroless plating procedure consists of the activation of a ceramic substrate by the sol-gel process of a Pd(II)-modified boehmite sol. Additionally, the infiltration of an electroless plating solution to a porous substrate during the deposition of palladium was employed with the filter device to improve adherence of a palladium layer to a substrate. The resulting membrane with a thickness of about 1 ??m has a high compactness. The membrane shows a hydrogen selectivity of 20-130 for H2/N2, and a hydrogen flux of 1.8-87 m3/m2.h, depending on operation conditions.},
author = {Zhao, H. B. and Pflanz, K. and Gu, J. H. and Li, A. W. and Stroh, N. and Brunner, H. and Xiong, G. X.},
doi = {10.1016/S0376-7388(97)00287-1},
file = {:Users/marc/Library/Application Support/Mendeley Desktop/Downloaded/Zhao et al. - 1998 - Preparation of palladium composite membranes by modified electroless plating procedure.pdf:pdf},
issn = {03767388},
journal = {Journal of Membrane Science},
keywords = {Electroless plating,Hydrogen separation,Pd membrane,Sol-gel process},
number = {2},
pages = {147--157},
title = {{Preparation of palladium composite membranes by modified electroless plating procedure}},
volume = {142},
year = {1998}
}
@article{Yudanov2003,
author = {Yudanov, I V and Sahnoun, R and Neyman, K M and R{\"{o}}sch, N and Hoffmann, J and Schauermann, S and Joh{\'{a}}nek, V and Unterhalt, H and Rupprechter, G and Libuda, J and Freund, H},
file = {:Users/marc/Library/Application Support/Mendeley Desktop/Downloaded/Yudanov et al. - 2003 - CO adsorption on Pd nanoparticles density functional and vibrational spectroscopy studies.pdf:pdf},
journal = {The Journal of Physical Chemistry B},
number = {1},
pages = {255--264},
title = {{CO adsorption on Pd nanoparticles: density functional and vibrational spectroscopy studies}},
url = {papers://8e66b560-2a31-474a-b520-48523ed3116f/Paper/p5558},
volume = {107},
year = {2003}
}
@article{Battersby2009,
abstract = {Cobalt silica membranes were fabricated using sol-gel techniques for separation of H2in a membrane reactor set up for the low temperature (up to 300 °C) water gas shift (WGS) reaction. Single dry gas testing prior to reaction showed He/N2and H2/CO2selectivities increasing from 75-400 to 45-160 as the temperature increased from 100 to 250 °C, respectively. During reaction the membrane delivered a H2permeation purity of 89-95{\%} at high conversions, with the higher water ratio conversion providing superior membrane operational performance. Characterisation of bulk gels indicated that the cobalt silica was hydrophilic and exposure to steam at 200 °C resulted in the densification of the film matrix. The cobalt doping allowed for the membrane structural microporosity to be maintained as H2selectivity was not affected by steam exposure, though the flux decreased due to pore collapse of the film matrix. A total of 8 thermal cycle testing were carried out from room temperature to 300 °C, and the membrane displayed good hydrothermal stability, maintaining a high H2selectivity for over 200 h of operation. {\textcopyright} 2008 Elsevier B.V. All rights reserved.},
author = {Battersby, Scott and Smart, Simon and Ladewig, Bradley and Liu, Shaomin and Duke, Mikel C. and Rudolph, Victor and da Costa, Jo{\~{a}}o C Diniz},
doi = {10.1016/j.seppur.2008.12.020},
file = {:Users/marc/Library/Application Support/Mendeley Desktop/Downloaded/Battersby et al. - 2009 - Hydrothermal stability of cobalt silica membranes in a water gas shift membrane reactor.pdf:pdf},
isbn = {1383-5866},
issn = {13835866},
journal = {Separation and Purification Technology},
keywords = {Cobalt silica,Hydrogen separation,Hydrothermal stability,Membrane reactor,Water gas shift},
number = {2},
pages = {299--305},
title = {{Hydrothermal stability of cobalt silica membranes in a water gas shift membrane reactor}},
volume = {66},
year = {2009}
}
@article{Ronsch2016,
abstract = {Methane production from syngas goes back to more than 100 years of research and process development. Early developments (1970-1980) using syngas from coal gasification plants primarily focused on fixed-bed and fluidized-bed methanation technologies. Temperature control and catalyst deactivation, e.g. caused by fouling and mechanical stress, were key issues of investigation. Due to the debate about a sustainable energy supply, research on methanation has been intensified during the last ten years. Novel reactor developments comprise e.g. micro reactors and three-phase reactors aiming at an advanced temperature control and a reduced complexity of future methanation plants. The developments are supported by detailed modeling and simulation work to optimize the design and dynamic behavior. To accompany and facilitate new methanation developments, the present work is aimed at giving researchers a comprehensive overview of methanation research conducted during the last century. On one hand, application-orientated research focusing on reactor developments, reactor modeling, and pilot plant investigation is reviewed. On the other hand, fundamentals such as reaction mechanisms and catalyst deactivation are presented.},
archivePrefix = {arXiv},
arxivId = {NIHMS150003},
author = {R{\"{o}}nsch, Stefan and Schneider, Jens and Matthischke, Steffi and Schl{\"{u}}ter, Michael and G{\"{o}}tz, Manuel and Lefebvre, Jonathan and Prabhakaran, Praseeth and Bajohr, Siegfried},
doi = {10.1016/j.fuel.2015.10.111},
eprint = {NIHMS150003},
isbn = {0016-2361},
issn = {00162361},
journal = {Fuel},
keywords = {Biomass-to-gas,Catalytic methanation,Power-to-gas,Reactor modeling,Sabatier reaction,Substitute Natural Gas (SNG)},
pages = {276--296},
pmid = {24335434},
publisher = {Elsevier Ltd},
title = {{Review on methanation - From fundamentals to current projects}},
volume = {166},
year = {2016}
}
@article{Li2008,
author = {Li, Hui and Xu, Hengyong and Li, Wenzhao},
doi = {10.1016/j.memsci.2008.06.053},
isbn = {03767388},
journal = {Journal of Membrane Science},
number = {1-2},
pages = {44--49},
title = {{Study of n value and $\alpha$/$\beta$ palladium hydride phase transition within the ultra-thin palladium composite membrane}},
volume = {324},
year = {2008}
}
@article{Hu2014a,
author = {Hu, Jianchen and Ji, Yanfeng and Shi, Yuanyuan},
file = {:Users/marc/Library/Application Support/Mendeley Desktop/Downloaded/Hu, Ji, Shi - 2014 - A Review on the use of Graphene as a Protective Coating against Corrosion.pdf:pdf},
journal = {Annals of Materials Science and Engineering},
keywords = {corrosion resistance,graphene,local oxidation,stability},
number = {3},
pages = {1--16},
title = {{A Review on the use of Graphene as a Protective Coating against Corrosion}},
volume = {1},
year = {2014}
}
@article{Mintova2001,
abstract = {Colloidal zeolites LTA and BEA sized below 100 nm were synthesized as building blocks for the controlled growth of thin microporous films on piezoelectric sensor devices (quartz crystal microbalance, QCM). The zeolite films were prepared on pre-seeded gold substrates on QCM devices. Initially, a layer of colloidal particles was deposited on the support through chemical bonding with a silane coupling agent, followed by hydrothermal growth. BEA- and LTA-type zeolite films with thicknesses of 250 and 450 nm, respectively, were prepared by optimizing the synthesis conditions. The application of these zeolite films in microsensors for water and organic compounds is presented. The importance of the zeolite structure type with respect to the sensitivity towards different organic and water vapors at various concentrations is discussed. Both zeolites are thermally stable and show reproducible responses during long-term experiments. Based on these results, it was concluded that both zeolite films could be used effectively as humidity sensor materials for water vapor sensing purposes. High sensitivity, good reversibility and long life were demonstrated for this type of zeolite film at low water concentrations. In comparison to LTA, the BEA films show a higher sorption capacity towards water vapor and no rejection of pentane, hexane and cyclohexane, due to the larger pore size of the BEA structure. ?? 2001 Elsevier Science B.V. All rights reserved.},
author = {Mintova, Svetlana and Bein, Thomas},
doi = {10.1016/S1387-1811(01)00443-7},
file = {:Users/marc/Library/Application Support/Mendeley Desktop/Downloaded/Mintova, Bein - 2001 - Nanosized zeolite films for vapor-sensing applications.pdf:pdf},
isbn = {1387-1811},
issn = {13871811},
journal = {Microporous and Mesoporous Materials},
keywords = {Chemical sensors,Nanosized zeolites,Thin films},
number = {2-3},
pages = {159--166},
title = {{Nanosized zeolite films for vapor-sensing applications}},
volume = {50},
year = {2001}
}
@article{Li2015,
author = {Li, Panyuan and Wang, Zhi and Qiao, Zhihua and Liu, Yanni and Cao, Xiaochang and Li, Wen and Wang, Jixiao and Wang, Shichang},
doi = {10.1016/j.memsci.2015.08.010},
file = {:Users/marc/Library/Application Support/Mendeley Desktop/Downloaded/Li et al. - 2015 - Recent developments in membranes for efficient hydrogen purification.pdf:pdf},
isbn = {03767388},
journal = {Journal of Membrane Science},
pages = {130--168},
title = {{Recent developments in membranes for efficient hydrogen purification}},
volume = {495},
year = {2015}
}
@article{Seshimo2009,
author = {Seshimo, Masahiro and Hirai, Takayuki and Rahman, Md Mizanur and Ozawa, Minoru and Sone, Masato and Sakurai, Makoto and Higo, Yakichi and Kameyama, Hideo},
doi = {10.1016/j.memsci.2009.07.007},
isbn = {03767388},
journal = {Journal of Membrane Science},
number = {1-2},
pages = {321--326},
title = {{Functionally graded Pd/{\$}\gamma{\$}-alumina composite membrane fabricated by electroless plating with emulsion of supercritical CO2}},
volume = {342},
year = {2009}
}
@article{Huang2017,
author = {Huang, Da and Li, Xiaolin and Wang, Shuai and He, Guili and Jiang, Wenkai and Hu, Jing and Wang, Yanjie and Hu, Nantao and Zhang, Yafei and Yang, Zhi},
doi = {10.1016/j.snb.2017.05.117},
file = {:Users/marc/Library/Application Support/Mendeley Desktop/Downloaded/Huang et al. - 2017 - Three-dimensional chemically reduced graphene oxide templated by silica spheres for ammonia sensing.pdf:pdf},
issn = {09254005},
journal = {Sensors and Actuators B: Chemical},
pages = {956--964},
publisher = {Elsevier B.V.},
title = {{Three-dimensional chemically reduced graphene oxide templated by silica spheres for ammonia sensing}},
url = {http://linkinghub.elsevier.com/retrieve/pii/S0925400517309401},
volume = {252},
year = {2017}
}
@article{Singh2012,
abstract = {We present a simplistic single step synthesis and a detailed study of the remarkable room temperature gas sensing and photoluminescence (PL) properties of zinc oxide (ZnO) decorated graphene oxide sheets (GrO). Investigation of opto-electronic properties reveal near UV to blue PL and semiconducting behavior of ZnO-GrO sheets. ZnO nano-crystallites serve the dual purpose of acting as a nano-spacer between dried graphene sheets as well as a primary sensing transducer for the gas sensing applications. PL has been used as a tool to study the defects associated with the surface of the nanocrystallite's trap levels and/or acceptor-donor recombinations. Time-resolved PL was used to determine free carrier or exciton lifetimes, a vital parameter related to quality of composite and device performance. Results are presented for the detection of common industrial toxins like CO, NH3 and NO for concentrations as low as 1 ppm at room temperature. A large sensor response and quick recovery time was observed at room temperature with preferred selectivity towards electron donor gases like CO and NH3. {\textcopyright} 2011 Elsevier Ltd. All rights reserved.},
author = {Singh, Gaurav and Choudhary, Anshul and Haranath, D. and Joshi, Amish G. and Singh, Nahar and Singh, Sukhvir and Pasricha, Renu},
doi = {10.1016/j.carbon.2011.08.050},
file = {:Users/marc/Library/Application Support/Mendeley Desktop/Downloaded/Singh et al. - 2012 - ZnO decorated luminescent graphene as a potential gas sensor at room temperature.pdf:pdf},
isbn = {0008-6223},
issn = {00086223},
journal = {Carbon},
number = {2},
pages = {385--394},
publisher = {Elsevier Ltd},
title = {{ZnO decorated luminescent graphene as a potential gas sensor at room temperature}},
url = {http://dx.doi.org/10.1016/j.carbon.2011.08.050},
volume = {50},
year = {2012}
}
@article{Straczewski2014,
author = {Straczewski, Grazyna and V{\"{o}}ller-Blumenroth, Johannes and Beyer, Hubert and Pfeifer, Peter and Steffen, Michael and Felden, Ingmar and Heinzel, Angelika and Wessling, Matthias and Dittmeyer, Roland},
doi = {10.1016/j.cep.2014.04.002},
isbn = {02552701},
journal = {Chemical Engineering and Processing: Process Intensification},
pages = {13--23},
title = {{Development of thin palladium membranes supported on large porous 310L tubes for a steam reformer operated with gas-to-liquid fuel}},
volume = {81},
year = {2014}
}
@article{Kingsbury2009a,
abstract = {Morphologies of ceramic hollow fibre membranes prepared from suspensions of Al2O3, NMP (N-methyl-2-pyrrolidone) and polyethersulfone (PESf) using a dry–wet spinning/sintering process have been studied experimentally. The results indicate that two types of membrane morphologies, i.e. finger-like and sponge-like structures can be expected. It is believed that finger-like void formation in asymmetric ceramic membranes is initiated by hydrodynamically unstable viscous fingering developed when a less viscous fluid (non-solvent) is in contact with a higher viscosity fluid (ceramic suspension containing invertible polymer binder). Finger-like void growth occurs only below a critical suspension viscosity, above which a sponge-like structure is observed over the entire hollow fibre cross-section. The effects of the air-gap, viscosity and non-solvent concentration on fibre morphology have been studied and it has been determined that viscosity is the dominating factor for ceramic systems.},
author = {Kingsbury, Benjamin F K and Li, K},
doi = {https://doi.org/10.1016/j.memsci.2008.11.050},
issn = {0376-7388},
journal = {Journal of Membrane Science},
keywords = {Alumina,Asymmetric membrane,Hollow fibres,Morphology},
number = {1},
pages = {134--140},
title = {{A morphological study of ceramic hollow fibre membranes}},
url = {http://www.sciencedirect.com/science/article/pii/S0376738808010235},
volume = {328},
year = {2009}
}
@misc{H2USALocationRoadmapWorkingGroup2017,
author = {{H2USA Location Roadmap Working Group}},
booktitle = {National Hydrogen Scenarios},
title = {{How many stations, where, when?}},
year = {2017}
}
@article{Tight2012,
abstract = {石墨烯 只 透水},
archivePrefix = {arXiv},
arxivId = {1112.3488},
author = {Tight, Through Helium-leak},
doi = {10.1126/science.1211694},
eprint = {1112.3488},
file = {:Users/marc/Library/Application Support/Mendeley Desktop/Downloaded/Tight - 2012 - Unimpeded Permeation of Water.pdf:pdf},
isbn = {1095-9203 (Electronic)$\backslash$r0036-8075 (Linking)},
issn = {0036-8075},
journal = {Science},
number = {January},
pages = {442--444},
pmid = {22282806},
title = {{Unimpeded Permeation of Water}},
volume = {335},
year = {2012}
}
@article{Ibarra2013,
abstract = {PCM-15 is a robust and recyclable sensor for the effective discrimination of a wide range of small molecules. Sensing is achieved by direct attenuation of the luminescence intensity of Tb(III) ions within the material. A competition study involving trace amounts of NH3 in H2 gas shows that PCM-15 can be used to quantitatively detect trace analytes.},
author = {Ibarra, Ilich A. and Hesterberg, Travis W. and Chang, Jong-San and Yoon, Ji Woong and Holliday, Bradley J. and Humphrey, Simon M.},
doi = {10.1039/c3cc44575e},
file = {:Users/marc/Library/Application Support/Mendeley Desktop/Downloaded/Ibarra et al. - 2013 - Molecular sensing and discrimination by a luminescent terbium–phosphine oxide coordination material.pdf:pdf},
isbn = {1364-548X (Electronic)$\backslash$n1359-7345 (Linking)},
issn = {1359-7345},
journal = {Chemical Communications},
number = {64},
pages = {7156},
pmid = {23835607},
title = {{Molecular sensing and discrimination by a luminescent terbium–phosphine oxide coordination material}},
url = {http://xlink.rsc.org/?DOI=c3cc44575e},
volume = {49},
year = {2013}
}
@article{Yuan2010,
abstract = {Dense ceramic membranes with protonic and electronic conductivity have attracted considerable interest in recent years. In this paper, the powders of SrCe0.75Zr0.20Tm0.05O3-?? were synthesized via the liquid citrate method, and the membranes of SrCe0.75Zr0.20Tm0.05O3-?? were prepared by pressing followed by sintering. X-ray diffraction (XRD) was used to characterize the phase structure of both the powder and sintered membrane. The microstructure of the sintered membranes was studied by scanning electron microscopy (SEM). Hydrogen permeation through the SrCe0.75Zr0.20Tm0.05O3-?? membranes was carried out using gas permeation setup at 900 ??C. Hydrogen permeation flux of SrCe0.75Zr0.20Tm0.05O3-?? membrane reaches up to 0.042 mL/min cm2 at H2 partial pressure of 0.4 atm. The hydrogen permeation fluxes obtained in this paper are similar to that of SrCe0.95Tm0.05O3-??, and Zr doping can increase mechanical strength of SrCe0.75Zr0.20Tm0.05O3-?? membranes and the resistance to reducing circumstance. ?? 2009 Wen Hui Yuan.},
author = {Yuan, Wen Hui and Mao, Ling Ling and Li, Li},
doi = {10.1016/j.cclet.2009.11.002},
file = {:Users/marc/Library/Application Support/Mendeley Desktop/Downloaded/Yuan, Mao, Li - 2010 - Novel SrCe0.75Zr0.20Tm0.05O3-alpha membrane for hydrogen separation.pdf:pdf},
issn = {10018417},
journal = {Chinese Chemical Letters},
keywords = {Ceramic membrane,Hydrogen permeation,Mixed conductor,Perovskite},
number = {3},
pages = {369--372},
title = {{Novel SrCe0.75Zr0.20Tm0.05O3-alpha membrane for hydrogen separation}},
volume = {21},
year = {2010}
}
@article{Chen2013,
abstract = {A new seeding method, namely, varying-temperature hot-dip coating (VTHDC), is proposed for synthesis of zeolite T membranes by secondary hydrothermal growth. The VTHDC method is composed ofhot-dip coating at higher temperature, rubbing offthe superfluous crystals, and hot-dip coating at lower temperature. It was found that the method was flexible and effective for combined control over the seed suspension concentration, seed size, and coating temperature, leading to combined control of properties of the seed layer over the seed size, thickness, coverage, and defect. A thin continuous, smooth defect-free asymmetric seed layer was achieved consisting of large and small zeolite T seed crystals. The resulting zeolite T membrane M5 exhibited high pervaporation performance with the flux reaching 2.12 and 2.52 kg/m2 h for the dehydration of90 wt {\%} EtOH/H2O and IPA/H2O mixture, respectively, at 348 K. The corresponding separation factor was up to 1301 and 10,000, respectively. VC 2012 American Institute of Chemical Engineers AIChE J, 59: 936–947, 2013 Keywords: zeolite T membrane, varying temperature hot dipping coating, pervaporation dehydration, EtOH/H2O mixture, IPA/H2O mixture Introduction},
author = {Chen, Xiaoxia and Wang, Jinqu and Yin, Dehong and Yang, Jianhua and Lu, Jinming and Zhang, Yan},
doi = {DOI 10.1002/aic.13851},
journal = {SEPARATIONS: MATERIALS, DEVICES, AND PROCESSES High-Performance},
keywords = {Electrostatic droplet actuation,Electrowetting,Lab-on-a-chip,MEMS,Microfluidics,Superhydrophobic surfaces},
pages = {936--947},
title = {{High-Performance Zeolite T Membrane for Dehydration of Organics by a New Varying Temperature Hot-Dip Coating Method}},
volume = {59},
year = {2013}
}
@article{Ockwig2007,
author = {{Nathan W. Ockwig}, Tina M Nenoff},
journal = {Chemical Reviews},
pages = {4078--4110},
title = {{Membranes for Hydrogen Separation}},
volume = {107},
year = {2007}
}
@article{Wang2007,
author = {Wang, Weiping and Pan, Xiulian and Zhang, Xiaoliang and Yang, Weishen and Xiong, Guoxing},
doi = {10.1016/j.seppur.2006.09.016},
isbn = {13835866},
journal = {Separation and Purification Technology},
number = {2},
pages = {262--271},
title = {{The effect of co-existing nitrogen on hydrogen permeation through thin Pd composite membranes}},
volume = {54},
year = {2007}
}
@article{Honda,
author = {Honda},
title = {{Honda Clarity Fuel Cell}},
url = {http://www.honda.co.uk/cars/new/coming-soon/clarity-fuel-cell/overview.html}
}
@article{Zhou2014b,
abstract = {An effective steam-assisted conversion (SAC) seeding method was proposed for the growth of a thin and high-quality SAPO-34 membrane on a low-cost and coarse macroporous $\alpha$-Al2O3tubular support. This seeding technique was composed of depositing the seeds-containing paste on a support and transforming the paste into continuous and compact seed layer by the SAC process. The paste could serve as the binder to prevent small seeds penetrating into the support. With the aid of the perfect seed layer, a high-quality SAPO-34 membrane with the thickness about 4 $\mu$m was synthesized by secondary growth. The as-synthesized membrane exhibited a high H2permeance of 6.96 × 10-6mol m-2s-1Pa-1at room temperature, with H2/CO2, H2/N2, H2/CH4, H2/C2H6, H2/C3H8, H2/n-C4H10and H2/i-C4H10ideal selectivities of 1.83, 7.58, 14.80, 18.24, 26.51, 40.15 and 53.02, respectively. {\textcopyright} 2014, Hydrogen Energy Publications, LLC. Published by Elsevier Ltd. All rights reserved.},
author = {Zhou, Liang and Yang, Jianhua and Li, Gang and Wang, Jinqu and Zhang, Yan and Lu, Jinming and Yin, Dehong},
doi = {10.1016/j.ijhydene.2014.06.159},
file = {:Users/marc/Library/Application Support/Mendeley Desktop/Downloaded/Zhou et al. - 2014 - Highly H2permeable SAPO-34 membranes by steam-assisted conversion seeding.pdf:pdf},
issn = {03603199},
journal = {International Journal of Hydrogen Energy},
keywords = {SAPO-34 zeolite membrane,Steam-assisted conversion},
number = {27},
pages = {14949--14954},
publisher = {Elsevier Ltd},
title = {{Highly H2permeable SAPO-34 membranes by steam-assisted conversion seeding}},
url = {http://dx.doi.org/10.1016/j.ijhydene.2014.06.159},
volume = {39},
year = {2014}
}
@article{Noordermeer1986,
abstract = {Thermal desorption spectroscopy and LEED have been used to investigate the interaction of CO and hydrogen with a Pd0.75Cu0.25(111) single crystal surface with surface composition of about Pd0.7Cu0.3. The main objective was to make a comparison with the previously studied Pd0.67Ag0.33(111) (surface composition Pd0.1Ag0.9) and Pd(111) surfaces. In addition, the effect of preadsorbed H on subsequent CO dosage and the effect of adsorbed CO on postdosed hydrogen are described. Marked differences were found in the adsorption behaviour of the three surfaces towards CO and hydrogen. The maximum amount of H and CO that can be adsorbed at 250 K and pressures below 10-9 mbar is much lower on the PdCu surface than expected on the basis of the surface composition. This effect appears to be caused by a low heat of adsorption of hydrogen and CO and Pd singlet sites. Arguments are presented that singlet Pd sites or isolated Pd atoms in a Cu or Ag matrix are able to trap and dissociate the hydrogen molecule at 250 K. The CO desorption spectra are not influenced by pre- or postexposed hydrogen. Adsorbed CO hampers the uptake of hydrogen upon subsequent exposure to hydrogen. Postdosed CO causes adsorbed H adatoms to move to the bulk (adsorbed H). CO exposure at 250 K results in a very broad desorption plateau between 310 and 425 K with hardly discernable maxima. The results can be explained in terms of the size and relative concentration of the various Pd sites present on the surface (triplet, doublet and singlet sites). It can be concluded that for Pd (111) the heat of adsorption of both CO and H differ appreciably for the triplet, doublet and singlet sites. The effect of site has a larger contribution to the decrease of the heat of adsorption with coverage than the effect of lateral interaction in the adlayer. For Pd(111), PdCu(111) and PdAg(111) the effect of the available Pd sites is the major effect that determines the heat of adsorption, followed by the effect of lateral interaction and for the alloy surfaces the electronic or ligand effect. ?? 1986.},
author = {Noordermeer, A. and Kok, G. A. and Nieuwenhuys, B. E.},
doi = {10.1016/0039-6028(86)90760-0},
file = {:Users/marc/Library/Application Support/Mendeley Desktop/Downloaded/Noordermeer, Kok, Nieuwenhuys - 1986 - Comparison between the adsorption properties of Pd(111) and PdCu(111) surfaces for carbon monoxid.pdf:pdf},
issn = {00396028},
journal = {Surface Science},
number = {2},
pages = {349--362},
title = {{Comparison between the adsorption properties of Pd(111) and PdCu(111) surfaces for carbon monoxide and hydrogen}},
volume = {172},
year = {1986}
}
@article{Ikuhara2007,
abstract = {Nickel (Ni) nanoparticle-dispersed amorphous silica (Si-O) powders were synthesized from chemical solution precursors. The high-temperature hydrogen adsorption property of the precursor-derived composite powders was investigated in comparison with the amorphous Si-O and Ni at 773 K. Among the three powder samples, Ni nanoparticle-dispersed amorphous Si-O exhibited a unique reversible hydrogen adsorption property that was hardly detected on the amorphous Si-O and Ni. The increase amount of the reversibly adsorbed hydrogen was the highest for the composite samples at around the Ni content with a Ni/(Si+Ni) ratio of 0.2-0.3. The results strongly suggested that when the composite material is used in the form of a gas separation membrane, the reversibly adsorbed hydrogen property is thought to contribute to the additional increase in the number of solubility sites for hydrogen, which leads to a selective enhancement in the high-temperature hydrogen permeance at 773 K. {\textcopyright} 2006 The American Ceramic Society.},
author = {Ikuhara, Yumi H. and Mori, Hiroshi and Saito, Tomohiro and Iwamoto, Yuji},
doi = {10.1111/j.1551-2916.2006.01434.x},
file = {:Users/marc/Library/Application Support/Mendeley Desktop/Downloaded/Ikuhara et al. - 2007 - High-temperature hydrogen adsorption properties of precursor-derived nickel nanoparticle-dispersed amorphous sil.pdf:pdf},
issn = {00027820},
journal = {Journal of the American Ceramic Society},
number = {2},
pages = {546--552},
title = {{High-temperature hydrogen adsorption properties of precursor-derived nickel nanoparticle-dispersed amorphous silica}},
volume = {90},
year = {2007}
}
@article{Gao2004,
author = {Gao, Huiyuan and Lin, Y S and Li, Yongdan and Zhang, Baoquan},
file = {:Users/marc/Library/Application Support/Mendeley Desktop/Downloaded/Gao et al. - 2004 - Chemical Stability and Its Improvement of Palladium-Based Metallic Membranes.pdf:pdf},
journal = {Ind. Eng. Chem. Res.},
pages = {6920--6930},
title = {{Chemical Stability and Its Improvement of Palladium-Based Metallic Membranes}},
volume = {43},
year = {2004}
}
@misc{Hawkins2001,
abstract = {Suspension-feeding behaviour in the scallop Chlamys farreri of 72 ± 1 mm shell length was studied in response to wide variations in the amount and composition of suspended seston. Clearance rates (CR; 1 h-1) with which scallops removed all particles larger than 3.9 $\mu$m equivalent spherical diameter varied over an order of magnitude, initially increasing to maxima in separate unimodal relations with different measures of dietary abundance that included total particulate volume (VOL; mm3 1-1), total particulate mass (TPM; mg 1-1), particulate organic mass (POM; mg 1-1), particulate organic carbon (POC; mg 1-1) and chlorophyll a (CHL; $\mu$g 1-1), before declining with further increases in seston concentration. Most variance in CR was associated with VOL (47{\%}), compared with TPM (29{\%}), POM (32{\%}), POC (29{\%}), PON (27{\%}) and CHL (20{\%}). This suggests morphological limits to feeding behaviour, as adjustments in CR were primarily dependent upon total seston volume, rather than any gravimetric measure of abundance. Nevertheless, stepwise regression indicated that VOL and CHL together accounted for as much as 58{\%} of the variance in CR; more than for VOL alone (p {\textless} 0.001), and more than twice that explained by gravimetric measures of TPM, POM or CHL. The associated relation was best described by a combination of two unimodal curves, predicting that CR increased with food availability to a maximum of 7.1 1 h-1 g-1 when VOL was 2.0 mm3 1-1 and CHL was 5.3 $\mu$g 1-1, before declining with further increases in either VOL or CHL. Response curves differed according to particle type. Most evidently, compared with silt-enriched diets, natural seawater contained about the same total chlorophyll 1-1, but with much lower average volume, with the result that maximal CR was significantly higher in natural seawater. These findings establish highly flexible regulation of CR. They also indicate that to understand regulatory responses in feeding to a mixed suspension of different particles, it is necessary to measure the separate significant influences of particle volume and seston composition, when the range of experimental conditions must span the lower extreme of naturally occurring concentrations. We impress how our findings help to reconcile hitherto apparent differences between previous reports of bivalve suspension-feeding behaviour. {\textcopyright} 2001 Elsevier Science B.V.},
author = {Hawkins, A J S and Fang, J G and Pascoe, P L and Zhang, J H and Zhang, X L and Zhu, M Y},
booktitle = {Journal of Experimental Marine Biology and Ecology},
doi = {10.1016/S0022-0981(01)00282-9},
number = {1},
title = {{Modelling short-term responsive adjustments in particle clearance rate among bivalve suspension-feeders: Separate unimodal effects of seston volume and composition in the scallop Chlamys farreri}},
volume = {262},
year = {2001}
}
@techreport{DouglasH.Read2016,
author = {{Douglas H. Read}, Colin H Sillerud},
booktitle = {Sandia Report},
publisher = {Sandia National Laboratories},
title = {{Metal-Organic Framework Thin Films as Stationary Phases in Microfabricated Gas-Chromatography Columns}},
year = {2016}
}
@article{Huang2014b,
abstract = {{\textless}p{\textgreater}Inspired by the bio-adhesive ability of the marine mussel, highly hydrogen permselective ZIF-8 membranes with a “reinforced concrete” structure were prepared on polydopamine functionalized inexpensive macroporous stainless-steel-nets.{\textless}/p{\textgreater}},
author = {Huang, Aisheng and Liu, Qian and Wang, Nanyi and Caro, J{\"{u}}rgen},
doi = {10.1039/C4TA00299G},
file = {:Users/marc/Library/Application Support/Mendeley Desktop/Downloaded/Huang et al. - 2014 - Highly hydrogen permselective ZIF-8 membranes supported on polydopamine functionalized macroporous stainless-steel.pdf:pdf},
isbn = {2050-7488},
issn = {2050-7488},
journal = {J. Mater. Chem. A},
number = {22},
pages = {8246--8251},
title = {{Highly hydrogen permselective ZIF-8 membranes supported on polydopamine functionalized macroporous stainless-steel-nets}},
url = {http://xlink.rsc.org/?DOI=C4TA00299G},
volume = {2},
year = {2014}
}
@article{OBrien2011,
author = {O'Brien, Casey P and Gellman, Andrew J and Morreale, Bryan D and Miller, James B},
doi = {10.1016/j.memsci.2011.01.044},
file = {:Users/marc/Library/Application Support/Mendeley Desktop/Downloaded/O'Brien et al. - 2011 - The hydrogen permeability of Pd4S.pdf:pdf},
isbn = {03767388},
journal = {Journal of Membrane Science},
number = {1-2},
pages = {263--267},
title = {{The hydrogen permeability of Pd4S}},
volume = {371},
year = {2011}
}
@incollection{Gugliuzza2015,
author = {Gugliuzza, A.},
booktitle = {Encyclopedia of Membranes},
editor = {E., Drioli and L., Giorno},
publisher = {Springer, Berlin, Heidelberg},
title = {{Membrane Swelling}},
year = {2015}
}
@article{Kang2013,
author = {Kang, Woo Ram and Lee, Ki Bong},
doi = {10.1007/s11814-013-0047-2},
isbn = {0256-1115
1975-7220},
journal = {Korean Journal of Chemical Engineering},
number = {7},
pages = {1386--1394},
title = {{Effect of operating parameters on methanation reaction for the production of synthetic natural gas}},
volume = {30},
year = {2013}
}
@article{Dunbar2015,
author = {Dunbar, Zachary W},
doi = {10.1016/j.jpowsour.2015.08.015},
file = {:Users/marc/Library/Application Support/Mendeley Desktop/Downloaded/Dunbar - 2015 - Hydrogen purification of synthetic water gas shift gases using microstructured palladium membranes.pdf:pdf},
isbn = {03787753},
journal = {Journal of Power Sources},
pages = {525--533},
title = {{Hydrogen purification of synthetic water gas shift gases using microstructured palladium membranes}},
volume = {297},
year = {2015}
}
@article{Densakulprasert2005,
abstract = {The effects of zeolite content, pore size and ion exchange capacity on electrical conductivity response to carbon monoxide (CO) of polyaniline/zeolite composites were investigated. Zeolite Y, 13X, and synthesized AlMCM41, all having the common cation Cu2+, were dry mixed with synthesized maleic acid (MA) doped polyaniline and compressed to form polyaniline (PANI)/zeolite pellet composites. The Y, 13X and AlMCM41 zeolite have the nominal pore sizes of 7, 10, 36 ??, and the Cu2+ exchange capacities of 0.161, 0.087, and 0.044 mol/g, respectively. With an addition of 13X zeolite to pristine polyaniline, the electrical conductivity sensitivity to CO/N2 gas increases with zeolite content. For the effect of zeolite type, the highest electrical conductivity sensitivity is obtained with the 13X zeolite, followed by the Y zeolite, and the AlMCM41 zeolite, respectively. Poor sensitivity of zeolite AlMCM41 is probably due to its very large pore size and its lowest Cu2+ exchange capacity. Y zeolite and 13X zeolite have comparable pore sizes but the latter has a greater pore free volume and a more favorable location distribution of the Cu2+ ions within the pore. The temporal response time increases with the amount of zeolite in the composites but it is inversely related to the amount of ion exchange capacity. ?? 2004 Published by Elsevier B.V.},
author = {Densakulprasert, Nataporn and Wannatong, Ladawan and Chotpattananont, Datchanee and Hiamtup, Piyanoot and Sirivat, Anuvat and Schwank, Johannes},
doi = {10.1016/j.mseb.2004.12.006},
file = {:Users/marc/Library/Application Support/Mendeley Desktop/Downloaded/Densakulprasert et al. - 2005 - Electrical conductivity of polyanilinezeolite composites and synergetic interaction with CO.pdf:pdf},
isbn = {0921-5107},
issn = {09215107},
journal = {Materials Science and Engineering B: Solid-State Materials for Advanced Technology},
keywords = {CO sensor,Conductive polymer,Polyaniline,Zeolite 13X,Zeolite AlMCM41,Zeolite Y},
number = {3},
pages = {276--282},
title = {{Electrical conductivity of polyaniline/zeolite composites and synergetic interaction with CO}},
volume = {117},
year = {2005}
}
@article{Koros1993,
abstract = {An exhaustive treatment of the past decade of intense activity in the membrane-based gas separation field would require an entire book and provide more detail than necessary to capture the essence of this dynamic field. This review seeks to define the current scientific, technological and commercial boundaries of the field of membrane-based gas separation and to project the position of these boundaries for the immediate future. The most understandable connection between these three areas lies in the material science and engineering achievements and limitations that dominate the application of the technology at the present time. Achievements over the past ten to fifteen have promoted the current strong interest in membrane-based gas separations, and the promise of steady progress toward removing remaining limitations suggests the likelihood of its long term growth as a field. Material selection, membrane formation, and trends in module and system characteristics are discussed in the context of the challenges and opportunities that exist for major commercial applications of the technology.},
author = {Koros, W.J. and Fleming, G.K.},
doi = {10.1016/0376-7388(93)80013-N},
file = {:Users/marc/Library/Application Support/Mendeley Desktop/Downloaded/Koros, Fleming - 1993 - Membrane-based gas separation.pdf:pdf},
isbn = {1022-9760},
issn = {0376-7388},
journal = {Journal of Membrane Science},
number = {1},
pages = {1--80},
title = {{Membrane-based gas separation}},
url = {https://www.sciencedirect.com/science/article/pii/037673889380013N},
volume = {83},
year = {1993}
}
@article{Wang2015b,
abstract = {Shape-controllable porous, hollow metal oxide cages are attracting more$\backslash$nand more attention due to their widespread applications. In this paper,$\backslash$noctahedral, truncated octahedral and cubic Cu2O/CuO cages were$\backslash$nsuccessfully fabricated by the thermal decomposition of the polyhedral$\backslash$ncrystals of Cu-based metal-organic frameworks (Cu-MOFs) as$\backslash$nself-sacrificial templates at 300 degrees C. The morphology of the$\backslash$nCu-MOF polyhedral precursors was well tuned by using lauric acid as the$\backslash$ngrowth modulator under solvothermal conditions. Gas-sensing measurements$\backslash$nrevealed that the octahedral Cu2O/CuO cages exhibited a gas-sensing$\backslash$nperformance far better than those exhibited by truncated octahedral and$\backslash$ncubic cages, which is attributed to the cooperative effect of the large$\backslash$nspecific surface area (150.3 m(2) g(-1)) and high capacity of$\backslash$nsurface-adsorbed oxygen of the octahedral Cu2O/CuO cages.},
author = {Wang, Yiting and Lu, Yinyun and Zhan, Wenwen and Xie, Zhaoxiong and Kuang, Qin and Zheng, Lansun},
doi = {10.1039/c5ta01108f},
file = {:Users/marc/Library/Application Support/Mendeley Desktop/Downloaded/Wang et al. - 2015 - Synthesis of porous Cu2OCuO cages using Cu-based metal-organic frameworks as templates and their gas-sensing proper.pdf:pdf},
isbn = {2050-7488},
issn = {2050-7488},
journal = {Journal of Materials Chemistry a},
number = {24},
pages = {12796--12803},
publisher = {Royal Society of Chemistry},
title = {{Synthesis of porous Cu2O/CuO cages using Cu-based metal-organic frameworks as templates and their gas-sensing properties}},
url = {http://dx.doi.org/10.1039/C5TA01108F},
volume = {3},
year = {2015}
}
@article{Hwang2013,
abstract = {This work reports on zeolitic imidazolate framework (ZIF)-coupled microscale resonators for highly sensitive and selective gas detection. The combination of microscale resonators and nanoscale materials simultaneously permits the benefit of larger capture area for adsorption from the resonator and enhanced surface adsorption capacity from the nanoscale ZIF structure. Dielectrophoresis (DEP) was demonstrated as a novel method for directly assembling concentrated ZIF nanoparticles on targeted regions of silicon resonant sensors. As part of the dielectrophoretic assembly process, the first ever measurements of the Clausius-Mossotti factor for ZIFs were conducted to determine optimal conditions for DEP assembly. The first ever real-time adsorption measurements of ZIFs were also performed to investigate the possibility of inherent gas selectivity. The ZIF-coupled resonators demonstrated sensitivity improvement up to 150 times over a bare silicon resonator with identical dimensions, and real-time adsorption measurements of ZIFs revealed different adsorption time constants for IPA and CO2.},
author = {Hwang, Yongha and Sohn, Hyunmin and Phan, Anh and Yaghi, Omar M. and Candler, Rob N.},
doi = {10.1021/nl4027692},
file = {:Users/marc/Library/Application Support/Mendeley Desktop/Downloaded/Hwang et al. - 2013 - Dielectrophoresis-assembled zeolitic imidazolate framework nanoparticle-coupled resonators for highly sensitive an.pdf:pdf},
isbn = {1530-6984},
issn = {15306984},
journal = {Nano Letters},
keywords = {Gas detectors,decay constant,dielectrophoresis,selectivity,sensitivity,zeolitic imidazolate framework},
number = {11},
pages = {5271--5276},
pmid = {24099583},
title = {{Dielectrophoresis-assembled zeolitic imidazolate framework nanoparticle-coupled resonators for highly sensitive and selective gas detection}},
volume = {13},
year = {2013}
}
@article{Lvov2000,
abstract = {The mechanism of low-temperature migration of analytes onto a palladium modifier and the mechanism of analyte retention on palladium in the pyrolysis stage have been interpreted on the basis of the method of absolute reaction rates and the mechanism of dissociative evaporation of solids. As has been shown previously by the author, the decomposition of solids, in particular, metal nitrates, occurs through the congruent gasification of all reaction products, irrespective of their saturated pressure (with the simultaneous condensation of low-volatility species). In the interval between gasification and condensation, these species could diffuse for some distance from the primary site. An application of the method of absolute reaction rates (the Hertz-Langmuir vaporization models) to the kinetics of analyte release in the presence of a palladium modifier permits the interpretation of the retention mechanism as dissociative chemisorption. The experimental data from the literature (the appearance temperatures and activation energies for Ag, As, Au, Bi, Cd, Cu, Se and Tl) were used in these calculations.},
author = {L'vov, Boris V.},
doi = {10.1007/s10812-014-9865-1},
file = {:Users/marc/Library/Application Support/Mendeley Desktop/Downloaded/L'vov - 2000 - Mechanism of Action of a Palladium Modifier.pdf:pdf},
isbn = {0584-8547},
issn = {00219037},
journal = {Spectrochimica Part B},
keywords = {atomic absorption spectrometry,energy of formation of free atoms,graphite furnace,modeling of an analytical signal,palladium modifier},
pages = {1659--1668},
title = {{Mechanism of Action of a Palladium Modifier}},
volume = {55},
year = {2000}
}
@article{Kanezashi2012,
abstract = {Bis(triethoxysilyl) ethane (BTESE), which consists of Si-C-C-Si bonds, was used as a silica precursor to prepare organic-inorganic hybrid silica membranes with loose amorphous networks. Single-gas permeation and binary-component gas separation characteristics for hybrid silica membranes were examined to discuss the effect of silica precursors on amorphous networks. The pore size distribution, as determined by single-gas permeation, suggested BTESE-derived silica membranes have loose amorphous structures compared to TEOS-derived silica membranes due to the differences in the minimum units of silica networks. For example, BTESE-derived silica membranes showed a high hydrogen permeance (0.2-1 ?? 10-5 mol m-2 s-1 Pa-1) with a high selectivity of H2 to SF6 (H2/SF6 permselectivity: 1000-25,500) and a low H2 to N2 permselectivity (???20). The binary-component gas separation of He and SF6 for a BTESE-derived silica membrane revealed that the swelling effect (adsorption-induced expansion of the zeolite crystals) by SF6 molecules, which has been suggested for zeolite membranes, was not observed in amorphous silica networks. In the present study, BTESE-derived silica membranes had high hydrothermal stability due to the presence of Si-C-C-Si bonds in the amorphous silica network. ?? 2009 Elsevier B.V. All rights reserved.},
author = {Kanezashi, Masakoto and Kawano, Mitsuki and Yoshioka, Tomohisa and Tsuru, Toshinori},
doi = {10.1021/ie201606k},
file = {:Users/marc/Library/Application Support/Mendeley Desktop/Downloaded/Kanezashi et al. - 2012 - Organic-inorganic hybrid silica membranes with controlled silica network size for propylenepropane separation.pdf:pdf},
isbn = {0376-7388},
issn = {08885885},
journal = {Industrial and Engineering Chemistry Research},
number = {2},
pages = {944--953},
title = {{Organic-inorganic hybrid silica membranes with controlled silica network size for propylene/propane separation}},
volume = {51},
year = {2012}
}
@article{Sun2016,
abstract = {Significant achievements have been made on the development of next-generation filtration and separation membranes using graphene materials, as graphene-based membranes can afford numerous novel mass-transport properties that are not possible in state-of-art commercial membranes, making them promising in areas such as membrane separation, water desalination, proton conductors, energy storage and conversion, etc. The latest developments on understanding mass transport through graphene-based membranes, including perfect graphene lattice, nanoporous graphene and graphene oxide membranes are reviewed here in relation to their potential applications. A summary and outlook is further provided on the opportunities and challenges in this arising field. The aspects discussed may enable researchers to better understand the mass-transport mechanism and to optimize the synthesis of graphene-based membranes toward large-scale production for a wide range of applications.},
annote = {Sun, Pengzhan
Wang, Kunlin
Zhu, Hongwei
ENG
Germany
2016/01/23 06:00
Adv Mater. 2016 Mar 23;28(12):2287-310. doi: 10.1002/adma.201502595. Epub 2016 Jan 21.},
author = {Sun, P and Wang, K and Zhu, H},
doi = {10.1002/adma.201502595},
isbn = {1521-4095 (Electronic)
0935-9648 (Linking)},
journal = {Adv Mater},
keywords = {filtration,graphene,mass transport,membranes,separation},
number = {12},
pages = {2287--2310},
pmid = {26797529},
title = {{Recent Developments in Graphene-Based Membranes: Structure, Mass-Transport Mechanism and Potential Applications}},
url = {http://www.ncbi.nlm.nih.gov/pubmed/26797529},
volume = {28},
year = {2016}
}
@article{Tosti2003,
author = {Tosti, S and Basile, A and Chiappetta, G and Rizzello, C and Violante, V},
file = {:Users/marc/Library/Application Support/Mendeley Desktop/Downloaded/Tosti et al. - 2003 - Pd – Ag membrane reactors for water gas shift reaction.pdf:pdf},
journal = {Chemical Engineering Journal},
keywords = {ag membrane reactors,hydrogenation and dehydrogenation processes,pd,permeation,permselectivity,tritiated water,water gas shift reaction},
pages = {23--30},
title = {{Pd – Ag membrane reactors for water gas shift reaction}},
volume = {93},
year = {2003}
}
@phdthesis{Løvvik2003,
abstract = {Palladium–silver alloy surfaces with and without adsorbed hydrogen have been studied through density functional theory within the generalized gradient approximations employing a slab representation of the surface. Our calculated lattice constants are in good agreement with experimental data, but we find a substantially lower surface energy for Ag?111? and Pd?111? than experiments. We have calculated adsorption energies of hydrogen on several sites on various alloy surfaces, and found that threefold hollow sites with as many palladium neighbors as possible are preferred. The difference in adsorption energy is so large that we expect trapping of hydrogen around palladium atoms in the surface, possibly resulting in a lower diffusion constant of hydrogen at low coverage on alloy surfaces than on the pure Pd and Ag surfaces. Assuming that the adsorption energy has contributions from geometric ?‘‘ensemble''? and electronic ?‘‘ligand''? effects, we found the geometric contribution to dominate. For the geometric contribution it is seen that the binding strength increases as the d-band center moves toward the Fermi level, a result also found by a number of other theoretical studies. However, for the electronic contribution we found that the variation of the adsorption energy as a function of the d-band center was opposite that reported by others:We saw that hydrogen binds less strongly to the surface as the d-band center moves toward the Fermi level. This could possibly be explained by a large variation of the interaction between the metal sp band and hydrogen.},
author = {L{\o}vvik, O. M. and {R. A. Olsen}},
booktitle = {JOURNAL OF CHEMICAL PHYSICS},
doi = {10.1063/1.1536955},
file = {:Users/marc/Library/Application Support/Mendeley Desktop/Downloaded/L{\o}vvik, R. A. Olsen - 2003 - Density functional calculations of hydrogen adsorption on palladium–silver alloy surfaces.pdf:pdf},
isbn = {0957-4484},
issn = {0957-4484},
number = {7},
pages = {3268--3276},
title = {{Density functional calculations of hydrogen adsorption on palladium–silver alloy surfaces}},
url = {http://stacks.iop.org/0957-4484/17/i=3/a=027?key=crossref.02539919c9b5a5bc50f7f4abfd29d6eb},
volume = {118},
year = {2003}
}
@article{Fu2017a,
abstract = {Carbon molecular sieve (CMS) membranes were prepared by controlled pyrolysis of polyetherimide (PEI) blended with polyimide (PI). The microstructure and gas separation properties of the precursor membranes and supported CMS membranes were characterized in terms of the different weight ratios of PI to PEI. The thermogravimetric analysis indicated that the co-pyrolysis in the PEI/PI appeared above 450 {\"{i}}¿½C. Both permeability and selectivity of supported CMS membranes increased with increasing PI content in the precursors. The positron annihilation lifetime (PAL) results showed that the free volume size and the pore size were enhanced as the PI content is increased and that there is a close relationship between the microstructures of polymer blend precursors and CMS membranes. These results confirmed that the microstructure and gas separation performance of CMS membranes can be adjusted by blending two thermally stable polymers. For CMS membrane derived from a precursor containing 75 wt{\%} PI and pyrolyzed at 700 {\"{i}}¿½C, the gas permeance for CO2was 40 GPU and CO2/N2selectivity was 39.},
author = {Fu, Ywu Jang and Hu, Chien Chieh and Lin, Di Wei and Tsai, Hui An and Huang, Shu Hsien and Hung, Wei Song and Lee, Kueir Rarn and Lai, Juin Yih},
doi = {10.1016/j.carbon.2016.11.026},
file = {:Users/marc/Library/Application Support/Mendeley Desktop/Downloaded/Fu et al. - 2017 - Adjustable microstructure carbon molecular sieve membranes derived from thermally stable polyetherimidepolyimide blen.pdf:pdf},
issn = {00086223},
journal = {Carbon},
pages = {10--17},
publisher = {Elsevier Ltd},
title = {{Adjustable microstructure carbon molecular sieve membranes derived from thermally stable polyetherimide/polyimide blends for gas separation}},
url = {http://dx.doi.org/10.1016/j.carbon.2016.11.026},
volume = {113},
year = {2017}
}
@article{Lee2009,
author = {Lee, Jeongyong and Farha, Omar K and Roberts, John and Scheidt, Karl A and Nguyen, Sonbinh T and Hupp, Joseph T},
doi = {10.1039/b807080f},
file = {:Users/marc/Library/Application Support/Mendeley Desktop/Downloaded/Lee et al. - 2009 - 2009 Metal – organic frameworks issue Metal – organic framework materials as catalysts w.pdf:pdf},
number = {5},
title = {{2009 Metal – organic frameworks issue Metal – organic framework materials as catalysts w}},
year = {2009}
}
@article{Urbiztondo2011,
abstract = {Inter-digital capacitors (IDCs) with electrode gaps of 10 or 50 microns have been coated with zeolite films consisting of different zeolites with Si/Al ratios ranging from 1.5 (zeolite A) to infinite (silicalite). The performance of the sensor in the measurement of humidity has been related to the electrical properties of the zeolites (relative permittivity, $\epsilon$r), which in turn is a function of their Si/Al ratio. With zeolites of a high Al content the limit of detection was under 0.5 ppmV. {\textcopyright} 2011 Elsevier B.V.},
author = {Urbiztondo, M. and Pellejero, I. and Rodriguez, A. and Pina, M. P. and Santamaria, J.},
doi = {10.1016/j.snb.2011.04.089},
file = {:Users/marc/Library/Application Support/Mendeley Desktop/Downloaded/Urbiztondo et al. - 2011 - Zeolite-coated interdigital capacitors for humidity sensing.pdf:pdf},
issn = {09254005},
journal = {Sensors and Actuators, B: Chemical},
keywords = {Humidity sensors,Interdigital capacitors (IDC),Zeolite},
number = {2},
pages = {450--459},
publisher = {Elsevier B.V.},
title = {{Zeolite-coated interdigital capacitors for humidity sensing}},
url = {http://dx.doi.org/10.1016/j.snb.2011.04.089},
volume = {157},
year = {2011}
}
@article{Murugan2015,
author = {Murugan, Arul and Brown, Andrew S},
doi = {10.1016/j.ijhydene.2015.01.041},
isbn = {03603199},
journal = {International Journal of Hydrogen Energy},
number = {11},
pages = {4219--4233},
title = {{Review of purity analysis methods for performing quality assurance of fuel cell hydrogen}},
volume = {40},
year = {2015}
}
@article{Jeon2011a,
author = {Jeon, S Y and Choi, M B and Park, C N and Wachsman, E D and Song, S J},
doi = {10.1016/j.memsci.2011.08.024},
isbn = {03767388},
journal = {Journal of Membrane Science},
number = {1-2},
pages = {323--327},
title = {{High sulfur tolerance dual-functional cermet hydrogen separation membranes}},
volume = {382},
year = {2011}
}
@article{Marti2013,
abstract = {Oriented thin films of a flexible, luminescent, metal organic framework (MOF), [Zn2(bpdc)2(bpee){\textperiodcentered}2DMF] (bpdc = 4,4'-biphenyldicarboxylate; bpee = 1,2-bipyridylethylene), also known as RPM3, were prepared on glass via pulsed laser deposition (PLD) followed by a solvothermal treatment. RPM3 thin films were then functionalized with AgNO3 by forming a $\pi$ complex with the bpee linker. The reversible binding of olefins such as propylene and 1-hexene to the Ag(+) functionalized RPM3 thin film was monitored by fluorescence spectroscopy. Adsorption of the olefins resulted in a fluorescence enhancement, while the corresponding paraffins either did not change or partially quenched the fluorescence. The RPM3 thin films hold promise as olefin sensors or adsorbents for olefin/paraffin separations.},
author = {Marti, Anne M. and Perera, Sanjaya D. and McBeath, Layne D. and Balkus, Kenneth J.},
doi = {10.1021/la400508y},
file = {:Users/marc/Library/Application Support/Mendeley Desktop/Downloaded/Marti et al. - 2013 - Fabrication of oriented silver-functionalized RPM3 films for the selective detection of olefins.pdf:pdf},
isbn = {0743-7463},
issn = {07437463},
journal = {Langmuir},
number = {19},
pages = {5927--5936},
pmid = {23594169},
title = {{Fabrication of oriented silver-functionalized RPM3 films for the selective detection of olefins}},
volume = {29},
year = {2013}
}
@article{Ho2012,
abstract = {Four iridium(III)-containing coordination polymers 1–4 using Ir(ppy)2(H2dcbpy)PF6 (L-H2, ppy = 2-phenylpyridine, H2dcbpy = 4,4′-dicarboxy-2,2′-bipyridine) as the bridging ligand, [ZnL2]{\textperiodcentered}3DMF{\textperiodcentered}5H2O (1), [CdL2(H2O)2]{\textperiodcentered}3DMF{\textperiodcentered}6H2O (2), [CoL2(H2O)2]{\textperiodcentered}2DMF{\textperiodcentered}8H2O (3) and [NiL2(H2O)2]{\textperiodcentered}3DMF{\textperiodcentered}6H2O (4), have been synthesized and structurally characterized. The emissions from 1–4 are ascribed to a metal-to-ligand charge transfer transition (MLCT). The absolute emission quantum yields for 1–4 in single crystals were measured in air to be 0.274, 0.193, 0.001 and 0.002, respectively. The noteworthy oxygen-sensing properties of 1–4 as well as L-H2 in a single crystal were also evaluated. The Stern–Volmer quenching constant, KSV values, of 1–4 and L-H2 can be deduced to be 0.834, 2.820, 1.328, 1.111 and 2.476, respectively. The results show promising KSV values (e.g.2) that are competitive or even larger than those of many known Ir-complexes. Moreover, the short response time (e.g. compound 2) and recovery times toward oxygen of 1–4 have been measured in their single crystal forms. The reversibility experiments for 1–4 were carried out for seven repeated cycles. As a result, {\textgreater}75{\%} recovery of intensity for 1 and 2 on each cycle demonstrates a high degree of reproducibility during the sensing process. It should be noted that iridium(III)-containing coordination polymers with high emission intensity and notable oxygen sensing properties are obscure, especially in the single crystal form. This, in combination with its fine reversibility, leads to success in single crystal oxygen recognition based on photoluminescence imaging. The detection limit could be 0.50{\%} for gaseous oxygen. Moreover, the temperature effect of compound 2 in a single crystal upon application as an oxygen sensor was expected.},
author = {Ho, Mei-Lin and Chen, Yi-An and Chen, Tsai-Chen and Chang, Pei-Jen and Yu, Yi-Ping and Cheng, Kum-Yi and Shih, Chien-Hung and Lee, Gene-Hsiang and Sheu, Hwo-Shuenn},
doi = {10.1039/c2dt11473a},
file = {:Users/marc/Library/Application Support/Mendeley Desktop/Downloaded/Ho et al. - 2012 - Synthesis, structure and oxygen-sensing properties of Iridium(iii)-containing coordination polymers with different ca.pdf:pdf},
isbn = {1477-9234 (Electronic)$\backslash$r1477-9226 (Linking)},
issn = {1477-9226},
journal = {Dalton Transactions},
number = {9},
pages = {2592},
pmid = {22222947},
title = {{Synthesis, structure and oxygen-sensing properties of Iridium(iii)-containing coordination polymers with different cations}},
url = {http://xlink.rsc.org/?DOI=c2dt11473a},
volume = {41},
year = {2012}
}
@article{Pomerantz2011,
author = {Pomerantz, Natalie and Ma, Yi Hua},
doi = {10.1016/j.memsci.2010.12.045},
file = {:Users/marc/Library/Application Support/Mendeley Desktop/Downloaded/Pomerantz, Ma - 2011 - Novel method for producing high H2 permeability Pd membranes with a thin layer of the sulfur tolerant PdCu fcc ph.pdf:pdf},
isbn = {03767388},
journal = {Journal of Membrane Science},
number = {1-2},
pages = {97--108},
title = {{Novel method for producing high H2 permeability Pd membranes with a thin layer of the sulfur tolerant Pd/Cu fcc phase}},
volume = {370},
year = {2011}
}
@article{Huang2011,
abstract = {An oriented, neutral and cation-free AlPO(4) LTA molecular sieve membrane with high hydrogen selectivity was prepared on porous $\alpha$-Al(2)O(3) supports through secondary growth of a highly oriented AlPO(4) LTA monolayer.},
author = {Huang, Aisheng and Caro, J{\"{u}}rgen},
doi = {10.1039/c1cc00029b},
file = {:Users/marc/Library/Application Support/Mendeley Desktop/Downloaded/Huang, Caro - 2011 - Highly oriented, neutral and cation-free AlPO4 LTA from a seed crystal monolayer to a molecular sieve membrane.pdf:pdf},
issn = {1359-7345},
journal = {Chemical Communications},
number = {14},
pages = {4201},
pmid = {21365116},
title = {{Highly oriented, neutral and cation-free AlPO4 LTA: from a seed crystal monolayer to a molecular sieve membrane}},
url = {http://xlink.rsc.org/?DOI=c1cc00029b},
volume = {47},
year = {2011}
}
@phdthesis{Gil2015,
address = {London},
author = {Gil, Ana Maria Gouveia},
booktitle = {Department of Chemical Engineering},
publisher = {Imperial College London},
title = {{Catalytic Hollow Fibre Membrane Reactors for H2 Production}},
volume = {PhD in Che},
year = {2015}
}
@article{Zuo2006,
abstract = {Composite membranes consisting of Ni metal and Ba(Zr0.1Ce0.7Y0.2)O3 (or Ni-BZCY7) have been developed for separation of hydrogen from gas mixtures to replace Ni-BCY20 (Ni-BaCe0.8Y0.2O3), which has poor stability in CO2 and H2O-containing atmosphere. Hydrogen fluxes through these cermet membranes were measured as a function of temperature, membrane thickness, and partial pressure of hydrogen in various atmospheres. Results indicated that the Ni-BZCY7 membrane is chemically stable and display high hydrogen permeability. A maximum flux of 0.805 cm3 min-1 cm2 was obtained for a dense cermet membrane of 266-??m-thick at 900 ??C using 100{\%} H2 as the feed gas and 100 ppm H2/N2 as the sweep gas. The stable performance of Ni-BZCY7 cermet membrane during exposure to a wet gas containing 30{\%} CO2 for about 80 h indicated that it is promising for practical applications. ?? 2006 Elsevier B.V. All rights reserved.},
author = {Zuo, Chendong and Lee, T. H. and Dorris, S. E. and Balachandran, U. and Liu, Meilin},
doi = {10.1016/j.jpowsour.2005.12.042},
file = {:Users/marc/Library/Application Support/Mendeley Desktop/Downloaded/Zuo et al. - 2006 - Composite Ni-Ba(Zr0.1Ce0.7Y0.2)O3 membrane for hydrogen separation.pdf:pdf},
isbn = {03787753},
issn = {03787753},
journal = {Journal of Power Sources},
keywords = {BaZr0.1Ce0.7Y0.2O3,Hydrogen separation,Mixed conductors,Permeation,Proton conductor,Stability},
number = {2},
pages = {1291--1295},
title = {{Composite Ni-Ba(Zr0.1Ce0.7Y0.2)O3 membrane for hydrogen separation}},
volume = {159},
year = {2006}
}
@article{Morreale2004,
abstract = {The hydrogen permeance of several 0.1 mm thick Pd-Cu alloy foils (80 wt.{\%} Pd-20 wt.{\%} Cu, 60 wt.{\%} Pd-40 wt.{\%} Cu and 53 wt.{\%} Pd-47 wt.{\%} Cu) was evaluated using transient flux measurements at temperatures ranging from 603 to 1123 K and pressures up to 620 kPa both in the presence and absence of 1000 ppm H 2S. Sulfur resistance, as evidenced by no significant change in permeance, was correlated with the temperatures associated with the face-centered-cubic crystalline structure for the alloys in this study. The permeance of the body-centered cubic phase, however, was up to two orders of magnitude lower when exposed to H2S. A smooth transition from sulfur poisoning to sulfur resistance with increasing temperature was correlated with the alloy transition from a body-centered-cubic structure to a face-centered-cubic structure. ?? 2004 Elsevier B.V. All rights reserved.},
author = {Morreale, B. D. and Ciocco, M. V. and Howard, B. H. and Killmeyer, R. P. and Cugini, A. V. and Enick, R. M.},
doi = {10.1016/j.memsci.2004.04.033},
file = {:Users/marc/Library/Application Support/Mendeley Desktop/Downloaded/Morreale et al. - 2004 - Effect of hydrogen-sulfide on the hydrogen permeance of palladium-copper alloys at elevated temperatures.pdf:pdf},
isbn = {0376-7388},
issn = {03767388},
journal = {Journal of Membrane Science},
keywords = {Gas separations,Hydrogen,Hydrogen-sulfide,Metal membranes,Palladium-copper},
number = {2},
pages = {219--224},
title = {{Effect of hydrogen-sulfide on the hydrogen permeance of palladium-copper alloys at elevated temperatures}},
volume = {241},
year = {2004}
}
@article{Peng2011,
abstract = {In this article, we report a novel fluorescent ammonia gas probe based on microstructured optical fiber (MOF) which is modified with eosin-doped cellulose acetate film. This probe was fabricated by liquid fluxion coating process. Polymer solution doped with eosin was directly inhaled into 18 array holes of MOF and then formed matrix film in them. The sensing properties of the optical fiber sensor to gaseous ammonia at room temperature were investigated. The sensing probe showed different fluorescence intensity at 576 nm to different concentrations of trace ammonia in carrier gas of nitrogen. The response range was 50-400 ppm, with short response time within 500 ms. Furthermore, the response range could be tailored through CTAB co-entrapment process in the sensing film. These test results demonstrated that low cost, simple structured fiber optic sensors for detecting ammonia gas samples could be developed based on MOF. {\textcopyright} 2011 Elsevier B.V.},
author = {Peng, Lirong and Yang, Xinghua and Yuan, Libo and Wang, Lili and Zhao, Enming and Tian, Fengjun and Liu, Yanxin},
doi = {10.1016/j.optcom.2011.06.015},
file = {:Users/marc/Library/Application Support/Mendeley Desktop/Downloaded/Peng et al. - 2011 - Gaseous ammonia fluorescence probe based on cellulose acetate modified microstructured optical fiber.pdf:pdf},
isbn = {0030-4018},
issn = {00304018},
journal = {Optics Communications},
keywords = {Ammonia,Fluorescence,Microstructured optical fiber,Optical sensing},
number = {19},
pages = {4810--4814},
publisher = {Elsevier B.V.},
title = {{Gaseous ammonia fluorescence probe based on cellulose acetate modified microstructured optical fiber}},
url = {http://dx.doi.org/10.1016/j.optcom.2011.06.015},
volume = {284},
year = {2011}
}
@article{Balachandran2006,
author = {Balachandran, U and Lee, T and Chen, L and Song, S and Picciolo, J and Dorris, S},
doi = {10.1016/j.fuel.2005.05.027},
isbn = {00162361},
journal = {Fuel},
number = {2},
pages = {150--155},
title = {{Hydrogen separation by dense cermet membranes}},
volume = {85},
year = {2006}
}
@article{Wilcox2010,
author = {Wilcox, Ekin Ozdogan and Jennifer},
doi = {10.1021/acs.energyfuels.5b01294},
file = {:Users/marc/Library/Application Support/Mendeley Desktop/Downloaded/Wilcox - 2010 - Investigation of H2 and H2S Adsorption on Niobium- and Copper-Doped Palladium Surfaces.pdf:pdf},
journal = {J. Phys. Cherm. B},
pages = {12851--12585},
title = {{Investigation of H2 and H2S Adsorption on Niobium- and Copper-Doped Palladium Surfaces}},
year = {2010}
}
@article{Li2008,
author = {Li, Hui and Xu, Hengyong and Li, Wenzhao},
doi = {10.1016/j.memsci.2008.06.053},
file = {:Users/marc/Library/Application Support/Mendeley Desktop/Downloaded/Li, Xu, Li - 2008 - Study of n value and $\alpha$$\beta$ palladium hydride phase transition within the ultra-thin palladium composite membrane.pdf:pdf},
isbn = {03767388},
journal = {Journal of Membrane Science},
number = {1-2},
pages = {44--49},
title = {{Study of n value and $\alpha$/$\beta$ palladium hydride phase transition within the ultra-thin palladium composite membrane}},
volume = {324},
year = {2008}
}
@article{Sasaki2002,
abstract = {Three different zeolites, which are zeolite A, silicalite-1 and sodalite, were deposited on quartz crystal microbalance (QCM) oscillators with fundamental resonance frequency of 4.7 MHz. Three QCMs, as well as one without zeolites as a reference, were exposed to single gases such as NO, SO2 and H2O in He at 443 K. The frequency shifts and their differential values to time were measured, and both of them are tested to elucidate which is more useful to quanlify and quatify these single gases using principal component analysis (PCA). Based on these results, mixed gases of NO and SO2 in several compositions in He are examined. Mixed-gas data are also qualified and quantified using PCA. These results suggest that the sensor system with the differential values which represent sorption-kinetics, instead of the frequency shifts representing sorption-equilibria, overcame the disadvantages of the conventional QCM sensors and showed favorable qualification and quantification performances for the gas mixture.},
author = {Sasaki, Isao and Tsuchiya, Hiroshi and Nishioka, Masateru and Sadakata, Masayoshi and Okubo, Tatsuya},
doi = {10.1016/S0925-4005(02)00132-6},
file = {:Users/marc/Library/Application Support/Mendeley Desktop/Downloaded/Sasaki et al. - 2002 - Gas sensing with zeolite-coated quartz crystal microbalances—principal component analysis approach.pdf:pdf},
issn = {09254005},
journal = {Sensors and Actuators B: Chemical},
keywords = {differential,gas sensor,principal component analysis,quartz crystal microbalance,zeolite},
number = {1},
pages = {26--33},
title = {{Gas sensing with zeolite-coated quartz crystal microbalances—principal component analysis approach}},
volume = {86},
year = {2002}
}
@article{Lu2010,
author = {Lu, Guang and Hupp, Joseph T},
doi = {10.1021/ja101415b},
file = {:Users/marc/Library/Application Support/Mendeley Desktop/Downloaded/Lu, Hupp - 2010 - Metal Organic Franeworks as Sensors A ZIF-8 Based Fabry Perot Device as a Selective Sensor for Chemical Vapours and Ga.pdf:pdf},
issn = {0002-7863},
journal = {Journal of the American Chemical Society},
number = {23},
pages = {7832--7833},
title = {{Metal Organic Franeworks as Sensors A ZIF-8 Based Fabry Perot Device as a Selective Sensor for Chemical Vapours and Gases}},
volume = {132},
year = {2010}
}
@misc{Peters2017,
author = {Peters, T.A. and R{\o}rvik, P.M. and Sunde, T.O. and Stange, M. and Ulla, S.R. and Roness, F. and Reinertsen, T.R. and R{\ae}der, J.H. and Larring, Y. and Bredesen, R.},
title = {{Palladium membranes as key-enabling technology for H2 production with CO2 capture - from innovation to industrial application}},
year = {2017}
}
@article{ArratibelPlazaola2017,
abstract = {Palladium-based membranes for hydrogen separation have been studied by several research groups during the last 40 years. Much effort has been dedicated to improving the hydrogen flux of these membranes employing different alloys, supports, deposition/production techniques, etc. High flux and cheap membranes, yet stable at different operating conditions are required for their exploitation at industrial scale. The integration of membranes in multifunctional reactors (membrane reactors) poses additional demands on the membranes as interactions at different levels between the catalyst and the membrane surface can occur. Particularly, when employing the membranes in fluidized bed reactors, the selective layer should be resistant to or protected against erosion. In this review we will also describe a novel kind of membranes, the pore-filled type membranes prepared by Pacheco Tanaka and coworkers that represent a possible solution to integrate thin selective membranes into membrane reactors while protecting the selective layer. This work is focused on recent advances on metallic supports, materials used as an intermetallic diffusion layer when metallic supports are used and the most recent advances on Pd-based composite membranes. Particular attention is paid to improvements on sulfur resistance of Pd based membranes, resistance to hydrogen embrittlement and stability at high temperature.},
author = {{Arratibel Plazaola}, Alba and {Pacheco Tanaka}, David and {Van Sint Annaland}, Martin and Gallucci, Fausto},
doi = {10.3390/molecules22010051},
file = {:Users/marc/Library/Application Support/Mendeley Desktop/Downloaded/Alba Arratibel Plazaola, David Alfredo Pacheco Tanaka - 2017 - Recent Advances in Pd-Based Membranes for Membrane Reactors.pdf:pdf},
issn = {1420-3049},
journal = {Molecules},
keywords = {inorganic membranes,membrane reactor,membranes,palladium membranes,pore-filled},
number = {1},
pages = {51},
pmid = {28045434},
title = {{Recent Advances in Pd-Based Membranes for Membrane Reactors}},
url = {http://www.mdpi.com/1420-3049/22/1/51},
volume = {22},
year = {2017}
}
@article{Wang2012,
author = {Wang, Haibo and Maiyalagan, Thandavarayan and Wang, Xin},
doi = {10.1021/cs200652y},
isbn = {2155-5435
2155-5435},
journal = {ACS Catalysis},
number = {5},
pages = {781--794},
title = {{Review on Recent Progress in Nitrogen-Doped Graphene: Synthesis, Characterization, and Its Potential Applications}},
volume = {2},
year = {2012}
}
@article{Ali1994,
abstract = {Experiments on commercially available palladium-silver (PdAg) tubular membranes were carried out to: (1) verify the validity of both Sievert's and Fick's laws for H2 permeation through the membrane; (2) quantify the deactivating effects of methylcyclohexane (MCH), toluene (TOL), sulphur (S), and chlorine (Cl) on membrane permeability; and (3) develop regeneration procedures to restore the original H2 permeation rates. The results showed that the H2 permeability data were represented by Sievert's and Fick's laws. Initial exposure of the fresh membrane to MCH or TOL caused inhibiton of H2 permeation but after air oxidation, the membrane permeability was restored and constant for 32 h at 633 K in the presence of MCH. The H2 permeation rates through the membrane were 14{\%} lower in the presence of MCH or TOL. S and Cl poisoned the membrane strongly but redox procedures for S and a water vapour treatment for Cl were sufficient to restore the original H2 permeation rates. The application of a catalytic membrane reactor (CMR) with a sulphided, monometallic, commercial catalyst is illustrated for MCH dehydrogenation where twice the equilibrium value of TOL was obtained. {\textcopyright} 1994.},
author = {Ali, Jawad K. and Newson, E. J. and Rippin, D. W T},
doi = {10.1016/0376-7388(93)E0219-A},
file = {:Users/marc/Library/Application Support/Mendeley Desktop/Downloaded/Ali, Newson, Rippin - 1994 - Deactivation and regeneration of PdAg membranes for dehydrogenation reactions.pdf:pdf},
isbn = {0376-7388},
issn = {03767388},
journal = {Journal of Membrane Science},
keywords = {Chlorine,Deactivation,Hydrogen permeation,Methylcyclohexane,PdAg membrane,Regeneration,Sulphur},
number = {1-2},
pages = {171--184},
title = {{Deactivation and regeneration of PdAg membranes for dehydrogenation reactions}},
volume = {89},
year = {1994}
}
@article{Tan2012,
abstract = {BaCe0.95Tb0.05O3−$\alpha$ (BCTb) perovskite hollow fibre membranes were fabricated by spinning the slurry mixture containing 66.67wt{\%} BCTb powder, 6.67wt{\%} polyethersulphone (PESf) and 26.67wt{\%} N-methyl-2-pyrrolidone (NMP) followed by sintering at elevated temperatures. The influence of sintering temperature on the membrane properties was investigated in terms of crystal phase, morphology, porosity and mechanical strength. In order to obtain gas-tight hollow fibres with sufficient mechanical strength, the sintering temperature should be controlled between 1350 and 1450°C. Hydrogen permeation through the BCTb hollow fibre membranes was carried out between 700 and 1000°C using 50{\%} H2–He mixture as feed on the shell side and N2 as sweep gas in the fibre lumen. The measured hydrogen permeation flux through the BCTb hollow fibre membranes reached up to 0.422$\mu$molcm−2s−1 at 1000°C when the flow rates of the H2–He feed and the nitrogen sweep were 40mLmin−1 and 30mLmin−1, respectively.},
author = {Tan, Xiaoyao and Song, Jian and Meng, Xiuxia and Meng, Bo},
doi = {10.1016/j.jeurceramsoc.2012.03.004},
file = {:Users/marc/Library/Application Support/Mendeley Desktop/Downloaded/Tan et al. - 2012 - Preparation and characterization of BaCe0.95Tb0.05O3−$\alpha$ hollow fibre membranes for hydrogen permeation.pdf:pdf},
isbn = {0955-2219},
issn = {09552219},
journal = {Journal of the European Ceramic Society},
keywords = {extrusion,hydrogen permeation,membranes,perovskite,sintering},
number = {10},
pages = {2351--2357},
publisher = {Elsevier Ltd},
title = {{Preparation and characterization of BaCe0.95Tb0.05O3−$\alpha$ hollow fibre membranes for hydrogen permeation}},
url = {http://dx.doi.org/10.1016/j.jeurceramsoc.2012.03.004},
volume = {32},
year = {2012}
}
@article{Nickerl2015,
abstract = {Dihydro-1,2,4,5-tetrazine-3,6-dicarboxylate was introduced into the chemically stable UiO-66 structure by a postsynthetic linker exchange reaction to create an optical sensor material for the detection of oxidative agents such as nitrous gases. The incorpo-rated tetrazine unit can be reversibly oxidized and reduced, which is accompanied by a drastic colour change from yellow to pink and vice versa. The high stability of the framework during redox reaction was proven by powder X-ray diffraction and nitrogen physisorption measurements. Gas sensors are widely used for monitoring the concentration of different gases such as methane in coal mines or carbon monoxide for personal protection using portable devices. An ideal sensor should be sensitive and stable enough to operate under ambient conditions and additionally produce a signal with a short response time that is easy to sense without the need for further equipment, the so-called wireless gas sensor. 1 One of the simplest, and arguably most powerful, means of transducing a sensing signal is via a visible change in the color of the material. 2 Furthermore, a sensor needing no power to work is independent of the availability of power sources being operational anytime and anywhere. NO x (nitrous gases) is a generic term for nitric oxide (NO) and nitrogen dioxide (NO 2), which are poisonous gases often released into the atmosphere by motor vehicles and industrial and domestic combustion processes. Therefore detection and monitoring of nitrous gases is of great interest and many NO x gas sensors have been investigated, for example, metal oxide semiconductors, solid electrolytes, and conducting polymers such as polypyrrole and polyaniline. The modular design of metal–organic frameworks (MOFs) along with a high porosity (allowing a rapid accessibility for guest molecules) opens the way for the targeted synthesis of desired sensor materials.},
author = {Nickerl, Georg and Senkovska, Irena and Kaskel, Stefan},
doi = {10.1039/C4CC08136F},
file = {:Users/marc/Library/Application Support/Mendeley Desktop/Downloaded/Nickerl, Senkovska, Kaskel - 2015 - Tetrazine functionalized zirconium MOF as an optical sensor for oxidizing gases.pdf:pdf},
issn = {1359-7345},
journal = {Chem. Commun.},
number = {12},
pages = {2280--2282},
pmid = {25558479},
publisher = {Royal Society of Chemistry},
title = {{Tetrazine functionalized zirconium MOF as an optical sensor for oxidizing gases}},
url = {http://xlink.rsc.org/?DOI=C4CC08136F},
volume = {51},
year = {2015}
}
@article{Paglieri1999,
abstract = {Pd/alumina composite membranes were fabricated using the generally practiced electroless plating process involving two-step activation of a symmetric 0.2 m -alumina microfilter with tin (Sn) chloride sensitizer (containing SnCl2 and SnCl4) and palladium(II) chloride (PdCl2). Pd films were deposited on these activated supports with a hydrazine- (N2H4-) and PdCl2-containing electroless plating bath. When these membranes were tested at 823 K for several days, the ideal H2/N2 separation factor (pure gas permeability ratio) declined substantially, depending on the membrane thickness. Modifications to the activation procedure minimized the amount of Sn chloride used in the sensitizing step. This reduced the selectivity decline, although the problem was not eliminated. The amount of Sn present at the Pd/ceramic interface was qualitatively related to the high-temperature performance. Possible routes for pore formation and selectivity decline are suggested. Sn chloride was removed from the process entirely with a new activation technique utilizing palladium(II) acetate (Pd(O2CCH3)2). Prior to electroless plating, substrates were dip-coated in a chloroform solution of Pd acetate, dried, calcined, and then reduced in flowing H2. At 973 K, nitrogen flux through these membranes remained constant for a period of at least a week. However, hydrogen permeability decreased at 873 K and above because of annealing.},
author = {Paglieri, S N and Foo, K Y and Way, J D and Collins, J P and Harper-Nixon, D L},
doi = {10.1021/ie980199c},
file = {:Users/marc/Library/Application Support/Mendeley Desktop/Downloaded/Paglieri et al. - 1999 - A New Preparation Technique for PdAlumina Membranes with Enhanced High-Temperature Stability.pdf:pdf},
issn = {0888-5885},
journal = {Industrial {\&} Engineering Chemistry Research},
keywords = {Pd,Sn,activation,alumina,flux,high temperature,hydrogen,membrane,nitrogen,performance,process,selectivity,separation,stability},
number = {5},
pages = {1925--1936},
title = {{A New Preparation Technique for Pd/Alumina Membranes with Enhanced High-Temperature Stability}},
url = {http://dx.doi.org/10.1021/ie980199c},
volume = {38},
year = {1999}
}
@misc{Sarashina2001,
abstract = {The complete primary structure of MSP-1, a major water-soluble glycoprotein in the foliated calcite shell layer of the scallop Patinopecten yessoensis, is reported. The full-length complementary DNA for MSP-1 isolated by polymerase chain reaction contained a sequence for a signal peptide of 20 amino acids followed by a polypeptide of 820 amino acids with calculated molecular mass of 74.5 kDa. The deduced amino acid sequence of MSP-1 includes a high proportion of Ser (32{\%}), Gly (25{\%}), and Asp (20{\%}), and the predicted isoelectric point is 3.2; in these respects, MSP-1 is a typical acidic glycoprotein of mineralized tissues. A repeated modular structure characterizes MSP-1, with a sequence unit between 158 and 177 amino acids in length being repeated 4 times in tandem in the middle part of the protein. The repeated unit comprises 3 modules (SG, D, and K domains), each having a distinct amino acid composition and sequence. The SG domain is almost exclusively composed of Ser and Gly residues. The D domain is rich in Asp residues, potential N-glycosylation and phosphorylation sites. The K domain is rich in Gly residues and has a core of basic residues. The Asp residues are arranged more or less regularly in the D domains, exhibiting some repeated motifs such as Asp-Gly-Ser-Asp and Asp-Ser-Asp. Further, the 4 D domains indicate remarkable overall sequence similarities to each other. These observations suggest that the regular arrangements of COO(-) groups in the D domain side chains may be important for specific control of crystal growth.},
author = {Sarashina, Isao and Endo, Kazuyoshi},
booktitle = {Marine Biotechnology},
doi = {10.1007/s10126-001-0013-6},
number = {4},
title = {{The complete primary structure of molluscan shell protein 1 (MSP-1), an acidic glycoprotein in the shell matrix of the scallop Patinopecten yessoensis}},
volume = {3},
year = {2001}
}
@article{Smith2015,
abstract = {The chemical reaction between hydrogen and purely sp(2)-bonded graphene to form graphene's purely sp(3)-bonded analogue, graphane, potentially allows the synthesis of a much wider variety of novel two-dimensional materials by opening a pathway to the application of conventional chemistry methods in graphene. Graphene is currently hydrogenated by exposure to atomic hydrogen in a vacuum, but these methods have not yielded a complete conversion of graphene to graphane, even with graphene exposed to hydrogen on both sides of the lattice. By heating graphene in molecular hydrogen under compression to modest high pressure in a diamond anvil cell (2.6-5.0 GPa), we are able to react graphene with hydrogen and propose a method whereby fully hydrogenated graphane may be synthesized for the first time.},
author = {Smith, Dean and Howie, Ross T. and Crowe, Iain F. and Simionescu, Cristina L. and Muryn, Chris and Vishnyakov, Vladimir and Novoselov, Konstantin S. and Kim, Yong Jin and Halsall, Matthew P. and Gregoryanz, Eugene and Proctor, John E.},
doi = {10.1021/acsnano.5b02712},
file = {:Users/marc/Library/Application Support/Mendeley Desktop/Downloaded/Smith et al. - 2015 - Hydrogenation of Graphene by Reaction at High Pressure and High Temperature.pdf:pdf},
isbn = {1936-0851},
issn = {1936086X},
journal = {ACS Nano},
keywords = {diamond anvil cell,functionalized graphene,graphane,graphene,hydrogenated graphene},
number = {8},
pages = {8279--8283},
title = {{Hydrogenation of Graphene by Reaction at High Pressure and High Temperature}},
volume = {9},
year = {2015}
}
@article{Sakamoto1997,
author = {Sakamoto, F and Kinari, Y and Chen, F L and Sakamoto, Y},
journal = {International Journal of Hydrogen Energy},
number = {4},
pages = {369--375},
title = {{Hydrogen permeation through palladium alloy membranes in mixture gases of 10{\%} Nitrogen and Ammonia in the Hydrogen}},
volume = {22},
year = {1997}
}
@article{Nishimura2002,
author = {Nishimura, C and Komaki, M and Hwang, S and Amano, M},
file = {:Users/marc/Library/Application Support/Mendeley Desktop/Downloaded/Nishimura et al. - 2002 - V-Ni alloy membranes for hydrogen purification.pdf:pdf},
journal = {Journal of Alloys and Compounds},
pages = {902--906},
title = {{V-Ni alloy membranes for hydrogen purification}},
volume = {330-332},
year = {2002}
}
@incollection{Volkov2014,
author = {Volkov, A},
booktitle = {Encycolpedia of Membranes},
editor = {{Drioli E.}, Giorno L},
publisher = {Springer, Berlin, Heidelberg},
title = {{Membrane Compaction}},
year = {2014}
}
@article{Liu2005,
author = {Liu, Shaomin and Gavalas, George R},
journal = {Industrial and Engineering Chemistry Research},
pages = {7633--7637},
title = {{Preparation of Oxygen Ion Conducting Ceramic Hollow-Fiber Membranes}},
volume = {44},
year = {2005}
}
@article{Fukui1997,
abstract = {The CO selective gas sensor based on La2O3–Au/SnO2 ceramics had a high selectivity to CO among H2, CH4, i-C4H10 and C2H4; while a high sensitivity to C2H5OH. An acidic catalyst layer converting C2H5OH into C2H4 with a low sensitivity by dehydration was coated on the sensing layer of the La2O3–Au/SnO2 ceramics to reduce the sensitivity to C2H5OH. Ferrierite, one of siliceous zeolites, was found to have a prominent ethanol filtering effect; such a prominent performance was related to a strong acid strength and a large acid amount. The siliceous zeolites have a hydrophobic property, a specific area of more than a few hundreds in m2 g−1 and a thermal stability. As for the ethanol filtering effects, ka/D≥105 was estimated on the basis of a simple model of chemical dynamics, k: a rate constant of dehydration, a: an effective reaction area of the catalyst layer, D: a diffusion coefficient in the filtering layer of ferrierite. As a result, a CO selective gas sensor with a sensitivity to CO over ten times higher than that to the other gases was obtained at about 300°C.},
author = {Fukui, Kiyoshi and Nishida, Sachiko},
doi = {10.1016/S0925-4005(97)00280-3},
file = {:Users/marc/Library/Application Support/Mendeley Desktop/Downloaded/Fukui, Nishida - 1997 - CO gas sensor based on Au–La2O3 added SnO2 ceramics with siliceous zeolite coat.pdf:pdf},
issn = {09254005},
journal = {Sensors and Actuators B: Chemical},
keywords = {A simple model of chemical dynamics,Acidic catalyst,CO selective gas sensor,Dehydration,Ethanol filtering effects,La2O3–Au/SnO2 ceramics,Siliceous zeolites},
number = {2},
pages = {101--106},
title = {{CO gas sensor based on Au–La2O3 added SnO2 ceramics with siliceous zeolite coat}},
url = {http://www.sciencedirect.com/science/article/pii/S0925400597002803},
volume = {45},
year = {1997}
}
@article{Flanagan2011,
abstract = {H permeabilities have been measured for a series of fcc Pd{\^{a}}€“Au alloy membranes in the temperature range from 393 to 573 K. The maximum permeability is found at the atom fraction of Au, XAu = 0.11 {\^{A}}± 0.01, which closely agrees with a theoretically predicted value (Sonwane, C.; Wilcox, J.; Ma, Y. J. Chem. Phys.2006, 125, 184714). For a series of Pd{\^{a}}€“Au alloy membranes with different Au contents, diffusion constants, activation energies, and pre-exponential factors have been determined as a function of H content. H concentration-independent values of these parameters have been obtained by extrapolation to H{\^{a}}†'0 where the H solution behaves ideally.},
author = {Flanagan, Ted B. and Wang, D.},
doi = {10.1021/jp201988u},
file = {:Users/marc/Library/Application Support/Mendeley Desktop/Downloaded/Flanagan, Wang - 2011 - Hydrogen permeation through fcc Pd-Au alloy membranes.pdf:pdf},
isbn = {1932-7447},
issn = {19327447},
journal = {Journal of Physical Chemistry C},
number = {23},
pages = {11618--11623},
title = {{Hydrogen permeation through fcc Pd-Au alloy membranes}},
volume = {115},
year = {2011}
}
@article{Chen2010,
author = {Chen, Chao-Huang and Ma, Yi Hua},
doi = {10.1016/j.memsci.2010.07.002},
isbn = {03767388},
journal = {Journal of Membrane Science},
number = {1-2},
pages = {535--544},
title = {{The effect of H2S on the performance of Pd and Pd/Au composite membrane}},
volume = {362},
year = {2010}
}
@article{Antoniazzi1989,
author = {Antoniazzi, A B and Haasz, A A and Stangeby, P C},
journal = {Journal of Nuclear Materials},
pages = {1065--1070},
title = {{The Effect of Adsorbed Carbon and Sulphur on Hydrogen Permeation Through Palladium}},
volume = {162-164},
year = {1989}
}
@article{Wang2015a,
abstract = {First principles calculation reveals that the Pd3Ag/TiAl interface structure terminated with Pd atoms at the top of the octahedral interstitials of TiAl possesses higher interface strength and better thermal stability than other interface structures, which should be fundamentally due to a stronger chemical bonding formed in the interface. Calculation also shows that hydrogen diffusion between surface layer and inter-site seems energetically more favorable than diffusion within the surface layer, and that the small energy barrier suggests the easy diffusion of hydrogen across the Pd3Ag/TiAl interface. Moreover, hydrogen diffusion would have a small effect to increase the interface energy and decrease the interface strength. The calculated results are in good agreement with experimental observations in the literature, and demonstrate that the Pd3Ag/TiAl membranes should be a good candidate for hydrogen permeation.},
author = {Wang, Ji Wei and He, Y. H. and Gong, H. R.},
doi = {10.1016/j.memsci.2014.10.022},
file = {:Users/marc/Library/Application Support/Mendeley Desktop/Downloaded/Wang, He, Gong - 2015 - Various properties of Pd3AgTiAl membranes from density functional theory.pdf:pdf},
issn = {18733123},
journal = {Journal of Membrane Science},
keywords = {Hydrogen permeation,Interface atomic configuration,Interface cohesion,Pd3Ag membranes,TiAl support},
pages = {406--413},
title = {{Various properties of Pd3Ag/TiAl membranes from density functional theory}},
volume = {475},
year = {2015}
}
@article{Dikin2007,
abstract = {Free-standing paper-like or foil-like materials are an integral part of our technological society. Their uses include protective layers, chemical filters, components of electrical batteries or supercapacitors, adhesive layers, electronic or optoelectronic components, and molecular storage. Inorganic 'paper-like' materials based on nanoscale components such as exfoliated vermiculite or mica platelets have been intensively studied and commercialized as protective coatings, high-temperature binders, dielectric barriers and gas-impermeable membranes. Carbon-based flexible graphite foils composed of stacked platelets of expanded graphite have long been used in packing and gasketing applications because of their chemical resistivity against most media, superior sealability over a wide temperature range, and impermeability to fluids. The discovery of carbon nanotubes brought about bucky paper, which displays excellent mechanical and electrical properties that make it potentially suitable for fuel cell and structural composite applications. Here we report the preparation and characterization of graphene oxide paper, a free-standing carbon-based membrane material made by flow-directed assembly of individual graphene oxide sheets. This new material outperforms many other paper-like materials in stiffness and strength. Its combination of macroscopic flexibility and stiffness is a result of a unique interlocking-tile arrangement of the nanoscale graphene oxide sheets.},
annote = {Dikin, Dmitriy A
Stankovich, Sasha
Zimney, Eric J
Piner, Richard D
Dommett, Geoffrey H B
Evmenenko, Guennadi
Nguyen, SonBinh T
Ruoff, Rodney S
ENG
England
2007/07/27 09:00
Nature. 2007 Jul 26;448(7152):457-60.},
author = {Dikin, D A and Stankovich, S and Zimney, E J and Piner, R D and Dommett, G H and Evmenenko, G and Nguyen, S T and Ruoff, R S},
doi = {10.1038/nature06016},
file = {:Users/marc/Library/Application Support/Mendeley Desktop/Downloaded/Dikin et al. - 2007 - Preparation and characterization of graphene oxide paper.pdf:pdf},
isbn = {1476-4687 (Electronic)
0028-0836 (Linking)},
journal = {Nature},
number = {7152},
pages = {457--460},
pmid = {17653188},
title = {{Preparation and characterization of graphene oxide paper}},
url = {http://www.ncbi.nlm.nih.gov/pubmed/17653188},
volume = {448},
year = {2007}
}
@book{Holleman2001,
author = {Holleman, A. F. and Wiberg, E.},
title = {{Inorganic Chemistry}},
year = {2001}
}
@article{Lewis1928,
author = {Lewis, W. K. and Frolich, Per K.},
doi = {10.1021/ie50219a021},
file = {:Users/marc/Library/Application Support/Mendeley Desktop/Downloaded/Lewis, Frolich - 1928 - Synthesis of Methanol from Carbon Monoxide and Hydrogen'.pdf:pdf},
issn = {0019-7866},
journal = {Industrial and Engineering Chemistry},
number = {3},
pages = {285--290},
title = {{Synthesis of Methanol from Carbon Monoxide and Hydrogen'}},
volume = {20},
year = {1928}
}
@article{Dunbar2012,
author = {Dunbar, Zachary W and Chu, Deryn},
doi = {10.1016/j.jpowsour.2012.05.044},
file = {:Users/marc/Library/Application Support/Mendeley Desktop/Downloaded/Dunbar, Chu - 2012 - Thin palladium membranes supported on microstructured nickel for purification of reformate gases.pdf:pdf},
isbn = {03787753},
journal = {Journal of Power Sources},
pages = {47--53},
title = {{Thin palladium membranes supported on microstructured nickel for purification of reformate gases}},
volume = {217},
year = {2012}
}
@article{Nelson1926,
author = {Nelson, WM. L. and Engleder, C RL J.},
file = {:Users/marc/Library/Application Support/Mendeley Desktop/Downloaded/Nelson, Engleder - 1926 - The Thermal Decomposition of Formic Acid.pdf:pdf},
journal = {J. Phys. Chem},
number = {4},
pages = {470--475},
title = {{The Thermal Decomposition of Formic Acid}},
volume = {30},
year = {1926}
}
@incollection{Richter2015,
author = {Richter, H},
booktitle = {Palladium Membrane Technology for Hydrogen Production, Carbon Capture and Other Application},
doi = {10.1533/9781782422419.1.69},
editor = {Doukelis, A and Panopoulos, K and Koumanakos, A and Kakaras, E},
pages = {69--82},
publisher = {Woodhead Publishing},
title = {{Large-scale ceramic support fabrication for palladium membranes}},
year = {2015}
}
@article{Vøllestad2014,
abstract = {Hydrogen permeation in 30{\%} Mo-substituted lanthanum tungsten oxide membranes, La27Mo1.5W3.5O55.5 (LWMo), has been measured as a function of temperature, hydrogen partial pressure gradient, and water vapor pressure in the sweep gas. Transport of hydrogen by means of ambipolar proton-electron conductivity and - with wet sweep gas - water splitting contributes to the measured hydrogen content in the permeate. At 700°C under dry sweep conditions, the H2 permeability in LWMo was 6 × 10 -4mLmin-1cm-1, which is significantly higher than that for state-of-the-art SrCeO3-based membranes. Proton conductivity was identified as rate limiting for ambipolar bulk transport across the membrane. On these bases it is evident that Mo-substitution is a successful doping strategy to increase the n-type conductivity and H2 permeability compared to nominally unsubstituted lanthanum tungsten oxide. A steady-state model based on the Wagner transport theory with partial conductivities as input parameters predicted H2 permeabilities in good agreement with the measured data. LWMo is a highly competitive mixed proton-electron conducting oxide for hydrogen transport membrane applications provided that long term stability can be ensured. {\textcopyright} 2014 Elsevier B.V.},
author = {V{\o}llestad, Einar and Vigen, Camilla K. and Magras{\'{o}}, Anna and Haugsrud, Reidar},
doi = {10.1016/j.memsci.2014.03.011},
file = {:Users/marc/Library/Application Support/Mendeley Desktop/Downloaded/V{\o}llestad et al. - 2014 - Hydrogen permeation characteristics of La27Mo1.5W3.5O55.5.pdf:pdf},
issn = {18733123},
journal = {Journal of Membrane Science},
keywords = {Ambipolar transport,Hydrogen flux,Lanthanum tungstate,Membrane,Mixed proton-electron conductor},
pages = {81--88},
publisher = {Elsevier},
title = {{Hydrogen permeation characteristics of La27Mo1.5W3.5O55.5}},
url = {http://dx.doi.org/10.1016/j.memsci.2014.03.011},
volume = {461},
year = {2014}
}
@article{Wang2014,
abstract = {Hierarchical $\alpha$-Fe2O3/NiO composites with a hollow nanostructure were synthesized by a facile hydrothermal method. The structures and morphologies of the composites were investigated by different kinds of techniques, including X-ray diffraction, field-emission electron scanning microscopy, transmission electron microscopy, and energy dispersive spectroscopy. Hierarchical $\alpha$-Fe2O3/NiO composites were fabricated by growing the $\alpha$-Fe2O3 nanorods on the surfaces of porous NiO nanosheets with a thickness of ∼12 nm. The gas sensing properties of hierarchical $\alpha$-Fe2O3/NiO composites toward toluene were investigated using a static system. The response of $\alpha$-Fe2O3/NiO composites to 100 ppm toluene was ∼18.68, which was 13.18 times higher than that of pure NiO at 300 °C. The enhanced response can be attributed to heterojunction. Meanwhile, the rapid response and recovery characteristics were observed because of the porous hollow structural characteristics and catalytic actions of $\alpha$-Fe2O3 and NiO.},
author = {Wang, Chen and Cheng, Xiaoyang and Zhou, Xin and Sun, Peng and Hu, Xiaolong and Shimanoe, Kengo and Lu, Geyu and Yamazoe, Noboru},
doi = {10.1021/am501063z},
file = {:Users/marc/Library/Application Support/Mendeley Desktop/Downloaded/Wang et al. - 2014 - Hierarchical -Fe2O3NiO composites with a hollow structure for a gas sensor.pdf:pdf},
isbn = {1944-8244},
issn = {19448252},
journal = {ACS Applied Materials and Interfaces},
keywords = {gas sensor,heterojunction,hollow nanostructure,hydrothermal},
number = {15},
pages = {12031--12037},
pmid = {25007348},
title = {{Hierarchical ??-Fe2O3/NiO composites with a hollow structure for a gas sensor}},
volume = {6},
year = {2014}
}
@article{Kang2017a,
abstract = {CrystEngComm PAPER Lili Fan, Ming Xue et al. In situ confinement of free linkers within a stable MOF membrane for highly improved gas separation properties A stable MOF membrane with guest molecules encapsulated in the pores by in situ synthesis has been suc-cessfully fabricated. The in situ confinement of linkers in the channels of the MOF membrane improves its gas separation properties, which may provide a general method for fine-tuning the pore size of MOF membranes and develop the functional applications of porous MOF materials.},
author = {Kang, Zixi and Fan, Lili and Wang, Sasa and Sun, Daofeng and Xue, Ming and Qiu, Shilun},
doi = {10.1039/c7ce00102a},
file = {:Users/marc/Library/Application Support/Mendeley Desktop/Downloaded/Kang et al. - 2017 - In situ confinement of free linkers within a stable MOF membrane for highly improved gas separation properties.pdf:pdf},
issn = {14668033},
journal = {CrystEngComm},
number = {12},
pages = {1601--1606},
title = {{In situ confinement of free linkers within a stable MOF membrane for highly improved gas separation properties}},
volume = {19},
year = {2017}
}
@article{Seund-EunNam2001,
author = {{Seund-Eun Nam}, Kew-Ho Lee},
chapter = {177},
file = {:Users/marc/Library/Application Support/Mendeley Desktop/Downloaded/Seund-Eun Nam - 2001 - Hydrogen separation by Pd alloy composite membranes introduction of diffusion barrier.pdf:pdf},
journal = {Journal of Membrane Science},
pages = {177--185},
title = {{Hydrogen separation by Pd alloy composite membranes: introduction of diffusion barrier}},
volume = {192},
year = {2001}
}
@article{Leo2011,
abstract = {Ba0.5Sr0.5Co0.8Fe0.2O3-?? (BSCF) hollow fibres are very attractive to deliver high oxygen fluxes at low manufacturing costs. These membranes efficiently separate oxygen from air via ionic diffusion, and have been exclusively prepared with sulphur containing binders, a vital ingredient to aggregate ceramic particles to shape hollow fibre geometries. Here we show that the choice of binder influences the stoichiometry of the crystal structure with the formation of non-ionic conduction domains, ultimately affecting oxygen ionic diffusion. Our synthesis method used sulphur free binders for the preparation of BSCF hollow fibres, resulting in an oxygen flux increase of 105{\%} to 9.5mlmin-1cm-2 at 950??C. In addition, catalytic surface modification of the BSCF hollow fibres further increased oxygen fluxes to 14.5mlmin-1cm-2. More importantly, oxygen fluxes significantly increased by a factor of 10 below 700??C, thus reducing operating temperature requirements by 250??C. ?? 2010 Elsevier B.V.},
author = {Leo, Adrian and Smart, Simon and Liu, Shaomin and {Diniz da Costa}, Jo{\~{a}}o C.},
doi = {10.1016/j.memsci.2010.11.002},
file = {:Users/marc/Library/Application Support/Mendeley Desktop/Downloaded/Leo et al. - 2011 - High performance perovskite hollow fibres for oxygen separation.pdf:pdf},
isbn = {0376-7388},
issn = {03767388},
journal = {Journal of Membrane Science},
keywords = {High oxygen flux,Hollow fibres,Perovskite,Sulphur free binder},
number = {1-2},
pages = {64--68},
publisher = {Elsevier B.V.},
title = {{High performance perovskite hollow fibres for oxygen separation}},
url = {http://dx.doi.org/10.1016/j.memsci.2010.11.002},
volume = {368},
year = {2011}
}
@article{Bhatt2013,
author = {Bhatt, R and Bhattacharya, S and Basu, R and Singh, A and Deshpande, U and Surger, C and Basu, S and Aswal, D K and Gupta, S K},
doi = {10.1016/j.tsf.2013.04.143},
file = {:Users/marc/Library/Application Support/Mendeley Desktop/Downloaded/Bhatt et al. - 2013 - Growth of Pd4S, PdS and PdS2 films by controlled sulfurization of sputtered Pd on native oxide of Si.pdf:pdf},
isbn = {00406090},
journal = {Thin Solid Films},
pages = {41--46},
title = {{Growth of Pd4S, PdS and PdS2 films by controlled sulfurization of sputtered Pd on native oxide of Si}},
volume = {539},
year = {2013}
}
@article{He2017,
abstract = {H2production from a biomass fermentation process provides a great opportunity for green and sustainability economy compared to use fossil fuels (e.g., steam methane reforming process), but the process usually produces a humidified gas containing 35–65 (vol.{\%}) H2and 40–50 (vol.{\%}) CO2. A novel, energy efficient two-stage carbon membrane system for H2purification based on the combination of H2selective carbon molecular sieve membranes in the first stage and CO2selective carbon membranes in the second stage was proposed. The concept is the first unique technology providing a low H2purification cost {\textless}1 {\$}/kg H2produced at a feed pressure of 20 bara based on process simulation and cost estimation. The produced high purity CO2({\textgreater}95 vol.{\%}) and H2({\textgreater}99.5 vol.{\%}) can be integrated into a gasification reformer and Fischer-Tropsch reactors to improve the biodiesel productivity from biomass.},
author = {He, Xuezhong},
doi = {10.1016/j.seppur.2017.05.034},
file = {:Users/marc/Library/Application Support/Mendeley Desktop/Downloaded/He - 2017 - Techno-economic feasibility analysis on carbon membranes for hydrogen purification.pdf:pdf},
issn = {18733794},
journal = {Separation and Purification Technology},
keywords = {Biomass,Carbon membranes,Cost estimation,Hydrogen purification,Process simulation},
pages = {117--124},
publisher = {Elsevier B.V.},
title = {{Techno-economic feasibility analysis on carbon membranes for hydrogen purification}},
url = {http://dx.doi.org/10.1016/j.seppur.2017.05.034},
volume = {186},
year = {2017}
}
@article{Kolb1986,
author = {Kolb, B. and Liebhardt, B. and Ettre, L. S.},
doi = {10.1007/BF02311600},
file = {:Users/marc/Library/Application Support/Mendeley Desktop/Downloaded/Kolb, Liebhardt, Ettre - 1986 - Cryofocusing in the combination of gas chromatography with equilibrium headspace sampling.pdf:pdf},
issn = {00095893},
journal = {Chromatographia},
keywords = {Capillary columns,Cryogenic trapping,Gas chromatography,Headspace analysis,Trace analysis},
number = {6},
pages = {305--311},
title = {{Cryofocusing in the combination of gas chromatography with equilibrium headspace sampling}},
volume = {21},
year = {1986}
}
@incollection{Peters2015,
author = {Peters, T A and Stange, M and Bredesen, R},
booktitle = {Palladium Membrane Technology for Hydrogen Production, Carbon Capture and Other Application},
doi = {10.1533/9781782422419.1.25},
editor = {Doukelis, A and Panopoulos, K and Koumanakos, A and Kakaras, E},
file = {:Users/marc/Library/Application Support/Mendeley Desktop/Downloaded/Peters, Stange, Bredesen - 2015 - Fabrication of palladium-based membranes by magnetron sputtering.pdf:pdf},
pages = {25--41},
publisher = {Woodhead Publishing},
title = {{Fabrication of palladium-based membranes by magnetron sputtering}},
year = {2015}
}
@article{Li2013a,
abstract = {Two cluster-based microporous MOFs containing coordination unsaturated metal(ii) sites (or open metal(ii) sites) have been constructed from boxlike or cagelike {\{}M3{\}}x SBUs. They both exhibit highly selective uptake for CO2 over CH4 and N2 under ambient conditions. More importantly, MOF-1 also provides an ideal luminescence sensor for selective probing Ba2+ and Cu2+ ions based on significant luminescence enhancement or quenching. This journal is {\textcopyright} 2013 The Royal Society of Chemistry.},
author = {Li, Yun-Wu and Li, Jian-Rong and Wang, Li-Fu and Zhou, Bo-Yu and Chen, Qiang and Bu, Xian-He},
doi = {10.1039/C2TA00635A},
file = {:Users/marc/Library/Application Support/Mendeley Desktop/Downloaded/Li et al. - 2013 - Microporous metal–organic frameworks with open metal sites as sorbents for selective gas adsorption and fluorescenc.pdf:pdf},
isbn = {2050-7488},
issn = {2050-7488},
journal = {J. Mater. Chem. A},
number = {3},
pages = {495--499},
title = {{Microporous metal–organic frameworks with open metal sites as sorbents for selective gas adsorption and fluorescence sensors for metal ions}},
url = {http://xlink.rsc.org/?DOI=C2TA00635A},
volume = {1},
year = {2013}
}
@article{Pozzo2009,
author = {Pozzo, M and Alfe, D},
doi = {10.1016/j.ijhydene.2008.11.109},
file = {:Users/marc/Library/Application Support/Mendeley Desktop/Downloaded/Pozzo, Alfe - 2009 - Hydrogen dissociation and diffusion on transition metal ( Ti , Zr , V , Fe , Ru , Co , Rh , Ni , Pd , Cu , Ag ) -d.pdf:pdf},
issn = {0360-3199},
journal = {International Journal of Hydrogen Energy},
number = {4},
pages = {1922--1930},
publisher = {Elsevier Ltd},
title = {{Hydrogen dissociation and diffusion on transition metal ( [ Ti , Zr , V , Fe , Ru , Co , Rh , Ni , Pd , Cu , Ag ) -doped Mg ( 0001 ) surfaces}},
url = {http://dx.doi.org/10.1016/j.ijhydene.2008.11.109},
volume = {34},
year = {2009}
}
@article{PachecoTanaka2005,
author = {{Pacheco Tanaka}, David A and {Llosa Tanco}, Margot A and Niwa, Shu-ichi and Wakui, Yoshito and Mizukami, Fujio and Namba, Takemi and Suzuki, Toshishige M},
doi = {10.1016/j.memsci.2004.06.002},
isbn = {03767388},
journal = {Journal of Membrane Science},
number = {1-2},
pages = {21--27},
title = {{Preparation of palladium and silver alloy membrane on a porous {\$}\alpha{\$}-alumina tube via simultaneous electroless plating}},
volume = {247},
year = {2005}
}
@article{Torkkeli2003a,
abstract = {superhydrophobic surfaces can be used to reduce the minimum voltage and to increase the maximum speed under certain conditions, but there are some harmful side-effects. First of all, the electrostatic pressure can push water into the surface pores, which hinders actuation. The phenomenon can also be treated as a vertical electrowetting effect. Another drawback is that the use of superhydrophobic surfaces makes actuation more critical to the properties of the liquid. For example, actuation of biological buffer solutions was not successful. For these reasons, it is concluded that it is more beneficial to use a smooth surface with low hysteresis than a superhydrophobic surface in droplet actuation. Electrostatic droplet actuation is a potential method for manipulating liquid on a microscopic scale, but there is still work to do. This work contains a detailed examination of the droplet actuation mechanism, and trapping of charges in the solid-liquid interface is found to be the most severe problem that needs to be solved.},
archivePrefix = {arXiv},
arxivId = {arXiv:physics/0201037v1},
author = {Ismail, Wan Norharyati Wan Salleh and Ahmad Fauzi},
doi = {10.1002/aic},
eprint = {0201037v1},
file = {:Users/marc/Library/Application Support/Mendeley Desktop/Downloaded/Ismail - 2012 - Fabrication and Characterization of PEIPVP-Based Carbon Hollow Fiber Membranes for CO2CH4 and CO2N2 Separation.pdf:pdf},
isbn = {9513862380},
issn = {12350621},
journal = {AIChE Journal},
keywords = {Electrostatic droplet actuation,Electrowetting,Lab-on-a-chip,MEMS,Microfluidics,Superhydrophobic surfaces},
number = {10},
pmid = {23641116},
primaryClass = {arXiv:physics},
title = {{Fabrication and Characterization of PEI/PVP-Based Carbon Hollow Fiber Membranes for CO2/CH4 and CO2/N2 Separation}},
volume = {58},
year = {2012}
}
@article{Melaina2013,
abstract = {Hydrogen is being pursued as a sustainable energy carrier for fuel cell electric vehicles (FCEVs) and as a means of storing renewable energy at utility scale. Hydrogen can also be used as a fuel in stationary fuel cell systems for buildings, backup power, or distributed generation. Blending hydrogen into the existing natural gas pipeline network has been proposed as a means of increasing the output of renewable energy systems such as large wind farms. If implemented with relatively low concentrations, less than 5{\%}–15{\%} hydrogen by volume, this strategy of storing and delivering renewable energy to markets appears to be viable without significantly increasing risks associated with utilization of the gas blend in end-use devices (such as household appliances), overall public safety, or the durability and integrity of the existing natural gas pipeline network. However, the appropriate blend concentration may vary significantly between pipeline network systems and natural gas compositions and must therefore be assessed on a case-by-case basis. Any introduction of a hydrogen blend concentration would require extensive study, testing, and modifications to existing pipeline monitoring and maintenance practices (e.g., integrity management systems). Additional cost would be incurred as a result, and this cost must be weighed against the benefit of providing a more sustainable and low-carbon gas product to consumers.},
author = {Melaina, M W and Antonia, O and Penev, M},
doi = {10.2172/1068610},
file = {:Users/marc/Library/Application Support/Mendeley Desktop/Downloaded/Melaina, Antonia, Penev - 2013 - Blending Hydrogen into Natural Gas Pipeline Networks A Review of Key Issues Blending Hydrogen into Nat.pdf:pdf},
keywords = {Hydrogen,March 2013,NREL/TP-5600-51995,National Renewable Energy Laboratory,U.S. natural gas pipeline network,hydrogen pipelines,study and assessment},
number = {March},
pages = {131},
title = {{Blending Hydrogen into Natural Gas Pipeline Networks : A Review of Key Issues Blending Hydrogen into Natural Gas Pipeline Networks : A Review of Key Issues}},
url = {http://www.nrel.gov/docs/fy13osti/51995.pdf},
year = {2013}
}
@article{Arblaster2012,
abstract = {The crystallographic properties of palladium at temperatures from absolute zero to the freezing point are assessed following a review of the literature published between 1901 to date. However values above 1100 K are considered to be highly tentative since they are based on only one set of measurements. Selected values of the thermal expansion coeffi cient and measurements of length change due to thermal expansion have been used to calculate the variation with temperature of the lattice parameter, interatomic distance, atomic and molar volumes and density. The data is presented in the form of Equations and in Tables whilst a comparison between selected and experimental values is shown in the Figures. This},
author = {Arblaster, John W. and Wombourne},
doi = {10.1595/147106712X646113},
file = {:Users/marc/Library/Application Support/Mendeley Desktop/Downloaded/Arblaster, Wombourne - 2012 - Crystallographic properties of Palladium.pdf:pdf},
number = {3},
pages = {181--189},
title = {{Crystallographic properties of Palladium}},
url = {http://www.tfp.ethz.ch/{~}haug/pv/{\%}5Cnhttp://www.google.com/url?sa=t{\&}source=web{\&}ct=res{\&}cd=5{\&}ved=0CBEQFjAE{\&}url=http://www.tfp.ethz.ch/{~}haug/pv/cigs-cell.pdf{\&}ei=dd8OS{\_}3DMM3K{\_}gahlpSvBQ{\&}usg=AFQjCNF9VZcxe2HBdx55zIHKbP3SiCwyqQ},
volume = {56},
year = {2012}
}
@article{Oh2009,
abstract = {The europium dopant concentration in strontium cerate was studied to achieve maximum hydrogen permeation. In order to determine high ambipolar conductivity, total conductivity and open circuit potential measurements were performed. Among the three different compositions of Eu-doped SrCe1 - xEuxO3 - ?? (x = 0.1, 0.15 and 0.2) studied, SrCe0.9Eu0.1O3 - ?? showed highest total conductivity between 600????C and 900????C. However, transference number measurements showed increasing electronic conductivity with increasing dopant concentration and a stronger temperature dependence for electronic conduction. Therefore, the highest ambipolar conductivity was obtained over the compositional range from SrCe0.85Eu0.15O3 - ?? to SrCe0.8Eu0.2O3 - ?? depending on temperature. Finally, the hydrogen permeation flux was calculated based on the ambipolar conductivity and compared with experimental results. ?? 2009 Elsevier B.V. All rights reserved.},
author = {keun Oh, Tak and Yoon, Heesung and Wachsman, E D},
doi = {10.1016/j.ssi.2009.07.001},
file = {:Users/marc/Library/Application Support/Mendeley Desktop/Downloaded/Oh, Yoon, Wachsman - 2009 - Effect of Eu dopant concentration in SrCe1 - xEuxO3 - on ambipolar conductivity.pdf:pdf},
isbn = {0167-2738},
issn = {01672738},
journal = {Solid State Ionics},
keywords = {Ambipolar,Conductivity,Membrane,Proton,SrCeO3},
number = {23-25},
pages = {1233--1239},
title = {{Effect of Eu dopant concentration in SrCe1 - xEuxO3 - ?? on ambipolar conductivity}},
volume = {180},
year = {2009}
}
@article{Methanation1975,
author = {Methanation, Catalytic and Plant, Pilot},
file = {:Users/marc/Library/Application Support/Mendeley Desktop/Downloaded/Methanation, Plant - 1975 - Design and Operation of Catalytic Methanation in the H Y G A S Pilot Plant.pdf:pdf},
title = {{Design and Operation of Catalytic Methanation in the H Y G A S Pilot Plant}},
year = {1975}
}
@article{Yamaura2001a,
abstract = {We prepared the melt-spun (Ni0.6Nb0.4)100−xZrx (x=0 to 40 at{\%}) and other amorphous alloy membranes and examined the permeation of hydrogen through those alloy membranes. The interatomic spacing in the Ni–Nb–Zr amorphous structure increased with increasing Zr content. The crystallization temperature of the Ni–Nb–Zr amorphous alloys decreased with increasing Zr content. The hydrogen flow increased with an increase of the temperature or the difference in the square-roots of hydrogen pressures across the membrane, $\Delta$$\backslash$$\backslash$sqrtp. At relatively higher temperature up to 673 K or at relatively higher hydrogen pressure difference, $\Delta$$\backslash$$\backslash$sqrtp up to 550 Pa1⁄2, the hydrogen flow was more strictly proportional to $\Delta$$\backslash$$\backslash$sqrtp. This indicates that the diffusion of hydrogen through the membrane is a rate-controlling factor for hydrogen permeation. The permeability of the Ni–Nb–Zr amorphous alloys was strongly dependent on alloy compositions and increased with increasing Zr content. However, it was difficult to investigate the hydrogen permeability of the (Ni0.6Nb0.4)60Zr40 amorphous alloy in this work due to the embrittlement during the measurement. The maximum hydrogen permeability was 1.3×10−8 (mol{\textperiodcentered}m−1{\textperiodcentered}s−1{\textperiodcentered}Pa−1⁄2) at 673 K for the (Ni0.6Nb0.4)70Zr30 amorphous alloy. It is noticed that the hydrogen permeability of the (Ni0.6Nb0.4)70Zr30 amorphous alloy is higher than that of pure Pd metal. These permeation characteristics indicate the possibility of future practical use of the melt-spun amorphous alloys as a hydrogen permeable membrane.},
author = {Yamaura, Shin-ichi and Shimpo, Yoichiro (Fukuda Metal Foil {\&} Powder Co. Ltd. ) and Okouchi, Hitoshi (Fukuda Metal Foil {\&} Powder Co. Ltd. ) and Nishida, Motonori (Fukuda Metal Foil {\&} Powder Co. Ltd. ) and Kajita, Osamu (Fukuda Metal Foil {\&} Powder Co. Ltd. ) and Kimura, Hisamichi and Inoue, Akihisa},
doi = {10.2320/matertrans.42.1885},
file = {:Users/marc/Library/Application Support/Mendeley Desktop/Downloaded/Yamaura et al. - 2001 - Hydrogen Permeation Characteristics of Melt-Spun Ni-Nb-Zr Amorphous Alloy Membranes.pdf:pdf},
issn = {1345-9678},
journal = {Materials Transactions},
keywords = {amorphous alloy,hydrogen permeation,hydrogen separation,melt-spinning},
number = {9},
pages = {1885--1890},
title = {{Hydrogen Permeation Characteristics of Melt-Spun Ni-Nb-Zr Amorphous Alloy Membranes}},
url = {http://ci.nii.ac.jp/naid/130004451634/},
volume = {42},
year = {2001}
}
@article{Plazaola2017,
abstract = {Palladium-based membranes for hydrogen separation have been studied by several research groups during the last 40 years. Much effort has been dedicated to improving the hydrogen flux of these membranes employing different alloys, supports, deposition/production techniques, etc. High flux and cheap membranes, yet stable at different operating conditions are required for their exploitation at industrial scale. The integration of membranes in multifunctional reactors (membrane reactors) poses additional demands on the membranes as interactions at different levels between the catalyst and the membrane surface can occur. Particularly, when employing the membranes in fluidized bed reactors, the selective layer should be resistant to or protected against erosion. In this review we will also describe a novel kind of membranes, the pore-filled type membranes prepared by Pacheco Tanaka and coworkers that represent a possible solution to integrate thin selective membranes into membrane reactors while protecting the selective layer. This work is focused on recent advances on metallic supports, materials used as an intermetallic diffusion layer when metallic supports are used and the most recent advances on Pd-based composite membranes. Particular attention is paid to improvements on sulfur resistance of Pd based membranes, resistance to hydrogen embrittlement and stability at high temperature.},
author = {Plazaola, Alba Arratibel and Tanaka, David Alfredo Pacheco and Annaland, Martin Van Sint and Gallucci, Fausto},
doi = {10.3390/molecules22010051},
file = {:Users/marc/Library/Application Support/Mendeley Desktop/Downloaded/Alba Arratibel Plazaola, David Alfredo Pacheco Tanaka - 2017 - Recent Advances in Pd-Based Membranes for Membrane Reactors.pdf:pdf},
isbn = {9781608055050},
issn = {14203049},
journal = {Molecules},
keywords = {Inorganic membranes,Membrane reactor,Membranes,Palladium membranes,Pore-filled},
number = {1},
pages = {1--53},
pmid = {28045434},
title = {{Recent advances in pd-based membranes for membrane reactors}},
volume = {22},
year = {2017}
}
@article{Rodatz2003,
author = {Rodatz, Paul Hendrik},
file = {:Users/marc/Library/Application Support/Mendeley Desktop/Downloaded/Rodatz - 2003 - Dynamics of the Polymer Electrolyte Fuel Cell Experiments and Model-Based Analysis.pdf:pdf},
number = {15320},
title = {{Dynamics of the Polymer Electrolyte Fuel Cell: Experiments and Model-Based Analysis}},
year = {2003}
}
@article{Peters2013,
annote = {NULL},
author = {Peters, T A and Kaleta, T and Stange, M and Bredesen, R},
doi = {10.1016/j.memsci.2012.11.062},
isbn = {03767388},
journal = {Journal of Membrane Science},
pages = {448--458},
title = {{Development of ternary Pd–Ag–TM alloy membranes with improved sulphur tolerance}},
volume = {429},
year = {2013}
}
@techreport{Hayter2014,
author = {Hayter, Dennis and Energy, Intelligent},
number = {September},
title = {{Global H2Mobility initiatives – what they mean for FCEV introduction}},
year = {2014}
}
@article{Darmawan2016a,
abstract = {This work investigates a modified sol-gel method for the preparation of interlayer-free nickel oxide silica membranes for desalination applications. The sol–gels were synthesized using TEOS, nickel nitrate hexahydrate, ethanol as solvent and water with and without peroxide (H2O2). The effect of the nickel embedded in the silica matrix as Ni/Si molar ratio was varied from 5 to 50 mol{\%} and systematically studied. The sols prepared with H2O2resulted in microporous structures and lower pore volume, contrary to the mesoporous structures derived from sols without H2O2. The modified sol-gel method proved to be robust enough for coating directly on $\alpha$-alumina substrates, as opposed to conventional methods which required substrates with interlayers. All interlayer-free nickel oxide silica membranes delivered high salt rejection ranging from 91.5 to 99.9{\%}, and reaching water flux as high as 7.3 kg m− 2 h− 1. The membranes prepared using sols with H2O2gave lower water flux and slightly higher salt rejection, attributed to lower pore volume and smaller pore size, respectively. The membranes prepared with Ni/Si molar ratio of 25{\%} achieved the highest water flux, though salt rejection slightly decreased with the increase of feed salt concentration from brackish (NaCl 0.3 wt{\%}) to sea water (NaCl 3.5 wt{\%}).},
author = {Darmawan, Adi and Karlina, Linda and Astuti, Yayuk and Sriatun and Motuzas, Julius and Wang, David K. and da Costa, Jo{\~{a}}o C.Diniz},
doi = {10.1016/j.jnoncrysol.2016.05.031},
file = {:Users/marc/Library/Application Support/Mendeley Desktop/Downloaded/Darmawan et al. - 2016 - Structural evolution of nickel oxide silica sol-gel for the preparation of interlayer-free membranes.pdf:pdf},
issn = {00223093},
journal = {Journal of Non-Crystalline Solids},
keywords = {Desalination,Membranes,Nickel oxide,Silica,Sol-gel},
pages = {9--15},
publisher = {Elsevier B.V.},
title = {{Structural evolution of nickel oxide silica sol-gel for the preparation of interlayer-free membranes}},
url = {http://dx.doi.org/10.1016/j.jnoncrysol.2016.05.031},
volume = {447},
year = {2016}
}
@incollection{RoaF.ThoenP.M.GadeS.K.WayJ.G.DeVossS.andAlptekin2009,
author = {{Roa, F., Thoen, P.M., Gade, S.K., Way, J.G., De Voss, S., and Alptekin}, G.},
booktitle = {Inorganic Membranes for Energy and Environmental Applications},
pages = {211},
title = {{Palladium-Copper and Palladium-Gold alloy Composite membranes for hydrogen separations}},
year = {2009}
}
@misc{Mineralprices.com2016,
author = {Mineralprices.com},
number = {06/12/2016},
title = {{Rare Earth Metal Prices}},
url = {http://mineralprices.com/default.aspx{\#}rar},
volume = {2016},
year = {2016}
}
@article{Cheng2002,
author = {Cheng, Y S and Pena, M A and Fierro, J L and Hui, D C W and Yeung, K L},
file = {:Users/marc/Library/Application Support/Mendeley Desktop/Downloaded/Cheng et al. - 2002 - Performance of alumina, zeolite, palladium, Pd–Ag alloy membranes for hydrogen separation from Towngas mixture.pdf:pdf},
journal = {Journal of Membrane Science},
pages = {329--340},
title = {{Performance of alumina, zeolite, palladium, Pd–Ag alloy membranes for hydrogen separation from Towngas mixture}},
volume = {204},
year = {2002}
}
@article{Zhang2017b,
abstract = {A sub-ppm-level CO gas sensor based on copper oxide (CuO)-decorated graphene hybrid nanocomposite was reported in this paper. The CuO/graphene hierarchical nanocomposite was successfully deposited on a substrate with interdigital microelectrodes via layer-by-layer self-assembly technique. The morphologies, microstructures and compositions of the as-prepared CuO/rGO film were sufficiently characterized by scanning electron microscopy (SEM), transmission electron microscope (TEM) and X-ray diffraction (XRD). The gas sensing properties of the CuO/graphene nanocomposite was investigated at room temperature over a wide range concentration of CO gas from 0.25 ppm to 1000 ppm. The experimental results exhibited not only an unprecedented detection abilities, but also fast response and recovery times, excellent repeatability, good stability and selectivity. The superior sensing mechanism for the presented sensor was ascribed to the hierarchical porous nanostructure and the formed heterojunction at the interfaces between CuO nanoflowers and rGO nanosheets.},
author = {Zhang, Dongzhi and Jiang, Chuanxing and Liu, Jingjing and Cao, Yuhua},
doi = {10.1016/j.snb.2017.03.108},
file = {:Users/marc/Library/Application Support/Mendeley Desktop/Downloaded/Zhang et al. - 2017 - Carbon monoxide gas sensing at room temperature using copper oxide-decorated graphene hybrid nanocomposite prepare.pdf:pdf},
issn = {09254005},
journal = {Sensors and Actuators, B: Chemical},
keywords = {Carbon monoxide,Copper oxide,Gas sensor,Reduced graphene oxide},
pages = {875--882},
publisher = {Elsevier B.V.},
title = {{Carbon monoxide gas sensing at room temperature using copper oxide-decorated graphene hybrid nanocomposite prepared by layer-by-layer self-assembly}},
url = {http://dx.doi.org/10.1016/j.snb.2017.03.108},
volume = {247},
year = {2017}
}
@article{Briceno2012,
abstract = {A high molecular weight polyimide (Matrimid) was used as a precursor for fabricating supported carbon molecular sieve membranes without crack formation at 550-700°C pyrolysis temperature. A one-step polymer (polyimide) coating method as precursor of carbon layer was used without needing a prior modification of a TiO2macroporous support. The following fabrication variables were optimized and studied to determine their effect on the carbon structure: polymeric solution concentration, solvent extraction, heating rate and pyrolysis temperature. Two techniques (Thermogravimetric analysis and Raman spectroscopy) were used to determine these effects on final carbon structure. Likewise, the effect of the support was also reported as an additional and important variable in the design of supported carbon membranes. Atomic force microscopy and differential scanning calorimetry quantified the degree of influence. Pure gas permeation tests were performed using CH4, CO, CO2and H2. The presence of a molecular sieving mechanism was confirmed after defects were plugged with PDMS solution at 12wt{\%}. Gas selectivities higher than Knudsen theoretical values were reached with membranes obtained over 650°C, showing as best values 4.46, 4.70 and 10.62 for H2/N2,H2/CO and H2/CH4ratio, respectively. Permeance values were over 9.82×10-9mol/(m2Pas)during pure hydrogen permeation tests. {\textcopyright} 2012 Elsevier B.V.},
author = {Brice{\~{n}}o, Kelly and Montan{\'{e}}, Daniel and Garcia-Valls, Ricard and Iulianelli, Adolfo and Basile, Angelo},
doi = {10.1016/j.memsci.2012.05.015},
file = {:Users/marc/Library/Application Support/Mendeley Desktop/Downloaded/Brice{\~{n}}o et al. - 2012 - Fabrication variables affecting the structure and properties of supported carbon molecular sieve membranes for.pdf:pdf},
issn = {03767388},
journal = {Journal of Membrane Science},
keywords = {Carbon supported membranes,Ceramic tubular support,Fabrication variables,Gas separation,Hydrogen separation,Materials for membrane reactors,Matrimid membranes,Polyimide membranes},
pages = {288--297},
title = {{Fabrication variables affecting the structure and properties of supported carbon molecular sieve membranes for hydrogen separation}},
volume = {415-416},
year = {2012}
}
@article{Ostwal2018,
abstract = {Graphene oxide – molybdenum disulfide hybrid membranes were prepared using vacuum filtration technique. The thickness and the MoS2content in the membranes were varied and their H2permeance and H2/CO2selectivity are reported. A 60 nm hybrid membrane containing {\~{}} 75{\%} by weight of MoS2exhibited the highest H2permeance of 804 × 10−9mol/m2s Pa with corresponding H2/CO2selectivity of 26.7; while a 150 nm hybrid membrane with {\~{}} 29{\%} MoS2showed the highest H2/CO2selectivity of 44.2 with corresponding H2permeance of 287 × 10−9mol/m2s Pa. The hybrid membranes exhibited much higher H2permeance compared to graphene oxide membranes and higher selectivity compared to MoS2membranes, which fully demonstrated the synergistic effect of both nanomaterials. The membranes also displayed excellent operational long-term stability.},
author = {Ostwal, Mayur and Shinde, Digambar B. and Wang, Xinbo and Gadwal, Ikhlas and Lai, Zhiping},
doi = {10.1016/j.memsci.2017.12.063},
file = {:Users/marc/Library/Application Support/Mendeley Desktop/Downloaded/Ostwal et al. - 2018 - Graphene oxide – molybdenum disulfide hybrid membranes for hydrogen separation.pdf:pdf},
issn = {18733123},
journal = {Journal of Membrane Science},
keywords = {Composite membranes,Gas separation,Graphene oxide,Molybdenum disulfide,Vacuum filtration},
number = {December 2017},
pages = {145--154},
publisher = {Elsevier B.V.},
title = {{Graphene oxide – molybdenum disulfide hybrid membranes for hydrogen separation}},
url = {https://doi.org/10.1016/j.memsci.2017.12.063},
volume = {550},
year = {2018}
}
@article{PLUNKEYTY2018b,
abstract = {hgjvhgjkhkjghkjbn},
author = {PLUNKEYTY, MARC and PLUNKEYTY, MARC and PLUNKEYTY, MARC and PLUNKEYTY, MARC},
doi = {sdfdsfsd},
journal = {JMS},
title = {sdfdsfsd},
url = {http://sdfsdfsd},
year = {2018}
}
@article{Mardilovich2002,
abstract = {The influence of the support properties on the characteristics of Pd/PSS composite membranes has been evaluated for a large group of membrane samples prepared by electroless plating. The H2 permeation in the membranes was found to follow Sievert's law between 325-500??C. The steady state H2 flux was stable for over 200 h. The hydrogen permeance was measured at 350??C with a 1-atm pressure difference and ranged from 20.0 to 5.1 m3/(m2??h??atm0.5) for a Pd layer thickness of 11.7 and 33.8 ??m, respectively. The thickness of the membranes and therefore the permeance obtained was dependent on the size of the largest pores present in the support. The thickness of the Pd layer was approximately three times the dimension of the largest pores in the support, consistent with the previous theoretical results of Ma et al. [3].},
author = {Mardilovich, Ivan P. and Engwall, Erik and Ma, Yi Hua},
doi = {10.1016/S0011-9164(02)00293-X},
file = {:Users/marc/Library/Application Support/Mendeley Desktop/Downloaded/Mardilovich, Engwall, Ma - 2002 - Dependence of hydrogen flux on the pore size and plating surface topology of asymmetric Pd-porous stai.pdf:pdf},
issn = {00119164},
journal = {Desalination},
keywords = {Composite palladium membrane,Electroless plating,Porous stainless steel},
number = {1-3},
pages = {85--89},
title = {{Dependence of hydrogen flux on the pore size and plating surface topology of asymmetric Pd-porous stainless steel membranes}},
volume = {144},
year = {2002}
}
@article{Garcia-Garcia2012,
author = {Garc{\'{i}}a-Garc{\'{i}}a, F R and Torrente-Murciano, L and Chadwick, D and Li, K},
doi = {10.1016/j.memsci.2012.02.031},
isbn = {03767388},
journal = {Journal of Membrane Science},
pages = {30--37},
title = {{Hollow fibre membrane reactors for high H2 yields in the WGS reaction}},
volume = {405-406},
year = {2012}
}
@article{Coulter2012,
abstract = {Palladium (Pd) based alloy membranes of PdAu and PdAuPt were fabricated using magnetron sputtering. Membranes ranging in thickness from 9 to 33$\mu$m were tested under pure gas and mixed gas conditions including U.S. Department of Energy (DOE) specified simulated synthesis gas mixtures containing H 2, H 2O, CO and CO 2, and up to 75ppmv H 2S. In pure hydrogen gas, the PdAu alloy exhibited higher permeabilities than pure Pd in agreement with published values for cold rolled and electrodeposited membranes of similar compositions. The addition of Pt in concentrations ranging from 4.3 to 12.9mass{\%} lowered the permeability in pure hydrogen gas to slightly below pure Pd. For thicker (25-33$\mu$m) PdAuPt membranes pretreated in air (versus nitrogen) at 400°C before exposure to hydrogen, an increase in permeability of 10-20{\%} was observed, which is lower than that reported in the literature suggesting that some surface properties of thinner foils may boost permeability. A thick (33$\mu$m) Pd-10Au-10Pt foil evaluated under the DOE conditions, showed good recovery in flux after exposure to syngas containing sulfur. The flux dropped from 0.212 to 0.167molm -2s -1 when 20ppmv of H 2S was added to the gas mixture. When the H 2S was removed from the gas mixture the flux returned to its original value of 0.212molm -2s -1. However, in the presence of H 2S the permeate purity somewhat degraded over time. The hydrogen permeability, as observed for other membranes, was depressed both by water-gas shift mixtures (with or without H 2S) but it appears that either high quantities of Au or the presence of Pt, or both, are inhibiting sulfide formation, thereby improving both permeability and stability under mixture gas conditions with H 2S. {\textcopyright} 2012 Elsevier B.V.},
author = {Coulter, Kent E. and Way, J. Douglas and Gade, Sabina K. and Chaudhari, Saurabh and Alptekin, G{\"{o}}khan O. and DeVoss, Sarah J. and Paglieri, Stephen N. and Pledger, Bill},
doi = {10.1016/j.memsci.2012.02.018},
file = {:Users/marc/Library/Application Support/Mendeley Desktop/Downloaded/Coulter et al. - 2012 - Sulfur tolerant PdAu and PdAuPt alloy hydrogen separation membranes.pdf:pdf},
issn = {03767388},
journal = {Journal of Membrane Science},
keywords = {Hydrogen,PdAu,PdAuPt,Sulfur tolerant},
pages = {11--19},
publisher = {Elsevier B.V.},
title = {{Sulfur tolerant PdAu and PdAuPt alloy hydrogen separation membranes}},
url = {http://dx.doi.org/10.1016/j.memsci.2012.02.018},
volume = {405-406},
year = {2012}
}
@article{Lin2013,
abstract = {Zeolite membranes possess a highly defined microporous structure which is capable of very unique separation principally based around molecular size exclusion, adsorption, and diffusion. Polycrystalline membranes of different zeolites have been prepared in the laboratory. These zeolite membranes show excellent separation performance for mixture with a strongly adsorbing component. These unique properties have let to industrial application of zeolite membranes for pervaporation separation of liquid mixtures. However, zeolite membranes exhibit somewhat lower diffusion-controlled selectivity for gas mixtures or unstable separation characteristics due to adsorption induced microstructural changes. These problems together with high membrane costs have hindered industrial applications of zeolite membranes for gas separations. Recent research has been directed toward synthesis of zeolite membranes by more efficient methods or on more cost-effective supports and microstructural engineering of the zeolite membranes. These efforts include understanding the formation of the non-zeolitic pores, developing methods to seal these defects, and synthesis of thin oriented zeolite membranes for improving separation performance. Research was also conducted to extend applications of zeolite membranes from gas and liquid separation to nanofiltration or desalination. {\textcopyright} 2013 Elsevier Ltd.},
author = {Lin, Ys and Duke, Mikel C.},
doi = {10.1016/j.coche.2013.03.002},
file = {:Users/marc/Library/Application Support/Mendeley Desktop/Downloaded/Lin, Duke - 2013 - Recent progress in polycrystalline zeolite membrane research.pdf:pdf},
isbn = {2211-3398},
issn = {22113398},
journal = {Current Opinion in Chemical Engineering},
number = {2},
pages = {209--216},
publisher = {Elsevier Ltd},
title = {{Recent progress in polycrystalline zeolite membrane research}},
url = {http://dx.doi.org/10.1016/j.coche.2013.03.002},
volume = {2},
year = {2013}
}
@article{Wang2015c,
abstract = {A highly permselective ZIF-100 molecular sieve membrane has been prepared on a polydopamine (PDA)-modified support. Attributed to the formation of strong covalent and non-covalent bonds between PDA and ZIF-100{\{},{\}} the ZIF-100 nutrients are attracted and bound to the support surface{\{},{\}} thus promoting the growth of well-intergrown and phase-pure ZIF-100 membranes. The developed ZIF-100 membranes show high H2/CO2 selectivity due to the outstanding CO2 adsorption capacity of ZIF-100.},
author = {Wang, Nanyi and Liu, Yi and Qiao, Zhiwei and Diestel, Lisa and Zhou, Jian and Huang, Aisheng and Caro, J{\"{u}}rgen},
doi = {10.1039/c4ta06763k},
file = {:Users/marc/Library/Application Support/Mendeley Desktop/Downloaded/Wang et al. - 2015 - Polydopamine-based synthesis of a zeolite imidazolate framework ZIF-100 membrane with high H2CO2 selectivity.pdf:pdf},
isbn = {2050-7488},
issn = {20507496},
journal = {Journal of Materials Chemistry A},
number = {8},
pages = {4722--4728},
title = {{Polydopamine-based synthesis of a zeolite imidazolate framework ZIF-100 membrane with high H2/CO2 selectivity}},
volume = {3},
year = {2015}
}
@article{Cuong2010,
abstract = {We report a solution-processed gas sensor based on vertically aligned ZnO nanorods (NRs) on a chemically converted graphene (CCG) film. The prepared sensor device effectively detected 2 ppm of H
                        2S in oxygen at room temperature. A high sensitivity of the gas sensor resulted from the growth of highly dense vertical ZnO NRs on the CCG film with numerous tiny white dots on its surface, which may provide a sufficient number of sites for the nucleation and growth of the ZnO NRs. The adsorption of oxygen on the surface of the ZnO NRs was found to be crucial for obtaining an excellent gas sensing performance of the ZnO NRs-CCG sensor. ?? 2010 Elsevier B.V. All rights reserved.},
author = {Cuong, Tran Viet and Pham, Viet Hung and Chung, Jin Suk and Shin, Eun Woo and Yoo, Dae Hwang and Hahn, Sung Hong and Huh, Jeung Soo and Rue, Gi Hong and Kim, Eui Jung and Hur, Seung Hyun and Kohl, Paul A.},
doi = {10.1016/j.matlet.2010.08.027},
file = {:Users/marc/Library/Application Support/Mendeley Desktop/Downloaded/Cuong et al. - 2010 - Solution-processed ZnO-chemically converted graphene gas sensor.pdf:pdf},
isbn = {0167577X},
issn = {0167577X},
journal = {Materials Letters},
keywords = {Graphene films,Sensors,Solution synthesis,ZnO nanorods},
number = {22},
pages = {2479--2482},
publisher = {Elsevier B.V.},
title = {{Solution-processed ZnO-chemically converted graphene gas sensor}},
url = {http://dx.doi.org/10.1016/j.matlet.2010.08.027},
volume = {64},
year = {2010}
}
@article{Løvvik2005,
abstract = {The surface segregation in Pd based alloys has been investigated by density-functional band-structure calculations. Twelve different metals were substituted in Pd(1 1 1) slabs at a level of 5{\{}{\%}{\}}: Ag, Au, Cd, Cu, Fe, Mn, Ni, Pb, Pt, Rh, Ru, and Sn. The segregation energy (the difference in calculated total free energy between surface sites and bulk-like sites) was calculated for each alloy, and the results were in very good agreement with experimental data where available. Particularly, we predict an oscillatory depth profile for Cu and Ni, similar to what has been found experimentally for the PdNi(1 0 0) surface. There are more or less pronounced correlations between the segregation energy and the relaxed position at the surface of the substituted atom, the metal radius of the substituted atom, and the experimental surface energy of the metal. It is proposed that the segregation energy is an indirect measure of the stability of Pd based hydrogen permeable membranes. {\{}{\textcopyright}{\}} 2005 Elsevier B.V. All rights reserved.},
author = {L{\o}vvik, O M},
doi = {10.1016/j.susc.2005.03.028},
issn = {00396028},
journal = {Surface Science},
keywords = {Alloys,Density functional calculations,Palladium,Surface segregation},
number = {1},
pages = {100--106},
title = {{Surface segregation in palladium based alloys from density-functional calculations}},
volume = {583},
year = {2005}
}
@article{Barbosa2017,
abstract = {The gas sensor response of tin monoxide micro-disks, functionalized with noble metal nanoparticles (Pd and Ag), to NO2, H2 and CO were studied by monitoring changes in their resistance upon exposure to the various gases. The tin monoxide, with unusually low Sn oxidation state, was synthetized by carbothermal reduction. Surface modification by Pd and Ag catalysts was achieved by coating the micro-disks by metallic nanoparticle dispersions, prepared by the polyol reduction process, followed by thermal treatment. SEM and TEM analysis showed nanoparticles to be well-dispersed over the SnO surfaces. The decorated SnO micro-disks exhibited high sensor response to reducing gases such as H2 and CO. On the other hand, the catalytic particles tended to reduce the sensor response to oxidizing gases such as NO2. The catalytic activity of Pd nanoparticles was tied to chemical sensitization while that of Ag nanoparticles to electronic sensitization. Impedance spectroscopy enabled deconvolution of different contributions to the sensor response with only the Ag-decorated specimens exhibiting two RC time constants. Thus, in contrast to undecorated and Pd-decorated specimens, nearly 80{\%} of Ag modified SnO's response to H2 was controlled by changes in the interface between particles and disks. Sensor response to H2 was optimal at higher temperatures (300????C), NO2 at 200????C while that for Pd-decorated materials; maximum sensor response to CO was observed at lower temperatures (under 150????C), where CO absorption by metal nanoparticles is favored.},
author = {Barbosa, Martin S. and Suman, Pedro H. and Kim, Jae J. and Tuller, Harry L. and Varela, Jos?? A. and Orlandi, Marcelo O.},
doi = {10.1016/j.snb.2016.07.157},
file = {:Users/marc/Library/Application Support/Mendeley Desktop/Downloaded/Barbosa et al. - 2017 - Gas sensor properties of Ag- and Pd-decorated SnO micro-disks to NO2, H2 and CO Catalyst enhanced sensor respons.pdf:pdf},
issn = {09254005},
journal = {Sensors and Actuators, B: Chemical},
keywords = {Catalyst,Gas sensors,Impedance spectroscopy,Sensitization,SnO},
pages = {253--261},
publisher = {Elsevier B.V.},
title = {{Gas sensor properties of Ag- and Pd-decorated SnO micro-disks to NO2, H2 and CO: Catalyst enhanced sensor response and selectivity}},
url = {http://dx.doi.org/10.1016/j.snb.2016.07.157},
volume = {239},
year = {2017}
}
@article{Bryden2002,
abstract = {Hydrogen permselectivity and poisoning resistance of nanostructured palladium-iron alloy films were examined to evaluate these materials for potential membrane applications. At 200??C, nanostructured palladium-iron films had hydrogen/helium selectivities of up to 35:1 and hydrogen fluxes of up to 10sccm/cm2. The rate-limiting step for hydrogen transport was diffusion through the bulk metal. Hydrogen flux in the nanostructured palladium-based membranes decreased with increasing iron content. The nanocrystalline membranes had higher hydrogen fluxes than coarse-grained polycrystalline systems of similar compositions. Nanostructured membranes also exhibited better resistance to hydrogen sulfide poisoning than polycrystalline membranes. When applied for the hydrogenation of ethene, the palladium-iron nanocrystalline membranes displayed stable activities over long time periods. ?? 2002 Elsevier Science B.V. All rights reserved.},
author = {Bryden, Kenneth J. and Ying, Jackie Y.},
doi = {10.1016/S0376-7388(01)00736-0},
file = {:Users/marc/Library/Application Support/Mendeley Desktop/Downloaded/Bryden, Ying - 2002 - Nanostructured palladium-iron membranes for hydrogen separation and membrane hydrogenation reactions.pdf:pdf},
isbn = {1617452343},
issn = {03767388},
journal = {Journal of Membrane Science},
keywords = {Gas separations,Membrane reactors,Metal membranes,Nanostructured materials,Poisoning},
number = {1-2},
pages = {29--42},
title = {{Nanostructured palladium-iron membranes for hydrogen separation and membrane hydrogenation reactions}},
volume = {203},
year = {2002}
}
@article{Chen2008,
author = {Chen, S C and Tu, G C and Hung, Caryat C Y and Huang, C A and Rei, M H},
doi = {10.1016/j.memsci.2007.12.066},
file = {:Users/marc/Library/Application Support/Mendeley Desktop/Downloaded/Chen et al. - 2008 - Preparation of palladium membrane by electroplating on AISI 316L porous stainless steel supports and its use for me.pdf:pdf},
isbn = {03767388},
journal = {Journal of Membrane Science},
number = {1-2},
pages = {5--14},
title = {{Preparation of palladium membrane by electroplating on AISI 316L porous stainless steel supports and its use for methanol steam reformer}},
volume = {314},
year = {2008}
}
@techreport{Brown2011,
author = {Brown, Andrew S and Vargha, Gergely M and Downer, Michaek L and Hart, Nick J and Ferrier, Gordon G and Hall, Karen I},
booktitle = {NPL Report AS 64},
file = {:Users/marc/Library/Application Support/Mendeley Desktop/Downloaded/Brown et al. - 2011 - NPL report AS 64 - Methods for the analysis of trace-level impurities in hydrogen for fuel cell applications.pdf:pdf},
publisher = {National Physical Laboratory},
title = {{Methods for the analysis of trace-level impurities in hydrogen for fuel cell applications}},
year = {2011}
}
@article{Foletto2008,
author = {Foletto, E. L. and {Wirbitzki Da Silveira}, J. V. and Jahn, S. L.},
doi = {10.1080/00986440213885},
file = {:Users/marc/Library/Application Support/Mendeley Desktop/Downloaded/Foletto, Wirbitzki Da Silveira, Jahn - 2008 - Preparation of palladium-silver alloy membranes for hydrogen permeation.pdf:pdf},
issn = {03270793},
journal = {Latin American Applied Research},
keywords = {Hydrogen permeation,Membrane preparation,Palladium-silver},
number = {1},
pages = {79--84},
title = {{Preparation of palladium-silver alloy membranes for hydrogen permeation}},
volume = {38},
year = {2008}
}
@article{Xu2014,
author = {Xu, Nong and Kim, Sung Su and Li, Anwu and Grace, John R. and Lim, C. Jim and Boyd, Tony},
doi = {10.1002/cjce.21954},
file = {:Users/marc/Library/Application Support/Mendeley Desktop/Downloaded/Xu et al. - 2014 - Preparation and characterization of palladium-ruthenium composite membrane on alumina-modified PSS substrate.pdf:pdf},
issn = {1939019X},
journal = {Canadian Journal of Chemical Engineering},
keywords = {Composite membrane,Electroless plating,Hydrogen separation,Palladium (Pd),Ruthenium (Ru)},
number = {6},
pages = {1041--1047},
title = {{Preparation and characterization of palladium-ruthenium composite membrane on alumina-modified PSS substrate}},
volume = {92},
year = {2014}
}
@article{Li2012,
abstract = {A review. This article reviews the metal-org. frameworks (MOFs), a new class of porous solid materials. Their performances for applications in sepns. and purifications is attracting intense interest of researchers working in the fields of chem., chem. engineering, materials science, and others. Despite being in its infancy, the research progress in this subject has already shown that MOFs are promising for sepn. applications. [on SciFinder(R)]},
archivePrefix = {arXiv},
arxivId = {arXiv:1408.1149},
author = {Li, Jian Rong and Sculley, Julian and Zhou, Hong Cai},
doi = {10.1021/cr200190s},
eprint = {arXiv:1408.1149},
file = {:Users/marc/Library/Application Support/Mendeley Desktop/Downloaded/Li, Sculley, Zhou - 2012 - Metal-organic frameworks for separations.pdf:pdf},
isbn = {0009-2665},
issn = {00092665},
journal = {Chemical Reviews},
number = {2},
pages = {869--932},
pmid = {21978134},
title = {{Metal-organic frameworks for separations}},
volume = {112},
year = {2012}
}
@article{She2014,
author = {She, Ying and Emerson, Sean C and Magdefrau, Neal J and Opalka, Susanne M and Thibaud-Erkey, Catherine and Vanderspurt, Thomas H},
doi = {10.1016/j.memsci.2013.09.025},
file = {:Users/marc/Library/Application Support/Mendeley Desktop/Downloaded/She et al. - 2014 - Hydrogen permeability of sulfur tolerant Pd–Cu alloy membranes.pdf:pdf},
isbn = {03767388},
journal = {Journal of Membrane Science},
pages = {203--211},
title = {{Hydrogen permeability of sulfur tolerant Pd–Cu alloy membranes}},
volume = {452},
year = {2014}
}
@article{Ebrahim2015,
abstract = {New composites contg. Cu-BTC and S- and N-doped graphite oxides (GOs) were synthesized.  The composites were evaluated as adsorbents of H2S under ambient conditions.  The texture and chem. of the initial samples and those exposed to H2S were analyzed using a wide range of anal. techniques (XRD, SEM-EDX, FTIR, thermal anal.-MS, and nitrogen adsorption).  The performance of the new composites as H2S adsorbents was much better than that of the parent MOFs.  It was owing to the formation of new microporosity as a result of linkages between the sulfonic acids and amine groups of modified GO and copper centers Cu-BTC.  The presence of moisture in the pore system increased the amt. adsorbed.  Phys. adsorption and reactive adsorption play an important role in the mechanism of retention.  The removal of H2S is favored on the composites with a higher degree of surface heterogeneity, which facilitates the retention of H2S mols. mostly in the form of sulfides.  Given the differences in the chem. of the N- and S-contg. groups in the composites, distinct mechanisms of adsorption occur, which result in sulfides/sulfates of unique morphologies.  Nitrogen functional groups catalyze the formation of superoxide ions on the graphene phase, resulting in the partial oxidn. of H2S and in the release of SO2. [on SciFinder(R)]},
author = {Ebrahim, Amani M. and Jagiello, Jacek and Bandosz, Teresa J.},
doi = {10.1039/C5TA01359C},
file = {:Users/marc/Library/Application Support/Mendeley Desktop/Downloaded/Ebrahim, Jagiello, Bandosz - 2015 - Enhanced reactive adsorption of H sub2sub S on Cu–BTC S- and N-doped GO composites.pdf:pdf},
isbn = {2050-7488},
issn = {2050-7488},
journal = {J. Mater. Chem. A},
number = {15},
pages = {8194--8204},
publisher = {Royal Society of Chemistry},
title = {{Enhanced reactive adsorption of H {\textless}sub{\textgreater}2{\textless}/sub{\textgreater} S on Cu–BTC/ S- and N-doped GO composites}},
url = {http://xlink.rsc.org/?DOI=C5TA01359C},
volume = {3},
year = {2015}
}
@article{Kim2013,
abstract = {Graphene is a distinct two-dimensional material that offers a wide range of opportunities for membrane applications because of ultimate thinness, flexibility, chemical stability, and mechanical strength. We demonstrate that few- and several-layered graphene and graphene oxide (GO) sheets can be engineered to exhibit the desired gas separation characteristics. Selective gas diffusion can be achieved by controlling gas flow channels and pores via different stacking methods. For layered (3- to 10-nanometer) GO membranes, tunable gas transport behavior was strongly dependent on the degree of interlocking within the GO stacking structure. High carbon dioxide/nitrogen selectivity was achieved by well-interlocked GO membranes in high relative humidity, which is most suitable for postcombustion carbon dioxide capture processes, including a humidified feed stream.},
annote = {Kim, Hyo Won
Yoon, Hee Wook
Yoon, Seon-Mi
Yoo, Byung Min
Ahn, Byung Kook
Cho, Young Hoon
Shin, Hye Jin
Yang, Hoichang
Paik, Ungyu
Kwon, Soongeun
Choi, Jae-Young
Park, Ho Bum
ENG
Research Support, Non-U.S. Gov't
Research Support, U.S. Gov't, Non-P.H.S.
2013/10/05 06:00
Science. 2013 Oct 4;342(6154):91-5. doi: 10.1126/science.1236098.},
author = {Kim, H W and Yoon, H W and Yoon, S M and Yoo, B M and Ahn, B K and Cho, Y H and Shin, H J and Yang, H and Paik, U and Kwon, S and Choi, J Y and Park, H B},
doi = {10.1126/science.1236098},
isbn = {1095-9203 (Electronic)
0036-8075 (Linking)},
journal = {Science},
number = {6154},
pages = {91--95},
pmid = {24092738},
title = {{Selective gas transport through few-layered graphene and graphene oxide membranes}},
url = {http://www.ncbi.nlm.nih.gov/pubmed/24092738},
volume = {342},
year = {2013}
}
@article{Rangnekar2015,
abstract = {The latest developments in zeolite membranes are reviewed, with an$\backslash$nemphasis on the synthesis techniques, including seed assembly and$\backslash$nsecondary growth methods. This review also discusses the current$\backslash$nindustrial applications of zeolite membranes, the feasibility of their$\backslash$nuse in membrane reactors and their hydrothermal stability. Finally,$\backslash$nzeolite membranes are compared with metal-organic framework (MOF)$\backslash$nmembranes and the latest advancements in MOF and mixed matrix membranes$\backslash$nare highlighted.},
author = {Rangnekar, Neel and Mittal, Nitish and Elyassi, Bahman and Caro, Juergen and Tsapatsis, Michael},
doi = {10.1039/c5cs00292c},
isbn = {0306-0012},
issn = {14604744},
journal = {Chemical Society Reviews},
number = {20},
pages = {7128--7154},
pmid = {26155855},
publisher = {Royal Society of Chemistry},
title = {{Zeolite membranes - a review and comparison with MOFs}},
url = {http://dx.doi.org/10.1039/C5CS00292C},
volume = {44},
year = {2015}
}
@article{Musket1976,
author = {Musket, R G},
file = {:Users/marc/Library/Application Support/Mendeley Desktop/Downloaded/Musket - 1976 - Effects of contamination on the interaction of hydrogen gas with palladium A Review.pdf:pdf},
journal = {Journal of Less Common Metals},
pages = {173--183},
title = {{Effects of contamination on the interaction of hydrogen gas with palladium: A Review}},
volume = {45},
year = {1976}
}
@misc{JohnsomMattheyPreciousMetalsManagement2016,
author = {{Johnsom Matthey Precious Metals Management}},
number = {06/12/2016},
title = {{Price Tables}},
volume = {2016},
year = {2016}
}
@article{Uhlmann2010,
abstract = {In this work we investigate the effect of high temperature steam on cobalt and cobalt oxide derived silica with an aim to providing an understanding of the permeation and gas separation performance of cobalt silica membranes exposed to simulated industrial wet gas streams. Cobalt silica (CoSi) and cobalt oxide silica (CoOxSi) xerogels were synthesized and exposed to steam at 500 °C for varying time periods. Subsequent characterization with FTIR and N2adsorption revealed that CoOxSi xerogels were significantly more hydrostable than CoSi xerogels, with few structural changes observed and only moderate densification (∼43{\%}) of the CoOxSi matrix experienced even after the longest steam exposure. In comparison the CoSi matrix experienced severe densification (∼89{\%}) after only short term steam exposure. CoOxSi was therefore selected as the optimal material for further membrane tests and CoOxSi membranes were subsequently synthesized using sol-gel techniques and exposed to steam in both high temperature long term stability studies and during temperature cycling tests. Exposure to steam had an adverse effect on membrane performance with the largest effect occurring during the initial stages where He permeance dropped from 3.95 × 10-8to 2.05 × 10-10mol m-2s-1Pa-1and He/N2ideal selectivities from 123 to 45, respectively after 135 h of testing. However, following this initial period of steam conditioning, these membranes were able to oppose further deterioration and maintain acceptable steady state performance. Further tests with an increased steam rate showed that nitrogen permeation decreased more significantly than helium, leading to a rise in membrane ideal selectivity, suggesting that steam was competitively blocking the pores available to nitrogen adsorption and/or reducing the diffusion of nitrogen. Steam testing under cycling temperature conditions showed that below 400 °C, the membrane ideal selectivity was low. However, above 400 °C the membrane consistently exhibited a molecular sieving mechanism. In addition water permeation through the membrane varied with temperature suggesting that the membrane matrix continuously underwent reversible structural modification by the combined effect of water and temperature. {\textcopyright} 2010 Elsevier B.V. All rights reserved.},
author = {Uhlmann, David and Smart, Simon and {Diniz Da Costa}, Jo{\~{a}}o C.},
doi = {10.1016/j.seppur.2010.10.004},
file = {:Users/marc/Library/Application Support/Mendeley Desktop/Downloaded/Uhlmann, Smart, Diniz Da Costa - 2010 - High temperature steam investigation of cobalt oxide silica membranes for gas separation.pdf:pdf},
isbn = {1383-5866},
issn = {13835866},
journal = {Separation and Purification Technology},
keywords = {Cobalt oxide silica,Gas separation,Hydrothermal stability},
number = {2},
pages = {171--178},
publisher = {Elsevier B.V.},
title = {{High temperature steam investigation of cobalt oxide silica membranes for gas separation}},
url = {http://dx.doi.org/10.1016/j.seppur.2010.10.004},
volume = {76},
year = {2010}
}
@article{Li2009,
abstract = {Adsorptive separation is very important in industry. Generally, the process uses porous solid materials such as zeolites, activated carbons, or silica gels as adsorbents. With an ever increasing need for a more efficient, energy-saving, and environmentally benign procedure for gas separation, adsorbents with tailored structures and tunable surface properties must be found. Metal-organic frameworks (MOFs), constructed by metal-containing nodes connected by organic bridges, are such a new type of porous materials. They are promising candidates as adsorbents for gas separations due to their large surface areas, adjustable pore sizes and controllable properties, as well as acceptable thermal stability. This critical review starts with a brief introduction to gas separation and purification based on selective adsorption, followed by a review of gas selective adsorption in rigid and flexible MOFs. Based on possible mechanisms, selective adsorptions observed in MOFs are classified, and primary relationships between adsorption properties and framework features are analyzed. As a specific example of tailor-made MOFs, mesh-adjustable molecular sieves are emphasized and the underlying working mechanism elucidated. In addition to the experimental aspect, theoretical investigations from adsorption equilibrium to diffusion dynamics via molecular simulations are also briefly reviewed. Furthermore, gas separations in MOFs, including the molecular sieving effect, kinetic separation, the quantum sieving effect for H2/D2 separation, and MOF-based membranes are also summarized (227 references).},
author = {Li, Jian-Rong and Kuppler, Ryan J. and Zhou, Hong-Cai},
doi = {10.1039/b802426j},
file = {:Users/marc/Library/Application Support/Mendeley Desktop/Downloaded/Li, Kuppler, Zhou - 2009 - Selective gas adsorption and separation in metal–organic frameworks.pdf:pdf},
isbn = {0306-0012},
issn = {0306-0012},
journal = {Chemical Society Reviews},
number = {5},
pages = {1477},
pmid = {19384449},
title = {{Selective gas adsorption and separation in metal–organic frameworks}},
url = {http://xlink.rsc.org/?DOI=b802426j},
volume = {38},
year = {2009}
}
@article{Bossard,
author = {Bossard, Peter R and Mettes, Jacques and Breziner, Luis and Gornick, Emeritus Fred},
journal = {Power {\&} Energy, USA},
title = {{New Sensor for Measuring Trace Impurities in Ultra Pure Hydrogen}}
}
@article{Al-Mufachi2015,
author = {Al-Mufachi, N A and Rees, N V and Steinberger-Wilkens, R},
doi = {10.1016/j.rser.2015.03.026},
isbn = {13640321},
journal = {Renewable and Sustainable Energy Reviews},
pages = {540--551},
title = {{Hydrogen selective membranes: A review of palladium-based dense metal membranes}},
volume = {47},
year = {2015}
}
@article{Ban2011,
abstract = {Molecular simulations were performed to characterize hydrated Nafion membranes in terms of gas adsorption, diffusion, and permeation. The experimental results validate the molecular model of Nafion with respect to material density, morphology, free volume, and water diffusivity. Nafion's adsorption property is examined in terms of the solubility and adsorption isotherms for gases, including H(2), O(2), and N(2). The adsorption capacity of hydrated Nafion is shown to be strong for O(2) and N(2) but not for H(2). Due to the dilution effect, N(2) is able to suppress the loading of O(2) and protect the fuel cell from fuel crossover. The dynamic behaviors of H(2) and O(2) are represented by self-diffusion coefficients, with the results showing that H(2) diffusion in Nafion membranes is nearly 1 order of magnitude faster than O(2) diffusion. The effects of water content and the concentration of adsorbed gases were verified, and a close correlation of Nafion free volume to gas transport properties was revealed. On the basis of the solution-diffusion mechanism, the permeabilities of H(2) and O(2) in hydrated Nafion membranes are calculated and compared with corresponding experiments, and the permeability of H(2) is found to be approximately twice that of O(2).},
author = {Ban, Shuai and Huang, Cheng and Yuan, Xiao-Zi and Wang, Haijiang},
doi = {10.1021/jp204141b},
file = {:Users/marc/Library/Application Support/Mendeley Desktop/Downloaded/Ban et al. - 2011 - Molecular Simulation of Gas Adsorption, Diffusion, and Permeation in Hydrated Nafion Membranes.pdf:pdf},
issn = {1520-6106},
journal = {The Journal of Physical Chemistry B},
number = {39},
pages = {11352--11358},
pmid = {21875104},
title = {{Molecular Simulation of Gas Adsorption, Diffusion, and Permeation in Hydrated Nafion Membranes}},
url = {http://pubs.acs.org/doi/abs/10.1021/jp204141b},
volume = {115},
year = {2011}
}
@article{Wang2017,
author = {Wang, Tao and Sun, Zhen and Huang, Da and Yang, Zhi and Ji, Qian and Hu, Nantao and Yin, Guilin and He, Dannong and Wei, Hao and Zhang, Yafei},
doi = {10.1016/j.snb.2017.05.162},
file = {:Users/marc/Library/Application Support/Mendeley Desktop/Downloaded/Wang et al. - 2017 - Studies on NH 3 gas sensing by zinc oxide nanowire-reduced graphene oxide nanocomposites.pdf:pdf},
issn = {09254005},
journal = {Sensors and Actuators B: Chemical},
keywords = {reduced graphene oxide,zinc oxide nanowires},
pages = {284--294},
publisher = {Elsevier B.V.},
title = {{Studies on NH 3 gas sensing by zinc oxide nanowire-reduced graphene oxide nanocomposites}},
url = {http://linkinghub.elsevier.com/retrieve/pii/S0925400517309942},
volume = {252},
year = {2017}
}
@article{Saufi2004,
abstract = {Carbon membrane materials are becoming more important in the new era of membrane technology for gas separation due to their higher selectivity, permeability and stability in corrosive and high temperature operations. Carbon membranes can be produced by pyrolysis of a suitable polymeric precursor under controlled conditions. This paper reviews the fabrication aspects of carbon membranes, which can be divided into six steps: precursor selection, polymeric membrane preparation, pretreatment of the precursor, pyrolysis process, post-treatment of pyrolyzed membranes and module construction. The manipulation of the pretreatment variables, pyrolysis process parameters and post-treatment conditions were shown to provide an opportunity to enhance the separation performance of carbon membranes in the future. By understanding the available methods, one can choose and optimize the best technique during the fabrication of carbon membranes. Furthermore, areas of future potential in carbon membrane research for gas separation were also briefly identified. {\textcopyright} 2003 Elsevier Ltd. All rights reserved.},
author = {Saufi, S. M. and Ismail, A. F.},
doi = {10.1016/j.carbon.2003.10.022},
file = {:Users/marc/Library/Application Support/Mendeley Desktop/Downloaded/Saufi, Ismail - 2004 - Fabrication of carbon membranes for gas separation - A review.pdf:pdf},
isbn = {0008-6223},
issn = {00086223},
journal = {Carbon},
keywords = {A. Porous carbon,B. Pyrolysis,Carbon precursor,Heat treatment},
number = {2},
pages = {241--259},
title = {{Fabrication of carbon membranes for gas separation - A review}},
volume = {42},
year = {2004}
}
@article{Peters2008,
author = {Peters, T A and Stange, M and Klette, H and Bredesen, R},
doi = {10.1016/j.memsci.2007.08.056},
isbn = {03767388},
journal = {Journal of Membrane Science},
number = {1-2},
pages = {119--127},
title = {{High pressure performance of thin Pd–23{\{}{\%}{\}}Ag/stainless steel composite membranes in water gas shift gas mixtures; influence of dilution, mass transfer and surface effects on the hydrogen flux}},
volume = {316},
year = {2008}
}
@article{Kim2013a,
author = {Kim, Hyo Won and Yoon, Hee Wook and Yoon, Seon-mi and Yoo, Byung Min and Ahn, Byung Kook and Cho, Young Hoon and Shin, Hye Jin and Yang, Hoichang and Paik, Ungyu and Kwon, Soongeun},
file = {:Users/marc/Library/Application Support/Mendeley Desktop/Downloaded/Kim et al. - 2013 - Selective Gas Transport Through Few-Layered Graphene and Graphene Oxide Membranes.pdf:pdf},
journal = {Science},
number = {October},
pages = {91--96},
title = {{Selective Gas Transport Through Few-Layered Graphene and Graphene Oxide Membranes}},
volume = {342},
year = {2013}
}
@article{Wei2008,
author = {Wei, Qi and Wang, Fei and Nie, Zuo-ren and Song, Chun-lin and Wang, Yan-li and Li, Qun-yan},
doi = {10.1021/jp711573f},
file = {:Users/marc/Library/Application Support/Mendeley Desktop/Downloaded/Wei et al. - 2008 - Highly Hydrothermally Stable Microporous Silica Membranes for Hydrogen Separation Highly Hydrothermally Stable Micro.pdf:pdf},
journal = {Society},
number = {grade 100},
pages = {9354--9359},
title = {{Highly Hydrothermally Stable Microporous Silica Membranes for Hydrogen Separation Highly Hydrothermally Stable Microporous Silica Membranes for Hydrogen Separation}},
year = {2008}
}
@article{Keurentjes2004,
author = {Keurentjes, Jos T. F. and Gielens, Frank C. and Tong, H. D. and van Rijn, C. J. M. and Vorstman, Marius a. G.},
doi = {10.1021/ie0341202},
file = {:Users/marc/Library/Application Support/Mendeley Desktop/Downloaded/Keurentjes et al. - 2004 - High-Flux Palladium Membranes Based on Microsystem Technology.pdf:pdf},
issn = {0888-5885},
journal = {Industrial {\&} Engineering Chemistry Research},
number = {16},
pages = {4768--4772},
title = {{High-Flux Palladium Membranes Based on Microsystem Technology}},
url = {http://pubs.acs.org/doi/abs/10.1021/ie0341202},
volume = {43},
year = {2004}
}
@article{Lin2012,
abstract = {We report a transformative, all inorganic synthesis method of preparing supported bimetallic Pd(3)Ag alloy nanoparticles. The method involves breaking down bulk Pd(3)Ag alloy into the nanoparticles in liquid lithium, converting metallic Li to LiOH, and transferring Pd(3)Ag nanoparticles/LiOH mixture onto non-water-soluble supports, followed by leaching off the LiOH with water under ambient conditions. The size of the resulting Pd(3)Ag nanoparticles was found narrowly distributed around 2.3 nm characterized by transmission electron microscope (TEM). In addition, studies by X-ray diffraction (XRD), extended X-ray absorption fine structure (EXAFS) spectroscopy, and X-ray absorption near edge structure (XANES) spectroscopy showed that the resulting Pd(3)Ag nanoparticles inherited similar atomic ratio and alloy structure as the starting material. The synthesized Pd(3)Ag nanoparticles exhibited excellent catalytic activity toward hydrogenation of acrolein to propanal.},
author = {Lin, Chi-Kai and Lin, Yan-Gu and Wu, Tianpin and Barkholtz, Heather M and Lin, Qiyin and Wei, Haojuan and Brewe, Dale L and Miller, Jeffrey T and Liu, Di-Jia and Ren, Yang and Ito, Yasuo and Xu, Tao},
doi = {10.1021/ic301940g},
file = {:Users/marc/Library/Application Support/Mendeley Desktop/Downloaded/Lin et al. - 2012 - Direct synthesis of bimetallic Pd3Ag nanoalloys from bulk Pd3Ag alloy.pdf:pdf},
issn = {1520-510X},
journal = {Inorganic chemistry},
number = {24},
pages = {13281--8},
pmid = {23186229},
title = {{Direct synthesis of bimetallic Pd3Ag nanoalloys from bulk Pd3Ag alloy.}},
url = {http://www.ncbi.nlm.nih.gov/pubmed/23186229},
volume = {51},
year = {2012}
}
@article{Zhang2008,
abstract = {Composite carbon membranes were prepared from poly(phthalazinone ether sulfone ketone) (PPESK) by incorporating with polyvinylpyrrolidone (PVP) or zeolite (ZSM-5) through stabilization and pyrolysis processes. The thermal stability of composite polymeric membranes was measured by thermal gravimetric analysis. The resultant composite carbon membranes were characterized by scanning electron microscopy, X-ray diffraction and gas permeation technique, respectively. The results illustrated that the thermal stability of composite polymeric membranes was enhanced by addition of ZSM-5 or reduced by PVP. For ZSM-5 or PVP composite carbon membranes prepared at 650 °C, the O2permeability is 199.70 Barrer or 124.89 Barrer, and the O2/N2selectivity is 10.3 or 4.2, respectively. Compared with carbon membranes from pure PPESK, the O2permeability of ZSM-5 or PVP composite carbon membranes increases by 18.5 or 11.6 times, together with the O2/N2selectivity decreasing by 35.2{\%} or 73.6{\%}, respectively. The gas separation mechanism of composite carbon membranes is molecular sieving. Adsorption effect also plays a significant role for CO2permeating through ZSM-5 composite carbon membranes. {\textcopyright} 2007 Elsevier B.V. All rights reserved.},
author = {Zhang, Bing and Wang, Tonghua and Wu, Yonghong and Liu, Qingling and Liu, Shili and Zhang, Shouhai and Qiu, Jieshan},
doi = {10.1016/j.seppur.2007.08.022},
file = {:Users/marc/Library/Application Support/Mendeley Desktop/Downloaded/Zhang et al. - 2008 - Preparation and gas permeation of composite carbon membranes from poly(phthalazinone ether sulfone ketone).pdf:pdf},
isbn = {1383-5866},
issn = {13835866},
journal = {Separation and Purification Technology},
keywords = {Carbon molecular sieves,Composite membrane,Gas separation,Pyrolysis},
number = {3},
pages = {259--263},
title = {{Preparation and gas permeation of composite carbon membranes from poly(phthalazinone ether sulfone ketone)}},
volume = {60},
year = {2008}
}
@article{Hara2002,
author = {Hara, S and Hatakeyama, N and Itoh, N and Kimura, H -M. and Inoue, A},
file = {:Users/marc/Library/Application Support/Mendeley Desktop/Downloaded/Hara et al. - 2002 - Hydrogen permeation through palladium-coated amorphous Zr-M-Ni (M=Ti, Hf) alloy membranes.pdf:pdf},
journal = {Desalination},
pages = {115--120},
title = {{Hydrogen permeation through palladium-coated amorphous Zr-M-Ni (M=]Ti, Hf) alloy membranes}},
volume = {1444},
year = {2002}
}
@article{Tsapatsis2011,
author = {Tsapatsis, Michael},
journal = {Science},
number = {November},
pages = {11--12},
title = {{Toward High-Throughput Zeolite Membranes}},
volume = {334},
year = {2011}
}
@article{Liu2015b,
abstract = {This work shows for the first time that the hydrothermal stability of cobalt oxide silica membranes is very dependent upon the cobalt oxide phase. Xerogels with the same cobalt loading (Co/Si=0.1) were characterised by N2sorption, CP/MAS29Si NMR, FTIR, Raman, XPS and DR UV-vis spectroscopy. It was found that the xerogels containing tetrahedrally coordinated cobalt (Co2+) in the silica matrix were hydrothermally unstable leading to a sharp loss of pore volume within 10h when exposed to 75mol{\%} water vapour and 550°C, followed by complete densification after 40h. However, silica xerogels containing a high content of octahedrally coordinated cobalt (Co3+) in the form of Co3O4were able to oppose structural densification and were consequently much more hydrothermally stable. The Co3O4and Co2+silica membranes were tested for single gas permeation using He, H2, CO2and N2, delivering similar He permeance (2.48 × 10-7and 2.85 × 10-7molm-2s-1Pa-1) and a He/N2selectivity of 50 and 41 at 500°C, respectively. Upon exposure to the same harsh hydrothermal conditions as the xerogels, the membranes were tested again for single gas permeation. The high content Co3O4silica membrane saw only a marginal decrease in He/N2selectivity to 39 (22{\%} loss) whilst the tetrahedral cobalt coordination silica membrane had a dramatic decline in selectivity to only 11 (73{\%} loss).},
author = {Liu, Liang and Wang, David K. and Martens, Dana L. and Smart, Simon and {Diniz da Costa}, Jo{\~{a}}o C.},
doi = {10.1016/j.memsci.2014.10.037},
file = {:Users/marc/Library/Application Support/Mendeley Desktop/Downloaded/Liu et al. - 2015 - Influence of sol-gel conditioning on the cobalt phase and the hydrothermal stability of cobalt oxide silica membrane.pdf:pdf},
issn = {18733123},
journal = {Journal of Membrane Science},
keywords = {Cobalt oxide phase,Hydrothermal stability,Silica,Water vapour},
pages = {425--432},
publisher = {Elsevier},
title = {{Influence of sol-gel conditioning on the cobalt phase and the hydrothermal stability of cobalt oxide silica membranes}},
url = {http://dx.doi.org/10.1016/j.memsci.2014.10.037},
volume = {475},
year = {2015}
}
@article{Zhang2014,
abstract = {A new ZIF-8 membrane architecture with high performance supported on vertically aligned ZnO nanorods was successfully prepared. The vertically aligned, single crystal ZnO nanorods were grown seamlessly from porous ceramic support to form an intermediate support layer for the ZIF-8 membrane. They provide multiple anchorages for the ZIF-8 membrane that are both strong and flexible. The nanorods were activated to induce a uniform nucleation of ZIF nuclei on their surface to initiate and guide the growth of a defect-free ZIF-8 membrane. Single gas permeations and binary separations carried out to investigate the transport properties of these new membrane architectures confirmed that the ZIF-8 membranes were free of defects and stable at a higher temperature (473 K).$\backslash$nA new ZIF-8 membrane architecture with high performance supported on vertically aligned ZnO nanorods was successfully prepared. The vertically aligned, single crystal ZnO nanorods were grown seamlessly from porous ceramic support to form an intermediate support layer for the ZIF-8 membrane. They provide multiple anchorages for the ZIF-8 membrane that are both strong and flexible. The nanorods were activated to induce a uniform nucleation of ZIF nuclei on their surface to initiate and guide the growth of a defect-free ZIF-8 membrane. Single gas permeations and binary separations carried out to investigate the transport properties of these new membrane architectures confirmed that the ZIF-8 membranes were free of defects and stable at a higher temperature (473 K).},
author = {Zhang, Xiongfu and Liu, Yaguang and Li, Shaohui and Kong, Lingyin and Liu, Haiou and Li, Yanshuo and Han, Wei and Yeung, King Lun and Zhu, Weidong and Yang, Weishen and Qiu, Jieshan},
doi = {10.1021/cm500269e},
file = {:Users/marc/Library/Application Support/Mendeley Desktop/Downloaded/Zhang et al. - 2014 - New membrane architecture with high performance ZIF-8 membrane supported on vertically aligned ZnO nanorods for ga.pdf:pdf},
isbn = {0897-4756},
issn = {08974756},
journal = {Chemistry of Materials},
number = {5},
pages = {1975--1981},
title = {{New membrane architecture with high performance: ZIF-8 membrane supported on vertically aligned ZnO nanorods for gas permeation and separation}},
volume = {26},
year = {2014}
}
@article{Abetz2006,
abstract = {A review.  The aim of this review paper is to give an overview of the research activities of GKSS in the field of polymer based membranes.  After summarizing the historic development of membrane science at GKSS, it describes different classes of polymeric materials for membranes, and the characterization of membranes.  The design of membrane-based sepn. processes followed by examples of applications will also be presented.  In the last chapter we will try to give an outlook for the future activities in membrane research. [on SciFinder(R)]},
author = {Abetz, Volker and Brinkmann, Torsten and Dijkstra, Marga and Ebert, Katrin and Fritsch, Detlev and Ohlrogge, Klaus and Paul, Dieter and Peinemann, Klaus Viktor and Nunes, Suzana Pereira and Scharnagl, Nico and Schossig, Michael},
doi = {10.1002/adem.200600032},
isbn = {1438-1656},
issn = {14381656},
journal = {Advanced Engineering Materials},
number = {5},
pages = {328--358},
title = {{Developments in membrane research: From material via process design to industrial application}},
volume = {8},
year = {2006}
}
@article{Fernandez2015,
author = {Fernandez, E and Coenen, K and Helmi, A and Melendez, J and Zu{\~{n}}iga, J and {Pacheco Tanaka}, D A and {van Sint Annaland}, M and Gallucci, F},
doi = {10.1016/j.ijhydene.2015.08.050},
isbn = {03603199},
journal = {International Journal of Hydrogen Energy},
number = {39},
pages = {13463--13478},
title = {{Preparation and characterization of thin-film Pd–Ag supported membranes for high-temperature applications}},
volume = {40},
year = {2015}
}
@article{Gascon2012,
abstract = {In this work, the concept of zeolite (zeolitic) membrane is discussed from a practical perspective. We consider the limitations of the existing synthesis methods and speculate on new opportunities of zeolites and zeolite-type materials such as metal organic frameworks for the production of membranes. This paper focuses on the barriers that need to be eliminated before the commercialization of these membranes becomes attractive. Additional opportunities for commercialization may arise in the shape either of mixed matrix membranes, taking advantage of composites with polymers, or as zeolite coatings useful for a plethora of new applications.$\backslash$nIn this work, the concept of zeolite (zeolitic) membrane is discussed from a practical perspective. We consider the limitations of the existing synthesis methods and speculate on new opportunities of zeolites and zeolite-type materials such as metal organic frameworks for the production of membranes. This paper focuses on the barriers that need to be eliminated before the commercialization of these membranes becomes attractive. Additional opportunities for commercialization may arise in the shape either of mixed matrix membranes, taking advantage of composites with polymers, or as zeolite coatings useful for a plethora of new applications.},
author = {Gascon, Jorge and Kapteijn, Freek and Zornoza, Beatriz and Sebasti{\'{a}}n, V{\'{i}}ctor and Casado, Clara and Coronas, Joaq{\'{u}}n},
doi = {10.1021/cm301435j},
file = {:Users/marc/Library/Application Support/Mendeley Desktop/Downloaded/Gascon et al. - 2012 - Practical approach to zeolitic membranes and coatings State of the art, opportunities, barriers, and future persp.pdf:pdf},
isbn = {0897-4756},
issn = {08974756},
journal = {Chemistry of Materials},
keywords = {MOF,gas separation,mixed matrix membrane,pervaporation,zeolite,zeolite coating,zeolite membrane},
number = {15},
pages = {2829--2844},
title = {{Practical approach to zeolitic membranes and coatings: State of the art, opportunities, barriers, and future perspectives}},
volume = {24},
year = {2012}
}
@article{Kirchner2008,
author = {Kirchner, A and Brown, I W M and Bowden, M E and Kemmitt, T and Smith, G},
doi = {10.1016/j.cap.2007.10.037},
isbn = {15671739},
journal = {Current Applied Physics},
number = {3-4},
pages = {451--454},
title = {{Preparation and high-temperature characterisation of nanostructured alumina ceramic membranes for gas purification}},
volume = {8},
year = {2008}
}
@article{Escolastico2014,
abstract = {Mixed electronic- and protonic-conducting composites made up of physical mixtures of La5.5WO11.25−$\delta$–La0.87Sr0.13CrO3−$\delta$ (LWO–LSC) have been evaluated as H2 separation membranes for operation at temperatures greater than 550 °C. The mixture of these two ion-conducting phases led to non-linear synergetic effects; i.e. unexpected enhancement of the total conductivity and well-balanced ambipolar conductivity, resulting in appealing H2 permeation fluxes through robust ceramic membranes. The preparation, primary characterization, H2 permeation and stability studies of various composites is presented. Mixing LWO and LSC phases makes it possible (1) to improve the LSC sintering behavior and to achieve very high membrane densities and (2) to obtain compounds with high total conductivity, higher than that shown for LWO and LSC, separately. The highest permeation rate is achieved for the 50 vol{\%}-LWO–LSC membrane, though other composite compositions showed higher total conductivity. Moreover, the influence on the H2 permeation of the composite composition, the humidification of gas streams, temperature and the use of various catalytic coatings on the membrane surface is evaluated. The nature of the transport mechanism is investigated by the permeation studies using deuterium tracers. The H2 permeation rates reported in this investigation for a 370 $\mu$m thick 50 vol{\%}-LWO–LSC membrane, e.g. 0.15 mL min−1 cm−2 at 700 °C, are the highest reported values, up to date, for any bulk mixed protonic-electronic membranes. The H2 permeation magnitude achieved at moderate temperatures along with the proven stability in CO2-rich atmospheres are firm steps towards the future application of this type of membrane for industrial processes.},
author = {Escol{\'{a}}stico, S. and Sol{\'{i}}s, C. and Kj{\o}lseth, C. and Serra, J. M.},
doi = {10.1039/C4EE02066A},
file = {:Users/marc/Library/Application Support/Mendeley Desktop/Downloaded/Escol{\'{a}}stico et al. - 2014 - Outstanding hydrogen permeation through CO sub2sub -stable dual-phase ceramic membranes.pdf:pdf},
issn = {1754-5692},
journal = {Energy Environ. Sci.},
number = {11},
pages = {3736--3746},
title = {{Outstanding hydrogen permeation through CO {\textless}sub{\textgreater}2{\textless}/sub{\textgreater} -stable dual-phase ceramic membranes}},
url = {http://xlink.rsc.org/?DOI=C4EE02066A},
volume = {7},
year = {2014}
}
@article{AlbaArratibelPlazaolaDavidAlfredoPachecoTanaka2017,
author = {{Alba Arratibel Plazaola, David Alfredo Pacheco Tanaka}, Martin Van Sint Annaland and Fausto Gallucci},
doi = {10.3390/molecules22010051},
file = {:Users/marc/Library/Application Support/Mendeley Desktop/Downloaded/Alba Arratibel Plazaola, David Alfredo Pacheco Tanaka - 2017 - Recent Advances in Pd-Based Membranes for Membrane Reactors.pdf:pdf},
journal = {Molecules},
keywords = {inorganic membranes,membrane reactor,membranes,palladium membranes,pore-filled},
number = {51},
pages = {1--53},
title = {{Recent Advances in Pd-Based Membranes for Membrane Reactors}},
volume = {22},
year = {2017}
}
@article{Kareva2017,
abstract = {The solubility of tin in the phases of Pd–Au–Sn and Pd–Cu–Sn ternary systems and a Pd–Au–Cu–Sn quaternary system with a fixed Pd: Au: Cu ratio of 11.1: 1: 4.6 is studied via microstructural, X-ray diffraction, and energy dispersive analysis. It is found that a quaternary alloy in equilibrium with a solid solution based on Pd, Au, and Sn contains a $\tau$1 compound with structure which is derivative of the In type. It contains {\~{}}15 at {\%} Sn and is a solid solution of the same compounds identified earlier in Pd–Au–Sn and Pd–Cu–Sn ternary systems. In addition, a quaternary alloy with a content of 20 at {\%} Sn also contains a $\tau$2 compound with the Pd2CuSn own type and can barely dissolve gold. The obtained data are used to construct a three-dimensional model of the Pd-rich part of the isothermal tetrahedron of the Pd–Au–Cu–Sn system and diagrams of the tin solubility isolines in palladium-rich alloys of the quaternary system at 500°С.},
author = {Kareva, M A and Kabanova, E G and Zhmurko, G P and Kuznetsov, V N},
doi = {10.1134/S0036024417020145},
issn = {1531-863X},
journal = {Russian Journal of Physical Chemistry A},
number = {2},
pages = {255--259},
title = {{Phase equilibria in the palladium-rich part of the Pd–Au–Cu–Sn quaternary system}},
url = {https://doi.org/10.1134/S0036024417020145},
volume = {91},
year = {2017}
}
@article{Itoh1993,
author = {Itoh, N and Xu, W.-C.},
journal = {Applied Catalysis A: General},
pages = {83--100},
title = {{Selective hydrogenation of phenol to cyclohexane using palladium-based membranes as catalysts}},
volume = {107},
year = {1993}
}
@article{FernandoRoaJ.DouglasWay2002,
author = {{Fernando Roa J. Douglas Way}, Michael J Block},
chapter = {411},
file = {:Users/marc/Library/Application Support/Mendeley Desktop/Downloaded/Fernando Roa J. Douglas Way - 2002 - The infleuance of alloy composition on the H2 flux of composite Pd-Cu membranes.pdf:pdf},
journal = {Desalination},
pages = {411--416},
title = {{The infleuance of alloy composition on the H2 flux of composite Pd-Cu membranes}},
volume = {147},
year = {2002}
}
@article{Roa2003a,
author = {Roa, Fernando and Way, J Douglas and McCormick, Robert L and Paglieri, Stephen N},
doi = {10.1016/s1385-8947(02)00106-7},
isbn = {13858947},
journal = {Chemical Engineering Journal},
number = {1},
pages = {11--22},
title = {{Preparation and characterization of Pd–Cu composite membranes for hydrogen separation}},
volume = {93},
year = {2003}
}
@article{Gil2013,
abstract = {Asymmetric supported La28 − xW4 + xO54 + 3x/2 (La/W ≈ 5.6) membranes were investigated for their hydrogen permeation properties as a function of temperature and feed gas conditions. Dense membranes of thickness 25–30 $\mu$m supported on substrates with 25 and 40 vol.{\%} porosity were compared. Above 850 °C under dry conditions, the hydrogen permeation rate was approximately constant as a function of temperature due to low concentration of protons in the material at high temperatures. Under humid conditions and above 960 °C enhanced permeation rates were observed. A hydrogen permeation as high as 0.14 NmL min−1 cm−2 was recorded at 1000 °C with 10 vol.{\%} H2 (2.5 vol.{\%} H2O) as feed gas.},
author = {Gil, Vanesa and Gurauskis, Jonas and Kj{\o}lseth, Christian and Wiik, Kjell and Einarsrud, Mari-Ann},
doi = {10.1016/j.ijhydene.2012.12.105},
file = {:Users/marc/Library/Application Support/Mendeley Desktop/Downloaded/Gil et al. - 2013 - Hydrogen permeation in asymmetric La28 − xW4 xO54 3x2 membranes.pdf:pdf},
issn = {03603199},
journal = {International Journal of Hydrogen Energy},
number = {7},
pages = {3087--3091},
title = {{Hydrogen permeation in asymmetric La28 − xW4 + xO54 + 3x/2 membranes}},
url = {http://linkinghub.elsevier.com/retrieve/pii/S0360319912028406},
volume = {38},
year = {2013}
}
@article{Sato1991,
author = {Sato, Noboru and Ando, Hiroshi and Kude, Yukinori},
file = {:Users/marc/Library/Application Support/Mendeley Desktop/Downloaded/Sato, Ando, Kude - 1991 - Separation of Hydrogen Through Palladium Thin Film supported on a Porous Glass Tube.pdf:pdf},
journal = {Journal of Membrane Science},
keywords = {composite membranes,gas separation,inorganic,membrane preparation and structure,membranes,metal membranes},
pages = {303--313},
title = {{Separation of Hydrogen Through Palladium Thin Film supported on a Porous Glass Tube}},
volume = {56},
year = {1991}
}
@article{Gu2009,
abstract = {A hydrothermally stable and H2-selective composite membrane was successfully prepared on a mesoporous alumina support by thermal decomposition of titanium isopropoxide (TIP) and tetraethylorthosilicate (TEOS) at high temperature. The membrane with a H2-selective silica-titania top layer of 10-20 nm in thickness had a H2permeance 3 × 10-7mol m-2s-1Pa-1and a H2selectivity over CH4and CO2in the range of 40-60. After 130 h exposure in 75 mol{\%} H2O at 923 K, the H2permeance was reduced by only 30{\%}, compared to a pure silica membrane which suffered a loss of 90{\%}. The use of the opposing reactants technique improved the quality of the composite membrane. Although considerable more efforts are needed to increase surface area, the stability properties suggest applicability of the composition at high temperature in the presence of high concentrations of steam. {\textcopyright} 2009 Elsevier B.V. All rights reserved.},
author = {Gu, Yunfeng and Oyama, S. Ted},
doi = {10.1016/j.memsci.2009.09.009},
file = {:Users/marc/Library/Application Support/Mendeley Desktop/Downloaded/Gu, Oyama - 2009 - Permeation properties and hydrothermal stability of silica-titania membranes supported on porous alumina substrates.pdf:pdf},
isbn = {0376-7388},
issn = {03767388},
journal = {Journal of Membrane Science},
keywords = {CVD,Composite membrane,Hydrogen separation,Hydrothermal stability,Silica,Titania},
number = {1-2},
pages = {267--275},
title = {{Permeation properties and hydrothermal stability of silica-titania membranes supported on porous alumina substrates}},
volume = {345},
year = {2009}
}
@article{Unemoto2006,
author = {Unemoto, A and Kaimai, A and Otake, T and Yashiro, K and Kawada, T and Mizusaki, J and Tsuneki, T and Yasuda, I},
file = {:Users/marc/Library/Application Support/Mendeley Desktop/Downloaded/Unemoto et al. - 2006 - Hydrogen Permeability of Palladium Alloy Membrane at High Temperatures in the Impurity Gases Co-existing Atmosph.pdf:pdf},
isbn = {9781622765409},
keywords = {high,impurity gas,membrane reformer,palladium alloy membrane,surface reaction},
number = {June},
pages = {1--9},
title = {{Hydrogen Permeability of Palladium Alloy Membrane at High Temperatures in the Impurity Gases Co-existing Atmospheres}},
year = {2006}
}
@article{Seshimo2008,
author = {Seshimo, Masahiro and Ozawa, Minoru and Sone, Masato and Sakurai, Makoto and Kameyama, Hideo},
doi = {10.1016/j.memsci.2008.07.007},
isbn = {03767388},
journal = {Journal of Membrane Science},
number = {1-2},
pages = {181--187},
title = {{Fabrication of a novel Pd/{\$}\gamma{\$}-alumina graded membrane by electroless plating on nanoporous {\$}\gamma{\$}-alumina}},
volume = {324},
year = {2008}
}
@misc{Hickson1999,
abstract = {The Queen Scallop Aequipecten opercularis has been an important part of the marine fauna of the northeast Atlantic region since the Miocene, and is a suitable candidate for palaeoenvironmental studies using stable isotopes. Modern shells were investigated to determine if carbonate is precipitated in isotopic equilibrium in this species, and to confirm that growth continues through the year. Specimens of A. opercularis were cultured under monitored semi-natural conditions over an autumn/winter period, and the analyses of these specimens were supplemented with isotope data from preserved indigenous North Sea specimens. The results indicated that the cultured specimens grew in relatively low temperatures, that oxygen isotopes of shell-exterior carbonate were precipitated in isotopic equilibrium, and that a large proportion of the ambient temperature range was recorded in shell carbonate. The indigenous North Sea specimens exhibited a good correspondence to predicted oxygen isotope values for summer; the isotopic record related to winter was, however, somewhat truncated due to growth decelerations in the specimens. Carbon isotopes of shell-exterior carbonate were calculated as being derived primarily from external sources, and showed comparatively little influence from metabolically derived carbon; this suggests that carbon isotopes were at least close to equilibrium with ambient waters. The limited incorporation of metabolic products suggests that A. opercularis would be a good monitor of changing carbon isotope ratios in seawater-dissolved inorganic carbon. The occurrence of equilibrium shell precipitation is explained using a 'pteriomorph shell-calcification model', which proposes that transport of ions through the periostracum caused the composition of the extrapallial fluid to be effectively the same as seawater. Carbonate secretion is, therefore, not influenced significantly by biological (vital) effects.},
author = {Hickson, Jon A and Johnson, Andrew L A and Heaton, Tim H E and Balson, Peter S},
booktitle = {Palaeogeography, Palaeoclimatology, Palaeoecology},
doi = {10.1016/S0031-0182(99)00120-0},
number = {4},
title = {{The shell of the Queen Scallop Aequipecten opercularis (L.) as a promising tool for palaeoenvironmental reconstruction: Evidence and reasons for equilibrium stable-isotope incorporation}},
volume = {154},
year = {1999}
}
@article{Shustova2013,
abstract = {We show that fluorescent molecules incorporated as ligands in rigid, porous metal-organic frameworks (MOFs) maintain their fluorescence response to a much higher temperature than in molecular crystals. The remarkable high-temperature ligand-based fluorescence, demonstrated here with tetraphenylethylene- and dihydroxyterephthalate-based linkers, is essential for enabling selective and rapid detection of analytes in the gas phase. Both Zn2(TCPE) (TCPE = tetrakis(4-carboxyphenyl)ethylene) and Mg(H2DHBDC) (H2DHBDC(2-) = 2,5-dihydroxybenzene-1,4-dicarboxylate) function as selective sensors for ammonia at 100 °C, although neither shows NH3 selectivity at room temperature. Variable-temperature diffuse-reflectance infrared spectroscopy, fluorescence spectroscopy, and X-ray crystallography are coupled with density-functional calculations to interrogate the temperature-dependent guest-framework interactions and the preferential analyte binding in each material. These results describe a heretofore unrecognized, yet potentially general property of many rigid, fluorescent MOFs and portend new applications for these materials in selective sensors, with selectivity profiles that can be tuned as a function of temperature.},
author = {Shustova, Natalia B. and Cozzolino, Anthony F. and Reineke, Sebastian and Baldo, Marc and Dinc??, Mircea},
doi = {10.1021/ja407778a},
file = {:Users/marc/Library/Application Support/Mendeley Desktop/Downloaded/Shustova et al. - 2013 - Selective turn-on ammonia sensing enabled by high-temperature fluorescence in metal-organic frameworks with ope.pdf:pdf},
isbn = {0002-7863},
issn = {00027863},
journal = {Journal of the American Chemical Society},
number = {36},
pages = {13326--13329},
pmid = {23981174},
title = {{Selective turn-on ammonia sensing enabled by high-temperature fluorescence in metal-organic frameworks with open metal sites}},
volume = {135},
year = {2013}
}
@article{Yildirim2013,
author = {Yildirim, Taner},
file = {:Users/marc/Library/Application Support/Mendeley Desktop/Downloaded/Yildirim - 2013 - Methane Storage in Metal − Organic Frameworks Current Records, Surprise Findings, and Challenges.pdf:pdf},
title = {{Methane Storage in Metal − Organic Frameworks: Current Records, Surprise Findings, and Challenges}},
year = {2013}
}
@article{Canepa2015,
abstract = {We combine infrared spectroscopy, nano-indentation measurements, and ab initio simulations to study the evolution of structural, elastic, thermal, and electronic responses of the metal–organic framework MOF-74-Zn when loaded with H2, CO2, CH4, and H2O. We find that molecular adsorption in this MOF triggers remarkable responses in all these properties of the host material, with specific signatures for each of the guest molecules. With this comprehensive study, we are able to clarify and correlate the underlying mechanisms regulating these responses with changes of physical and chemical environments. Our findings suggest that metal–organic framework materials in general, and MOF-74-Zn in particular, can be very promising materials for novel transducers and sensor applications, including highly selective small-molecule detection in gas mixtures.},
author = {Canepa, Pieremanuele and Tan, Kui and Du, Yingjie and Lu, Hongbing and Chabal, Yves J. and Thonhauser, Timo},
doi = {10.1039/C4TA03968H},
file = {:Users/marc/Library/Application Support/Mendeley Desktop/Downloaded/Canepa et al. - 2015 - Structural, elastic, thermal, and electronic responses of small-molecule-loaded metal–organic framework materia.pdf:pdf},
isbn = {2050-7488$\backslash$r2050-7496},
issn = {2050-7488},
journal = {J. Mater. Chem. A},
number = {3},
pages = {986--995},
publisher = {Royal Society of Chemistry},
title = {{Structural, elastic, thermal, and electronic responses of small-molecule-loaded metal–organic framework materials}},
url = {http://pubs.rsc.org/en/content/articlehtml/2015/ta/c4ta03968h},
volume = {3},
year = {2015}
}
@article{Yang2014,
abstract = {{\textless}p{\textgreater}Porous conducting polymer and reduced graphene oxide (GO) composites was prepared as sensing materials for enhanced sensitivity and selectivity.{\textless}/p{\textgreater}},
author = {Yang, Yajie and Yang, Xiaojie and Yang, Wenyao and Li, Shibin and Xu, Jianhua and Jiang, Yadong},
doi = {10.1039/C4RA06560C},
file = {:Users/marc/Library/Application Support/Mendeley Desktop/Downloaded/Yang et al. - 2014 - Porous conducting polymer and reduced graphene oxide nanocomposites for room temperature gas detection.pdf:pdf},
issn = {2046-2069},
journal = {RSC Adv.},
number = {80},
pages = {42546--42553},
publisher = {Royal Society of Chemistry},
title = {{Porous conducting polymer and reduced graphene oxide nanocomposites for room temperature gas detection}},
url = {http://xlink.rsc.org/?DOI=C4RA06560C},
volume = {4},
year = {2014}
}
@article{Saleh1969,
author = {Saleh, M},
doi = {10.1039/TF9706600242},
file = {:Users/marc/Library/Application Support/Mendeley Desktop/Downloaded/Saleh - 1969 - Interaction of Sulphur Compounds with Palladium.pdf:pdf},
issn = {00147672},
pages = {242--250},
title = {{Interaction of Sulphur Compounds with Palladium}},
year = {1969}
}
@article{Zhao1998,
abstract = {A thin palladium composite membrane was produced by modified electroless plating procedure. Compared with the conventional electroless plating procedure, the modified electroless plating procedure consists of the activation of a ceramic substrate by the sol-gel process of a Pd(II)-modified boehmite sol. Additionally, the infiltration of an electroless plating solution to a porous substrate during the deposition of palladium was employed with the filter device to improve adherence of a palladium layer to a substrate. The resulting membrane with a thickness of about 1 ??m has a high compactness. The membrane shows a hydrogen selectivity of 20-130 for H2/N2, and a hydrogen flux of 1.8-87 m3/m2.h, depending on operation conditions.},
author = {Zhao, H. B. and Pflanz, K. and Gu, J. H. and Li, A. W. and Stroh, N. and Brunner, H. and Xiong, G. X.},
doi = {10.1016/S0376-7388(97)00287-1},
file = {:Users/marc/Library/Application Support/Mendeley Desktop/Downloaded/Zhao et al. - 1998 - Preparation of palladium composite membranes by modified electroless plating procedure(2).pdf:pdf},
issn = {03767388},
journal = {Journal of Membrane Science},
keywords = {Electroless plating,Hydrogen separation,Pd membrane,Sol-gel process},
number = {2},
pages = {147--157},
title = {{Preparation of palladium composite membranes by modified electroless plating procedure}},
volume = {142},
year = {1998}
}
@article{Lee2016a,
author = {Lee, Melanie and Wang, Bo and Li, K},
doi = {10.1016/j.memsci.2015.12.047},
isbn = {03767388},
journal = {Journal of Membrane Science},
pages = {48--58},
title = {{New designs of ceramic hollow fibres toward broadened applications}},
volume = {503},
year = {2016}
}
@article{Zhang2012a,
abstract = {Hydrogen-based energy is a promising renewable and clean resource. Thus, hydrogen selective microporous membranes with high performance and high stability are demanded. Novel NH2-MIL-53(Al) membranes are evaluated for hydrogen separation for this goal. Continuous NH2-MIL-53(Al) membranes have been prepared successfully on macroporous glass frit discs assisted with colloidal seeds. The gas sorption ability of NH2-MIL-53(Al) materials is studied by gas adsorption measurement. The isosteric heats of adsorption in a sequence of CO2 {\textgreater} N2 {\textgreater} CH4 H2 indicates different interactions between NH2-MIL-53(Al) framework and these gases. As-prepared membranes are measured by single and binary gas permeation at different temperatures. The results of singe gas permeation show a decreasing permeance in an order of H2 {\textgreater} CH4 {\textgreater} N2 {\textgreater} CO2, suggesting that the diffusion and adsorption properties make significant contributions in the gas permeation through the membrane. In binary gas permeation, the NH2-MIL-53(Al) membrane shows high selectivity for H2 with separation factors of 20.7, 23.9 and 30.9 at room temperature (288 K) for H2 over CH4, N2 and CO2, respectively. In comparison to single gas permeation, a slightly higher separation factor is obtained due to the competitive adsorption effect between the gases in the porous MOF membrane. Additionally, the NH2-MIL-53(Al) membrane exhibits very high permeance for H2 in the mixtures separation (above 1.5 x 10-6 mol m-2 s-1 Pa-1) due to its large cavity, resulting in a very high separation power. The details of the temperature effect on the permeances of H2 over other gases are investigated from 288 to 353 K. The supported NH2-MIL-53(Al) membranes with high hydrogen separation power possess high stability, resistance to cracking, temperature cycling and show high reproducibility, necessary for the potential application to hydrogen recycling.},
author = {Zhang, Feng and Zou, Xiaoqin and Gao, Xue and Fan, Songjie and Sun, Fuxing and Ren, Hao and Zhu, Guangshan},
doi = {10.1002/adfm.201200084},
file = {:Users/marc/Library/Application Support/Mendeley Desktop/Downloaded/Zhang et al. - 2012 - Hydrogen selective NH2-MIL-53(Al) MOF membranes with high permeability.pdf:pdf},
isbn = {1616-301X},
issn = {1616301X},
journal = {Advanced Functional Materials},
keywords = {high permeability,hydrogen separation,metal-organic framework membrane},
number = {17},
pages = {3583--3590},
title = {{Hydrogen selective NH2-MIL-53(Al) MOF membranes with high permeability}},
volume = {22},
year = {2012}
}
@article{Peters2016,
author = {Peters, T A and Polfus, J M and Stange, M and Veenstra, P and Nijmeijer, A and Bredesen, R},
doi = {10.1016/j.fuproc.2016.06.012},
file = {:Users/marc/Library/Application Support/Mendeley Desktop/Downloaded/Peters et al. - 2016 - H2 flux inhibition and stability of Pd-Ag membranes under exposure to trace amounts of NH3.pdf:pdf},
isbn = {03783820},
journal = {Fuel Processing Technology},
pages = {259--265},
title = {{H2 flux inhibition and stability of Pd-Ag membranes under exposure to trace amounts of NH3}},
volume = {152},
year = {2016}
}
@article{Amandusson2001,
abstract = {The hydrogen permeation through surface modified Pd and Pd70Ag30 membranes has been studied at temperatures between 100 and 350°C. Silver has been evaporated on Pd and Pd70Ag30 foils with a thickness of 25$\mu$m in order to study the role of the surface composition in comparison with the membrane bulk composition. The Pd70Ag30-based membranes display the largest permeation rates at temperatures below 200°C, while Pd membranes with 20{\AA} silver evaporated on the upstream side show the largest permeation rates above 200°C. There are, consequently, different rate limiting processes above and below 200°C: at temperatures below 200°C, the bulk diffusion through the membrane is rate limiting, while at temperatures above 200°C, the influence of the surface composition starts to become significant. It has further been concluded that a sharp silver concentration gradient from the surface to the bulk is important for the hydrogen permeation rate at temperatures above 200°C. Adding oxygen to the hydrogen supply will almost totally inhibit the hydrogen permeation rate when a pure Pd membrane surface is facing the upstream side, while for silver-containing surfaces the presence of oxygen has almost no effect. On a clean Pd surface, oxygen effectively consumes adsorbed hydrogen in a water forming reaction. With Ag on the surface, no water formation is detected. Co-supplied CO inhibits the permeation of hydrogen in a similar manner on all studied membrane surfaces, independent of surface silver content. {\textcopyright} 2001 Elsevier Science B.V. All rights reserved.},
author = {Amandusson, H. and Ekedahl, L. G. and Dannetun, H.},
doi = {10.1016/S0376-7388(01)00414-8},
file = {:Users/marc/Library/Application Support/Mendeley Desktop/Downloaded/Amandusson, Ekedahl, Dannetun - 2001 - Hydrogen permeation through surface modified Pd and PdAg membranes.pdf:pdf},
isbn = {03767388 (ISSN)},
issn = {03767388},
journal = {Journal of Membrane Science},
keywords = {Hydrogen,Oxygen,Palladium,Permeation,Silver},
number = {1},
pages = {35--47},
title = {{Hydrogen permeation through surface modified Pd and PdAg membranes}},
volume = {193},
year = {2001}
}
@article{Ozaki2003b,
abstract = {We prepared a series of V-(15 - x)Ni-xAl (0.09 ??? x ??? 4.5) alloys by arc melting and investigated their hydrogen permeation characteristics using palladium-coated disk samples. In x ??? 0.9 alloys, precipitates of the sigma phase were observed in the matrix of the bcc phase. In alloys with higher Al content, a single bcc phase was observed. The hydrogen permeability increased in parallel with the Al content at temperatures higher than 473 K. The hydrogen permeability of the V-10.5Ni-4.5Al membrane was 6.29 ?? 10
                        -8 mol H
                        2 m
                        -1 s
                        -1 Pa
                        -1/2 at 623 K, or about twice that of the V-15Ni membrane. The hydrogen diffusivity was only weakly dependent on the Al content, hence the increase in the hydrogen permeability in parallel with the Al content was concluded to be mostly due to the increase in hydrogen solubility. The hydrogen permeability of the V-10.5Ni-4.5Al membrane decreased to 59{\%} of the initial value during a permeation test conducted for 102 h. However, a subsequent baking treatment with air introduced into the permeation system resulted in a complete recovery of the hydrogen permeability to the initial value. ?? 2003 International Association for Hydrogen Energy. Published by Elsevier Ltd. All rights reserved.},
author = {Ozaki, Tetsuya and Zhang, Yi and Komaki, Masao and Nishimura, Chikashi},
doi = {10.1016/S0360-3199(02)00251-3},
file = {:Users/marc/Library/Application Support/Mendeley Desktop/Downloaded/Ozaki et al. - 2003 - Hydrogen permeation characteristics of V-Ni-Al alloys.pdf:pdf},
isbn = {03603199},
issn = {03603199},
journal = {International Journal of Hydrogen Energy},
keywords = {Aluminum,Diffusion,Hydrogen permeation,Nickel,Vanadium},
number = {11},
pages = {1229--1235},
title = {{Hydrogen permeation characteristics of V-Ni-Al alloys}},
volume = {28},
year = {2003}
}
@article{Didenko2016,
author = {Didenko, L P and Savchenko, V I and Sementsova, L A and Bikov, L A},
doi = {10.1016/j.ijhydene.2015.10.107},
file = {:Users/marc/Library/Application Support/Mendeley Desktop/Downloaded/Didenko et al. - 2016 - Hydrogen flux through the membrane based on the Pd–In–Ru foil.pdf:pdf},
isbn = {03603199},
journal = {International Journal of Hydrogen Energy},
number = {1},
pages = {307--315},
title = {{Hydrogen flux through the membrane based on the Pd–In–Ru foil}},
volume = {41},
year = {2016}
}
@article{Chen1996,
author = {Chen, F L and Kinari, Y and Sakamoto, F and Nakayama, Y and Sakamoto, Y},
file = {:Users/marc/Library/Application Support/Mendeley Desktop/Downloaded/Chen et al. - 1996 - Hydrogen permeation through palladium-based alloy membranes in mixtures of 10{\%} Methane and Ethylene in the Hydrogen.pdf:pdf},
journal = {International Journal of Hydrogen Energy},
number = {7},
pages = {555--561},
title = {{Hydrogen permeation through palladium-based alloy membranes in mixtures of 10{\%} Methane and Ethylene in the Hydrogen}},
volume = {21},
year = {1996}
}
@article{Petit2010a,
abstract = {Composites of the metal-organic framework (MOF), MOF-5, and graphite oxide (GO) with different ratios of the two components are prepared and tested in ammonia removal under dry conditions. The parent and composite materials are characterized before and after exposure to ammonia by sorption of N(2), X-ray diffraction, thermal analyses, and FT-IR spectroscopy. The results show a synergetic effect resulting in an increase in the ammonia uptake compared to the parent materials. It is linked to enhanced dispersive forces in the pore space of the composites. Additionally, ammonia interacts with zinc oxide tetrahedra via hydrogen bonding and is intercalated between the layers of GO. Retention of a large quantity of ammonia eventually leads to a collapse of the MOF-5 structure in the composites. The effect resembles that observed when MOF-5 is exposed to water. Taking into account the similarity of ammonia and water molecules, it is hypothesized that ammonia causes a destruction of the MOF-5 and composite structure as a result of its hydrogen bonding with the zinc oxide clusters.},
author = {Petit, Camille and Bandosz, Teresa J.},
doi = {10.1002/adfm.200900880},
file = {:Users/marc/Library/Application Support/Mendeley Desktop/Downloaded/Petit, Bandosz - 2010 - Enhanced adsorption of ammonia on metal-organic frameworkgraphite oxide composites Analysis of surface interacti.pdf:pdf},
isbn = {1616-301X},
issn = {1616301X},
journal = {Advanced Functional Materials},
number = {1},
pages = {111--118},
title = {{Enhanced adsorption of ammonia on metal-organic framework/graphite oxide composites: Analysis of surface interactions}},
volume = {20},
year = {2010}
}
@article{Chen2005,
author = {Chen, Huey‐Ing and Chu, Chin‐Yi and Huang, Ting‐Chia},
doi = {10.1081/ss-120030789},
file = {:Users/marc/Library/Application Support/Mendeley Desktop/Downloaded/Chen, Chu, Huang - 2005 - Comprehensive Characterization and Permeation Analysis of Thin PdAl2O3Composite Membranes Prepared by Suctio.pdf:pdf},
isbn = {0149-6395
1520-5754},
journal = {Separation Science and Technology},
number = {7},
pages = {1461--1483},
title = {{Comprehensive Characterization and Permeation Analysis of Thin Pd/Al2O3Composite Membranes Prepared by Suction‐Assisted Electroless Deposition}},
volume = {39},
year = {2005}
}
@article{Okazaki2006,
abstract = {A series of Pd-Ag membranes with different atomic ratio (Ag = 0, 5, 10, 15, 20 and 23{\%}) was fabricated by controlling the chemical components in the electroless plating bath followed by thermal annealing of the deposited metals. Gas permeation of the membranes was examined at the temperature ranges at 100-300 °C. The hydrogen flux for the membranes of low Ag content gave a marked change giving a peak in the range of 170-200 °C, which was attributed to the $\alpha$-$\beta$ crystal phase transition of the hydride. The peak in hydrogen flux became less significant and appears as shoulder along with the increase of silver content. This observation coincides with the decrease of lattice size difference between $\alpha$ and $\beta$ phase by the increase of Ag content in the Pd-Ag alloy. Improved durability of the alloy membrane (Ag {\textgreater} 20{\%}) was demonstrated by cyclic change of gas and temperature and was attributed to the suppression of lattice expansion by alloying with more than 20{\%} of silver. {\textcopyright} 2006 Elsevier B.V. All rights reserved.},
author = {Okazaki, Junya and Tanaka, David A Pacheco and Tanco, Margot A Llosa and Wakui, Yoshito and Mizukami, Fujio and Suzuki, Toshishige M.},
doi = {10.1016/j.memsci.2006.05.042},
file = {:Users/marc/Library/Application Support/Mendeley Desktop/Downloaded/Okazaki et al. - 2006 - Hydrogen permeability study of the thin Pd-Ag alloy membranes in the temperature range across the $\alpha$-$\beta$ phase tr.pdf:pdf},
issn = {03767388},
journal = {Journal of Membrane Science},
keywords = {Durability,Hydrogen permeation,Pd-Ag alloy membrane,$\alpha$-$\beta$ phase transition},
number = {1-2},
pages = {370--374},
title = {{Hydrogen permeability study of the thin Pd-Ag alloy membranes in the temperature range across the $\alpha$-$\beta$ phase transition}},
volume = {282},
year = {2006}
}
@article{Long2009,
abstract = {RSC Publishing Logo View PDF VersionView Previous ArticleView Next Article DOI:10.1039/B903811F (Editorial) Chem. Soc. Rev., 2009, 38, 1213-1214. The pervasive chemistry ofmetalorganic frameworks. ... DOI: 10.1039/b817735j; Zaworotko et al. DOI: 10.1039/b807086p). ...},
author = {Long, Jeffrey R. and Yaghi, Omar M.},
doi = {10.1039/b903811f},
file = {:Users/marc/Library/Application Support/Mendeley Desktop/Downloaded/Long, Yaghi - 2009 - The pervasive chemistry of metal–organic frameworks.pdf:pdf},
isbn = {0306-0012},
issn = {0306-0012},
journal = {Chemical Society Reviews},
number = {5},
pages = {1213},
pmid = {19384431},
title = {{The pervasive chemistry of metal–organic frameworks}},
url = {http://xlink.rsc.org/?DOI=b903811f},
volume = {38},
year = {2009}
}
@article{Bhandari2009,
author = {Bhandari, Rajkumar and Ma, Yi Hua},
doi = {10.1016/j.memsci.2009.02.014},
isbn = {03767388},
journal = {Journal of Membrane Science},
number = {1-2},
pages = {50--63},
title = {{Pd–Ag membrane synthesis: The electroless and electro-plating conditions and their effect on the deposits morphology}},
volume = {334},
year = {2009}
}
@article{Chen2012,
author = {Chen, Chee and Gobina, Edward},
doi = {10.1016/s0958-2118(12)70252-5},
file = {:Users/marc/Library/Application Support/Mendeley Desktop/Downloaded/Chen, Gobina - 2012 - Trial designs of ultra-thin palladium alloy membrane purifiers for high-density hydrogen production Pt. 1.pdf:pdf},
isbn = {09582118},
journal = {Membrane Technology},
number = {12},
pages = {7--12},
title = {{Trial designs of ultra-thin palladium alloy membrane purifiers for high-density hydrogen production Pt. 1}},
volume = {2012},
year = {2012}
}
@misc{BOC2018,
author = {BOC},
title = {{ECOVAR – onsite gas generation solutions}},
year = {2018}
}
@article{Itta2011,
abstract = {Through a spin-coating technique, a high performance carbon molecular sieve (CMS) membranes were fabricated from thermally stable polymer polyphenylene oxide (PPO) and thermally labile polymer poly vinylpyrrolidone (PVP). The permeation results show that the small gas molecules (H2, CO2, N2, and CH4) transport mechanism is dominated by the molecular sieving effect. The permeation performances have a strong dependency upon polymer concentration and pyrolysis temperature. The best performance for hydrogen permeability obtained with PPO 15 PVP pyrolyzed at 700°C was 1121 Barrer (1 Barrer=1×10-10cm3(STP) cm/[cm2scm Hg]) and the values of selectivity for gas pairs such as H2/N2and H2/CH4were 163.9 and 160.9, respectively. The correlation factor of permselectivity of H2/N2and H2/CH4gas pairs obtained from PPO and PPO/PVP derived CMS membranes were above the Robeson (2008) upper bound. The addition of thermally labile PVP creates diffusion pathways and controls selectivities for the CMS membranes derived from PPO 10 PVP and PPO 15 PVP. {\textcopyright} 2011 Elsevier B.V.},
author = {Itta, Arun Kumar and Tseng, Hui Hsin and Wey, Ming Yen},
doi = {10.1016/j.memsci.2011.02.027},
file = {:Users/marc/Library/Application Support/Mendeley Desktop/Downloaded/Itta, Tseng, Wey - 2011 - Fabrication and characterization of PPOPVP blend carbon molecular sieve membranes for H 2N 2 and H 2CH 4 separ.pdf:pdf},
issn = {03767388},
journal = {Journal of Membrane Science},
keywords = {Blending,CMS membrane,Gas separation,Polymer,Spin coating},
number = {1-2},
pages = {387--395},
publisher = {Elsevier B.V.},
title = {{Fabrication and characterization of PPO/PVP blend carbon molecular sieve membranes for H 2/N 2 and H 2/CH 4 separation}},
url = {http://dx.doi.org/10.1016/j.memsci.2011.02.027},
volume = {372},
year = {2011}
}
@article{Escolastico2013,
abstract = {Tungstates Ln6WO12 are proton conducting crystalline materials which show significant mixed protonic and electronic conductivity and stability in moist CO2 environments. These materials are promising candidates for hydrogen separation membranes at temperatures above 700°C, especially when integrated in intensified processes as catalytic membrane reactors. From this family of compounds, La6-xWOy (0.4{\textless}x{\textless}0.7) presents the highest protonic conductivity and hydrogen permeability. This contribution presents the preparation, hydrogen permeation and stability study of La5.5WO11.25-$\delta$ membranes. Hydrogen separation properties of La5.5WO11.25-$\delta$ were systematically analyzed, i.e. the influence of the H2 concentration in feed stream, humidification degree and operating temperature were investigated. The integrity of the La5.5WO11.25-$\delta$ after the hydrogen permeation experiments was confirmed by X-ray diffraction (XRD), scanning electron microscopy (SEM) and Raman spectroscopy. The stability of La5.5WO11.25-$\delta$ permeation when CO2 containing atmosphere was used as sweep gas was evaluated. Furthermore, the stability of this compound in CO2-rich and sulfur-containing environments was confirmed by XRD. {\textcopyright} 2013 Elsevier B.V.},
author = {Escol{\'{a}}stico, Sonia and Sol{\'{i}}s, Cecilia and Scherb, Tobias and Schumacher, Gerhard and Serra, Jose M.},
doi = {10.1016/j.memsci.2013.05.005},
file = {:Users/marc/Library/Application Support/Mendeley Desktop/Downloaded/Escol{\'{a}}stico et al. - 2013 - Hydrogen separation in La5.5WO11.25-$\delta$ membranes.pdf:pdf},
issn = {03767388},
journal = {Journal of Membrane Science},
keywords = {CO2 stable,Hydrogen membrane,Lanthanum tungstate,Mixed conductor,Proton conductor},
pages = {276--284},
publisher = {Elsevier},
title = {{Hydrogen separation in La5.5WO11.25-$\delta$ membranes}},
url = {http://dx.doi.org/10.1016/j.memsci.2013.05.005},
volume = {444},
year = {2013}
}
@article{Vellingiri2016,
abstract = {In this research, we investigated the sorptive behavior of a mixture of 14 volatile and semi-volatile organic compounds (four aromatic hydrocarbons (benzene, toluene, p-xylene, and styrene), six C 2-C 5 volatile fatty acids (VFAs), two phenols, and two indoles) against three metal-organic frameworks (MOFs), i.e., MOF-5, Eu-MOF, and MOF-199 at 5 to 10 mPa VOC partial pressures (25 °C). The selected MOFs exhibited the strongest affinity for semi-volatile (polar) VOC molecules (skatole), whereas the weakest affinity toward was volatile (non-polar) VOC molecules (i.e., benzene). Our experimental results were also supported through simulation analysis in which polar molecules were bound most strongly to MOF-199, reflecting the presence of strong interactions of Cu 2+ with polar VOCs. In addition, the performance of selected MOFs was compared to three well-known commercial sorbents (Tenax TA, Carbopack X, and Carboxen 1000) under the same conditions. The estimated equilibrium adsorption capacity (mg.g-1) for the all target VOCs was in the order of; MOF-199 (71.7) {\textgreater}Carboxen-1000 (68.4) {\textgreater}Eu-MOF (27.9) {\textgreater}Carbopack X (24.3) {\textgreater}MOF-5 (12.7) {\textgreater}Tenax TA (10.6). Hopefully, outcome of this study are expected to open a new corridor to expand the practical application of MOFs for the treatment diverse VOC mixtures.},
author = {Vellingiri, Kowsalya and Szulejko, Jan E. and Kumar, Pawan and Kwon, Eilhann E. and Kim, Ki-Hyun and Deep, Akash and Boukhvalov, Danil W. and Brown, Richard J. C.},
doi = {10.1038/srep27813},
file = {:Users/marc/Library/Application Support/Mendeley Desktop/Downloaded/Vellingiri et al. - 2016 - Metal organic frameworks as sorption media for volatile and semi-volatile organic compounds at ambient condit.pdf:pdf},
issn = {2045-2322},
journal = {Scientific Reports},
number = {1},
pages = {27813},
publisher = {Nature Publishing Group},
title = {{Metal organic frameworks as sorption media for volatile and semi-volatile organic compounds at ambient conditions}},
url = {http://www.nature.com/articles/srep27813},
volume = {6},
year = {2016}
}
@article{Paillard2012,
author = {Paillard, Claude-etienne},
doi = {10.1016/j.ijhydene.2012.08.026},
file = {:Users/marc/Library/Application Support/Mendeley Desktop/Downloaded/Paillard, Naudet - 2012 - Risk assessment for the preparation of compressed oxidant-fuel gas mixtures. Application to the H 2air mixture.pdf:pdf},
title = {{Risk assessment for the preparation of compressed oxidant e fuel gas mixtures . Application to the H 2 / air mixture manufacturing}},
volume = {7},
year = {2012}
}
@article{Xomeritakis1997,
abstract = {Thin (0.1-1.5 ??m) Pd and Pd/Ag alloy membranes have been prepared on porous ceramic substrates consisting of a macroporous ??-Al2O3 disk coated with a sol-gel derived, mesoporous ??-Al2O3 top layer. Metallorganic chemical vapor deposition (MOCVD) and magnetron sputtering were used to coat the thin metallic membranes employing Pd(II) acetylacetonate and a 75{\%} Pd-25{\%} Ag alloy target, respectively. The gas transport properties of the thin metallic membranes were determined by multicomponent permeation experiments with He, H2 and Ar at 25-300??C and 1 arm total pressure. The H2 permeance and H2: He selectivity were in the range 1.0-2.0 x 10-7 mol m-2s-1Pa-1 and 30-200 at 300??C, respectively. The dependence of H2 permeation rates on membrane thickness and temperature suggest that surface reaction steps are rate-limiting for H2 transport through the thin, ceramic-supported metallic membranes made by MOCVD and sputtering.},
author = {Xomeritakis, George and Lin, Y. S.},
doi = {10.1016/S0376-7388(97)00084-7},
file = {:Users/marc/Library/Application Support/Mendeley Desktop/Downloaded/Xomeritakis, Lin - 1997 - Fabrication of thin metallic membranes by MOCVD and sputtering.pdf:pdf},
issn = {03767388},
journal = {Journal of Membrane Science},
keywords = {Chemical vapor deposition,Gas separations,Magnetron sputtering,Metallic membranes},
number = {2},
pages = {217--230},
title = {{Fabrication of thin metallic membranes by MOCVD and sputtering}},
volume = {133},
year = {1997}
}
@article{Khatib2011,
abstract = {The application of inorganic silica membranes for sulfur hexafluoride selectivity separation at elevated temperatures has attracted much attention due to their good permselectivity and mechanical strength. These membranes are usually in the form of a thin layer of silica (selective layer) deposited on a thick porous support. One of the methods which is successfully used for deposition of the silica layers is the chemical vapor deposition (CVD), due to its versatility and reproducibility as well as high selectivities obtained with the membranes formed by this method. This chapter starts with a brief description of the basic principles of CVD and its application in the preparation of silica membranes, followed by a complete literature review which surveys the studies that have been carried out on supported silica membranes prepared through CVD and applied in hydrogen separation with two of the most commonly used supports, Vycor glass and alumina. {\textcopyright} 2011 Elsevier B.V.},
author = {Khatib, Sheima Jatib and Oyama, Ted T. and de Souza, K{\'{a}}tia R. and Noronha, F{\'{a}}bio B.},
doi = {10.1016/B978-0-444-53728-7.00002-1},
file = {:Users/marc/Library/Application Support/Mendeley Desktop/Downloaded/Khatib et al. - 2011 - Review of Silica Membranes for Hydrogen Separation Prepared by Chemical Vapor Deposition.pdf:pdf},
isbn = {978-0-444-53728-7},
issn = {09275193},
journal = {Membrane Science and Technology},
keywords = {Alumina,CVD,Hydrogen permeability,Hydrogen separation,Inorganic membrane,Permselectivity,Silica membrane,Vycor},
pages = {25--60},
title = {{Review of Silica Membranes for Hydrogen Separation Prepared by Chemical Vapor Deposition}},
volume = {14},
year = {2011}
}
@article{AirProducts,
author = {{Air Products}},
file = {:Users/marc/Library/Application Support/Mendeley Desktop/Downloaded/Air Products - Unknown - PRISM {\textregistered} Membrane Systems for ammonia plants.pdf:pdf},
title = {{PRISM {\textregistered} Membrane Systems for ammonia plants}}
}
@article{Liu2013b,
abstract = {Inspired by the bioadhesive ability of the marine mussel, a simple, versatile, and powerful synthesis strategy was developed to prepare highly reproducible and permselective molecular sieve membranes by using polydopamine as a novel covalent linker. Attributing to the formation of strong covalent and noncovalent bonds, ZIF-8 nutrients are attracted and bound to the support surface, thus promoting the ZIF-8 nucleation and the growth of uniform, well intergrown, and phase-pure ZIF-8 molecular sieve membranes. The developed ZIF-8 membranes show high hydrogen selectivity and thermal stability. At 150 °C and 1 bar, the mixture separation factors of H2/CO2, H2/N2, H2/CH4, and H2/C3H8 are 8.9, 16.2, 31.5 and 712.6, with H2 permeances higher than 1.8 × 10–7 mol{\textperiodcentered}m–2{\textperiodcentered}s–1{\textperiodcentered}Pa–1, which is promising for hydrogen separation and purification.},
author = {Liu, Qian and Wang, Nanyi and Caro, J{\"{u}}rgen and Huang, Aisheng},
doi = {10.1021/ja4080562},
file = {:Users/marc/Library/Application Support/Mendeley Desktop/Downloaded/Liu et al. - 2013 - Bio-inspired polydopamine A versatile and powerful platform for covalent synthesis of molecular sieve membranes.pdf:pdf},
isbn = {0002-7863},
issn = {00027863},
journal = {Journal of the American Chemical Society},
number = {47},
pages = {17679--17682},
pmid = {24224527},
title = {{Bio-inspired polydopamine: A versatile and powerful platform for covalent synthesis of molecular sieve membranes}},
volume = {135},
year = {2013}
}
@article{Løvvik2014,
author = {L{\o}vvik, O M and Peters, T A and Bredesen, R},
doi = {10.1016/j.memsci.2013.11.035},
file = {:Users/marc/Library/Application Support/Mendeley Desktop/Downloaded/L{\o}vvik, Peters, Bredesen - 2014 - First-principles calculations on sulfur interacting with ternary Pd – Ag-transition metal alloy mem.pdf:pdf},
issn = {0376-7388},
journal = {Journal of Membrane Science},
keywords = {Atomistic modeling,H2S,Membrane,Pd-alloy},
pages = {525--531},
publisher = {Elsevier},
title = {{First-principles calculations on sulfur interacting with ternary Pd – Ag-transition metal alloy membrane alloys}},
url = {http://dx.doi.org/10.1016/j.memsci.2013.11.035},
volume = {453},
year = {2014}
}
@article{Drahushuk2012,
abstract = {Graphene has enormous potential as a unique molecular barrier material with atomic layer thickness, enabling new types of membranes for separation and manipulation. However, the conventional analysis of diffusive transport through a membrane fails in the case of single layer graphene (SLG) and other 2D atomically thin membranes. In this work, analytical expressions are derived for gas permeation through such atomically thin membranes in various limits of gas diffusion, surface adsorption, or pore translocation as the rate-limiting step. Gas permeation can proceed via direct gas-phase interaction with the pore, or interaction via the adsorbed phase on the membrane exterior surface. A series of van der Waals force fields allows for the estimation of the energy barriers present for various types of graphene nanopores. These analytical models will assist in the understanding of molecular dynamics and experimental studies of such membranes.},
author = {Drahushuk, Lee W. and Strano, Michael S.},
doi = {10.1021/la303468r},
file = {:Users/marc/Library/Application Support/Mendeley Desktop/Downloaded/Drahushuk, Strano - 2012 - Mechanisms of gas permeation through single layer graphene membranes.pdf:pdf},
isbn = {0743-7463},
issn = {07437463},
journal = {Langmuir},
number = {48},
pages = {16671--16678},
pmid = {23101879},
title = {{Mechanisms of gas permeation through single layer graphene membranes}},
volume = {28},
year = {2012}
}
@article{Binions2009,
abstract = {The use of zeolites as transformation layers to enhance the response and discriminating power of solid-state metal-oxide-semiconductor gas sensors is demonstrated. Thick film sensors were prepared by screen printing layers of tungsten trioxide or chromium titanium oxide with various zeolites as overlayers. The sensors gas response was tested against carbon monoxide and ethanol in varying concentrations. Experimental result.,; show that it is possible to dramatically alter the response behavior of the devices: in the instance of ethanol gas with a zeolite Y-modified sensor. the response was increased by 2 orders of magnitude compared to the unmodified sensor. Computational modelling studies show that a combination of catalytic reaction and diffusion behavior are responsible for these changes. Such discriminatory behavior should prove useful in electronic noses and sensor arrays. (c) 2009 The Electrochemical Society. DOI: 10.1149/1.3065436 All rights reserved.},
author = {Binions, Russell and Davies, Helen and Afonja, Ayo and Dungey, Sheena and Lewis, Dewi and Williams, David E. and Parkin, Ivan P.},
doi = {10.1149/1.3065436},
file = {:Users/marc/Library/Application Support/Mendeley Desktop/Downloaded/Binions et al. - 2009 - Zeolite-Modified Discriminating Gas Sensors.pdf:pdf},
isbn = {0013-4651},
issn = {00134651},
journal = {Journal of The Electrochemical Society},
number = {3},
pages = {J46},
pmid = {1000253636},
title = {{Zeolite-Modified Discriminating Gas Sensors}},
url = {http://jes.ecsdl.org/cgi/doi/10.1149/1.3065436},
volume = {156},
year = {2009}
}
@article{Dou2014,
abstract = {Luminescent metal-organic framework films, CPM-5 superset of Tb3+ and MIL-100(In)superset of Tb3+, have been constructed by postfunctionalization of two porous indium-organic frameworks with different structures, respectively. The MIL-100(In)superset of Tb3+ film shows high oxygen sensitivity (K-SV = 7.59) and short response/recovery time (6 s/53 s).},
author = {Dou, Zhongshang and Yu, Jiancan and Cui, Yuanjing and Yang, Yu and Wang, Zhiyu and Yang, Deren and Qian, Guodong},
doi = {10.1021/ja411224j},
file = {:Users/marc/Library/Application Support/Mendeley Desktop/Downloaded/Dou et al. - 2014 - Luminescent metal-organic framework films as highly sensitive and fast-response oxygen sensors.pdf:pdf},
isbn = {0002-7863},
issn = {15205126},
journal = {Journal of the American Chemical Society},
number = {15},
pages = {5527--5530},
pmid = {24697214},
title = {{Luminescent metal-organic framework films as highly sensitive and fast-response oxygen sensors}},
volume = {136},
year = {2014}
}
@article{Løvvik2008,
abstract = {It is well known that silver segregates to the surface of pure and ideal Pd-Ag alloy surfaces. By first-principles band-structure calculations it is shown in this paper how this may be changed when hydrogen is adsorbed on a Pd-Ag(1 1 1) surface. Due to hydrogen binding more strongly to palladium than to silver, there is a clear energy gain from a reversal of the surface segregation. Hydrogen-induced segregation may provide a fundamental explanation for the hydrogen or reducing treatments that are required to activate hydrogen-selective membrane or catalyst performance. ?? 2008 Elsevier B.V. All rights reserved.},
author = {L{\o}vvik, O. M. and Opalka, Susanne M.},
doi = {10.1016/j.susc.2008.07.016},
file = {:Users/marc/Library/Application Support/Mendeley Desktop/Downloaded/L{\o}vvik, Opalka - 2008 - Reversed surface segregation in palladium-silver alloys due to hydrogen adsorption.pdf:pdf},
isbn = {0039-6028},
issn = {00396028},
journal = {Surface Science},
keywords = {Density-functional calculations,Hydrogen,Palladium,Silver,Surface segregation},
number = {17},
pages = {2840--2844},
title = {{Reversed surface segregation in palladium-silver alloys due to hydrogen adsorption}},
volume = {602},
year = {2008}
}
@misc{Girling2008,
author = {Girling, R. D.},
booktitle = {Agricultural and Forest Entomology},
doi = {10.1111/j.1461-9563.2008.00379.x},
isbn = {0875844162},
issn = {0021-8561},
number = {4},
pages = {1163--306},
pmid = {27075100},
title = {{Myzus persicae}},
volume = {25},
year = {2008}
}
@article{Yan2010,
abstract = {The cermet consisting of electronic conductor Ni and proton conductor La2Ce2O7 (LDC) shows good chemical stability but poor hydrogen permeability. In order to improve the hydrogen permeability, novel Ni-La2-xSmxCe2O7 (x = 0, 0.025, 0.05, 0.075, 0.1 and 0.2) cermets were developed for hydrogen separation. The results show that Sm element doping of LDC can affect the rate of hydrogen permeation, with Ni-La1.95Sm0.05Ce2O7 possessing the highest hydrogen permeation fluxes. {\textcopyright} 2010 Professor T. Nejat Veziroglu.},
author = {Yan, Litao and Sun, Wenping and Bi, Lei and Fang, Shumin and Tao, Zetian and Liu, Wei},
doi = {10.1016/j.ijhydene.2010.02.134},
file = {:Users/marc/Library/Application Support/Mendeley Desktop/Downloaded/Yan et al. - 2010 - Effect of Sm-doping on the hydrogen permeation of Ni-La2Ce2O7 mixed protonic-electronic conductor.pdf:pdf},
issn = {03603199},
journal = {International Journal of Hydrogen Energy},
keywords = {Cermet,Hydrogen separation,Proton conductor},
number = {10},
pages = {4508--4511},
publisher = {Elsevier Ltd},
title = {{Effect of Sm-doping on the hydrogen permeation of Ni-La2Ce2O7 mixed protonic-electronic conductor}},
url = {http://dx.doi.org/10.1016/j.ijhydene.2010.02.134},
volume = {35},
year = {2010}
}
@article{Kim2018,
abstract = {Metal-organic frameworks (MOFs) have attracted a special attention to the selective capture of harmful gases from air, owing to the presence of a high density of active surfaces that can be tailored by an appropriate modification. In this paper, recent studies on appropriate approaches for the selective capture of harmful gases (NH3, CO, H2S, NOx, SOx, Cl2, etc.) performed via experimental and computational methods are comprehensively reviewed with the aim of establishing well-designed strategies for the specific tasks. Three primary conclusions regarding the design strategy of MOFs are highlighted from the reviewed studies: the introduction of appropriate open metal sites for the selective capture of polar harmful gases, inefficiency of open metal sites introduced for the selective capture of non-polar harmful gases, and introduction of appropriate surface functionality for individual harmful gas. It is believed that the review will play a critical role in designing promising MOFs with appropriate surface chemistry for the selective capture of harmful gases from air.},
author = {Kim, Ki Chul},
doi = {10.1016/j.jorganchem.2017.11.017},
file = {:Users/marc/Library/Application Support/Mendeley Desktop/Downloaded/Kim - 2018 - Design strategies for metal-organic frameworks selectively capturing harmful gases.pdf:pdf},
issn = {0022328X},
journal = {Journal of Organometallic Chemistry},
keywords = {Design strategy,Harmful gas,Metal-organic frameworks,Open metal sites,Separations,Surface functionality},
pages = {94--105},
publisher = {Elsevier B.V},
title = {{Design strategies for metal-organic frameworks selectively capturing harmful gases}},
url = {https://doi.org/10.1016/j.jorganchem.2017.11.017},
volume = {854},
year = {2018}
}
@article{Gade2009,
abstract = {The addition of gold to palladium membranes produces many desirable effects for hydrogen purification, including improved tolerance of sulfur compounds, reduction in hydride phase formation, and, for certain compositions, improved hydrogen permeability. The focus of this work is to determine if sequential plating can be used to produce self-supported alloy membranes with equivalent properties to membranes produced by conventional metallurgical techniques such as cold-working. Sequential electroplating and electroless plating were used to produce freestanding planar Pd-Au membranes with Au contents ranging from 0 to 20 wt{\%}, consisting of Au layers on both sides of a pure Pd core. Membranes were characterized by single-gas permeation measurements, scanning electron microscopy with energy dispersive X-ray spectroscopy (SEM/EDS), and high temperature, controlled-atmosphere XRD (HTXRD). Sequentially plated foils tested without any prior annealing had significantly lower H2 permeabilities than either measured or literature values for homogeneous foils of equivalent composition. This effect appears to be due to the formation of stable gold-enriched surface layers. Pretreatment of membranes to 1023 K created membranes with hydrogen permeabilities equivalent to literature values, despite the fact that trace amounts of surface gold remained detectable with XRD. ?? 2009 Elsevier B.V. All rights reserved.},
annote = {NULL},
author = {Gade, Sabina K. and Payzant, E. Andrew and Park, Helen J. and Thoen, Paul M. and Way, J. Douglas},
doi = {10.1016/j.memsci.2009.05.034},
file = {:Users/marc/Library/Application Support/Mendeley Desktop/Downloaded/Gade et al. - 2009 - The effects of fabrication and annealing on the structure and hydrogen permeation of Pd-Au binary alloy membranes.pdf:pdf},
isbn = {0376-7388},
issn = {03767388},
journal = {Journal of Membrane Science},
keywords = {Electroless plating,Hydrogen separation,Metal membranes,Palladium membranes,Palladium-gold alloy},
number = {1-2},
pages = {227--233},
title = {{The effects of fabrication and annealing on the structure and hydrogen permeation of Pd-Au binary alloy membranes}},
volume = {340},
year = {2009}
}
@article{Gharibi2013,
author = {Gharibi, Hussein and Saadatinasab, Mohammadamir and Zolfaghari, Alireza},
doi = {10.1016/j.memsci.2013.07.037},
isbn = {03767388},
journal = {Journal of Membrane Science},
pages = {355--361},
title = {{Hydrogen permeability and sulfur tolerance of a novel dual membrane of PdAg/PdCu layers deposited on porous stainless steel}},
volume = {447},
year = {2013}
}
@article{Zhang2017c,
abstract = {This paper reports metal oxide (MOx)-decorated graphene-based sensor array combining with back-propagation (BP) neural network toward the detection of indoor air pollutant exposure. Tin dioxide (SnO2) nanospheres and copper oxide (CuO) nanoflowers-decorated graphene were used as candidates for formaldehyde and ammonia gas sensing, respectively. The as-synthesized sensing materials were characterized in terms of their nanostructural, morphological and compositional features by SEM, Raman spectra, and XRD. The sensor array was fabricated via one-step hydrothermal route and layer-by-layer (LbL) self-assembly technique on the substrate with interdigital microelectrodes. The sensing properties of MOx/graphene composite toward the mixture gas of ammonia and formaldehyde, such as dynamic response, sensitivity, response/recovery time, and stability, were investigated at room temperature. And furthermore, this work successfully achieved the recognition and quantitative prediction of components in the gas mixture of formaldehyde and ammonia through the combination of MOx/graphene-based sensor array and neural network-based signal processing technologies.},
author = {Zhang, Dongzhi and Liu, Jingjing and Jiang, Chuanxing and Liu, Aiming and Xia, Bokai},
doi = {10.1016/j.snb.2016.08.085},
file = {:Users/marc/Library/Application Support/Mendeley Desktop/Downloaded/Zhang et al. - 2017 - Quantitative detection of formaldehyde and ammonia gas via metal oxide-modified graphene-based sensor array combin.pdf:pdf},
issn = {09254005},
journal = {Sensors and Actuators, B: Chemical},
keywords = {Graphene,Layer-by-layer self-assembly,Neural network model,Sensor array},
pages = {55--65},
publisher = {Elsevier B.V.},
title = {{Quantitative detection of formaldehyde and ammonia gas via metal oxide-modified graphene-based sensor array combining with neural network model}},
url = {http://dx.doi.org/10.1016/j.snb.2016.08.085},
volume = {240},
year = {2017}
}
@article{Thoen2006,
author = {Thoen, Paul M and Roa, Fernando and Way, J Douglas},
doi = {10.1016/j.desal.2005.09.025},
file = {:Users/marc/Library/Application Support/Mendeley Desktop/Downloaded/Thoen, Roa, Way - 2006 - High flux palladium–copper composite membranes for hydrogen separations.pdf:pdf},
isbn = {00119164},
journal = {Desalination},
number = {1-3},
pages = {224--229},
title = {{High flux palladium–copper composite membranes for hydrogen separations}},
volume = {193},
year = {2006}
}
@misc{Undefinedhowtoevokesuchemotionsthroughaproduct.Inotherwordsthisisaninvestigationofthemeaningthatcouldbedesignedintoaproductinordertocommunicatewiththeuseratanemotionallevel.Aliteraturesurveyofrecentdesigntra,
author = {in one's mind; which produc {undefined  how to evoke such emotions through a product. In other words, this is an investigation of the "meaning" that could be designed into a product in order to "communicate" with the user at an emotional level. A literature survey of recent design tr}, undefined P Y - This paper explores theoretical issues in ergonomics related to semantics and the emotional content of design. The aim is to find answers to the following questions: how to design products triggering "happiness" and undefined Undefined and response surface methodology {undefined  and the allocation and evaluation of tuning parameters in separation from the evolutionary process. A combination of a genetic algorithm and a gradient-based algorithm is used for tuning of the approximation functions. The problem of the choice}, undefined P Y - This thesis addresses two problems arising in many real-life design optimization applications: the high computational cost of function evaluations and the presence of numerical noise in the function values. The and undefined Undefined and undefined Undefined and particular {undefined  engineering design methods must integrate the many different aspects of designing and the priorities of the end-user. Engineering Design (3rd edition) describes a systematic approach to engineering design. The authors argue that such an approac}, undefined P Y - Engineering design must be carefully planned and systematically executed. In and this paper undefined  we introduce a general framework for product experience that applies to all affective responses that can be experienced in human-product interaction. Three distinct components or levels of product experiences are discussed: aesthetic experience, undefined P Y - In and of products requiring chemical undefined  a new tape that one can remove without peeling off paint, computer chips and an oxygen- enriching device. We find more of our students involved in product design, both in the traditional chemical industry, where the move to high value added che, undefined P Y - We respond to a request to examine the area of product design and its impact on chemical engineering. We focus on how we are educating our undergraduates and deliberately adopt a controversial stand. Examples and study aims to elucidate how industrial designers and engineering designers collaborate {undefined  and how such an alliance reflects in the design process. We conducted in-depth interviews about actual product design projects with 34 industrial and engineering designers from six consumer product manufacturers. We firstly identified individua}, undefined P Y - This and such contemporary design and development issues as identifying customer needs {undefined  design for manufacturing, prototyping, and industrial design, Product Design and Development by Ulrich and Eppinger presents in a clear and detailed way a set of product development techniques aimed at bringing together the marketing, design, a}, undefined P Y - Treating and {undefined  we test the small molecule flexible ligand docking program Glide on a set of 19 non-$\alpha$-helical peptides and systematically improve pose prediction accuracy by enhancing Glide sampling for flexible polypeptides. In addition, scoring of the poses}, undefined P Y - Predicting the binding mode of flexible polypeptides to proteins is an important task that falls outside the domain of applicability of most small molecule and protein−protein docking tools. Here and undefined Undefined and on an extensive literature review and consumer interviews {undefined  the authors define product design and its dimensions. Using data from three samples (6,418 U.S. consumers and 1,083 and 583 European consumers), the authors develop and validate a new scale to measure product design along the dimensions of aest}, undefined P Y - Product design is a source of competitive advantage for companies and is an important driver of company performance. Drawing and fundamental principles of chemical product design and associated systematic tools undefined  within a broad domain of chemical products including molecules, formulations and devices, are still under development. In this paper, we propose a simple and fundamental conceptual model that defines the chemical product design problem as the i, undefined P Y - The and undefined Undefined and fixed product platform upon which derivative products are created through substitution of add-on modules {undefined  the approach here permits the platform itself to be one of several possible options. We first develop function structures for each product. After comparing function structures for common and unique functions, rules are applied to determine poss}, undefined P Y - This paper presents an approach to architecting a product family that shares inter-changeable modules. Rather than a},
title = {{No Title}}
}
@techreport{Kluiters2004,
author = {Kluiters, S C A},
title = {{Status review on membrane systems for hydrogen separation. Intermediate report EU Project MIGRYED NNES-2001}},
year = {2004}
}
@misc{Hyundai2015,
author = {Hyundai},
title = {{Hyundai ix35 Hydrogen Fuel Cell Vehicle}},
url = {https://www.hyundai.co.uk/about-us/environment/hydrogen-fuel-cell},
year = {2015}
}
@article{Wang2014a,
abstract = {This work shows for the first time that sol-gel derived cobalt oxide silica membranes can be produced on large scale, tubular supports through successful implementation of the rapid thermal processing (RTP) techniques. The combination of fast sol-gel methods and RTP techniques reduced the overall membrane fabrication time from a minimum of 7 days to approximately 12h. The successful RTP technique was developed by changing the silica precursor from tetraethyl orthosilicate to ethyl silicate 40 and adjusting the nitric acid and water ratios in the sol-gel process. The optimized sol-gel conditions for the best membranes were found to deliver average He permeances of 3×10-7molm-2s-1Pa-1and He/N2permselectivities of 69 at 450°C. This work has enormous potential for transforming the traditional processing pathways associated with sol-gel derived membranes for both research and commercial separation applications. {\textcopyright} 2014 Elsevier B.V.},
author = {Wang, David K. and {Diniz da Costa}, Jo{\~{a}}o C. and Smart, Simon},
doi = {10.1016/j.memsci.2014.01.014},
file = {:Users/marc/Library/Application Support/Mendeley Desktop/Downloaded/Wang, Diniz da Costa, Smart - 2014 - Development of rapid thermal processing of tubular cobalt oxide silica membranes for gas separation.pdf:pdf},
issn = {03767388},
journal = {Journal of Membrane Science},
keywords = {Ethyl silica 40,Gas separation,Rapid thermal processing,Silica membranes,Sol-gel},
pages = {192--201},
publisher = {Elsevier},
title = {{Development of rapid thermal processing of tubular cobalt oxide silica membranes for gas separations}},
url = {http://dx.doi.org/10.1016/j.memsci.2014.01.014},
volume = {456},
year = {2014}
}
@article{Tao2015,
abstract = {Abstract This paper provides a comprehensive overview of developments and recent trends in H{\textless}inf{\textgreater}2{\textless}/inf{\textgreater} separation technology that uses dense proton-electron conducting ceramic materials and their associated membranes. Various proton-electron conducting materials and their associated membranes are summarized and classified into several important categories, such as Ni-composite proton-conducting materials, as well as tungstate-based, BaPrO{\textless}inf{\textgreater}3{\textless}/inf{\textgreater}-based, LaGaO{\textless}inf{\textgreater}3{\textless}/inf{\textgreater}-based, and niobate/tantalite composite metal oxide-based ceramic materials/membranes. Various membrane designs, including asymmetric ceramic membranes (supported and self-supported) and surface-modified membranes, are also reviewed. Several important properties of ceramic materials and membranes, such as proton and electron conductivity and performance (i.e., H{\textless}inf{\textgreater}2{\textless}/inf{\textgreater} transport flux and lifetime stability), are also discussed. To highlight the technical progress in this area, all possible ceramic materials and associated membranes are summarized, along with their properties and performance, to help readers quickly locate the information they are looking for. Based on this review, several challenges hindering the maturation of this technology are analyzed in depth, and possible research directions for overcoming these challenges are suggested.},
author = {Tao, Zetian and Yan, Litao and Qiao, Jinli and Wang, Baolin and Zhang, Lei and Zhang, Jiujun},
doi = {10.1016/j.pmatsci.2015.04.002},
file = {:Users/marc/Library/Application Support/Mendeley Desktop/Downloaded/Tao et al. - 2015 - A review of advanced proton-conducting materials for hydrogen separation.pdf:pdf;:Users/marc/Library/Application Support/Mendeley Desktop/Downloaded/Tao et al. - 2015 - A review of advanced proton-conducting materials for hydrogen separation(2).pdf:pdf},
isbn = {0079-6425},
issn = {00796425},
journal = {Progress in Materials Science},
keywords = {Ceramic materials and membranes,Hydrogen separation,Proton and electron conduction},
pages = {1--50},
publisher = {Elsevier Ltd},
title = {{A review of advanced proton-conducting materials for hydrogen separation}},
url = {http://dx.doi.org/10.1016/j.pmatsci.2015.04.002},
volume = {74},
year = {2015}
}
@article{Kikuchi1991,
author = {Kikuchi, E and Uemiya, S},
chapter = {261},
journal = {Gas Separation and Purification},
title = {{Preparation of supported thin palladium-silver alloy membranes and their characteristics for hydrogen separation}},
volume = {5},
year = {1991}
}
@article{Peng2009,
author = {Peng, Lixia and Rao, Yongchu and Luo, Lizhu and Chen, Chang'An},
doi = {10.1016/j.jallcom.2009.06.158},
file = {:Users/marc/Library/Application Support/Mendeley Desktop/Downloaded/Peng et al. - 2009 - The poisoning of Pd–Y alloy membranes by carbon monoxide.pdf:pdf},
isbn = {09258388},
journal = {Journal of Alloys and Compounds},
number = {1-2},
pages = {74--77},
title = {{The poisoning of Pd–Y alloy membranes by carbon monoxide}},
volume = {486},
year = {2009}
}
@article{Brown2015,
abstract = {A purity specification in an international standard (ISO 14687) for hydrogen for use in proton exchange membrane fuel cells is discussed. The specification, which is referenced in a recent EU directive, sets maximum limits for 14 impurities in hydrogen. We consider one entry in the specification, total halogenated compounds, and conclude that no currently available analytical method is able to measure total halogenated compounds in a robust, traceable and accurately quantifiable manner. Three suggestions for addressing this problem when the standard is revised are given, namely (1) replacing `total halogenated compounds' with a list of key halogenated compounds that can be measured individually, (2) identifying compounds whose measurement is more routine to act as `canary species' for hydrogen produced from different sources and (3) setting different specifications for hydrogen produced from different sources. Prior to the revision of ISO 14687, we propose that inductively coupled plasma mass spectroscopy provides the best currently available analytical solution to estimate the amount fraction of halogenated compounds in hydrogen.},
author = {Brown, Andrew S. and Murugan, Arul and Brown, Richard J.C.},
doi = {10.1007/s00769-015-1135-2},
file = {:Users/marc/Library/Application Support/Mendeley Desktop/Downloaded/Brown, Murugan, Brown - 2015 - Measurement of ‘total halogenated compounds' in hydrogen Is the ISO 14687 specification achievable.pdf:pdf},
issn = {09491775},
journal = {Accreditation and Quality Assurance},
keywords = {Analysis,Halogenated compounds,Hydrogen,Purity,Traceability},
number = {3},
pages = {223--227},
publisher = {Springer Berlin Heidelberg},
title = {{Measurement of ‘total halogenated compounds' in hydrogen: Is the ISO 14687 specification achievable?}},
volume = {20},
year = {2015}
}
@article{Yukawa2002,
abstract = {The alloying effects on the stability of vanadium hydrides have been investigated systematically for binary V–M and ternary V–Ti–M alloys, where M's are various alloying elements. The PCT curves for the $\beta$ phase (V2H or VH) are measured at low hydrogen pressures using an electrochemical method. On the other hand, for the $\gamma$ phase (VH2), the PCT curves are measured at high hydrogen pressures using an ordinary Sieverts-type apparatus. It is found that the phase stability of vanadium hydrides is affected strongly by the presence of a small amount of elements added to vanadium metal. The alloying effects on the hydride stability are different between the $\gamma$ phase and the $\beta$ phase. For example, the $\gamma$ phase becomes most unstable when the group 8 elements in the periodic table, Fe, Ru and Os, are added to vanadium metal. On the other hand, the stability of the $\beta$ phase changes monotonously following the order of elements in the periodic table. Also, the efficiency of alloying element in modifying the hydride stability is found to be different between the $\beta$1 phase (V2H) and the $\beta$2 phase (VH). In addition, it is shown that the logarithm of the plateau pressures for the $\beta$1 and the $\gamma$ phases change almost linearly with the amount of alloying element in the alloy. Using this linear relationship for the $\gamma$ phase, the second plateau pressures of ternary V–Ti–Ru and V–Ti–Cr alloys can be estimated quantitatively in the wide compositional range.},
author = {Yukawa, Hiroshi and Yamashita, Daisuke and Ito, Shigeyuki and Morinaga, Masahiko and Yamaguchi, Shu},
doi = {10.2320/matertrans.43.2757},
file = {:Users/marc/Library/Application Support/Mendeley Desktop/Downloaded/Yukawa et al. - 2002 - Alloying effects on the phase stability of hydrides formed in vanadium alloys.pdf:pdf},
issn = {1345-9678},
journal = {Materials Transactions},
keywords = {alloying effects,hydride stability,hydrogen storage alloys,vanadium alloys,vanadium hydride},
number = {11},
pages = {2757--2762},
title = {{Alloying effects on the phase stability of hydrides formed in vanadium alloys}},
url = {http://www.jim.or.jp/journal/e/pdf3/43/11/2757.pdf{\%}5Cnpapers2://publication/uuid/3C0A7ADC-9BA8-471E-9CF3-40C116284411{\%}5Cnhttp://joi.jlc.jst.go.jp/JST.JSTAGE/matertrans/43.2757?from=CrossRef},
volume = {43},
year = {2002}
}
@article{Garcia-Garcia2012,
author = {Garc{\'{i}}a-Garc{\'{i}}a, F R and Torrente-Murciano, L and Chadwick, D and Li, K},
doi = {10.1016/j.memsci.2012.02.031},
isbn = {03767388},
journal = {Journal of Membrane Science},
pages = {30--37},
title = {{Hollow fibre membrane reactors for high H2 yields in the WGS reaction}},
volume = {405-406},
year = {2012}
}
@article{Mintova2001a,
abstract = {LTA-type zeolite films on mass sensors (quartz crystal microbalances) were prepared using secondary growth on a precursor seed layer. Zeolite seeds with a mean size of 40 nm were obtained at room temperature and used to form precursor layers via several identical adsorption steps on silane-modified sensor surfaces. Oriented LTA-type films with thicknesses in the range of 65-980 nm were prepared after additional hydrothermal treatment of the precursor seed layers at elevated temperature (100 degreesC) for different crystallization times (from 3 to 24 h). The application of LTA-type zeolite films in humidity microsensors is presented. The importance of film thickness, type of zeolite structure, and the preparation parameters of the sensor layers with respect to the sensitivity of the humidity sensors at different water vapor concentrations are discussed. The zeolite films are hydrophilic and thermally stable and show reproducible response. On the basis of these results, it was concluded that the above LTA zeolite films can be used effectively as humidity sensor materials for water vapor sensing purposes. High sensitivity, good reversibility, and long life were demonstrated for thin LTA-type zeolite films at low water concentrations.},
author = {Mintova, S. and Mo, S. and Bein, T.},
doi = {10.1021/cm000671w},
file = {:Users/marc/Library/Application Support/Mendeley Desktop/Downloaded/Mintova, Mo, Bein - 2001 - Humidity sensing with ultrathin LTA-type molecular sieve films grown on piezoelectric devices.pdf:pdf},
isbn = {0897-4756},
issn = {08974756},
journal = {Chemistry of Materials},
number = {3},
pages = {901--905},
title = {{Humidity sensing with ultrathin LTA-type molecular sieve films grown on piezoelectric devices}},
volume = {13},
year = {2001}
}
@techreport{EdlundD.J.Henry1996,
author = {{Edlund, D. J., Henry}, H.},
pages = {Phase II Final Report, US DOE DE--FG03--91ER81228},
title = {{A membrane process for hot gas clean-up and decomposition of H2S to elemental sulfur}},
year = {1996}
}
@article{Zhang2017,
abstract = {A novel methane sensor based on platinum (Pt)-loaded cobalt oxide (Co3O4)/molybdenum disulfide (MoS2) nanocomposite was reported in this paper. The sensor was fabricated via layer-by-layer (LbL) self-assembly method for the first time, and was characterized by X-ray diffraction (XRD), scanning electron microscope (SEM), energy dispersive spectrometer (EDS), elemental mapping, transmission electron microscope (TEM) and X-ray photoelectron spectroscopy (XPS). The gas sensing properties of the as-prepared Pt-Co3O4/MoS2 composite toward methane gas was investigated under various operating temperature, and the optimal working temperature of 170 °C was determined. The Pt-Co3O4/MoS2 sensor exhibits superior gas sensing performance toward methane as compared to the Co3O4, Co3O4/MoS2 counterparts. The underlying gas sensing mechanism of the Pt-Co3O4/MoS2 sensor was systematically discussed, which demonstrates that the enhanced sensing performance of the sensor is attributed to the good synergistic effect of the ternary materials, including high availability of oxygen species, active catalytic effect, and special interactions at MoS2/Co3O4 heterojunction.},
author = {Zhang, Dongzhi and Chang, Hongyan and Sun, Yan'e and Jiang, Chuanxing and Yao, Yao and Zhang, Yong},
doi = {10.1016/j.snb.2017.06.063},
file = {:Users/marc/Library/Application Support/Mendeley Desktop/Downloaded/Zhang et al. - 2017 - Fabrication of platinum-loaded cobalt oxidemolybdenum disulfide nanocomposite toward methane gas sensing at low te.pdf:pdf},
issn = {09254005},
journal = {Sensors and Actuators, B: Chemical},
keywords = {Layer-by-layer self-assembly,Methane gas detection,Operating temperature,Pt-Co3O4/MoS2 nanocomposite},
pages = {624--632},
publisher = {Elsevier B.V.},
title = {{Fabrication of platinum-loaded cobalt oxide/molybdenum disulfide nanocomposite toward methane gas sensing at low temperature}},
url = {http://dx.doi.org/10.1016/j.snb.2017.06.063},
volume = {252},
year = {2017}
}
@article{Chang2013,
abstract = {SnO2-reduced graphene oxide (SnO2-rGO) composites were prepared via a hydro-thermal reaction of graphene oxide (GO) and SnCl2{\{}bullet operator{\}}2H2O in the mixed solvent of ethylene glycol and water. During the redox reaction, GO was reduced to rGO while Sn2+ was oxidized to SnO2, uniformly depositing on the surface of rGO sheets. The composites were characterized by X-ray diffraction (XRD), scanning electron microscopy (SEM), thermogravimetric analysis (TGA), infrared spectra analysis (IR) and transmission electron microscopy (TEM), respectively, and their gas sensing properties were further investigated. Compared with pure SnO2 nanoparticles, the as-prepared SnO2-rGO gas sensor showed much better gas sensing behavior in sensitivity and response-recovery time to ethanol and H2S at low concentrations. Overall, the highly sensitive, quick-responding and low cost SnO2-rGO gas sensor could be potentially applied in environmental monitoring area. {\textcopyright} 2013.},
author = {Chang, Yanhong and Yao, Yunfeng and Wang, Bin and Luo, Hui and Li, Tianyi and Zhi, Linjie},
doi = {10.1016/j.jmst.2012.11.007},
file = {:Users/marc/Library/Application Support/Mendeley Desktop/Downloaded/Chang et al. - 2013 - Reduced Graphene Oxide Mediated SnO2 Nanocrystals for Enhanced Gas-sensing Properties.pdf:pdf},
isbn = {1005-0302},
issn = {10050302},
journal = {Journal of Materials Science and Technology},
keywords = {Gas sensor,Response-recovery time,Sensitivity,SnO2-rGO},
number = {2},
pages = {157--160},
publisher = {Elsevier Ltd},
title = {{Reduced Graphene Oxide Mediated SnO2 Nanocrystals for Enhanced Gas-sensing Properties}},
url = {http://dx.doi.org/10.1016/j.jmst.2012.11.007},
volume = {29},
year = {2013}
}
@article{Wang2016c,
abstract = {Graphene-based gas/vapor sensors have attracted much attention in recent years due to their variety of structures, unique sensing performances, room-temperature working conditions, and tremendous application prospects, etc. Herein, we summarize recent advantages in graphene preparation, sensor construction, and sensing properties of various graphene-based gas/vapor sensors, such as NH3, NO2, H2, CO, SO2, H2S, as well as vapor of volatile organic compounds. The detection mechanisms pertaining to various gases are also discussed. In conclusion part, some existing problems which may hinder the sensor applications are presented. Several possible methods to solve these problems are proposed, for example, conceived solutions, hybrid nanostructures, multiple sensor arrays, and new recognition algorithm.},
author = {Wang, Tao and Huang, Da and Yang, Zhi and Xu, Shusheng and He, Guili and Li, Xiaolin and Hu, Nantao and Yin, Guilin and He, Dannong and Zhang, Liying},
doi = {10.1007/s40820-015-0073-1},
file = {:Users/marc/Library/Application Support/Mendeley Desktop/Downloaded/Wang et al. - 2016 - A Review on Graphene-Based GasVapor Sensors with Unique Properties and Potential Applications.pdf:pdf},
isbn = {2311-6706
2150-5551},
issn = {21505551},
journal = {Nano-Micro Letters},
keywords = {Chemiresistor,Detection mechanism,Gas/Vapor sensor,Graphene},
number = {2},
pages = {95--119},
publisher = {Springer Berlin Heidelberg},
title = {{A Review on Graphene-Based Gas/Vapor Sensors with Unique Properties and Potential Applications}},
volume = {8},
year = {2016}
}
@article{Zhou2013,
abstract = {Stable Cu(2)O nanocrystals of around 3 nm were uniformly and densely grown on functionalized graphene sheets (FGS), which act as molecular templates instead of surfactants for controlled nucleation; the distribution density of nanocrystals can be easily controlled by FGS with different C/O ratios. The nanocomposite displays improved stability of the crystalline phase in wet air, which is attributed to finite-size effects that the high-symmetry crystalline phase is to be more stable at smaller size. Meanwhile, we conjecture that the oxygen adsorbed on the interfacial surface prefers to extract electrons from FGS, thus the interfacial bonding also makes a contribution in alleviating the process of corrosion to some extent. More importantly, the Cu(2)O-FGS nanocomposite based sensor realizes room temperature sensing to H(2)S with fantastic sensitivity (11{\%}); even at the exposed concentration of 5 ppb, the relative resistance changes show good linearity with the logarithm of the concentration. The enhancement of sensitivity is attributed to the synergistic effect of Cu(2)O and FGS; on the one hand, surfactant-free capped Cu(2)O nanocrystals display higher surface activity to adsorb gas molecules, and on the other hand, FGS acting as conducting network presents greater electron transfer efficiency. These observations show that the Cu(2)O-FGS nanocomposite based sensors have potential applications for monitoring air pollution at room temperature with low cost and power consumption.},
author = {Zhou, Lisha and Shen, Fangping and Tian, Xike and Wang, Donghong and Zhang, Ting and Chen, Wei},
doi = {10.1039/c2nr33164k},
file = {:Users/marc/Library/Application Support/Mendeley Desktop/Downloaded/Zhou et al. - 2013 - Stable Cu2O nanocrystals grown on functionalized graphene sheets and room temperature H2S gas sensing with ultrahig.pdf:pdf},
isbn = {2040-3364},
issn = {2040-3364},
journal = {Nanoscale},
number = {4},
pages = {1564},
pmid = {23325161},
title = {{Stable Cu2O nanocrystals grown on functionalized graphene sheets and room temperature H2S gas sensing with ultrahigh sensitivity}},
url = {http://xlink.rsc.org/?DOI=c2nr33164k},
volume = {5},
year = {2013}
}
@article{Zhang2012,
abstract = {The effects of heat treatment in air on H 2 sorption over the free-standing 25-??m cold-worked Pd-Ag25wt{\%} and Pd-Au10wt{\%} membrane surfaces were investigated. The equilibrium sorption results demonstrated that H 2 solubility did not change for both Pd-Ag and Pd-Au membranes before and after air oxidation at 300??C. The sorption kinetics data revealed a much faster H 2 sorption rates for the as-received Pd-Au membrane than the as-received Pd-Ag alloy. The heat treatment in air accelerated the H 2 sorption kinetics for both membranes at small H 2 equilibrium pressure (EP) of 0.16kPa. However, the oxidation enhancement extent is greater for Pd-Ag and the disparity becomes more significant when EP increases. The sorption rate enhancement for Pd-Ag membrane existed over a wide range of EPs even though the enhancement effect decreased with increasing EPs. For Pd-Au membrane, the enhancement induced by air oxidation decreased sharply when EP increased because the sorption rate is already very fast for the as-received Pd-Au alloy and there is less latitude for further improvement than Pd-Ag membrane. It is found that the variation of H 2 sorption rates over membrane surfaces correlates to the change of H 2 permeability. After heat treatment in air, the Pd-Ag membrane experienced a systematic increase in H 2 permeability with higher permeability increase at lower pressures while there was basically no permeation increase due to air oxidation effect for the Pd-Au membrane within the pressure range of practical H 2 permeation tests, which are consistent with the previously reported sorption rates hypothesis and results. ?? 2012 Elsevier B.V.},
author = {Zhang, Ke and Gade, Sabina K. and Way, J. Douglas},
doi = {10.1016/j.memsci.2012.02.025},
file = {:Users/marc/Library/Application Support/Mendeley Desktop/Downloaded/Zhang, Gade, Way - 2012 - Effects of heat treatment in air on hydrogen sorption over Pd-Ag and Pd-Au membrane surfaces.pdf:pdf},
isbn = {0376-7388},
issn = {03767388},
journal = {Journal of Membrane Science},
keywords = {Air oxidation,Palladium membrane,Palladium-gold,Palladium-silver,Sorption rate},
pages = {78--83},
publisher = {Elsevier B.V.},
title = {{Effects of heat treatment in air on hydrogen sorption over Pd-Ag and Pd-Au membrane surfaces}},
url = {http://dx.doi.org/10.1016/j.memsci.2012.02.025},
volume = {403-404},
year = {2012}
}
@article{Chen1996,
author = {Chen, F L and Kinari, Y and Sakamoto, F and Nakayama, Y and Sakamoto, Y},
journal = {International Journal of Hydrogen Energy},
number = {7},
pages = {555--561},
title = {{Hydrogen permeation through palladium-based alloy membranes in mixtures of 10{\{}{\%}{\}} Methane and Ethylene in the Hydrogen}},
volume = {21},
year = {1996}
}
@article{Farrauto2003,
abstract = {The hydrogen economy is fast approaching as petroleum reserves are rapidly consumed. The fuel cell promises to deliver clean and efficient power by combining hydrogen and oxygen in a simple electrochemical device that directly converts chemical energy to electrical energy. Hydrogen, the most plentiful element available, can be extracted from water by electrolysis. One can imagine capturing energy from the sun and wind and/or from the depths of the earth to provide the necessary power for electrolysis. Alternative energy sources such as these are the promise for the future, but for now they are not feasible for power needs across the globe. A transitional solution is required to convert certain hydrocarbon fuels to hydrogen. These fuels must be available through existing infrastructures such as the natural gas pipeline. The present review discusses the catalyst and adsorbent technologies under development for the extraction of hydrogen from natural gas to meet the requirements for the proton exchange membrane (PEM) fuel cell. The primary market is for residential applications, where pipeline natural gas will be the source of H2 used to power the home. Other applications including the reforming of methanol for portable power applications such as laptop computers, cellular phones, and personnel digital equipment are also discussed. Processing natural gas containing sulfur requires many materials, for example, adsorbents for desulfurization, and heterogeneous catalysts for reforming (either autothermal or steam reforming) water gas shift, preferential oxidation of CO, and anode tail gas combustion. All these technologies are discussed for natural gas and to a limited extent for reforming methanol.},
author = {Farrauto, R. and Hwang, S. and Shore, L. and Ruettinger, W. and Lampert, J. and Giroux, T. and Liu, Y. and Ilinich, O.},
doi = {10.1146/annurev.matsci.33.022802.091348},
isbn = {1531-7331},
issn = {1531-7331},
journal = {Annual Review of Materials Research},
keywords = {adsorbents,catalysts,h 2 generation,natural gas,pem fuel cell},
number = {1},
pages = {1--27},
title = {{New Material Needs for Hydrocarbon Fuel Processing: Generating Hydrogen for the PEM Fuel Cell}},
url = {http://www.annualreviews.org/doi/10.1146/annurev.matsci.33.022802.091348},
volume = {33},
year = {2003}
}
@article{Chen2015,
author = {Chen, Yan and Liao, Qing and Li, Zhong and Wang, Haihui and Wei, Yanying and Feldhoff, Armin and Caro, J{\"{u}}rgen},
doi = {10.1002/aic.14772},
isbn = {00011541},
journal = {AIChE Journal},
number = {6},
pages = {1997--2007},
title = {{A CO2-stable hollow-fiber membrane with high hydrogen permeation flux}},
volume = {61},
year = {2015}
}
@article{Nagarkar2015,
abstract = {The toxic gas H{\textless}inf{\textgreater}2{\textless}/inf{\textgreater}S has recently emerged as one of the important signaling molecules in biological systems. Thus understanding the production, distribution, and mode of action of H{\textless}inf{\textgreater}2{\textless}/inf{\textgreater}S in biological system is important, but the fleeting and reactive nature of H{\textless}inf{\textgreater}2{\textless}/inf{\textgreater}S makes it a daunting task. Herein we report a biocompatible, nitro-functionalized metal-organic framework as reaction-based fluorescence turn-on probe for fast and selective H{\textless}inf{\textgreater}2{\textless}/inf{\textgreater}S detection. The selective turn-on performance of MOF remains unaffected even in presence of competing biomolecules. Highly selective H{\textless}inf{\textgreater}2{\textless}/inf{\textgreater}S detection is achieved by using a nitro-functionalized metal-organic framework as a reaction-based fluorescence turn-on probe. The nitro groups selectively undergo reduction by H{\textless}inf{\textgreater}2{\textless}/inf{\textgreater}S to amine moieties, giving a fluorescence turn-on response under physiological conditions. The fast and sensitive detection performance is unaffected even in the presence of competing biomolecules. {\textcopyright} 2015 WILEY-VCH Verlag GmbH {\&} Co. KGaA, Weinheim.},
author = {Nagarkar, Sanjog S. and Desai, Aamod V. and Ghosh, Sujit K.},
doi = {10.1002/chem.201501043},
file = {:Users/marc/Library/Application Support/Mendeley Desktop/Downloaded/Nagarkar, Desai, Ghosh - 2015 - A Nitro-Functionalized Metal-Organic Framework as a Reaction-Based Fluorescence Turn-On Probe for Rapid.pdf:pdf},
issn = {15213765},
journal = {Chemistry - A European Journal},
keywords = {fluorescent probes,hydrogen sulfide,metal-organic frameworks,molecular recognition,sensors},
number = {28},
pages = {9994--9997},
title = {{A Nitro-Functionalized Metal-Organic Framework as a Reaction-Based Fluorescence Turn-On Probe for Rapid and Selective H2S Detection}},
volume = {21},
year = {2015}
}
@article{Yoshimune2005,
author = {Yoshimune, M and Fujiwara, I and Suda, H and Haraya, K},
doi = {10.1246/cl.2005.958},
file = {:Users/marc/Library/Application Support/Mendeley Desktop/Downloaded/Yoshimune et al. - 2005 - Novel carbon molecular sieve membranes derived from poly (phenylene oxide) and its derivatives for gas separat.pdf:pdf},
issn = {0366-7022},
journal = {Chemistry Letters},
number = {7},
pages = {958--959},
title = {{Novel carbon molecular sieve membranes derived from poly (phenylene oxide) and its derivatives for gas separation}},
volume = {34},
year = {2005}
}
@article{DosSantos1998,
abstract = {Samples of amorphous Fe 40 Ni 38 Mo 4 B 18 and Ni 81 P 19 alloys were submitted to electrochemical hydrogen permeation tests, under cathodic charging conditions, which resulted in elevated hydrogen fugacity. Two different permeation techniques were used: the double-potentiostatic and the potentiostatic step methods. The main difference between these techniques is the form of the resulting hydrogen concentration profile across the sample thickness. The hydrogen permeation curves for the Ni 81 P 19 samples, for both techniques, had shapes identical to the theoretical curves. The experimental curves for the Fe 40 Ni 38 Mo 4 B 18 samples, however, deviated from the theoretical curves. These deviations are attributed to the formation of a hydride phase due to the high hydrogen fugacity imposed during cathodic charging. The potential use of these electrochemical hydrogen permeation techniques is demonstrated in the study of phase separation through the analysis of the growth of a new phase. The diffusivity of hydrogen in the hydride formed in the Fe 40 Ni 38 Mo 4 B 18 amorphous metallic alloy was found to be 5.7 ± 0.3 × 10 -15 m 2 s -1 and the thickness of the hydride layer, formed during the double-potentiostatic test, for a cathodic charging level of -2000 mV/SCE saturated calomel electrode (SCE) was found to be 18.9 ± 0.5 $\mu$m. {\textcopyright} 1998 Published by Elsevier Science B.V. All rights reserved.},
author = {dos Santos, D.S and de Miranda, P.E.V},
doi = {10.1016/S0022-3093(98)00487-6},
file = {:Users/marc/Library/Application Support/Mendeley Desktop/Downloaded/dos Santos, de Miranda - 1998 - The use of electrochemical hydrogen permeation techniques to detect hydride phase separation in amorphou.pdf:pdf},
issn = {00223093},
journal = {Journal of Non-Crystalline Solids},
pages = {133--139},
title = {{The use of electrochemical hydrogen permeation techniques to detect hydride phase separation in amorphous metallic alloys}},
url = {http://linkinghub.elsevier.com/retrieve/pii/S0022309398004876},
volume = {232-234},
year = {1998}
}
@article{AhmadiFeijani2015,
abstract = {In this study, poly(vinylidene fluoride) (PVDF) was modified by a mixture of KOH and KMNO4 in order to effect HF elimination. During this reaction, some functional groups were created in the PVDF structure. The metal–organic frameworks, MIL-53(Al) and NH2-MIL-35(Al), were embedded in modified PVDF (M-PVDF) to fabricate mixed matrix membranes (MMMs). Different characterization techniques such as FT-IR, XRD, SEM, DSC, TGA, and contact angle tests have been implemented to identify the prepared membranes. Pure and mixed (1:1) CO2 and CH4 were used to test gas separation performance of MMMs. CO2 permeance was increased for pure M-PVDF membrane compared with the pristine PVDF by 31.2{\%}. Upon 10 wt {\%} loading of MIL-53(Al) and NH2-MIL-35(Al) in M-PVDF, CO2 permeability increased 104.33{\%} and 80.02{\%} relative to the unfilled M-PVDF. The highest CO2/CH4 ideal selectivity and separation factor of 43.9 and 42.65, respectively, were reported for NH2-MIL-35(Al)/M-PVDF with 10 wt {\%} loading. These improvements could be attr...},
author = {{Ahmadi Feijani}, Elahe and Tavasoli, Ahmad and Mahdavi, Hossein},
doi = {10.1021/acs.iecr.5b02549},
file = {:Users/marc/Library/Application Support/Mendeley Desktop/Downloaded/Ahmadi Feijani, Tavasoli, Mahdavi - 2015 - Improving Gas Separation Performance of Poly(vinylidene fluoride) Based Mixed Matrix Membra.pdf:pdf},
isbn = {0888-5885$\backslash$r1520-5045},
issn = {0888-5885},
journal = {Industrial {\&} Engineering Chemistry Research},
number = {48},
pages = {12124--12134},
title = {{Improving Gas Separation Performance of Poly(vinylidene fluoride) Based Mixed Matrix Membranes Containing Metal–Organic Frameworks by Chemical Modification}},
url = {http://pubs.acs.org/doi/10.1021/acs.iecr.5b02549},
volume = {54},
year = {2015}
}
@article{Li1999,
author = {Li, Anwu and Liang, Weiqiang and Hughes, Ronald},
file = {:Users/marc/Library/Application Support/Mendeley Desktop/Downloaded/Li, Liang, Hughes - 1999 - Fabrication of defect-free PdAl2O3 composite membranes for hydrogen separation.pdf:pdf},
journal = {Thin Solid Films},
pages = {106--112},
title = {{Fabrication of defect-free Pd/Al2O3 composite membranes for hydrogen separation}},
volume = {350},
year = {1999}
}
@article{Song2013,
abstract = {Co-doped BaCe0.85Tb0.05Co0.1O 3-?? (BCTCo) nanopowder was synthesized via a sol-gel method using ethylenediaminetetraacetic acid (EDTA) and citric acid as the chelating agents. Using the resultant powder, BCTCo perovskite hollow fibre membranes were then fabricated by the combined phase inversion and sintering technique. Properties of the BCTCo powder and the hollow fibre membranes in terms of crystalline phase, morphology, electrical conductivity, porosity, mechanical strength and hydrogen/oxygen permeation were investigated by a variety of characterization methods. The results indicated that doping of cobalt in the BCTb oxide led to a higher electrical conductivity and lower calcination temperature for the powder precursor to a perovskite structure as well as sintering temperature for the hollow fibre precursors to gastight membranes. In order to obtain gastight and robust hollow fibre membranes, the sintering temperature should be controlled between 1300 and 1450 C. The maximum hydrogen flux through the BCTCo hollow fibre membranes reached up to 0.385 mL cm -2 min-1 at 1000 C under 50{\%} H2-He/N 2 gradient, which is higher than that of the un-doped BCTb hollow fibre membranes with the same effective thickness, and especially much higher than that obtained from other proton conductors due to the asymmetric structure of the membrane designed. Moreover, the BCTCo hollow fibre membrane also exhibited noticeable oxygen permeation fluxes, i.e. 0.122 mL cm-2 min-1 at 1000 C under the air/He gradient. However, doping of cobalt might damage the mechanical stability of the perovskite membranes in the hydrogen-containing atmosphere. ?? 2013, Hydrogen Energy Publications, LLC. Published by Elsevier Ltd. All rights.},
author = {Song, Jian and Li, Liping and Tan, Xiaoyao and Li, K.},
doi = {10.1016/j.ijhydene.2013.04.104},
file = {:Users/marc/Library/Application Support/Mendeley Desktop/Downloaded/Song et al. - 2013 - BaCe0.85Tb0.05Co0.1O3- perovskite hollow fibre membranes for hydrogenoxygen permeation.pdf:pdf},
isbn = {0360-3199},
issn = {03603199},
journal = {International Journal of Hydrogen Energy},
keywords = {Hollow fibre membrane,Hydrogen permeation,Oxygen permeation,Proton conductor},
number = {19},
pages = {7904--7912},
title = {{BaCe0.85Tb0.05Co0.1O3-?? perovskite hollow fibre membranes for hydrogen/oxygen permeation}},
volume = {38},
year = {2013}
}
@article{Xu2016b,
abstract = {In the present work, we report a new road to enhance the gas separation performances of the zeolite Na-LTA membrane by tuning the pore size of zeolite LTA through silver cation exchange. Through the functionalization of the alumina support by using 3-aminopropyltriethoxysilane (APTES), a thin, phase-pure and well intergrown zeolite Na-LTA membrane with a thickness of about 5.0 $\mu$m can be facilely prepared on the APTES-modified macroporous $\alpha$-Al2O3tube. After a following silver exchange treatment of the as-synthesized zeolite Na-LTA membranes, the sodium ions in zeolite Na-LTA framework are replaced by silver ions, thus forming zeolite Ag-LTA membranes with a narrower pore diameter. The zeolite Ag-LTA membranes were characterized by scanning electron microscopy (SEM), X-ray diffraction (XRD), and X-ray photoelectron spectroscopy (XPS). It is found that both the morphology and structure of the zeolite Na-LTA membrane keep unchanged after silver-exchange, and no cracks, pinholes or other defects are observed in the membrane layer. The zeolite Ag-LTA membrane shows high hydrogen selectivity due to the reduction of pore size of the zeolite Ag-LTA. For the separation of the binary mixture at 50 °C and 2.0 bar, the separation factor of H2/C3H8is $\alpha$≈120.8, which by far exceeds $\alpha$≈19.4 of the starting Na-LTA membrane, and are also much higher than the separation factors previously reported for the H2/C3H8mixture on zeolite membranes.},
author = {Xu, Kai and Yuan, Chenfang and Caro, J{\"{u}}rgen and Huang, Aisheng},
doi = {10.1016/j.memsci.2016.03.036},
file = {:Users/marc/Library/Application Support/Mendeley Desktop/Downloaded/Xu et al. - 2016 - Silver-exchanged zeolite LTA molecular sieving membranes with enhanced hydrogen selectivity.pdf:pdf},
issn = {18733123},
journal = {Journal of Membrane Science},
keywords = {Gas separation,Ion exchange,Molecular sieve membrane,Zeolite Ag-LTA membrane,Zeolite LTA membrane},
pages = {1--8},
publisher = {Elsevier},
title = {{Silver-exchanged zeolite LTA molecular sieving membranes with enhanced hydrogen selectivity}},
url = {http://dx.doi.org/10.1016/j.memsci.2016.03.036},
volume = {511},
year = {2016}
}
@misc{Cited2012,
archivePrefix = {arXiv},
arxivId = {arXiv:1208.5721},
author = {Cited, References and City, Oklahoma and Data, Related U-s- Application},
doi = {10.1016/j.(73)},
eprint = {arXiv:1208.5721},
file = {:Users/marc/Library/Application Support/Mendeley Desktop/Downloaded/Cited, City, Data - 2012 - Process for preparing palladium alloy composite membranes for use in hydrogen separation, palladium alloy com.pdf:pdf},
isbn = {2004001828},
number = {12},
pages = {0--4},
pmid = {1000182772},
title = {{Process for preparing palladium alloy composite membranes for use in hydrogen separation, palladium alloy composite membranes and products incorporating or made from membranes}},
volume = {1},
year = {2012}
}
@article{Wang2016b,
author = {Wang, Ming and Huang, Ming-Ling and Cao, Yue and Ma, Xiao-Hua and Xu, Zhen-Liang},
doi = {10.1016/j.memsci.2016.05.038},
isbn = {03767388},
journal = {Journal of Membrane Science},
pages = {144--153},
title = {{Fabrication, characterization and separation properties of three-channel stainless steel hollow fiber membrane}},
volume = {515},
year = {2016}
}
@article{Venkatasubramanian2012,
abstract = {We report an experimental investigation of the adsorption properties of two important small-pore metal-organic framework (MOF) materials recently identified for gas separation applications, through the development and use of a high-pressure/high-temperature quartz crystal microbalance (QCM) device. In particular, we characterize in detail the CO2, CH4, and N-2 adsorption characteristics of the MOFs Cu(4,4'-(hexafluoroisopropylidene)bisbenzoate)(1.5) (referred to as Cu-hfipbb) and zeolitic imidazolate framework-90 (ZIF-90). We first describe the construction of a QCM-based adsorption measurement apparatus. Single-component adsorption isotherms of CO2, CH4, and N-2 in the two MOFs were then measured at temperatures ranging from 30 to 70 C and pressures ranging from 0.3 to 110 psi. In both materials, the order of adsorption strength is CO2 {\textgreater} CH4 {\textgreater} N-2. We find that adsorption in the 1-D channels of Cu-hfipbb can be well described by a single-site Langmuir model. On the other hand, adsorption in ZIF-90 follows a more complex behavior, commensurate with its pore structure consisting of large porous cages connected in three dimensions by small windows. The nongravimetric QCM-based measurement techniques are shown to be a valuable microanalytical tool for the study of molecular adsorption in MOFs.},
author = {Venkatasubramanian, Anandram and Navaei, Milad and Bagnall, Kevin R. and McCarley, Ken C. and Nair, Sankar and Hesketh, Peter J.},
doi = {10.1021/jp304631m},
file = {:Users/marc/Library/Application Support/Mendeley Desktop/Downloaded/Venkatasubramanian et al. - 2012 - Gas adsorption characteristics of metal-organic frameworks via quartz crystal microbalance techniques.pdf:pdf},
isbn = {1932-7447},
issn = {19327447},
journal = {Journal of Physical Chemistry C},
number = {29},
pages = {15313--15321},
title = {{Gas adsorption characteristics of metal-organic frameworks via quartz crystal microbalance techniques}},
volume = {116},
year = {2012}
}
@article{Basile2008a,
author = {Basile, A and Gallucci, F and Iulianelli, A and Tereschenko, G F and Ermilova, M M and Orekhova, N V},
doi = {10.1016/j.memsci.2007.10.028},
isbn = {03767388},
journal = {Journal of Membrane Science},
number = {1-2},
pages = {44--50},
title = {{Ti–Ni–Pd dense membranes—The effect of the gas mixtures on the hydrogen permeation}},
volume = {310},
year = {2008}
}
@article{Yukawa2011,
abstract = {The alloying effects of tungsten on the hydrogen solubility, the resistance to hydrogen embrittlement and the hydrogen permeability are investigated for Ta-based hydrogen permeable membranes. The hydrogen solubility is found to decrease by the addition of tungsten into tantalum or by increasing the temperature. It is also found that the mechanical properties (i.e., strength and ductility) for Ta-based alloy is better than that for Nb-based alloy in hydrogen atmosphere at high temperature. It is demonstrated that the Ta-5 mol{\%}W alloy possesses excellent hydrogen permeability without showing any hydrogen embrittlement when used under appropriate permeation conditions. For example, the hydrogen flux for Ta-5 mol{\%}W alloy measured at 773 K under the pressure condition of inlet/outlet = 0.15/0.01 MPa is about 5 times higher than that for Pd-27 mol{\%}Ag alloy measured at the same testing condition. {\textcopyright}2011 The Japan Institute of Metals.},
author = {Yukawa, H. and Nambu, T. and Matsumoto, Y.},
doi = {10.2320/matertrans.MA201007},
file = {:Users/marc/Library/Application Support/Mendeley Desktop/Downloaded/Yukawa, Nambu, Matsumoto - 2011 - Ta-W Alloy for Hydrogen Permeable Membranes.pdf:pdf},
issn = {1347-5320},
journal = {Materials Transactions},
keywords = {brittle transition hydrogen concentration,dbtc,ductile-to-,hydrogen permeability,hydrogen permeable membrane,hydrogen solubility,resistance to hydrogen embrittlement,tantalum-based alloy},
number = {4},
pages = {610--613},
title = {{Ta-W Alloy for Hydrogen Permeable Membranes}},
url = {http://joi.jlc.jst.go.jp/JST.JSTAGE/matertrans/MA201007?from=CrossRef},
volume = {52},
year = {2011}
}
@article{Dolan2009,
author = {Dolan, Michael and Dave, Narendra and Morpeth, Leigh and Donelson, Richard and Liang, Daniel and Kellam, Michael and Song, Song},
doi = {10.1016/j.memsci.2008.10.030},
file = {:Users/marc/Library/Application Support/Mendeley Desktop/Downloaded/Dolan et al. - 2009 - Ni-based amorphous alloy membranes for hydrogen separation at 400°C.pdf:pdf},
isbn = {03767388},
journal = {Journal of Membrane Science},
number = {2},
pages = {549--555},
title = {{Ni-based amorphous alloy membranes for hydrogen separation at 400°C}},
volume = {326},
year = {2009}
}
@misc{Linde,
author = {Linde},
title = {{Membrane plants}},
url = {https://www.linde-engineering.com/en/process{\_}plants/adsorption-and-membrane-plants/membrane-plants/index.html}
}
@article{Bandosz2011,
abstract = {Two types of metal-organic framework (MOF)/ graphite oxide hybrid materials were prepared. One is based on a zinc-containing, MOF-5 and the other on a copper- containing HKUST-1. The materials are characterized by X-ray diffraction, sorption of nitrogen, thermal analyses, Fourier Transform infrared spectroscopy (FT-IR) and scan- ning electron microscopy (SEM). Their features are com- pared to the ones of the parent materials. The water stabil- ity and ammonia adsorption capacity of the hybrid materi- als were also evaluated. It was found that the latter com- pounds exhibit features similar to the ones of the parent MOF. In most cases, their porosity increased compared to the one calculated considering the physical mixture ofMOF and GO. This new porosity likely located between the two components of the hybrid materials is responsible for the enhanced ammonia adsorption capacity of the compounds. However, for both the zinc-based and the copper-based ma- terials (MOFs and hybrid materials), a collapse of the frame- work was observed as a result of ammonia adsorption. This collapse is caused by the interactions of ammonia with the metallic centers ofMOFs either by hydrogen bonding (zinc- based materials) or coordination and subsequent complex- ation (copper-based materials). Whereas the MOF-5 based compounds collapse in presence of humidity, the copper- based materials are stable.},
author = {Bandosz, Teresa J. and Petit, Camille},
doi = {10.1007/s10450-010-9267-5},
file = {:Users/marc/Library/Application Support/Mendeley Desktop/Downloaded/Bandosz, Petit - 2011 - MOFgraphite oxide hybrid materials Exploring the new concept of adsorbents and catalysts.pdf:pdf},
isbn = {0929-5607},
issn = {09295607},
journal = {Adsorption},
keywords = {Adsorption,Ammonia,Graphite oxide,Hybrid material,Metal-organic frameworks},
number = {1},
pages = {5--16},
title = {{MOF/graphite oxide hybrid materials: Exploring the new concept of adsorbents and catalysts}},
volume = {17},
year = {2011}
}
@article{Kingsbury2010,
abstract = {Asymmetric ceramic hollow fibre membranes and membrane supports have been prepared using a combined phase inversion and sintering technique for use in a multifunctional catalytic membrane reactor. The asymmetric structure is such that the fibre may simultaneously function either as a porous membrane and a matrix for catalyst deposition, or as a porous support for the coating of a gas separation layer and a matrix for catalyst deposition. The effectiveness of catalyst deposition depends strongly on the pore size distribution of the membrane or membrane support, which is bimodal in nature and is affected by the calcination temperature and the fibre preparation parameters. The effect of the calcination temperature and preparation parameters on the pore size distribution and fibre morphology have been studied systematically with regard to catalyst deposition and fibre mechanical strength and a route to optimizing the fibre structure has been suggested. ?? 2010 Elsevier B.V. All rights reserved.},
author = {Kingsbury, Benjamin F K and Wu, Zhentao and Li, K.},
doi = {10.1016/j.cattod.2010.02.039},
file = {:Users/marc/Library/Application Support/Mendeley Desktop/Downloaded/Kingsbury, Wu, Li - 2010 - A morphological study of ceramic hollow fibre membranes A perspective on multifunctional catalytic membrane r.pdf:pdf},
isbn = {0376-7388},
issn = {09205861},
journal = {Catalysis Today},
keywords = {Asymmetric hollow fibre,Catalyst deposition,Pore size distribution},
number = {3-4},
pages = {306--315},
title = {{A morphological study of ceramic hollow fibre membranes: A perspective on multifunctional catalytic membrane reactors}},
volume = {156},
year = {2010}
}
@article{Polfus2015,
abstract = {Some compositions of ceramic hydrogen permeable membranes are promising for integration in high temperature processes such as steam methane reforming due to their high chemical stability in large chemical gradients and CO2 containing atmospheres. In the present work, we investigate the hydrogen permeability of densely sintered ceramic composites (cercer) of two mixed ionic-electronic conductors: La27W3.5Mo1.5O55.5-delta (LWM) containing 30, 40 and 50wtpercent La0.87Sr0.13CrO3-delta (LSC). Hydrogen permeation was characterized as a function of temperature, feed side hydrogen partial pressure (0.1-0.9bar) with wet and dry sweep gas. In order to assess potentially limiting surface kinetics, measurements were also carried out after applying a catalytic Pt-coating to the feed and sweep side surfaces. The apparent hydrogen permeability, with contribution from both H2 permeation and water splitting on the sweep side, was highest for LWM70-LSC30 with both wet and dry sweep gas. The Pt-coating further enhances the apparent H2 permeability, particularly at lower temperatures. The apparent H2 permeability at 700?C in wet 50percent H2 was 1.1?10-3mLmin-1cm-1 with wet sweep gas, which is higher than for the pure LWM material. The present work demonstrates that designing dual-phase ceramic composites of mixed ionic-electronic conductors is a promising strategy for enhancing the ambipolar conductivity and gas permeability of dense ceramic membranes. ?H2 flux of ceramic composites of mixed ionic-electronic conductors measured.?La27W5O55.5-LaCrO3 composites are chemically compatible and stable.?H2 permeability found to be among state-of-the-art dense ceramic membranes.?H2 permeability significantly higher than the individual composite components.},
author = {Polfus, Jonathan M and Xing, Wen and Fontaine, Marie-Laure and Denonville, Christelle and Henriksen, Partow P and Bredesen, Rune},
doi = {10.1016/j.memsci.2015.01.027},
file = {:Users/marc/Library/Application Support/Mendeley Desktop/Downloaded/Polfus et al. - 2015 - Hydrogen separation membranes based on dense ceramic composites in the La27W5O55.5-LaCrO3 system.pdf:pdf},
issn = {03767388},
journal = {Journal of Membrane Science},
keywords = {Ceramic–ceramic composite,Dense ceramic membrane,Hydrogen separation,Lanthanum chromite,Lanthanum tungstate},
pages = {39--45},
publisher = {Elsevier},
title = {{Hydrogen separation membranes based on dense ceramic composites in the La27W5O55.5-LaCrO3 system}},
url = {http://dx.doi.org/10.1016/j.memsci.2015.01.027},
volume = {479},
year = {2015}
}
@article{Vilaseca2008,
abstract = {Zeolite A (LTA)-coated micromachined sensors have been prepared and used in the sensing of individual gases (H2, CH4, C2H5OH, C3H8 and CO, in the 10-1000 ppm range) and gas mixtures. Unlike previous works with conventional sensors, a hydrothermal synthesis was not used to prepare a zeolite film. Instead, a zeolite coating was formed on top of the Pd/SnO2 surface by microdropping from a zeolite suspension. In spite of this, the response of the sensor with zeolite is significantly different from that of unmodified sensors, and essentially reproduces the performance of zeolite-coated conventional sensors. By avoiding the use of a hydrothermal synthesis the integrity of the sensor is better preserved, and the resulting non-continuous zeolite film has the added advantage of a strong reduction in response times. ?? 2008 Elsevier B.V. All rights reserved.},
author = {Vilaseca, M. and Coronas, J. and Cirera, A. and Cornet, A. and Morante, J. R. and Santamaria, J.},
doi = {10.1016/j.snb.2008.03.002},
file = {:Users/marc/Library/Application Support/Mendeley Desktop/Downloaded/Vilaseca et al. - 2008 - Development and application of micromachined PdSnO2 gas sensors with zeolite coatings.pdf:pdf},
isbn = {0925-4005},
issn = {09254005},
journal = {Sensors and Actuators, B: Chemical},
keywords = {Gas sensor selectivity,Microsensors,SnO2,Zeolite coatings},
number = {2},
pages = {435--441},
title = {{Development and application of micromachined Pd/SnO2 gas sensors with zeolite coatings}},
volume = {133},
year = {2008}
}
@article{Wilson2012,
abstract = {In order to maximize peak capacity and detection sensitivity of fast gas chromatography (GC) separations, it is necessary to minimize band broadening, and in particular due to injection since this is often a major contributor. A high-speed cryo-focusing injection (HSCFI) system was constructed to first cryogenically focus analyte compounds in a 6 cm long section of metal MXT column, and second, reinject the focused analytes by rapidly resistively heating the metal column via an in-house built electronic circuit. Since the cryogenically cooled section of column is small (???750 nl) and the direct resistive heating is fast (???6000 ??C/s), HSCFI is demonstrated to produce an analyte peak with a 6.3 ms width at half height, w 1/2. This was achieved using a 1 m long column with a 180 ??m inner diameter (i.d.) operated at an absolute head pressure of 55 psi and an oven temperature of 60 ??C, with a 10 V pulse applied to the metal column for 50 ms. HSCFI was also used to demonstrate the head space sampling and fast GC analysis of an aqueous solution containing six test analytes (acetone, methanol, ethanol, toluene, chlorobenzene, pentanol). Using Henry's law constants for each of the analytes, injected mass limits of detection (LODs) were typically in the low pg levels (e.g., 1.2 pg for acetone) for the high speed separation. Finally, to demonstrate the use of HSCFI with a complex sample, a gasoline was separated using a 20 m??100 ??m i.d. column and the stock GC oven for temperature programming, which provided a separation time of 200 s and an average peak width at the base of 440 ms resulting in a total peak capacity of 460 peaks (at unit resolution). ?? 2012 Elsevier B.V. All rights reserved.},
author = {Wilson, Ryan B. and Fitz, Brian D. and Mannion, Brandyn C. and Lai, Tina and Olund, Roy K. and Hoggard, Jamin C. and Synovec, Robert E.},
doi = {10.1016/j.talanta.2012.03.054},
file = {:Users/marc/Library/Application Support/Mendeley Desktop/Downloaded/Wilson et al. - 2012 - High-speed cryo-focusing injection for gas chromatography Reduction of injection band broadening with concentrati.pdf:pdf},
isbn = {0039-9140},
issn = {00399140},
journal = {Talanta},
keywords = {Cyrogenic trapping,Gas chromatography,High speed,Thermal injection},
pages = {9--15},
pmid = {22841041},
publisher = {Elsevier},
title = {{High-speed cryo-focusing injection for gas chromatography: Reduction of injection band broadening with concentration enrichment}},
url = {http://dx.doi.org/10.1016/j.talanta.2012.03.054},
volume = {97},
year = {2012}
}
@article{Wei2014,
abstract = {Perfluorodecyl-modified silica membranes were prepared by the sol-gel technique using tetraethyl orthosilicate (TEOS) and 1H,1H,2H,2H-Perflouorodecyltriethoxysilane (PFDTES) as precursors under acidic and clean room conditions. The wettability and pore structure of the modified silica membranes were characterized by means of water contact angle measurement, Fourier translation infrared spectroscopy (FT-IR), solid state29Si magic angle spinning nuclear magnetic resonance (29Si MAS NMR) and nitrogen adsorption. The membrane performances including gas permeation/separation and long-term hydrothermal stability were also investigated in detail. The results show that perfluorodecyl groups have been successfully incorporated, resulting in a transformation from hydrophility to hydrophobicity for the modified silica membranes. With a low PFDTES/TEOS molar ratio of 0.2, the modified silica membranes exhibit a water contact angle of 112.6±0.5° and a pore size ranging from 0.45 to 0.9nm. The hydrogen permeance increases with increasing temperatures, leading to a considerably high value of 9.71×10-7molm-2s-1Pa-1at 300°C. At such a temperature, the H2/CO2permselectivity and binary gas separation factor are 7.19 and 12.11, respectively. Under a humid condition with a temperature of 250°C and a water vapor molar ratio of 5{\%}, the single H2permeance and H2/CO2permselectivity remain almost constant for at least 200h, indicating that the modified membranes possess an outstanding hydrothermal stability. {\textcopyright} 2014 Elsevier B.V.},
author = {Wei, Qi and Ding, Yuan Li and Nie, Zuo Ren and Liu, Xiang Ge and Li, Qun Yan},
doi = {10.1016/j.memsci.2014.04.036},
file = {:Users/marc/Library/Application Support/Mendeley Desktop/Downloaded/Wei et al. - 2014 - Wettability, pore structure and performance of perfluorodecyl-modified silica membranes.pdf:pdf},
issn = {18733123},
journal = {Journal of Membrane Science},
keywords = {Hydrothermal stability,Perfluorodecyl groups,Pore structure,Silica membrane,Wettability},
pages = {114--122},
title = {{Wettability, pore structure and performance of perfluorodecyl-modified silica membranes}},
volume = {466},
year = {2014}
}
@incollection{March1968,
author = {March, Jerry},
booktitle = {Advance Organic Chemistry: Reactions, Mechanisms and Structure},
pages = {533--534},
title = {{Advance Organic Chemistry: Reactions, Mechanisms and Structure}},
year = {1968}
}
@article{Lu1992,
abstract = {Some properties of a nanocrystalline Ni-P alloy and an amorphous alloy of the same chemical composition, including the thermal expansion coefficient, specific heat capacity, electrical resistance and thermal stability, are experimentally measured and compared. The results show that the properties of the nanocrystalline alloy are very different from those of the amorphous alloy.},
author = {Lu, K. and Wang, J. T. and Wei, W. D.},
doi = {10.1088/0022-3727/25/5/010},
file = {:Users/marc/Library/Application Support/Mendeley Desktop/Downloaded/Lu, Wang, Wei - 1992 - Comparison of Properties of Nanocrystalline and Amorphous Ni-P Alloys.pdf:pdf},
isbn = {0022-3727},
issn = {13616463},
journal = {Journal of Physics D: Applied Physics},
number = {5},
pages = {808--812},
title = {{Comparison of Properties of Nanocrystalline and Amorphous Ni-P Alloys}},
volume = {25},
year = {1992}
}
@article{Du2011,
author = {Du, Huailiang and Li, Jingyuan and Zhang, Jing and Su, Gang and Li, Xiaoyi and Zhao, Yuliang},
doi = {10.1021/jp206258u},
isbn = {1932-7447
1932-7455},
journal = {The Journal of Physical Chemistry C},
number = {47},
pages = {23261--23266},
title = {{Separation of Hydrogen and Nitrogen Gases with Porous Graphene Membrane}},
volume = {115},
year = {2011}
}
@article{Haas2009,
abstract = {The exchange-correlation functionals of the generalized gradient approximation (GGA) are still the most used for the calculations of the geometry and electronic structure of solids. The PBE functional [J. P. Perdew et al., Phys. Rev. Lett. 77, 3865 (1996)], the most common of them, provides excellent results in many cases. However, very recently other GGA functionals have been proposed and compete in accuracy with the PBE functional, in particular for the structure of solids. We have tested these GGA functionals, as well as the local-density approximation (LDA) and TPSS (meta-GGA approximation) functionals, on a large set of solids using an accurate implementation of the Kohn-Sham equations, namely, the full-potential linearized augmented plane-wave and local orbitals method. Often these recently proposed GGA functionals lead to improvement over LDA and PBE, but unfortunately none of them can be considered as good for all investigated solids.},
author = {Haas, Philipp and Tran, Fabien and Blaha, Peter},
doi = {10.1103/PhysRevB.79.085104},
file = {:Users/marc/Library/Application Support/Mendeley Desktop/Downloaded/Haas, Tran, Blaha - 2009 - Calculation of the lattice constant of solids with semilocal functionals.pdf:pdf},
isbn = {1098-0121},
issn = {10980121},
journal = {Physical Review B - Condensed Matter and Materials Physics},
number = {8},
pages = {1--10},
title = {{Calculation of the lattice constant of solids with semilocal functionals}},
volume = {79},
year = {2009}
}
@misc{EpoxyTechnology2008,
author = {{Epoxy Technology}},
file = {:Users/marc/Library/Application Support/Mendeley Desktop/Downloaded/Epoxy Technology - 2008 - EPO-TEK TV1002 Product information sheet.pdf:pdf},
isbn = {9780123705013},
title = {{EPO-TEK TV1002 Product information sheet}},
year = {2008}
}
@article{Lewis2013,
abstract = {Palladium-platinum (Pd-Pt) alloy membranes have been fabricated by sequential, electroless deposition onto porous yttria-stablized zirconia supports manufactured by Praxair, Inc. Membranes were synthesized with thicknesses of 4-12??m that contained up to 27wt{\%} Pt. Pd-Pt alloy membranes had lower pure-gas hydrogen flux compared with pure Pd membranes of equal thickness. However, when tested at 673K under an identical synthetic water-gas shift feed gas mixture composed of H2, H2O, CO2, and CO at 689.5kPa total pressure, the Pd-Pt alloy membranes had over 25{\%} higher hydrogen fluxes than a pure Pd membrane of similar thickness. Membrane films were analyzed after testing with SEM, EDS, and ICP-AES to corroborate membrane thickness and alloy compositions estimated by mass gain. SEM thickness estimates on membrane cross sections were similar to those estimated by mass gain. ICP-AES analysis was performed on two membranes and confirmed the composition estimated by gravimetric analysis. In five of six membranes the film composition estimated by gravimetric analysis was consistent with the surface composition estimated by EDS which indicated that the membranes had been adequately annealed. ?? 2013 Elsevier B.V.},
author = {Lewis, A. E. and Kershner, D. C. and Paglieri, S. N. and Slepicka, M. J. and Way, J. D.},
doi = {10.1016/j.memsci.2013.02.056},
file = {:Users/marc/Library/Application Support/Mendeley Desktop/Downloaded/Lewis et al. - 2013 - Pd-PtYSZ composite membranes for hydrogen separation from synthetic water-gas shift streams.pdf:pdf},
issn = {03767388},
journal = {Journal of Membrane Science},
keywords = {Electroless deposition,Fabricating palladium-platinum membranes,Hydrogen separation,Palladium-platinum alloy,Water-gas shift reaction},
pages = {257--264},
title = {{Pd-Pt/YSZ composite membranes for hydrogen separation from synthetic water-gas shift streams}},
volume = {437},
year = {2013}
}
@article{Wang2016d,
abstract = {In this paper, we present a facile one-step microwave assisted hydrothermal route for the synthesis of the unloaded and Pd-loaded SnO2 nanostructures. The Pd was grown in situ on the SnO2 nanostructure, constructing Pd/SnO2. A gas sensor based on the as-prepared Pd/SnO2 was fabricated and tested for response to carbon monoxide gas. The results indicated that the sensor using 3.0wt{\%} Pd-loaded SnO2 to 100ppm carbon monoxide performed a superior sensing properties compared to 0wt{\%}, 1.5wt{\%}, and 4.5wt{\%} Pd-loaded samples at a relatively low temperature (100°C). Such enhanced gas sensing performances could be attributed to both the contribution of Pd-loaded and the in situ method. In addition, the one-step in situ microwave assisted loading process provides a promising and versatile choice for the preparation of gas sensing materials.},
author = {Wang, Qingji and Wang, Chen and Sun, Hongbin and Sun, Peng and Wang, Yongzhi and Lin, Jun and Lu, Geyu},
doi = {10.1016/j.snb.2015.07.115},
file = {:Users/marc/Library/Application Support/Mendeley Desktop/Downloaded/Wang et al. - 2016 - Microwave assisted synthesis of hierarchical PdSnO2 nanostructures for CO gas sensor.pdf:pdf},
isbn = {8643185167},
issn = {09254005},
journal = {Sensors and Actuators B: Chemical},
keywords = {microwave-assisted hydrothermal method},
pages = {257--263},
publisher = {Elsevier B.V.},
title = {{Microwave assisted synthesis of hierarchical Pd/SnO2 nanostructures for CO gas sensor}},
url = {http://linkinghub.elsevier.com/retrieve/pii/S0925400515301726},
volume = {222},
year = {2016}
}
@article{Tran2014,
abstract = {Reduced graphene oxide (rGO) nanosheets as an over-coating layer on silver nanoparticles (rGO/AgNPs) and silver nanowires (rGO/AgNWs) have been fabricated using solution processes and tested for NH3 gas detection in nitrogen/air at room temperature (RT). When exposed to 100 ppm NH3, the rGO/AgNWs device exhibited an excellent sensitivity (15{\%}) compared to the rGO/AgNPs, rGO, and AgNWs devices. Our results revealed that the underneath AgNWs play a role as a network for providing a carrier transfer pathway, thus the rGO/AgNWs gas sensor device showed a stable, reproducible response and full recovery in nitrogen at RT. However, an excess amount of the AgNWs resulted in the agglomeration of the AgNWs. Consequently, a defective rGO over-coating layer with appearance of the pits was formed, reducing the sensitivity due to the adsorption deficiency. {\textcopyright} 2013 Elsevier B.V.},
author = {Tran, Quang Trung and Hoa, Huynh Tran My and Yoo, Dae Hwang and Cuong, Tran Viet and Hur, Seung Hyun and Chung, Jin Suk and Kim, Eui Jung and Kohl, Paul A.},
doi = {10.1016/j.snb.2013.12.062},
file = {:Users/marc/Library/Application Support/Mendeley Desktop/Downloaded/Tran et al. - 2014 - Reduced graphene oxide as an over-coating layer on silver nanostructures for detecting NH3 gas at room temperature.pdf:pdf},
isbn = {0925-4005},
issn = {09254005},
journal = {Sensors and Actuators, B: Chemical},
keywords = {Ag nanoparticles,Ag nanowires,NH3 gas sensors,Reduced graphene oxide},
pages = {45--50},
publisher = {Elsevier B.V.},
title = {{Reduced graphene oxide as an over-coating layer on silver nanostructures for detecting NH3 gas at room temperature}},
url = {http://dx.doi.org/10.1016/j.snb.2013.12.062},
volume = {194},
year = {2014}
}
@article{Zhang2017a,
abstract = {A new ternary nanocomposite of Ag-loaded zinc oxide (Ag-ZnO)/molybdenum disulfide (MoS2) was successfully prepared via layer-by-layer (LbL) self-assembly method for carbon monoxide (CO) sensing application. The microstructure and micromorphology of Ag-ZnO/MoS2 nanocomposite were fully characterized by various analytical techniques. The gas-sensing properties of the Ag-ZnO/MoS2 nanocomposite film were investigated upon exposure to different CO concentrations at room temperature, and compared with the pure ZnO, ZnO/MoS2 and Pt-ZnO/MoS2 sensors. The experimental results indicated that the Ag-ZnO/MoS2 nanocomposite sensor has excellent response, swift response/recovery characteristics, as well as acceptable repeatability and selectivity, which outstripped that of pure ZnO, ZnO/MoS2 and Pt-ZnO/MoS2 film sensors. The underlying sensing mechanism of Ag-ZnO/MoS2 nanocomposite film was attributed to catalytic activity of Ag and synergistic effect of ZnO and MoS2. This work demonstrated the Ag-ZnO/MoS2 is an excellent candidate for constructing room temperature CO gas sensor.},
author = {Zhang, Dongzhi and Sun, Yan'e and Jiang, Chuanxing and Yao, Yao and Dongyue, Wang and Zhang, Yong},
doi = {10.1016/j.snb.2017.07.173},
file = {:Users/marc/Library/Application Support/Mendeley Desktop/Downloaded/Zhang et al. - 2017 - Room-temperature highly sensitive CO gas sensor based on Ag-loaded zinc oxidemolybdenum disulfide ternary nanocomp.pdf:pdf},
issn = {09254005},
journal = {Sensors and Actuators, B: Chemical},
keywords = {CO gas sensor,Layer-by-layer self-assembly,Molybdenum disulfide,Nanocomposite,Room temperature},
number = {2},
pages = {1120--1128},
publisher = {Elsevier B.V.},
title = {{Room-temperature highly sensitive CO gas sensor based on Ag-loaded zinc oxide/molybdenum disulfide ternary nanocomposite and its sensing properties}},
url = {http://dx.doi.org/10.1016/j.snb.2017.07.173},
volume = {253},
year = {2017}
}
@article{Polotskaya2005,
abstract = {Three types of imide-containing polyamic acids (polybenzoxazinoneimide prepolymers), namely, a homopolymer, a copolymer, and a polymer-metal complex were synthesized and used for homogeneous membranes preparation. These membranes exhibited good physico-mechanical properties and also chemical and hydrolytical stability. Their gas separation properties were measured and analyzed by correlation with macromolecular packing density. Polybenzoxazinoneimide membranes were prepared by heating prepolymer membranes to 220 °C. Difference between gas separation properties of membranes based on polybenzoxazinoneimides and those of their prepolymers was estimated. Gas transport properties of all novel membranes were compared with those of known membranes by using Robeson's diagram. It was established that a polybenzoxazinoneimide membrane including a polymer-metal complex is the most effective among membranes studied here. {\textcopyright} 2005 Elsevier Ltd. All rights reserved.},
author = {Polotskaya, G. and Goikhman, M. and Podeshvo, I. and Kudryavtsev, V. and Pientka, Z. and Brozova, L. and Bleha, M.},
doi = {10.1016/j.polymer.2005.02.111},
isbn = {0032-3861},
issn = {00323861},
journal = {Polymer},
keywords = {Gas separation,Membranes,Polybenzoxazinoneimide},
number = {11},
pages = {3730--3736},
title = {{Gas transport properties of polybenzoxazinoneimides and their prepolymers}},
volume = {46},
year = {2005}
}
@article{Yimlamai2011,
author = {Yimlamai, Intira and Niamlang, Sumonman and Chanthaanont, Pojjawan and Kunanuraksapong, Ruksapong and Changkhamchom, Sairung and Sirivat, Anuvat},
doi = {10.1007/s11581-011-0545-3},
file = {:Users/marc/Library/Application Support/Mendeley Desktop/Downloaded/Yimlamai et al. - 2011 - Electrical conductivity response and sensitivity of ZSM-5, Y, and mordenite zeolites towards ethanol vapor.pdf:pdf},
issn = {09477047},
journal = {Ionics},
keywords = {Ethanol vapor,Mordenite zeolite,Sensing materials,Y Zeolite,ZSM-5 zeolite},
number = {7},
pages = {607--615},
title = {{Electrical conductivity response and sensitivity of ZSM-5, Y, and mordenite zeolites towards ethanol vapor}},
volume = {17},
year = {2011}
}
@misc{HydrogenAnalysisResourceCenter2018,
author = {{Hydrogen Analysis Resource Center}},
title = {{International hydrogen stations}},
year = {2018}
}
@article{Fang2010,
author = {Fang, Shumin and Bi, Lei and Yan, Litao and Sun, Wenping},
doi = {10.1021/jp102271v},
file = {:Users/marc/Library/Application Support/Mendeley Desktop/Downloaded/Fang et al. - 2010 - CO2 Resistant Hydrogen Permeation Membranes Based on Doped Ceria and Nickel.pdf:pdf},
issn = {1932-7447},
journal = {Journal of Physical Chemistry C},
pages = {10986--10991},
title = {{CO2 Resistant Hydrogen Permeation Membranes Based on Doped Ceria and Nickel}},
url = {http://pubs.acs.org/doi/abs/10.1021/jp102271v},
year = {2010}
}
@article{Song2015a,
abstract = {BaCe{\textless}inf{\textgreater}0.95{\textless}/inf{\textgreater}Tb{\textless}inf{\textgreater}0.05{\textless}/inf{\textgreater}O{\textless}inf{\textgreater}3-??{\textless}/inf{\textgreater} (BCTb) proton conducting perovskite hollow fibre membranes were fabricated by the combined phase-inversion and sintering technique. Surface modification was performed by H{\textless}inf{\textgreater}2{\textless}/inf{\textgreater}SO{\textless}inf{\textgreater}4{\textless}/inf{\textgreater} etching and/or by Pd-coating on the hollow fibre surfaces to improve the hydrogen permeation. Hydrogen permeation fluxes through the resultant hollow fibre membranes were measured between 650 and 1000 ??C, using 50-50{\%} H{\textless}inf{\textgreater}2{\textless}/inf{\textgreater}-He feed on the shell side and N{\textless}inf{\textgreater}2{\textless}/inf{\textgreater} as the sweep gas in the fibre lumen. The results indicate that coating porous Pd layers on the hollow fibre membrane surfaces could promote remarkably the hydrogen flux, i.e. from 0.046 mL cm{\textless}sup{\textgreater}-2{\textless}/sup{\textgreater} min{\textless}sup{\textgreater}-1{\textless}/sup{\textgreater} at 900 ??C in the original hollow fibre, to 0.272 mL cm{\textless}sup{\textgreater}-2{\textless}/sup{\textgreater} min{\textless}sup{\textgreater}-1{\textless}/sup{\textgreater} in the Pd-coated membrane, whereas the H{\textless}inf{\textgreater}2{\textless}/inf{\textgreater}SO{\textless}inf{\textgreater}4{\textless}/inf{\textgreater} etching modification would deteriorate the hydrogen permeation due to the contamination of sulphur-containing compounds. The surface exchange kinetics plays an important role in the hydrogen permeation through the BCTb perovskite membranes, which can be improved by coating porous Pd layers on either or both surfaces of the hollow fibre perovskite membranes.},
author = {Song, Jian and Meng, Bo and Tan, Xiaoyao and Liu, Shaomin},
doi = {10.1016/j.ijhydene.2015.03.057},
file = {:Users/marc/Library/Application Support/Mendeley Desktop/Downloaded/Song et al. - 2015 - Surface-modified proton conducting perovskite hollow fibre membranes by Pd-coating for enhanced hydrogen permeation.pdf:pdf},
issn = {03603199},
journal = {International Journal of Hydrogen Energy},
keywords = {Hollow fibre membrane,Hydrogen permeation,Proton conductor,Surface modification},
number = {18},
pages = {6118--6127},
publisher = {Elsevier Ltd},
title = {{Surface-modified proton conducting perovskite hollow fibre membranes by Pd-coating for enhanced hydrogen permeation}},
url = {http://dx.doi.org/10.1016/j.ijhydene.2015.03.057},
volume = {40},
year = {2015}
}
@misc{FCell2016,
author = {FCell, Symbio},
title = {{Zero emission mobility solutions}},
url = {https://www.symbio.one/en/},
year = {2016}
}
@incollection{Exter2015,
author = {den Exter, M J},
booktitle = {Palladium Membrane Technology for Hydrogen Production, Carbon Capture and Other Application},
doi = {10.1533/9781782422419.1.43},
editor = {Doukelis, A and Panopoulos, K and Koumanakos, A and Kakaras, E},
file = {:Users/marc/Library/Application Support/Mendeley Desktop/Downloaded/Exter - 2015 - The use of electroless plating as a deposition technology in the fabrication of palladium-based membranes.pdf:pdf},
pages = {43--67},
publisher = {Woodhead Publishing},
title = {{The use of electroless plating as a deposition technology in the fabrication of palladium-based membranes}},
year = {2015}
}
@article{David2011,
author = {David, E and Kopac, J},
doi = {10.1016/j.ijhydene.2010.12.032},
file = {:Users/marc/Library/Application Support/Mendeley Desktop/Downloaded/David, Kopac - 2011 - Devlopment of palladiumceramic membranes for hydrogen separation.pdf:pdf},
isbn = {03603199},
journal = {International Journal of Hydrogen Energy},
number = {7},
pages = {4498--4506},
title = {{Devlopment of palladium/ceramic membranes for hydrogen separation}},
volume = {36},
year = {2011}
}
@article{Okazaki2008,
author = {Okazaki, Junya and Tanaka, David Alfredo Pacheco and Tanco, Margot Anabel Llosa and Wakui, Yoshito and Ikeda, Takuji and Mizukami, Fujio and Suzuki, Toshishige M},
doi = {10.2320/matertrans.MBW200720},
file = {:Users/marc/Library/Application Support/Mendeley Desktop/Downloaded/Okazaki et al. - 2008 - Preparation and Hydrogen Permeation Properties of Thin Pd-Au Alloy Membranes Supported on Porous $\alpha$-Alumina Tube.pdf:pdf},
isbn = {1347-5320
1345-9678},
journal = {Materials Transactions},
number = {3},
pages = {449--452},
title = {{Preparation and Hydrogen Permeation Properties of Thin Pd-Au Alloy Membranes Supported on Porous $\alpha$-Alumina Tube}},
volume = {49},
year = {2008}
}
@article{Zenith2015,
author = {Zenith, Federico},
journal = {IFAC Papers Online},
number = {8},
pages = {445--450},
title = {{Reducing Fuel Cell Degedation in Micro Combined Heat and Power Systems}},
volume = {48},
year = {2015}
}
@article{Lu2012,
abstract = {Patterned metal-organic framework, ZIF-8 thin films can be generated by using standard photolithography or via selective growth with the aid of microcontact printing. The alternate chemical deposition (of ZIF-8) and physical deposition (of metallic materials) allow the insertion of metal layers in the ZIF-8 film that could serve as multifunctional chemical sensors for vapors and gases.},
author = {Lu, Guang and Farha, Omar K. and Zhang, Weina and Huo, Fengwei and Hupp, Joseph T.},
doi = {10.1002/adma.201202116},
file = {:Users/marc/Library/Application Support/Mendeley Desktop/Downloaded/Lu et al. - 2012 - Engineering ZIF-8 thin films for hybrid MOF-based devices.pdf:pdf},
isbn = {0935-9648},
issn = {09359648},
journal = {Advanced Materials},
keywords = {metal-organic frameworks,thin films},
number = {29},
pages = {3970--3974},
pmid = {22718482},
title = {{Engineering ZIF-8 thin films for hybrid MOF-based devices}},
volume = {24},
year = {2012}
}
@article{Escolastico2009,
abstract = {The tungstates Ln6WO12 are proton-conducting materials exhibiting sufficient electronic conductivity to consider them as potential candidates for the separation of hydrogen at high temperature. Hydrogen-permeable membranes will find application in power plants applying precombustion strategies, process intensification using high-temperature catalytic membrane reactors, and in components for electrochemical systems as proton conducting fuel cells (PCFCs) and electrolyzers. This work presents the preparation and characterization of nanocrystalline mixed conducting materials with three different nominal compositions (Nd6WO12-Eu 6WO12-Er6WO12) using a ol-gel complexation synthesis method. The evolution of the crystalline structure and crystallite size is studied as a function of the sintering temperature. Generally, the nanosized oxides show a (pseudo)-cubic crystalline fluorite structure which evolves into the most stable fluorite symmetry (tetragonal for Nd and rhombohedral for Er) with increasing sintering temperatures, i.e., crystallite sizes. Shrinkage behavior was analyzed for the three compositions in the range from 1000 to 1500 ?C and the nanosized Nd-based oxide showed a very high sintering activity even at relatively low temperatures (1100-1200 ?C). In addition, the total conductivity in different environments has been studied systematically for samples sintered at different temperatures and the highest total conductivity was obtained for the Eu-based compound having structure with tetragonal symmetry (0.009 S/ cm at 850 ?C). Hydrogen permeation was studied for a disk-shaped Nd6WO12 membrane in the range of 700-1000 ?C. Finally, stability of these materials at 700 and 800 C has been evaluated in contactwith a CO2-rich gas stream (dry or humidified) as well as thermochemical compatibility with yttriastabilized zirconia in the range 1200-1500 ?C.},
author = {Escol{\'{a}}stico, Sonia and Vert, Vicente B. and Serra, Jos{\'{e}} M.},
doi = {10.1021/cm900067k},
file = {:Users/marc/Library/Application Support/Mendeley Desktop/Downloaded/Escol{\'{a}}stico, Vert, Serra - 2009 - Preparation and characterization of nanocrystalline mixed proton-electronic conducting materials base.pdf:pdf},
isbn = {0897-4756},
issn = {08974756},
journal = {Chemistry of Materials},
number = {14},
pages = {3079--3089},
title = {{Preparation and characterization of nanocrystalline mixed proton-electronic conducting materials based on the system Ln6WO12}},
volume = {21},
year = {2009}
}
@article{Tanaka2006,
abstract = {Permeability and selectivity of pure gas H2, CO2, O2, N2and CH4as well as a mixture of CO2/N2for sulfonated homopolyimides prepared from 1,4,5,8-naphthalene tetracarboxylic dianhydride (NTDA) and 2,2-bis[4-(4-aminophenoxy)phenyl] hexafluoro propane disulfonic acid (BAPHFDS) were measured and compared to those of the non-sulfonated homopolyimide having the same polymer backbone. The polyimide in a proton form (NTDA-BAPHFDS(H)) displayed higher selectivity of H2over CH4without loss of H2permeability. Strong intermolecular interaction induced by sulfonic acid groups decreased diffusivity of the larger molecules. The CO2/N2(19/81) mixed gas permeation was investigated as a function of humidity. With increasing relative humidity from 0{\%} RH to 90{\%} RH, the CO2permeability for NTDA-BAPHFDS(H) polyimide increased by more than one order of magnitude, and the selectivity of CO2/N2also increased twice or more. On the other hand, the gas permeability for the non-sulfonated polyimide slightly decreased with increasing humidity. NTDA-BAPHFDS(H) polyimide displayed a CO2permeability of 290×10-10cm3(STP) cm/(cm2s cmHg) and a separation factor of CO2/N2of 51 at 96{\%} RH, 50 °C and total pressure of 1 atm. {\textcopyright} 2006 Elsevier Ltd. All rights reserved.},
author = {Tanaka, Kazuhiro and Islam, Md Nurul and Kido, Masayoshi and Kita, Hidetoshi and ichi Okamoto, Ken},
doi = {10.1016/j.polymer.2006.04.001},
isbn = {0032-3861},
issn = {00323861},
journal = {Polymer},
keywords = {Humidity,Separation of carbon dioxide,Sulfonated polyimides},
number = {12},
pages = {4370--4377},
title = {{Gas permeation and separation properties of sulfonated polyimide membranes}},
volume = {47},
year = {2006}
}
@article{Hu2011,
abstract = {The development of hydrogen energy systems has placed a high demand on hydrogen-permeable membranes as compact hydrogen separators and purifiers. Although Pd/Ceramic composite membranes are particularly effective in this role, the high cost of these membranes has greatly limited their applications; this high cost stems largely from the use of expensive substrate material. This problem may be solved by substrate recycling and the use of lower cost substrates. As a case study, we employed expensive asymmetric microporous Al2O3 and low-cost macroporous symmetric Al 2O3 as membrane substrates (average pore sizes are 0.2 and 3.3 ??m, respectively). The palladium membranes were fabricated by electroless plating, and substrate recycling was carried out by palladium dissolution with a hot HNO3 solution. The functional surface layer of the microporous Al2O3 was damaged during substrate recycling, and the reuse of the substrate led to poor membrane selectivity. With the assistance of pencil coating as a facile and environmentally benign surface treatment, the macroporous Al2O3 can be successfully utilized. Furthermore, the macroporous Al2O3 can be also recycled and reused as membrane substrate, yielding highly permeable, selective and stable palladium membranes. Consequently, the substrate cost can be further decreased, and the applications of this kind of membranes would expand. ?? 2011, Hydrogen Energy Publications, LLC. Published by Elsevier Ltd. All rights reserved.},
author = {Hu, Xiaojuan and Yu, Jian and Song, Jun and Wang, Xueguang and Huang, Yan},
doi = {10.1016/j.ijhydene.2011.09.040},
file = {:Users/marc/Library/Application Support/Mendeley Desktop/Downloaded/Hu et al. - 2011 - Toward low-cost Pdceramic composite membranes for hydrogen separation A case study on reuse of the recycled porous Al.pdf:pdf},
issn = {03603199},
journal = {International Journal of Hydrogen Energy},
keywords = {Asymmetric porous ceramics,Composite palladium membrane,Hydrogen separation,Macroporous ceramics,Pencil coating,Recycle},
number = {24},
pages = {15794--15802},
publisher = {Elsevier Ltd},
title = {{Toward low-cost Pd/ceramic composite membranes for hydrogen separation: A case study on reuse of the recycled porous Al2O3 substrates in membrane fabrication}},
url = {http://dx.doi.org/10.1016/j.ijhydene.2011.09.040},
volume = {36},
year = {2011}
}
@article{MartinezGaleano2016,
abstract = {NaA zeolite membranes were synthesized on top of both disk and tubular porous stainless steel substrates by the secondary growth method. The support seed stage was optimized, considering the seed synthesis conditions and the deposition method. The formation of the NaA zeolite on both the membrane and the residual powder collected at the bottom of the vessel after synthesis was corroborated by X-ray diffraction (XRD). Scanning electron microscopy (SEM) revealed the formation of a film with well-defined crystals with a cubic morphology characteristic of the NaA zeolite. The support modification with 3-aminopropyltriethoxysilane improved the morphology and stability of the NaA zeolite membranes synthesized on top of porous stainless steel substrates. When vacuum was applied only in the last hydrothermal synthesis stage, a smooth zeolite layer was obtained on top of porous stainless steel supports. The single gas (H2, N2, CO2and CH4) and mixture (H2/N2and H2/CH4) perm-selective properties of the tubular membranes were evaluated as a function of temperature and pressure. Separation factors higher than the Knudsen coefficient were obtained in all membranes. The single gas H2and CO2permeation flux of the membranes increased with temperature indicating a preponderance of the activated diffusion over the Knudsen diffusion. The higher H2/CO2ideal separation factors were 7.6 and 5.6 for the more selective tubular membranes, at 453 K and 100 kPa.},
author = {{Mart{\'{i}}nez Galeano}, Yohana and Cornaglia, Laura and Tarditi, Ana M.},
doi = {10.1016/j.memsci.2016.04.005},
file = {:Users/marc/Library/Application Support/Mendeley Desktop/Downloaded/Mart{\'{i}}nez Galeano, Cornaglia, Tarditi - 2016 - NaA zeolite membranes synthesized on top of APTES-modified porous stainless steel substra.pdf:pdf},
issn = {18733123},
journal = {Journal of Membrane Science},
keywords = {3-aminopropyltriethoxysilane,Hydrogen recovery,Inorganic membranes,NaA zeolite,Stainless steel support},
pages = {93--103},
publisher = {Elsevier},
title = {{NaA zeolite membranes synthesized on top of APTES-modified porous stainless steel substrates}},
url = {http://dx.doi.org/10.1016/j.memsci.2016.04.005},
volume = {512},
year = {2016}
}
@article{Gade2008,
abstract = {It is desirable to create thin ({\textless}25 ??m), unsupported, defect-free palladium and palladium alloy foils in a cost-effective manner in order to study intrinsic material properties exclusive of support effects. We have developed a novel technique for producing unsupported palladium films by electroless plating upon mirror-finished stainless steel supports followed by mechanical removal. High quality pure palladium films as thin as 7.2 ??m were produced. Single gas steady state permeation experiments were performed using hydrogen and nitrogen to examine permeability and selectivity. The pure palladium membranes showed hydrogen permeabilities comparable to cold-rolled unsupported foils, and high H
                        2/N
                        2 selectivity. Palladium-copper membranes were prepared by sequential electroless plating of copper onto palladium foils followed by in situ annealing. The annealing process produces films of desired composition with permeabilities comparable to those in the literature. The annealing process does not appear to produce defects in the film, and the membranes thus produced have performed 15 days without increased leak rates. ?? 2007 Elsevier B.V. All rights reserved.},
author = {Gade, Sabina K. and Thoen, Paul M. and Way, J. Douglas},
doi = {10.1016/j.memsci.2007.08.022},
file = {:Users/marc/Library/Application Support/Mendeley Desktop/Downloaded/Gade, Thoen, Way - 2008 - Unsupported palladium alloy foil membranes fabricated by electroless plating.pdf:pdf},
isbn = {03767388},
issn = {03767388},
journal = {Journal of Membrane Science},
keywords = {Electroless plating,Hydrogen separation,Membrane supports,Metal membranes,Palladium membranes},
number = {1-2},
pages = {112--118},
title = {{Unsupported palladium alloy foil membranes fabricated by electroless plating}},
volume = {316},
year = {2008}
}
@article{Løvvik2008,
abstract = {It is well known that silver segregates to the surface of pure and ideal Pd-Ag alloy surfaces. By first-principles band-structure calculations it is shown in this paper how this may be changed when hydrogen is adsorbed on a Pd-Ag(1 1 1) surface. Due to hydrogen binding more strongly to palladium than to silver, there is a clear energy gain from a reversal of the surface segregation. Hydrogen-induced segregation may provide a fundamental explanation for the hydrogen or reducing treatments that are required to activate hydrogen-selective membrane or catalyst performance. ?? 2008 Elsevier B.V. All rights reserved.},
author = {L{\o}vvik, O M and Opalka, Susanne M},
doi = {10.1016/j.susc.2008.07.016},
isbn = {0039-6028},
issn = {00396028},
journal = {Surface Science},
keywords = {Density-functional calculations,Hydrogen,Palladium,Silver,Surface segregation},
number = {17},
pages = {2840--2844},
title = {{Reversed surface segregation in palladium-silver alloys due to hydrogen adsorption}},
volume = {602},
year = {2008}
}
@article{Battersby2009a,
abstract = {In this work we investigate the performance of cobalt silica membranes for the separation of gas mixtures at various temperatures and partial pressures. The membranes were prepared by a sol-gel process using tetraethyl orthosilicate (TEOS) in ethanol and H2O2with cobalt nitrate hexahydrate (Co(NO3)26H2O). The membranes complied with molecular sieving transport mechanism, delivering high single gas selectivities for He/N2(4500) and H2/CO2(1000) and high activation energy for the smaller gas molecules (He, H2) whilst a negative activation energy for larger molecules (CO, CO2and N2). Molecular probing results strongly suggest membranes with a narrow pore size distribution with an average pore size of 3 {\AA}. The effect of gas composition on the membrane operation was studied over both binary and ternary gas mixtures, and compared with the single gas permeance results. It was found in both cases feed concentration had a large impact on both selectivities and flow rates. These followed a trade off inverse relationship, as increasing H2feed concentration led to a higher flow rate but a lower H2selectivity, though H2purity in the permeate stream increased. The use of sweep gas in the permeate stream to increase the driving force of gas permeation was beneficial as the H2flow rate and the H2recovery rate increased by a factor of 3. It is noteworthy to mention that though the ternary feed flow had only 27{\%} H2concentration, the permeate stream delivered CO and CO2at very low concentrations, 0.8 and 0.14{\%}, respectively. It was observed that the membrane selectivity in gas mixtures were 10-15{\%} of the single gas selectivity, while permeation decreased with the gas composition (Single {\textgreater} Binary {\textgreater} Ternary). Nevertheless, increased temperature and sweep flow rate allowed the membrane to deliver a permeate stream in excess of 99{\%} H2purity and a lower CO concentration of 700 ppm, indicating the quality of these membranes for gas mixture separation. {\textcopyright} 2008 Elsevier B.V. All rights reserved.},
author = {Battersby, Scott and Tasaki, Tsutomu and Smart, Simon and Ladewig, Bradley and Liu, Shaomin and Duke, Mikel C. and Rudolph, Victor and {Diniz da Costa}, Jo{\~{a}}o C.},
doi = {10.1016/j.memsci.2008.12.051},
file = {:Users/marc/Library/Application Support/Mendeley Desktop/Downloaded/Battersby et al. - 2009 - Performance of cobalt silica membranes in gas mixture separation.pdf:pdf},
issn = {03767388},
journal = {Journal of Membrane Science},
keywords = {Cobalt,Gas mixtures,H2separation,Silica membranes,Sweep and flow rates},
number = {1-2},
pages = {91--98},
title = {{Performance of cobalt silica membranes in gas mixture separation}},
volume = {329},
year = {2009}
}
@article{Norby2008,
author = {Norby, Truls and Haugsrud, Reidar},
file = {:Users/marc/Library/Application Support/Mendeley Desktop/Downloaded/Norby, Haugsrud - 2008 - Dense Ceramic Membranes for Hydrogen Separation, in Membranes for Energy Conversion.pdf:pdf},
isbn = {3527313427},
journal = {Nonporous inorganic membranes},
pages = {169--216},
title = {{Dense Ceramic Membranes for Hydrogen Separation, in Membranes for Energy Conversion}},
url = {http://books.google.com/books?hl=en{\&}lr={\&}id=gCQjw-cyulAC{\&}oi=fnd{\&}pg=PA1{\&}dq=Dense+Ceramic+Membranes+for+Hydrogen+Separation{\&}ots=e-3oiNW5EW{\&}sig=Iv9bFzwJWBTYe7brB66{\_}qQ5W{\_}u8},
year = {2008}
}
@article{Agrawal2015,
abstract = {A zeolite membrane fabrication process combining 2D-zeolite nanosheet seeding and gel-free secondary growth is described. This process produces selective molecular sieve films that are as thin as 100 nm and exhibit record high permeances for xylene- and butane-isomers.},
author = {Agrawal, Kumar Varoon and Topuz, Berna and Pham, Tung Cao Thanh and Nguyen, Thanh Huu and Sauer, Nicole and Rangnekar, Neel and Zhang, Han and Narasimharao, Katabathini and Basahel, Sulaiman Nasir and Francis, Lorraine F. and Macosko, Christopher W. and Al-Thabaiti, Shaeel and Tsapatsis, Michael and Yoon, Kyung Byung},
doi = {10.1002/adma.201405893},
isbn = {1521-4095 (Electronic)$\backslash$r0935-9648 (Linking)},
issn = {15214095},
journal = {Advanced Materials},
keywords = {butane and xylene isomer separation,secondary growth,zeolite membranes,zeolite nanosheets},
number = {21},
pages = {3243--3249},
pmid = {25866018},
title = {{Oriented MFI membranes by gel-less secondary growth of sub-100 nm MFI-nanosheet seed layers}},
volume = {27},
year = {2015}
}
@article{Kuntze1999,
abstract = {The (110) surface of the alloy Au3Pd has been investigated by quantitative low-energy electron diffraction and low-energy ion scattering spectroscopy to determine the structure and composition of the first three atomic layers. The structure and the segregation profile have also been modeled by means of the embedded atom method, combined with Monte Carlo simulations. Both the experimental and theoretical results indicate segregation of gold, resulting in (nearly) pure Au in the two topmost layers. The surface is (12) missing-row reconstructed similarly to Au(110), with a significant contraction of the first interlayer spacing and a buckling in the third layer.},
author = {Kuntze, J and Speller, S and Heiland, W and Deurinck, P and Creemers, C and Atrei, A and Bardi, U},
file = {:Users/marc/Library/Application Support/Mendeley Desktop/Downloaded/Kuntze et al. - 1999 - Surface structure and segregation profile of the alloy Au 3 Pd (110) Experiment and theory.pdf:pdf},
journal = {Physical Review B},
number = {12},
pages = {9010--9018},
title = {{Surface structure and segregation profile of the alloy Au 3 Pd (110): Experiment and theory}},
volume = {60},
year = {1999}
}
@article{Migliardini2016,
author = {Migliardini, F and {Di Palma}, T M and Gaele, M F and Corbo, P},
doi = {10.1016/j.ijhydene.2016.06.196},
isbn = {03603199},
journal = {International Journal of Hydrogen Energy},
title = {{Hydrogen purge and reactant feeding strategies in self-humidified PEM fuel cell systems}},
year = {2016}
}
@article{Caro2011,
abstract = {Despite much progress in the development of zeolite molecular sieve membranes, there is so far no industrial gas separation by zeolite membranes, with the exception of the de-watering of bio-ethanol by steam permeation using LTA membranes. During the last 5 years, metal-organic framework (MOF) membranes have been developed and tested in gas separation. The complete tool box of techniques originally developed for the preparation of zeolite membranes could be applied for the preparation of MOF membranes, such as the use of macroporous ceramic or metal supports, seeding, intergrowth-supporting additives, and microwave heating. There are some structure-related properties of MOFs which recommend them as suitable material for molecular sieve membranes. On the other hand, the structural flexibility of MOFs apparently prevents a sharp molecular sieving with a pore size estimated from the 'rigid' crystallographic structure by size exclusion. An application of MOFs is predicted in so-called mixed matrix membranes which show improved performance in comparison with the pure polymer membranes. Different from zeolites as organic-inorganic material, the MOF nanoparticles can be easily embedded into organic polymers, and standard shaping technologies to hollow fibers or spiral wound geometries can be applied. ?? 2011 Elsevier Ltd. All rights reserved.},
author = {Caro, Juergen},
doi = {10.1016/j.coche.2011.08.007},
file = {:Users/marc/Library/Application Support/Mendeley Desktop/Downloaded/Caro - 2011 - Are MOF membranes better in gas separation than those made of zeolites.pdf:pdf},
isbn = {22113398},
issn = {22113398},
journal = {Current Opinion in Chemical Engineering},
number = {1},
pages = {77--83},
publisher = {Elsevier Ltd},
title = {{Are MOF membranes better in gas separation than those made of zeolites?}},
url = {http://dx.doi.org/10.1016/j.coche.2011.08.007},
volume = {1},
year = {2011}
}
@article{Wei2011,
author = {Wei, Yanying and Liu, Hongfei and Xue, Jian and Li, Zhong and Wang, Haihui},
doi = {10.1002/aic.12321},
isbn = {00011541},
journal = {AIChE Journal},
number = {4},
pages = {975--984},
title = {{Preparation and oxygen permeation of U-shaped perovskite hollow-fiber membranes}},
volume = {57},
year = {2011}
}
@misc{Atures2012,
author = {Atures, E and Ations, Pplic},
file = {:Users/marc/Library/Application Support/Mendeley Desktop/Downloaded/Atures, Ations - 2012 - Br 680-3 Polyimide paste adhesive Technical Dtaa sheet.pdf:pdf},
title = {{Br 680-3 Polyimide paste adhesive Technical Dtaa sheet}},
year = {2012}
}
@article{Mao2012,
abstract = {We report a novel and selective gas-sensing platform with reduced graphene oxide (RGO) decorated with tin oxide (SnO2) nanocrystals (NCs). This hybrid SnO2 NC–RGO platform showed enhanced NO2 but weakened NH3 sensing compared with bare RGO, showing promise in tuning the sensitivity and selectivity of RGO-bas},
author = {Mao, Shun and Cui, Shumao and Lu, Ganhua and Yu, Kehan and Wen, Zhenhai and Chen, Junhong},
doi = {10.1039/c2jm30378g},
file = {:Users/marc/Library/Application Support/Mendeley Desktop/Downloaded/Mao et al. - 2012 - Tuning gas-sensing properties of reduced graphene oxide using tin oxide nanocrystals.pdf:pdf},
isbn = {0959-9428},
issn = {0959-9428},
journal = {Journal of Materials Chemistry},
number = {22},
pages = {11009},
title = {{Tuning gas-sensing properties of reduced graphene oxide using tin oxide nanocrystals}},
url = {http://xlink.rsc.org/?DOI=c2jm30378g},
volume = {22},
year = {2012}
}
@article{Zou2012,
abstract = {The stabilization of copper clusters in nanosized metal-organic framework crystals, Cu-Y(BTC), was achieved by a solvent-exchange approach, followed by hydrogen reduction. The formation of copper clusters in the Y(BTC) nanocrystals generated during the hydrogen reduction process was followed by UV-vis spectroscopy. The Cu-Y(BTC) nanocrystals were further assembled in thin films with a thickness of 250 nm. The distribution and size of the copper clusters in the films were studied by CO chemisorption, followed by FT-IR spectroscopy combined with transmission electron microscopy. It was shown that the copper clusters with a mean diameter of 6 nm were homogeneously distributed and stabilized in the Cu-Y(BTC) films. Further, the Cu-Y(BTC) films were utilized for detection of single harmful gases, such as CO, chloroform, and 2-ethylthiophene, or mixtures of two compounds. The high sensitivity, selectivity, and reversibility of the Cu-Y(BTC) films toward single CO, chloroform, and 2-ethylthiophene were demonstrated. Noteworthy, the Cu-Y(BTC) films exhibited a fast response toward CO, even in the presence of chloroform and 2-ethylthiophene, which was due to the high activity and accessibility of copper clusters. The response of Cu-Y(BTC) toward 2-ethylthiophene was slower in comparison with chloroform, which was attributed to the bigger size and higher viscosity of 2-ethylthiophene. {\textcopyright} 2012 American Chemical Society.},
author = {Zou, Xiaoqin and Goupil, Jean Michel and Thomas, S{\'{e}}bastien and Zhang, Feng and Zhu, Guangshan and Valtchev, Valentin and Mintova, Svetlana},
doi = {10.1021/jp3044016},
file = {:Users/marc/Library/Application Support/Mendeley Desktop/Downloaded/Zou et al. - 2012 - Detection of harmful gases by copper-containing metal-organic framework films.pdf:pdf},
isbn = {19327447},
issn = {19327447},
journal = {Journal of Physical Chemistry C},
number = {31},
pages = {16593--16600},
title = {{Detection of harmful gases by copper-containing metal-organic framework films}},
volume = {116},
year = {2012}
}
@article{Chen2016b,
abstract = {A mixed protonic-electronic conductor made of La5.5W0.45Nb0.15Mo0.4O11.25-$\delta$(LWNM) has been developed by partial co-substitution of W with Nb and Mo in La5.5WO11.25-$\delta$aiming to improve the hydrogen permeation property. A hydrogen permeation flux of 0.195 mL/mincm2was obtained at 1000 °C with a mixture of 50{\%} H2-50{\%} He as the feed gas under dry condition. More importantly, there was no obvious decline of the hydrogen flux during 80 h operation even with CO2in the feed gas.},
author = {Chen, Yan and Cheng, Shunfan and Chen, Li and Wei, Yanying and Ashman, Peter J. and Wang, Haihui},
doi = {10.1016/j.memsci.2016.02.065},
file = {:Users/marc/Library/Application Support/Mendeley Desktop/Downloaded/Chen et al. - 2016 - Niobium and molybdenum co-doped La5.5WO11.25-$\delta$membrane with improved hydrogen permeability.pdf:pdf},
issn = {18733123},
journal = {Journal of Membrane Science},
keywords = {CO2stability,Hydrogen separation,Lanthanum tungstate,Membrane,Proton conductor},
pages = {155--163},
publisher = {Elsevier},
title = {{Niobium and molybdenum co-doped La5.5WO11.25-$\delta$membrane with improved hydrogen permeability}},
url = {http://dx.doi.org/10.1016/j.memsci.2016.02.065},
volume = {510},
year = {2016}
}
@article{Y.S.Cheng2001,
author = {{Y.S. Cheng}, K L Yeung},
file = {:Users/marc/Library/Application Support/Mendeley Desktop/Downloaded/Y.S. Cheng - 2001 - Effects of electroless plating chemistry on the synthesis of palladium membranes.pdf:pdf},
journal = {Journal of Membrane Science},
pages = {195--203},
title = {{Effects of electroless plating chemistry on the synthesis of palladium membranes}},
volume = {182},
year = {2001}
}
@article{Hromadka2015,
abstract = {An optical fibre long period grating (LPG) modified with a thin film of ZIF-8, a zeolitic immidazol framework material and a subgroup of metal organic framework family, was employed for the detection of organic vapours. ZIF-8 films were deposited onto the surface of the LPG using an in-situ crystallization technique by mixing freshly prepared 12.5 mM zinc nitrate hexahydrate and 25 mM 2-metyl-imidazole solutions in methanol. The ZIF-8 film was characterized by scanning electron microscopy. The thickness and refractive index (RI) of films deposited after 1,2,3,5 and 10 growth cycles were determined by using an ellipsometer. The crystallinity of the films was confirmed by X-ray diffraction pattern. The LPG was designed to operate at the phase matching turning point to provide the highest sensitivity. The sensing mechanism is based on the measurement of the change in the RI of the film induced by the penetration of the chemical molecules into the ZIF-8 pores. The responses of LPGs modified with 2 and 5 growth cycles of ZIF-8 to exposure to methanol, ethanol, 2-propanol and acetone were characterised. The sensitivity of the measurements to humidity as an interfering parameter was also investigated.},
author = {Hromadka, Jiri and Tokay, Begum and James, Stephen and Tatam, Ralph P. and Korposh, Sergiy},
doi = {10.1016/j.snb.2015.07.027},
file = {:Users/marc/Library/Application Support/Mendeley Desktop/Downloaded/Hromadka et al. - 2015 - Optical fibre long period grating gas sensor modified with metal organic framework thin films.pdf:pdf},
isbn = {978-1-62841-839-2},
issn = {09254005},
journal = {Sensors and Actuators, B: Chemical},
keywords = {Long period grating (LPG),Metal organic framework (MOF),Organic vapour sensing,ZIF-8,Zeolitic imidazolate framework (ZIF)},
pages = {891--899},
publisher = {Elsevier B.V.},
title = {{Optical fibre long period grating gas sensor modified with metal organic framework thin films}},
url = {http://dx.doi.org/10.1016/j.snb.2015.07.027},
volume = {221},
year = {2015}
}
@article{Uemiya1991,
abstract = {The hydrogen permeability and selectivity of a composite membrane consisting of miscible palladium-silver alloy film supported on the outer surface of a porous alumina cylinder were investigated. The membrane showed much greater flux than commercially obtainable palladium-based membranes for hydrogen separation. The high hydrogen flux was due to both the thinness of the film and its high hydrogen solubility. In addition, hydrogen embrittlement was suppressed by alloying the thin palladium film with silver, and 100{\%} hydrogen selectivity was retained even at relatively low temperatures. ?? 1991.},
author = {Uemiya, Shigeyuki and Matsuda, Takeshi and Kikuchi, Eiichi},
doi = {10.1016/S0376-7388(00)83041-0},
file = {:Users/marc/Library/Application Support/Mendeley Desktop/Downloaded/Uemiya, Matsuda, Kikuchi - 1991 - Hydrogen permeable palladium-silver alloy membrane supported on porous ceramics.pdf:pdf},
isbn = {03767388 (ISSN)},
issn = {03767388},
journal = {Journal of Membrane Science},
keywords = {ceramic membrane,composite membrane,gas separation,hydrogen permeability,membrane preparation and structure,palladium-silver alloy},
number = {3},
pages = {315--325},
title = {{Hydrogen permeable palladium-silver alloy membrane supported on porous ceramics}},
volume = {56},
year = {1991}
}
@article{Guerreiro2016,
author = {Guerreiro, Bruno Honrado and Martin, Manuel H and Rou{\'{e}}, Lionel and Guay, Daniel},
doi = {10.1016/j.memsci.2016.02.040},
file = {:Users/marc/Library/Application Support/Mendeley Desktop/Downloaded/Guerreiro et al. - 2016 - Hydrogen permeability of PdCuAu membranes prepared from mechanically-alloyed powders.pdf:pdf},
isbn = {03767388},
journal = {Journal of Membrane Science},
pages = {68--82},
title = {{Hydrogen permeability of PdCuAu membranes prepared from mechanically-alloyed powders}},
volume = {509},
year = {2016}
}
@article{Kitiwan2010b,
abstract = {The present work was focused on the preparation of palladium alloy membranes and the effect of properties of ceramic support on the composited membrane morphology. Palladium-base membrane is known to have high selectivity and stability for hydrogen separation. In order to increase hydrogen permeation and separation factor, the membrane must be thinner and defect-free. Palladium membrane supported on a porous alumina prepared by electroless plating is the promising method to provide good hydrogen permeability. The alumina tube substrate was pre-seeded by immersing in the palladium acetate solution and followed by reduction in the alkaline hydrazine solution. After that, the deposition of palladium membrane could be achieved from the plating bath containing ethylenediamine tetraacetic acid (EDTA) stabilized palladium complex and hydrazine. The morphology of palladium film was observed to progress as a function of plating time and a dense layer membrane was available after plating for 3 h. The porosity of ceramic support exhibited an effect on the microstructure of deposited film such that the support with low porosity tended to achieve a defect free palladium membrane. ?? 2010 The Chinese Society for Metals.},
author = {Kitiwan, Mettaya and Atong, Duangduen},
doi = {10.1016/S1005-0302(11)60016-9},
file = {:Users/marc/Library/Application Support/Mendeley Desktop/Downloaded/Kitiwan, Atong - 2010 - Effects of Porous Alumina Support and Plating Time on Electroless Plating of Palladium Membrane(2).pdf:pdf},
issn = {10050302},
journal = {Journal of Materials Science and Technology},
keywords = {Electroless plating,Palladium composite membrane,Tubular porous support},
number = {12},
pages = {1148--1152},
publisher = {The Chinese Society for Metals},
title = {{Effects of Porous Alumina Support and Plating Time on Electroless Plating of Palladium Membrane}},
url = {http://dx.doi.org/10.1016/S1005-0302(11)60016-9},
volume = {26},
year = {2010}
}
@article{Bryden1995,
abstract = {Palladium membranes are of technological interest since they show a very high selectivity for hydrogen. Diffusion through the palladium is often the rate-limiting step in hydrogen transport through the membrane. Hydrogen flux can be improved by reducing the membrane thickness and increasing the hydrogen diffusivity. Nanostructured palladium has a higher hydrogen diffusivity than conventional palladium due to its large volume fraction of grain boundaries. They were generated as ultrathin membranes by d.c. magnetron sputtering onto porous Vycor?? glass substrates in an argon atmosphere. The nanostructured films exhibited no cracks on the submicron scale when examined with environmental scanning electron microscopy. The membranes delaminated from the substrates when exposed to hydrogen at room temperature, owing to the ?? ??? ?? phase transition. Heating the ultrathin nanostructured palladium membranes to 200 ??C led to some grain growth. Stabilization against the phase transition and grain coarsening is critical to applications of nanostructured membranes in hydrogen separations, and can be achieved by the alloying of palladium with another metal. ?? 1995.},
author = {Bryden, Kenneth J. and Ying, Jackie Y.},
doi = {10.1016/0921-5093(95)09950-6},
file = {:Users/marc/Library/Application Support/Mendeley Desktop/Downloaded/Bryden, Ying - 1995 - Nanostructured palladium membrane synthesis by magnetron sputtering.pdf:pdf},
issn = {09215093},
journal = {Materials Science and Engineering A},
keywords = {Inorganic membranes,Magnetron sputtering,Nanostructures,Palladium},
number = {1-2},
pages = {140--145},
title = {{Nanostructured palladium membrane synthesis by magnetron sputtering}},
volume = {204},
year = {1995}
}
@article{Ayturk2009,
author = {Ayturk, M Engin and Ma, Yi Hua},
doi = {10.1016/j.memsci.2008.12.062},
file = {:Users/marc/Library/Application Support/Mendeley Desktop/Downloaded/Ayturk, Ma - 2009 - Electroless Pd and Ag deposition kinetics of the composite Pd and PdAg membranes synthesized from agitated plating b.pdf:pdf},
isbn = {03767388},
journal = {Journal of Membrane Science},
number = {1-2},
pages = {233--245},
title = {{Electroless Pd and Ag deposition kinetics of the composite Pd and Pd/Ag membranes synthesized from agitated plating baths}},
volume = {330},
year = {2009}
}
@article{Zuo2006b,
abstract = {Successful development of hydrogen separation membranes based on mixed ionic and electronic conductors may improve the economics of hydrogen production. While high proton conductivity has been reported for many perovskite-type oxides in a humid atmosphere, materials with both high proton conductivity and good chemical stability under conditions for practical hydrogen separation are yet to be developed. In this paper, we report the effect of Zr-doping in BZCYs [Ba( Zr0.8-chi Ce chi Y0.2)O3-alpha] (0.4 {\textless}= chi {\textless}= 0.8) on the proton conductivity and chemical stability. A novel Ni-BZCY7 [Ni-Ba(Zr0.1Ce0.7Y0.2)-O3-alpha cermet (metal-ceramic composite) membrane appears to have not only high proton conductivity but also adequate stability in a CO2- and H2O-containing atmosphere},
author = {Zuo, Chendong and Dorris, S. E. and Balachandran, U. and Liu, Meilin},
doi = {10.1021/cm0518224},
file = {:Users/marc/Library/Application Support/Mendeley Desktop/Downloaded/Zuo et al. - 2006 - Effect of Zr-doping on the chemical stability and hydrogen permeation of the Ni-BaCe 0.8y 0.2O 3- mixed protonic-ele.pdf:pdf},
isbn = {0897-4756},
issn = {08974756},
journal = {Chemistry of Materials},
number = {19},
pages = {4647--4650},
title = {{Effect of Zr-doping on the chemical stability and hydrogen permeation of the Ni-BaCe 0.8y 0.2O 3-?? mixed protonic-electronic conductor}},
volume = {18},
year = {2006}
}
@article{Petit2010,
abstract = {Composites of a copper-based metal-org. framework (MOF) and graphite oxide (GO) were tested for hydrogen sulfide removal at ambient conditions.  In order to understand the mechanisms of adsorption, the initial and exhausted samples were analyzed by various techniques including X-ray diffraction, Fourier transform IR spectroscopy, thermogravimetric analyses, and sorption of nitrogen.  Compared to the parent materials, an enhancement in hydrogen sulfide adsorption was found.  It was the result of phys. adsorption of water and H2S in the pore space formed at the interface between the MOF units and the graphene layers where the dispersive forces are the strongest.  Besides physisorption, reactive adsorption was found as the main mechanism of retention.  H2S mols. bind to the copper centers of the MOF.  They progressively react with the MOF units resulting in the formation of copper sulfide.  This leads to the collapse of the MOF structure.  Water enhances adsorption in the composites as it allows the dissoln. of hydrogen sulfide. [on SciFinder(R)]},
author = {Petit, Camille and Mendoza, Barbara and Bandosz, Teresa J.},
doi = {10.1002/cphc.201000689},
file = {:Users/marc/Library/Application Support/Mendeley Desktop/Downloaded/Petit, Mendoza, Bandosz - 2010 - Hydrogen sulfide adsorption on MOFs and MOFGraphite oxide composites.pdf:pdf},
isbn = {1439-4235},
issn = {14394235},
journal = {ChemPhysChem},
keywords = {Adsorption,Composites,Desulfurization,Graphene,Metal-organic frameworks},
number = {17},
pages = {3678--3684},
pmid = {20945452},
title = {{Hydrogen sulfide adsorption on MOFs and MOF/Graphite oxide composites}},
volume = {11},
year = {2010}
}
@article{AbuElHawa2014,
author = {{Abu El Hawa}, Hani W and Paglieri, Stephen N and Morris, Craig C and Harale, Aadesh and {Douglas Way}, J},
doi = {10.1016/j.memsci.2014.04.029},
isbn = {03767388},
journal = {Journal of Membrane Science},
pages = {151--160},
title = {{Identification of thermally stable Pd-alloy composite membranes for high temperature applications}},
volume = {466},
year = {2014}
}
@article{Peng2017,
abstract = {Metal-organic framework (MOF) nanosheets could serve as ideal building blocks of molecular sieve membranes owing to their structural diversity and minimized mass-transfer barrier. To date, discovery of appropriate MOF nanosheets and facile fabrication of high performance MOF nanosheet-based membranes remain as great challenges. A modified soft-physical exfoliation method was used to disintegrate a lamellar amphiprotic MOF into nanosheets with a high aspect ratio. Consequently sub-10 nm-thick ultrathin membranes were successfully prepared, and these demonstrated a remarkable H-2/CO2 separation performance, with a separation factor of up to 166 and H-2 permeance of up to 8 X 10(-7) molm(-2) s(-1)Pa(-1) at elevated testing temperatures owing to a well-defined size-exclusion effect. This nanosheet-based membrane holds great promise as the next generation of ultrapermeable gas separation membrane.},
author = {Peng, Yuan and Li, Yanshuo and Ban, Yujie and Yang, Weishen},
doi = {10.1002/anie.201703959},
file = {:Users/marc/Library/Application Support/Mendeley Desktop/Downloaded/Peng et al. - 2017 - Two-Dimensional Metal–Organic Framework Nanosheets for Membrane-Based Gas Separation.pdf:pdf},
isbn = {1433-7851},
issn = {15213773},
journal = {Angewandte Chemie - International Edition},
keywords = {gas separation,metal–organic frameworks,molecular sieve membranes,nanosheets,ultrapermeable membranes},
number = {33},
pages = {9757--9761},
title = {{Two-Dimensional Metal–Organic Framework Nanosheets for Membrane-Based Gas Separation}},
volume = {56},
year = {2017}
}
@article{Roa2005,
author = {Roa, Fernando and Way, J Douglas},
doi = {10.1016/j.apsusc.2004.06.023},
isbn = {01694332},
journal = {Applied Surface Science},
number = {1-4},
pages = {85--104},
title = {{The effect of air exposure on palladium–copper composite membranes}},
volume = {240},
year = {2005}
}
@article{Basile2008,
author = {Basile, Angelo and Gallucci, Fausto and Tosti, Silvano},
doi = {10.1016/s0927-5193(07)13008-4},
isbn = {09275193},
pages = {255--323},
title = {{Synthesis, Characterization, and Applications of Palladium Membranes}},
volume = {13},
year = {2008}
}
@article{Chu2013,
abstract = {This paper presents a metal enhanced optical oxygen sensor that comprises an optical fiber coated at one end with platinum (II) meso- tetrakis(pentafluorophenyl)porphyrin (PtTFPP) and silver metal-coated nanoparticles embedded in an n-octyltriethoxysilane (Octyl-triEOS)/ tetraethylorthosilane (TEOS) composite xerogel. The sensitivity of the optical oxygen sensor is quantified in terms of the ratio IN2/IO2, where IN2 and IO2 represent the detected fluorescence intensities in pure nitrogen and pure oxygen environments, respectively. The experimental results show that the oxygen sensor has a sensitivity of 167. The response time was 2.6 s when switching from pure nitrogen to pure oxygen, and 36 s when switching in the reverse direction. The experimental results show that compared to oxygen sensor based on Pt(II) complex immobilized in the sol-gel matrix, the proposed optical fiber oxygen sensor has higher sensitivity. In addition to the increased surface area per unit mass of the sensing surface, the metal-coated silica nanoparticles also increase the sensitivity because the metal-enhanced fluorescence. The proposed optical sensor has the advantages of low cost and high sensitivity for oxygen monitoring using a cheap LED as a light source. {\textcopyright} 2013 Elsevier B.V.},
author = {Chu, Cheng Shane and Sung, Ti Wen and Lo, Yu Lung},
doi = {10.1016/j.snb.2013.05.011},
file = {:Users/marc/Library/Application Support/Mendeley Desktop/Downloaded/Chu, Sung, Lo - 2013 - Enhanced optical oxygen sensing property based on Pt(II) complex and metal-coated silica nanoparticles embedded i.pdf:pdf},
isbn = {0925-4005},
issn = {09254005},
journal = {Sensors and Actuators, B: Chemical},
keywords = {Metal-enhanced,Optical fiber oxygen sensor,Pt(II) complex,Sol-gel},
pages = {287--292},
publisher = {Elsevier B.V.},
title = {{Enhanced optical oxygen sensing property based on Pt(II) complex and metal-coated silica nanoparticles embedded in sol-gel matrix}},
url = {http://dx.doi.org/10.1016/j.snb.2013.05.011},
volume = {185},
year = {2013}
}
@misc{APMEX2016,
author = {APMEX},
number = {06/12/2016},
title = {{Silver Prices}},
url = {http://www.apmex.com/spotprices/silver-prices},
volume = {2016},
year = {2016}
}
@article{Smith2016,
author = {Smith, Merry K. and Jensen, Kennedy E. and Pivak, Polina A. and Mirica, Katherine A.},
doi = {10.1021/acs.chemmater.6b02528},
file = {:Users/marc/Library/Application Support/Mendeley Desktop/Downloaded/Smith et al. - 2016 - Direct Self-Assembly of Conductive Nanorods of Metal-Organic Frameworks into Chemiresistive Devices on Shrinkable.pdf:pdf},
issn = {15205002},
journal = {Chemistry of Materials},
number = {15},
pages = {5264--5268},
title = {{Direct Self-Assembly of Conductive Nanorods of Metal-Organic Frameworks into Chemiresistive Devices on Shrinkable Polymer Films}},
volume = {28},
year = {2016}
}
@article{Carneiro2008,
abstract = {B3LYP/LANL2DZ and B3LYP/6-31G(d)-restricted and -unrestricted calculations are employed to calculate energies and adsorption forms of formaldehyde adsorbed on planar and on tetrahedral Pd4 clusters and on a Pd4 cluster supported on Al10O15. Formaldehyde adsorbs on planar Pd4 in the eta(2)(C,O)-di-sigma adsorption mode, while on tetrahedral Pd4, it adsorbs in the eta(2)(C,O)-pi adsorption mode. The adsorption energy on planar Pd4 is -21.4 kcal x mol(-1), whereas for the tetrahedral Pd4 cluster, the adsorption energy is -13.2 kcal x mol(-1). The latter value is close to experimental findings (-12 to -14 kcal x mol(-1)). Adsorption of formaldehyde on Pd4 supported on an Al10O15 cluster leads essentially to the same result as that found for adsorption on the tetrahedral Pd4 cluster. Charge density analysis for the interaction between formaldehyde and the Pd4 clusters indicates strong backdonation in the eta(2) adsorption mode, leading to positive charge on the Pd4 cluster. NBO analysis shows that the highly coordinated octahedral aluminum atoms of Al10O15 donate electron density to the supported Pd4 cluster, while tetrahedral aluminum atoms with lower coordination number have acidic nature and therefore act as electron acceptors.},
author = {Carneiro, Jos{\'{e}} Walkimar de M and Cruz, Maur{\'{i}}cio T de M},
doi = {10.1021/jp801591z},
file = {:Users/marc/Library/Application Support/Mendeley Desktop/Downloaded/Carneiro, Cruz - 2008 - Density functional theory study of the adsorption of formaldehyde on Pd4 and on Pd4gamma-Al2O3 clusters.pdf:pdf},
issn = {1520-5215},
journal = {The journal of physical chemistry. A},
number = {38},
pages = {8929--37},
pmid = {18702460},
title = {{Density functional theory study of the adsorption of formaldehyde on Pd4 and on Pd4/gamma-Al2O3 clusters.}},
url = {http://www.ncbi.nlm.nih.gov/pubmed/18702460},
volume = {112},
year = {2008}
}
@article{FernandoRoaJ.DouglasWay2002,
author = {{Fernando Roa  J. Douglas Way}, Michael J Block},
chapter = {411},
file = {:Users/marc/Library/Application Support/Mendeley Desktop/Downloaded/Fernando Roa J. Douglas Way - 2002 - The infleuance of alloy composition on the H2 flux of composite Pd-Cu membranes.pdf:pdf},
journal = {Desalination},
pages = {411--416},
title = {{The infleuance of alloy composition on the H2 flux of composite Pd-Cu membranes}},
volume = {147},
year = {2002}
}
@article{Chuapradit2005,
abstract = {Electrical conductivity response of polyaniline/zeolite composites towards CO is investigated in terms of dopant type, dopant concentration, zeolite LTA content and zeolite pore size. Both MA and HCl doped polyanilines respond with comparable magnitudes towards CO; the latter responses are slightly smaller for the same doping level. Addition of zeolite 4A reduces the electrical conductivity response but improves the sensitivity towards CO with increasing zeolite concentration up to 40{\%} w/w. This concentration is evidently below the percolation threshold value, which is estimated to be above 50{\%} w/w. Composite with zeolite 3A has a comparable sensitivity value relative to that of pure polyaniline. Composites of 4A and 5A have greater sensitivity values over that of the pure polyaniline at the CO concentration range between 16 and 1000 ppm. Zeolite 5A is the most effective mesoporous material in promoting interaction between CO and polyaniline because of its largest pore size of 5 {\AA}, relative to the zeolite 3A and 4A which have the pore sizes of 4 and 3 {\AA}, respectively. {\textcopyright} 2004 Published by Elsevier Ltd.},
author = {Chuapradit, C. and Wannatong, L. Ruangchuay and Chotpattananont, D. and Hiamtup, Piyanoot and Sirivat, A. and Schwank, J.},
doi = {10.1016/j.polymer.2004.11.101},
file = {:Users/marc/Library/Application Support/Mendeley Desktop/Downloaded/Chuapradit et al. - 2005 - Polyanilinezeolite LTA composites and electrical conductivity response towards CO.pdf:pdf},
isbn = {00323861 (ISSN)},
issn = {00323861},
journal = {Polymer},
keywords = {Composite,Polyaniline,Zeolite LTA},
number = {3},
pages = {947--953},
title = {{Polyaniline/zeolite LTA composites and electrical conductivity response towards CO}},
volume = {46},
year = {2005}
}
@article{Aba2015,
author = {Aba, Nor Farah Diana and Chong, Jeng Yi and Wang, Bo and Mattevi, Cecilia and Li, K},
doi = {10.1016/j.memsci.2015.03.001},
file = {:Users/marc/Library/Application Support/Mendeley Desktop/Downloaded/Aba et al. - 2015 - Graphene oxide membranes on ceramic hollow fibers – Microstructural stability and nanofiltration performance.pdf:pdf},
isbn = {03767388},
journal = {Journal of Membrane Science},
pages = {87--94},
title = {{Graphene oxide membranes on ceramic hollow fibers – Microstructural stability and nanofiltration performance}},
volume = {484},
year = {2015}
}
@article{Luo2016,
abstract = {{\textless}p{\textgreater}Triptycene-containing PBO-based polymers with ultrafine microporosity promoting ultrafast and highly selective gas transport.{\textless}/p{\textgreater}},
author = {Luo, Shuangjiang and Liu, Junyi and Lin, Haiqing and Kazanowska, Barbara A. and Hunckler, Michael D. and Roeder, Ryan K. and Guo, Ruilan},
doi = {10.1039/c6ta03951k},
isbn = {2050-7488},
issn = {20507496},
journal = {Journal of Materials Chemistry A},
number = {43},
pages = {17050--17062},
publisher = {Royal Society of Chemistry},
title = {{Preparation and gas transport properties of triptycene-containing polybenzoxazole (PBO)-based polymers derived from thermal rearrangement (TR) and thermal cyclodehydration (TC) processes}},
volume = {4},
year = {2016}
}
@article{Gade2011,
abstract = {Self-supported Pd-Au membranes were produced by magnetron sputtering and cold working with compositions between 7 and 20wt{\%} Au. Permeation tests were performed in synthetic water-gas shift reaction mixtures with up to 50ppm H2S. Membranes with higher gold content showed less flux inhibition by either carbon or sulfur containing species, regardless of fabrication technique or thickness. A 20wt{\%} Au alloy had the same permeability in pure hydrogen as it did in a sulfur-free WGS mixture, and only lost 40{\%} of its permeability in a 20ppm H2S mixture. However, membranes produced by sputtering experienced irreversible loss of hydrogen selectivity when exposed to mixtures containing sulfur, caused by a significant decrease in membrane thickness. No equivalent decrease in thickness or selectivity was observed for cold-worked membranes. This metal loss is explained by a corrosive mechanism in which palladium sulfides flake off the feed surface and are entrained. Although sulfidation also occurs in cold-worked membranes only sputtered membranes display this corrosion, due to a different microstructure in which grains are oriented perpendicular to the surface. {\textcopyright} 2010 Elsevier B.V.},
annote = {NULL},
author = {Gade, Sabina K. and DeVoss, Sarah J. and Coulter, Kent E. and Paglieri, Stephen N. and Alptekin, G{\"{o}}khan O. and Way, J. Douglas},
doi = {10.1016/j.memsci.2010.11.044},
file = {:Users/marc/Library/Application Support/Mendeley Desktop/Downloaded/Gade et al. - 2011 - Palladium-gold membranes in mixed gas streams with hydrogen sulfide Effect of alloy content and fabrication techniq.pdf:pdf},
issn = {03767388},
journal = {Journal of Membrane Science},
keywords = {Cold-working,H2S tolerance,Magnetron sputtering,Palladium alloy membrane,Palladium-gold},
number = {1-2},
pages = {35--41},
publisher = {Elsevier B.V.},
title = {{Palladium-gold membranes in mixed gas streams with hydrogen sulfide: Effect of alloy content and fabrication technique}},
url = {http://dx.doi.org/10.1016/j.memsci.2010.11.044},
volume = {378},
year = {2011}
}
@article{Robeson2008,
abstract = {The empirical upper bound relationship for membrane separation of gases initially published in 1991 has been reviewed with the myriad of data now presently available. The upper bound correlation follows the relationship Pi= k $\alpha$i jn, where Piis the permeability of the fast gas, $\alpha$ij(Pi/Pj) is the separation factor, k is referred to as the "front factor" and n is the slope of the log-log plot of the noted relationship. Below this line on a plot of log $\alpha$ijversus log Pi, virtually all the experimental data points exist. In spite of the intense investigation resulting in a much larger dataset than the original correlation, the upper bound position has had only minor shifts in position for many gas pairs. Where more significant shifts are observed, they are almost exclusively due to data now in the literature on a series of perfluorinated polymers and involve many of the gas pairs comprising He. The shift observed is primarily due to a change in the front factor, k, whereas the slope of the resultant upper bound relationship remains similar to the prior data correlations. This indicates a different solubility selectivity relationship for perfluorinated polymers compared to hydrocarbon/aromatic polymers as has been noted in the literature. Two additional upper bound relationships are included in this analysis; CO2/N2and N2/CH4. In addition to the perfluorinated polymers resulting in significant upper bound shifts, minor shifts were observed primarily due to polymers exhibiting rigid, glassy structures including ladder-type polymers. The upper bound correlation can be used to qualitatively determine where the permeability process changes from solution-diffusion to Knudsen diffusion. {\textcopyright} 2008 Elsevier B.V. All rights reserved.},
archivePrefix = {arXiv},
arxivId = {arXiv:1402.6991v1},
author = {Robeson, Lloyd M.},
doi = {10.1016/j.memsci.2008.04.030},
eprint = {arXiv:1402.6991v1},
file = {:Users/marc/Library/Application Support/Mendeley Desktop/Downloaded/Robeson - 2008 - The upper bound revisited.pdf:pdf},
isbn = {0376-7388},
issn = {03767388},
journal = {Journal of Membrane Science},
keywords = {Gas separation,Membrane separation,Polymer permeability,Upper bound},
number = {1-2},
pages = {390--400},
pmid = {257834700044},
title = {{The upper bound revisited}},
volume = {320},
year = {2008}
}
@article{Celebi2014,
abstract = {A two-dimensional (2D) porous layer can make an ideal membrane for separation of chemical mixtures because its infinitesimal thickness promises ultimate permeation. Graphene--with great mechanical strength, chemical stability, and inherent impermeability--offers a unique 2D system with which to realize this membrane and study the mass transport, if perforated precisely. We report highly efficient mass transfer across physically perforated double-layer graphene, having up to a few million pores with narrowly distributed diameters between less than 10 nanometers and 1 micrometer. The measured transport rates are in agreement with predictions of 2D transport theories. Attributed to its atomic thicknesses, these porous graphene membranes show permeances of gas, liquid, and water vapor far in excess of those shown by finite-thickness membranes, highlighting the ultimate permeation these 2D membranes can provide.},
author = {Celebi, Kemal and Buchheim, Jakob and Wyss, Roman M and Droudian, Amirhossein and Gasser, Patrick and Shorubalko, Ivan and Kye, Jeong-il and Lee, Changho and Park, Hyung Gyu},
doi = {10.1126/science.1249097},
file = {:Users/marc/Library/Application Support/Mendeley Desktop/Downloaded/Celebi et al. - 2014 - Ultimate Permeation Across Atomically Thin Porous Graphene.pdf:pdf},
isbn = {2007005484},
issn = {0036-8075},
journal = {Science},
number = {6181},
pages = {289--293},
pmid = {24744372},
title = {{Ultimate Permeation Across Atomically Thin Porous Graphene}},
url = {http://www.ncbi.nlm.nih.gov/pubmed/24744372{\%}0Ahttp://www.sciencemag.org/cgi/doi/10.1126/science.1249097},
volume = {344},
year = {2014}
}
@misc{DepartmentforEnvironment2017,
author = {{Department for Environment}, Food {\&} Rural Affairs and {Department for Transport}},
title = {{Air quality plan for nitrogen dioxide (NO2) in UK}},
year = {2017}
}
@article{Nishimura2003,
abstract = {Hydrogen permeation and transmission electron microscope (TEM) observations were performed for V-Al alloys (10-40 mol{\%} Al). Hydrogen permeability of V-Al alloys decreased with aluminum content, but not in a monotonous manner. Below 20 mol{\%} of aluminum, hydrogen permeability of V-Al alloys decreased linearly with aluminum content. From 20 to 30 mol{\%} of aluminum, hydrogen permeability decreased abruptly. A15 phase which is shown in the V-Al phase diagram was not observed in any samples, quenched from 1373 K or aged at 853 K for 100-350 h. Instead, some precipitates of 200 nm to 1 ??m were observed at grain boundaries, sub-boundaries and inside the grains. The amount of the precipitates, however, was too small to explain the significant drop of hydrogen permeability observed in the alloys with more than 20 mol{\%} Al. The long-time permeation test showed that V-10Al sample possessed satisfactory durability to be used as hydrogen purification materials. ?? 2002 Elsevier B.V. All rights reserved.},
author = {Nishimura, C. and Ozaki, T. and Komaki, M. and Zhang, Y.},
doi = {10.1016/S0925-8388(02)01273-2},
file = {:Users/marc/Library/Application Support/Mendeley Desktop/Downloaded/Nishimura et al. - 2003 - Hydrogen permeation and transmission electron microscope observations of V-Al alloys.pdf:pdf},
issn = {09258388},
journal = {Journal of Alloys and Compounds},
keywords = {Aluminium,Hydrogen permeation,Membrane,Vanadium},
pages = {295--299},
title = {{Hydrogen permeation and transmission electron microscope observations of V-Al alloys}},
volume = {356-357},
year = {2003}
}
@article{Li2013b,
abstract = {Pilot-scale zeolite NaA membranes with high PV performance supported on cheap coarse macroporous supports were prepared by one single secondary growth using varying temperature hot dip-coating seeding method (VTHD). Through the VTHD method, a thin, dense and pinhole-free asymmetric NaA seed layer composed of large and small NaA seeds could be manipulated onto the surface of a coarse macroporous support. The large NaA seeds mainly acted as fillers to reduce the pore sizes of the support while the small NaA seeds acted as nuclei to provide sites for NaA crystal growth. The effects of the seed suspension concentrations, the seed sizes and coating temperature on the morphology of the seed layer and the separation performance of the resulting NaA membranes were investigated. The reproducibility of the VTHD method was as high as 70{\%}. The zeolite NaA membrane prepared by the VTHD method showed a water flux of 2.85kgm-2h-1with a separation factor over 10,000 in dehydrating the 90wt{\%} ethanol/10wt{\%} water mixture at 343K. The use of a cheap macroporous support and the high reproducibility of the VTHD method provide the feasibility for large-scale commercial production of low cost zeolite NaA membranes, promoting the broad application of zeolite NaA membranes. {\textcopyright} 2013 Elsevier B.V.},
author = {Li, Huazheng and Wang, Jinqu and Xu, Jing and Meng, Xiangdi and Xu, Bo and Yang, Jianhua and Li, Shiyang and Lu, Jinming and Zhang, Yan and He, Xiaolan and Yin, Dehong},
doi = {10.1016/j.memsci.2013.04.030},
isbn = {0376-7388},
issn = {03767388},
journal = {Journal of Membrane Science},
keywords = {Dehydration,High reproducibility,Pervaporation performance,Pilot-scale zeolite NaA membrane,Varying temperature hot dip-coating method},
pages = {513--522},
title = {{Synthesis of zeolite NaA membranes with high performance and high reproducibility on coarse macroporous supports}},
volume = {444},
year = {2013}
}
@article{Ziegler2001,
abstract = {The accessible pore system of polymeric ultrafiltration membranes was modified by titanium dioxide and treated further with palladium acetate to yield catalytically active, porous nanofiltration membranes. This procedure is in general applicable for any polymeric ultrafiltration membrane. To overcome the drawback of low thermal stability of common polymeric membranes, new ultrafiltration membranes were developed, based on a polyamideimide containing up to 40wt.{\%} of an inorganic filler admixed to the membrane casting solution. A similar treatment yielded catalytically active polymer membranes stable up to 200??C. The membranes were characterised by several methods and their performance was tested in a membrane reactor at 30??C. The hydrogenation of propene and the selective hydrogenation of propyne were examined as test reactions. For most effective membranes, 100{\%} conversion of propene at a maximum yield of 98{\%} of propane was obtained at the permeate side. In the selective hydrogenation of 5{\%} propyne in propene, a selectivity of 99{\%} for propene at 100{\%} conversion of propyne was achieved at a permeate flux of 0.08m3/m2hbar at 0.9bar pressure difference. ?? 2001 Elsevier Science B.V.},
author = {Ziegler, Silke and Theis, Juliane and Fritsch, Detlev},
doi = {10.1016/S0376-7388(00)00688-8},
file = {:Users/marc/Library/Application Support/Mendeley Desktop/Downloaded/Ziegler, Theis, Fritsch - 2001 - Palladium modified porous polymeric membranes and their performance in selective hydrogenation of propy.pdf:pdf},
isbn = {0376-7388},
issn = {03767388},
journal = {Journal of Membrane Science},
keywords = {Alkene,Alkyne,Catalytic membrane reactor,Palladium,Porous polymeric membrane,Selective hydrogenation,TiO2 pore modification},
number = {1-2},
pages = {71--84},
title = {{Palladium modified porous polymeric membranes and their performance in selective hydrogenation of propyne}},
volume = {187},
year = {2001}
}
@article{Li2013,
abstract = {Ultrathin, molecular-sieving membranes have great potential to realize high-flux, high-selectivity mixture separation at low energy cost. Current microporous membranes [pore size {\textless} 1 nanometer (nm)], however, are usually relatively thick. With the use of current membrane materials and techniques, it is difficult to prepare microporous membranes thinner than 20 nm without introducing extra defects. Here, we report ultrathin graphene oxide (GO) membranes, with thickness approaching 1.8 nm, prepared by a facile filtration process. These membranes showed mixture separation selectivities as high as 3400 and 900 for H2/CO2 and H2/N2 mixtures, respectively, through selective structural defects on GO.},
annote = {Li, Hang
Song, Zhuonan
Zhang, Xiaojie
Huang, Yi
Li, Shiguang
Mao, Yating
Ploehn, Harry J
Bao, Yu
Yu, Miao
ENG
Research Support, Non-U.S. Gov't
2013/10/05 06:00
Science. 2013 Oct 4;342(6154):95-8. doi: 10.1126/science.1236686.},
author = {Li, H and Song, Z and Zhang, X and Huang, Y and Li, S and Mao, Y and Ploehn, H J and Bao, Y and Yu, M},
doi = {10.1126/science.1236686},
isbn = {1095-9203 (Electronic)
0036-8075 (Linking)},
journal = {Science},
number = {6154},
pages = {95--98},
pmid = {24092739},
title = {{Ultrathin, molecular-sieving graphene oxide membranes for selective hydrogen separation}},
url = {http://www.ncbi.nlm.nih.gov/pubmed/24092739},
volume = {342},
year = {2013}
}
@article{Cao2017,
author = {Cao, Yanying and He, Yi and Zou, Xiaoxin and Li, Guo-Dong},
doi = {10.1016/j.snb.2017.05.181},
file = {:Users/marc/Library/Application Support/Mendeley Desktop/Downloaded/Cao et al. - 2017 - Tungsten oxide clusters decorated ultrathin In 2 O 3 nanosheets for selective detecting formaldehyde.pdf:pdf},
issn = {09254005},
journal = {Sensors and Actuators B: Chemical},
pages = {232--238},
title = {{Tungsten oxide clusters decorated ultrathin In 2 O 3 nanosheets for selective detecting formaldehyde}},
url = {http://linkinghub.elsevier.com/retrieve/pii/S0925400517310134},
volume = {252},
year = {2017}
}
@article{Liu2013a,
abstract = {Cermet membranes composited of Ni and doped barium cerate have been widely studied for hydrogen separation; however, their practical application is limited primarily by the relatively low permeation rate and instability of doped barium cerate in H2O and CO2 containing gases. Here we report our findings on the development of a thin-film cermet membrane consisting of Ni and BaZr0.1Ce0.7Y0.1Yb0.1O3−$\delta$ (BZCYYb), supported on a porous Ni–BZCYYb substrate. High fluxes of 1.12 and 0.49 ml min−1 cm−2 have been demonstrated at 900 °C and 700 °C, respectively, when hydrogen was used as the feed gas on one side and N2 as the sweep gas on the other side. Most importantly, the high-performance membrane can be easily fabricated by a cost-effective particle-suspension coating/co-firing process, offering great promise for large scale hydrogen separation applications.},
author = {Liu, Mingfei and Sun, Wenping and Li, Xiaxi and Feng, Shi and Ding, Dong and Chen, Dongchang and Liu, Meilin and Park, Hyeon Cheol},
doi = {10.1016/j.ijhydene.2013.09.057},
file = {:Users/marc/Library/Application Support/Mendeley Desktop/Downloaded/Liu et al. - 2013 - High-performance Ni–BaZr0.1Ce0.7Y0.1Yb0.1O3−$\delta$ (BZCYYb) membranes for hydrogen separation.pdf:pdf},
issn = {03603199},
journal = {International Journal of Hydrogen Energy},
number = {34},
pages = {14743--14749},
title = {{High-performance Ni–BaZr0.1Ce0.7Y0.1Yb0.1O3−$\delta$ (BZCYYb) membranes for hydrogen separation}},
url = {http://linkinghub.elsevier.com/retrieve/pii/S0360319913022805},
volume = {38},
year = {2013}
}
@article{Murray2009,
abstract = {New materials capable of storing hydrogen at high gravimetric and volumetric densities are required if hydrogen is to be widely employed as a clean alternative to hydrocarbon fuels in cars and other mobile applications. With exceptionally high surface areas and chemically-tunable structures, microporous metal-organic frameworks have recently emerged as some of the most promising candidate materials. In this critical review we provide an overview of the current status of hydrogen storage within such compounds. Particular emphasis is given to the relationships between structural features and the enthalpy of hydrogen adsorption, spectroscopic methods for probing framework-H(2) interactions, and strategies for improving storage capacity (188 references).},
annote = {Murray, Leslie J
Dinca, Mircea
Long, Jeffrey R
ENG
England
2009/04/23 09:00
Chem Soc Rev. 2009 May;38(5):1294-314. doi: 10.1039/b802256a. Epub 2009 Mar 25.},
author = {Murray, L J and Dinca, M and Long, J R},
doi = {10.1039/b802256a},
isbn = {0306-0012 (Print)
0306-0012 (Linking)},
journal = {Chem Soc Rev},
number = {5},
pages = {1294--1314},
pmid = {19384439},
title = {{Hydrogen storage in metal-organic frameworks}},
url = {http://www.ncbi.nlm.nih.gov/pubmed/19384439},
volume = {38},
year = {2009}
}
@article{Kingsbury2010a,
abstract = {Asymmetric ceramic hollow fibre membranes and membrane supports have been prepared using a combined phase inversion and sintering technique for use in a multifunctional catalytic membrane reactor. The asymmetric structure is such that the fibre may simultaneously function either as a porous membrane and a matrix for catalyst deposition, or as a porous support for the coating of a gas separation layer and a matrix for catalyst deposition. The effectiveness of catalyst deposition depends strongly on the pore size distribution of the membrane or membrane support, which is bimodal in nature and is affected by the calcination temperature and the fibre preparation parameters. The effect of the calcination temperature and preparation parameters on the pore size distribution and fibre morphology have been studied systematically with regard to catalyst deposition and fibre mechanical strength and a route to optimizing the fibre structure has been suggested.},
author = {Kingsbury, Benjamin F K and Wu, Zhentao and Li, K},
doi = {https://doi.org/10.1016/j.cattod.2010.02.039},
issn = {0920-5861},
journal = {Catalysis Today},
keywords = {Asymmetric hollow fibre,Catalyst deposition,Pore size distribution},
number = {3},
pages = {306--315},
title = {{A morphological study of ceramic hollow fibre membranes: A perspective on multifunctional catalytic membrane reactors}},
url = {http://www.sciencedirect.com/science/article/pii/S0920586110001240},
volume = {156},
year = {2010}
}
@article{AceitunoMelgar2015,
author = {{Aceituno Melgar}, V{\'{i}}ctor Manuel and Kim, Jinsoo and Othman, Mohd Roslee},
doi = {10.1016/j.jiec.2015.03.006},
isbn = {1226086X},
journal = {Journal of Industrial and Engineering Chemistry},
pages = {1--15},
title = {{Zeolitic imidazolate framework membranes for gas separation: A review of synthesis methods and gas separation performance}},
volume = {28},
year = {2015}
}
@article{Dolan2010,
author = {Dolan, M D},
doi = {10.1016/j.memsci.2010.06.068},
isbn = {03767388},
journal = {Journal of Membrane Science},
number = {1-2},
pages = {12--28},
title = {{Non-Pd BCC alloy membranes for industrial hydrogen separation}},
volume = {362},
year = {2010}
}
@article{Wales2015,
abstract = {Improvements in the efficiency of combustion within a vehicle can lead to reductions in the emission of harmful pollutants and increased fuel efficiency. Gas sensors have a role to play in this process, since they can provide real time feedback to vehicular fuel and emissions management systems as well as reducing the discrepancy between emissions observed in factory tests and ‘real world' scenarios. In this review we survey the current state-of-the-art in using porous materials for sensing the gases relevant to automotive emissions. Two broad classes of porous material – zeolites and metal–organic frameworks (MOFs) – are introduced, and their potential for gas sensing is discussed. The adsorptive, spectroscopic and electronic techniques for sensing gases using porous materials are summarised. Examples of the use of zeolites and MOFs in the sensing of water vapour, oxygen, NOx, carbon monoxide and carbon dioxide, hydrocarbons and volatile organic compounds, ammonia, hydrogen sulfide, sulfur dioxide and hydrogen are then detailed. Both types of porous material (zeolites and MOFs) reveal great promise for the fabrication of sensors for exhaust gases and vapours due to high selectivity and sensitivity. The size and shape selectivity of the zeolite and MOF materials are controlled by variation of pore dimensions, chemical composition (hydrophilicity/hydrophobicity), crystal size and orientation, thus enabling detection and differentiation between different gases and vapours.},
author = {Wales, Dominic J. and Grand, Julien and Ting, Valeska P. and Burke, Richard D. and Edler, Karen J. and Bowen, Chris R. and Mintova, Svetlana and Burrows, Andrew D.},
doi = {10.1039/C5CS00040H},
file = {:Users/marc/Library/Application Support/Mendeley Desktop/Downloaded/Wales et al. - 2015 - Gas sensing using porous materials for automotive applications.pdf:pdf},
isbn = {0306-0012},
issn = {0306-0012},
journal = {Chem. Soc. Rev.},
number = {13},
pages = {4290--4321},
pmid = {25982991},
publisher = {Royal Society of Chemistry},
title = {{Gas sensing using porous materials for automotive applications}},
url = {http://xlink.rsc.org/?DOI=C5CS00040H},
volume = {44},
year = {2015}
}
@misc{Toyota2015,
author = {Toyota},
title = {{The Toyota Mirai}},
url = {https://www.toyota.co.uk/new-cars/new-mirai/landing.json},
year = {2015}
}
@article{Wang2013,
abstract = {This work is the first demonstration of rapid thermal processing techniques as applied to metal oxide/silica membranes on tubular geometries. A procedure was developed which combined fast sol-gel synthesis, rapid calcination steps and a thermal annealing stage to reduce the membrane fabrication time by more than two-thirds, from a conventional process taking seven or more days to less than two. A significant aspect of this major development was the use of a pre-hydrolysed silica precursor ethyl silicate 40 (ES40), instead of the generally preferred tetraethyl orthosilicate (TEOS) which eliminated the need for the researcher specific and time consuming sol-gel reaction stages prior to membrane fabrication. As a result, modified-silica membranes containing cobalt oxides could be directly calcined at 600 C, instead of conventional thermal process which require slow ramping rates of ≤1 C min-1 to avoid cracking. As-prepared membranes delivered H2 permeances of 5 × 10-7 mol m-2 s-1 Pa-1 at 450 C and H2/N2 permselectivities of 54. The RTP techniques demonstrated in this work greatly reduced the production time and should both allow researchers to significantly increase their productivity and ultimately reduce the barriers for deployment of inorganic membranes into industrial applications. {\textcopyright} 2013, Hydrogen Energy Publications, LLC. Published by Elsevier Ltd. All rights.},
author = {Wang, David K. and Motuzas, Julius and {Diniz Da Costa}, Jo{\~{a}}o C. and Smart, Simon},
doi = {10.1016/j.ijhydene.2013.04.052},
file = {:Users/marc/Library/Application Support/Mendeley Desktop/Downloaded/Wang et al. - 2013 - Rapid thermal processing of tubular cobalt oxide silica membranes.pdf:pdf},
issn = {03603199},
journal = {International Journal of Hydrogen Energy},
keywords = {Hydrogen production,Rapid thermal processing,Silica membranes},
number = {18},
pages = {7394--7399},
title = {{Rapid thermal processing of tubular cobalt oxide silica membranes}},
volume = {38},
year = {2013}
}
@misc{LME2016,
author = {LME},
number = {06/12/2016},
title = {{LME Copper}},
url = {http://www.lme.com/en-gb/metals/non-ferrous/copper/},
volume = {2016},
year = {2016}
}
@article{Wang2009,
author = {Wang, T and Zhang, Y and Li, G and Li, H},
file = {:Users/marc/Library/Application Support/Mendeley Desktop/Downloaded/Wang et al. - 2009 - Preparation and characterization of alumina hollow fiber membranes.pdf:pdf},
journal = {Frontiers of Chemical Engineering in China},
keywords = {membranes,microfiltration,ultrathin alumina hollow fibers},
number = {3},
pages = {265--271},
title = {{Preparation and characterization of alumina hollow fiber membranes}},
url = {http://link.springer.com/article/10.1007/s11705-009-0010-2},
volume = {3},
year = {2009}
}
@article{Ruiz-Trejo2015,
abstract = {Silver- BaCe0.5Zr0.3Y0.16Zn0.04O3-$\delta$ (Ag/BCZYZ) composites were investigated due to their potential application as hydrogen separation membranes, with emphasis on their fabrication and characterization. A precursor powder of BCZYZ was prepared via a wet chemical route and characterized by XRD, SEM and dilatometry. The precursor powder was coated with silver using Tollens reaction and then sintered under a variety of conditions. It was possible to obtain dense samples with a low level of non-percolating silver (2 vol{\%}). Silver was present even if sintered at 1300 °C as it remained trapped in the ceramic matrix. The overall conductivity of a dense sample with 2 vol{\%} of silver increased when compared to pure BCZYZ, and in particular the grain boundary resistance decreased considerably. A measurement of the open circuit voltage in fuel cell mode indicates the presence of mixed electronic-protonic conductivity in the composite.},
author = {Ruiz-Trejo, Enrique and Zhou, Yuning and Brandon, Nigel P.},
doi = {10.1016/j.ijhydene.2015.01.146},
file = {:Users/marc/Library/Application Support/Mendeley Desktop/Downloaded/Ruiz-Trejo, Zhou, Brandon - 2015 - On the manufacture of silver-BaCe0.5Zr0.3Y0.16Zn0.04O3-$\delta$ composites for hydrogen separation membrane.pdf:pdf},
isbn = {0360-3199},
issn = {03603199},
journal = {International Journal of Hydrogen Energy},
keywords = {Cermet composites,Hydrogen separation,Mixed protonic electronic conduction,Proton conductors},
number = {11},
pages = {4146--4153},
title = {{On the manufacture of silver-BaCe0.5Zr0.3Y0.16Zn0.04O3-$\delta$ composites for hydrogen separation membranes}},
volume = {40},
year = {2015}
}
@article{Rezac1999,
abstract = {Blends of an acetylene-terminated monomer (ATM) with a commercially available poly(etherimide) (PEI, Ultem(TM)) were prepared, crosslinked, and characterized. Varying degrees of crosslinking were achieved through thermal treatment at 150-270°C. Incorporation of the uncrosslinked additive into the PEI resulted in reductions in the glass transition temperature, gas permeabilities and selectivities, and thermal stability. These behavior are consistent with antiplasticization of the polymer host by the ATM additive. Crosslinking of the actylene-terminated additive led to increases in thermal and chemical stability and improved gas selectivities as compared to the uncrosslinked blend. Gas transport properties are reported as a function of temperature. For the blend composition considered (9wt{\%} ATM in PEI), the fully crosslinked blend had transport properties which were essentially equivalent to the virgin PEI. Further, processing of the blend could be achieved in the same manner as for the virgin PEI. The resistance of the crosslinked blend to chemical dissolution or swelling was markedly improved as compared to PEI. Copyright (C) 1999 Elsevier Science B.V.},
author = {Rezac, Mary E. and Sch{\"{o}}berl, Birgit},
doi = {10.1016/S0376-7388(98)00346-9},
isbn = {0376-7388},
issn = {03767388},
journal = {Journal of Membrane Science},
keywords = {Crosslinking,Gas and vapor permeation,Polyimide},
number = {2},
pages = {211--222},
title = {{Transport and thermal properties of poly(ether imide)/acetylene-terminated monomer blends}},
volume = {156},
year = {1999}
}
@article{Catalano2010,
author = {Catalano, Jacopo and {Giacinti Baschetti}, Marco and Sarti, Giulio C},
doi = {10.1016/j.memsci.2010.06.055},
file = {:Users/marc/Library/Application Support/Mendeley Desktop/Downloaded/Catalano, Giacinti Baschetti, Sarti - 2010 - Hydrogen permeation in palladium-based membranes in the presence of carbon monoxide.pdf:pdf},
isbn = {03767388},
journal = {Journal of Membrane Science},
number = {1-2},
pages = {221--233},
title = {{Hydrogen permeation in palladium-based membranes in the presence of carbon monoxide}},
volume = {362},
year = {2010}
}
@article{Kreno2010,
author = {Kreno, L.E. and Hupp, J.T. and {Van Duyne}, R.P},
doi = {10.1021/ac102127p},
file = {:Users/marc/Library/Application Support/Mendeley Desktop/Downloaded/Kreno, Hupp, Van Duyne - 2010 - Metal− Organic Framework Thin Film for Enhanced Localized Surface Plasmon Resonance Gas Sensing.pdf:pdf},
isbn = {0003-2700},
issn = {0003-2700},
journal = {Analytical Chemistry},
number = {19},
pages = {8042--8046},
title = {{Metal− Organic Framework Thin Film for Enhanced Localized Surface Plasmon Resonance Gas Sensing}},
url = {http://scholar.google.com/scholar?q=Metal? Organic Framework Thin Film for Enhanced Localized Surface Plasmon Resonance Gas Sensing{\&}btnG={\&}hl=en{\&}num=20{\&}as{\_}sdt=0,22 VN  - readcube.com},
volume = {82},
year = {2010}
}
@article{Liguori2014,
abstract = {The present work is focused on the investigation of the performance and long-term stability of two composite palladium membranes under different operating conditions. One membrane (Pd/porous stainless steel (PSS)) is characterized by a {\~{}}10 microm-thick palladium layer on a porous stainless steel substrate, which is pretreated by means of surface modification and oxidation; the other membrane (Pd/Al2O3) is constituted by a {\~{}}7 microm-thick palladium layer on an asymmetric microporous Al2O3 substrate. The operating temperature and pressure ranges, used for studying the performance of these two kinds of membranes, are 350-450 degrees C and 200-800 kPa, respectively. The H2 permeances and the H2/N2 selectivities of both membranes were investigated and compared with literature data. At 400 degrees C and 200 kPa as pressure difference, Pd/PSS and Pd/Al2O3 membranes exhibited an H2/N2 ideal selectivity equal to 11700 and 6200, respectively, showing stability for 600 h. Thereafter, H2/N2 selectivity of both membranes progressively decreased and after around 2000 h, dropped dramatically to 55 and 310 for the Pd/PSS and Pd/Al2O3 membranes, respectively. As evidenced by Scanning Electron Microscope (SEM) analyses, the pinholes appear on the whole surface of the Pd/PSS membrane and this is probably due to release of sulphur from the graphite seal rings.},
annote = {Liguori, Simona
Iulianelli, Adolfo
Dalena, Francesco
Pinacci, Pietro
Drago, Francesca
Broglia, Maria
Huang, Yan
Basile, Angelo
ENG
Switzerland
2014/06/25 06:00
Membranes (Basel). 2014 Mar 6;4(1):143-62. doi: 10.3390/membranes4010143.},
author = {Liguori, S and Iulianelli, A and Dalena, F and Pinacci, P and Drago, F and Broglia, M and Huang, Y and Basile, A},
doi = {10.3390/membranes4010143},
file = {:Users/marc/Library/Application Support/Mendeley Desktop/Downloaded/Liguori et al. - 2014 - Performance and Long-Term Stability of PdPSS and PdAl2O3 Membranes for Hydrogen Separation.pdf:pdf},
isbn = {2077-0375 (Linking)},
journal = {Membranes (Basel)},
number = {1},
pages = {143--162},
pmid = {24957126},
title = {{Performance and Long-Term Stability of Pd/PSS and Pd/Al2O3 Membranes for Hydrogen Separation}},
url = {http://www.ncbi.nlm.nih.gov/pubmed/24957126},
volume = {4},
year = {2014}
}
@article{D.T.Hughes1978,
author = {{D.T. Hughes}, I R Harris},
chapter = {9},
file = {:Users/marc/Library/Application Support/Mendeley Desktop/Downloaded/D.T. Hughes - 1978 - A comparative study of hydrogen permeabilities and solubilities in some palladium solid solution alloys.pdf:pdf},
journal = {Journal of Less Common Metals},
pages = {9--21},
title = {{A comparative study of hydrogen permeabilities and solubilities in some palladium solid solution alloys}},
volume = {61},
year = {1978}
}
@article{Hayashi2005,
abstract = {Thermal expansion coefficients of yttria stabilized zirconia(YSZ) with the Y 2O 3 content of 3, 6, 8 and 10 mol{\%} were measured using a push-rod type dilatometer in the temperature range from 103 to 876 K. The thermal expansion coefficient of YSZ decreased with the increase of the Y 2O 3 content. The thermal expansion coefficient of YSZ was theoretically estimated for the various Y 2O 3 contents using the values of isochoric heat capacity, molar volume, isothermal bulk modulus and the Gruneisen constant in a good agreement with the experimental results. Among the physical properties, the bulk modulus is considered to be mainly responsible for decreasing the thermal expansion coefficient of YSZ with the increase of Y 2O 3 content. The doping effect of Y 2O 3 in YSZ on the thermal expansion coefficient was discussed using a molecular dynamics simulation. ?? 2004 Elsevier B.V. All rights reserved.},
author = {Hayashi, Hideko and Saitou, Tetsuya and Maruyama, Naotaka and Inaba, Hideaki and Kawamura, Katsuyuki and Mori, Masashi},
doi = {10.1016/j.ssi.2004.08.021},
file = {:Users/marc/Library/Application Support/Mendeley Desktop/Downloaded/Hayashi et al. - 2005 - Thermal expansion coefficient of yttria stabilized zirconia for various yttria contents.pdf:pdf},
isbn = {0167-2738},
issn = {01672738},
journal = {Solid State Ionics},
keywords = {Doping effect,Molecular dynamics simulation,Thermal expansion coefficient,YSZ,Yttria stabilized zirconia},
number = {5-6},
pages = {613--619},
title = {{Thermal expansion coefficient of yttria stabilized zirconia for various yttria contents}},
volume = {176},
year = {2005}
}

@article{Bryden1997,
author = {Bryden, Kenneth J and Ying, Jackie Y},
file = {:Users/marc/Library/Application Support/Mendeley Desktop/Downloaded/Bryden, Ying - 1997 - Electrodeposition sysnthesis and hydrogen absorption properties of nanostructured palladium-iron alloys.pdf:pdf},
journal = {Nanostructure},
pages = {485--488},
title = {{Electrodeposition sysnthesis and hydrogen absorption properties of nanostructured palladium-iron alloys}},
volume = {9},
year = {1997}
}
@article{Zhu2017,
abstract = {From the early of 1990s, intensive research has been started to seek a possible more efficient gas separation using inorganic membranes with more hopes on zeolite membrane. Although prohibitive difficulties have been encountered, our enthusiasm or imagination has never been quenched out, particular with the modern era of “magic” material-graphene. In this work, we explored the possibility of using supported graphene oxide (GO) membrane for gas separation. For this purpose, the porous YSZ hollow fiber ceramic support and GO flakes with size up to 5 µm were separately prepared; subsequently, vacuum-suction impregnation was applied to assemble the GO laminates on the fiber external surface with the thickness of 230 nm. The 2D nano-channels formed by the two neighboring one-atom-thick GO layers endow the membrane with molecular-sieving function. Gas permeation behavior was investigated by the measurement of gas permeances from the single gas components and the gas mixture (H2/N2). The ideal selectivity of H2/N2at 20 oC was up to 76, mirroring the molecular sieving function via molecular size limited diffusion and preferable adsorption. Compared to the ideal selectivity, the H2/N2separation factor is lower due to the blocking effect of the other components. At higher temperature, N2permeance was increased by a percentage more than that of H2thus decreasing the separation factor from 68 (20 oC) to 37 (100 oC). The H2separation test for 240 h highlights stable performance in maintaining the separation factor and the permeance. H2permeation using GO/YSZ hollow fiber was also theoretically probed. Simulation implies that in achieving the overall H2permeance, the front part (near the inlet) of the hollow fiber makes much more contribution than the rear part. To more effectively use the GO hollow fiber, the feed gas pressure can be increased or the vacuum pressure can be applied in the permeate side.},
author = {Zhu, Jingchang and Meng, Xiuxia and Zhao, Jinping and Jin, Yun and Yang, Naitao and Zhang, Shuguang and Sunarso, Jaka and Liu, Shaomin},
doi = {10.1016/j.memsci.2017.04.032},
file = {:Users/marc/Library/Application Support/Mendeley Desktop/Downloaded/Zhu et al. - 2017 - Facile hydrogennitrogen separation through graphene oxide membranes supported on YSZ ceramic hollow fibers.pdf:pdf},
isbn = {03767388},
issn = {18733123},
journal = {Journal of Membrane Science},
keywords = {Gas separation,Graphene oxide membrane,Hollow fiber,Inorganic membrane,Stability},
number = {January},
pages = {143--150},
publisher = {Elsevier B.V.},
title = {{Facile hydrogen/nitrogen separation through graphene oxide membranes supported on YSZ ceramic hollow fibers}},
url = {http://dx.doi.org/10.1016/j.memsci.2017.04.032},
volume = {535},
year = {2017}
}
@article{Li2015a,
author = {Li, Haixia and Song, Jian and Tan, Xiaoyao},
doi = {10.4172/2155-9589.1000136},
file = {:Users/marc/Library/Application Support/Mendeley Desktop/Downloaded/Li, Song, Tan - 2015 - Sintering of the Immersion-Induced Porous Stainless Steel Hollow Fiber Membranes.pdf:pdf},
journal = {J Membra Sci Technol},
keywords = {hollow fiber,inversion,phase,porous stainless steel membrane,sintering},
number = {2},
title = {{Sintering of the Immersion-Induced Porous Stainless Steel Hollow Fiber Membranes}},
volume = {5},
year = {2015}
}
@article{Huang2012c,
abstract = {Sandwich-structured composite zeolite membranes with enhanced hydrogen selectivity were prepared on porous $\alpha$-Al(2)O(3) supports by using 3-aminopropyltriethoxysilane as an interlayer.},
author = {Huang, Aisheng and Wang, Nanyi and Caro, J{\"{u}}rgen},
doi = {10.1039/c2cc17248h},
file = {:Users/marc/Library/Application Support/Mendeley Desktop/Downloaded/Huang, Wang, Caro - 2012 - Stepwise synthesis of sandwich-structured composite zeolite membranes with enhanced separation selectivity.pdf:pdf},
isbn = {1364-548X (Electronic)$\backslash$r1359-7345 (Linking)},
issn = {13597345},
journal = {Chemical Communications},
number = {29},
pages = {3542--3544},
pmid = {22378243},
title = {{Stepwise synthesis of sandwich-structured composite zeolite membranes with enhanced separation selectivity}},
volume = {48},
year = {2012}
}
@article{NguyenVANHIEUandJ.H.CRAIG1985,
author = {{Nguyen VAN HIEU and J.H. CRAIG}, Jr.},
file = {:Users/marc/Library/Application Support/Mendeley Desktop/Downloaded/Nguyen VAN HIEU and J.H. CRAIG - 1985 - EFFECT OF CO ON HYDROGEN ADSORPTION ON PALLADIUM.pdf:pdf},
journal = {SURFACE SCIENCE LETTERS},
pages = {483--487},
title = {{EFFECT OF CO ON HYDROGEN ADSORPTION ON PALLADIUM}},
volume = {160},
year = {1985}
}
@article{Erm??k2001,
abstract = {Hydrogen permeability through Ni3Al-M intermetallic membranes was studied by gas-permeation method in the temperature range 573-1223 K. The influence of the third element (M = Cr, Fe and Zr) upon the retardation of hydrogen transfer from the gas phase into the bulk was qualified by dependence of measured permeation flux J through the sample on sample thickness d. It was found that Zr supports and, on the contrary, Cr eliminates the inhibiting effect of the surface. The most expressive effect of Fe additions consists in a considerably raised grain boundary brittleness in a hydrogen environment. ?? 2001 Elsevier Science Ltd.},
author = {Erm??k, J. and Rothov??, V.},
doi = {10.1016/S0966-9795(01)00016-4},
file = {:Users/marc/Library/Application Support/Mendeley Desktop/Downloaded/Ermk, Rothov - 2001 - Surface barrier for hydrogen permeability in Ni3Al - Influence of Cr, Fe and Zr.pdf:pdf},
issn = {09669795},
journal = {Intermetallics},
keywords = {A. Intermetallics miscellaneuos,A. Nickel aluminides, based on Ni3Al,B. Hydrogen embrittlement},
number = {5},
pages = {403--408},
title = {{Surface barrier for hydrogen permeability in Ni3Al - Influence of Cr, Fe and Zr}},
volume = {9},
year = {2001}
}
@article{Li2010a,
author = {Li, Yanshuo and Liang, Fangyi and Bux, Helge and Yang, Weishen and Caro, J{\"{u}}rgen},
doi = {10.1016/j.memsci.2010.02.074},
file = {:Users/marc/Library/Application Support/Mendeley Desktop/Downloaded/Li et al. - 2010 - Zeolitic imidazolate framework ZIF-7 based molecular sieve membrane for hydrogen separation.pdf:pdf},
isbn = {03767388},
journal = {Journal of Membrane Science},
number = {1-2},
pages = {48--54},
title = {{Zeolitic imidazolate framework ZIF-7 based molecular sieve membrane for hydrogen separation}},
volume = {354},
year = {2010}
}
@article{Pham2013,
abstract = {Zeolite membranes: A promising method is reported for the fabrication of oriented silica MFI zeolite films (see picture; TPAOH=tetrapropylammonium hydroxide). The films synthesized by using this method exhibit an outstanding performance for the separation of p- and o-xylene.},
author = {Pham, Tung Cao Thanh and Nguyen, Thanh Huu and Yoon, Kyung Byung},
doi = {10.1002/anie.201301766},
file = {:Users/marc/Library/Application Support/Mendeley Desktop/Downloaded/Pham, Nguyen, Yoon - 2013 - Gel-free secondary growth of uniformly oriented silica MFI zeolite films and application for xylene separati.pdf:pdf},
isbn = {1433-7851},
issn = {14337851},
journal = {Angewandte Chemie - International Edition},
keywords = {membranes,mesoporous materials,silica,thin films},
number = {33},
pages = {8693--8698},
pmid = {23832590},
title = {{Gel-free secondary growth of uniformly oriented silica MFI zeolite films and application for xylene separation}},
volume = {52},
year = {2013}
}
@article{Adams2011,
abstract = {We are facing accelerated global warming due to the accumulation of greenhouse gases. A hydrogen-based economy is one potential approach toward maintaining our standard of living while lowering carbon dioxide emissions. Palladium is a unique material with a strong affinity to hydrogen owing to both its catalytic and hydrogen absorbing properties. Palladium has the potential to play a major role in virtually every aspect of the envisioned hydrogen economy, including hydrogen purification, storage, detection, and fuel cells. Major aspects of current research and potential applications of palladium-based nanomaterials in various hydrogen technologies are presented in this review. ?? 2011 Elsevier Ltd.},
author = {Adams, Brian D. and Chen, Aicheng},
doi = {10.1016/S1369-7021(11)70143-2},
file = {:Users/marc/Library/Application Support/Mendeley Desktop/Downloaded/Adams, Chen - 2011 - The role of palladium in a hydrogen economy.pdf:pdf},
isbn = {1369-7021},
issn = {13697021},
journal = {Materials Today},
number = {6},
pages = {282--289},
publisher = {Elsevier Ltd},
title = {{The role of palladium in a hydrogen economy}},
url = {http://dx.doi.org/10.1016/S1369-7021(11)70143-2},
volume = {14},
year = {2011}
}
@article{V.JayaramanM.PakalaR.Y.Lin1995,
author = {{V. Jayaraman   M. Pakala, R.Y. Lin}, Y S Lin},
file = {:Users/marc/Library/Application Support/Mendeley Desktop/Downloaded/V. Jayaraman M. Pakala, R.Y. Lin - 1995 - Fabrication of ultrathin metallic membranes on ceramic supports by sputter deposition.pdf:pdf},
journal = {Journal of Membrane Science},
pages = {89--100},
title = {{Fabrication of ultrathin metallic membranes on ceramic supports by sputter deposition}},
volume = {99},
year = {1995}
}
@article{Weinkauf1992,
author = {Weinkauf, D.H. and Paul, D.R.},
journal = {J. Polym. Sci.: Part B: Polym. Phys.},
pages = {837},
title = {{Gas transport properties of thermotropic liquid-crystalline copolyesters. II. The effects of copolymer composition}},
volume = {30},
year = {1992}
}
@article{Qi2013,
abstract = {Doping very small amounts of Ru(II) into a flexible, ultramicroporous, fluorescent Zn(II) coordination polymer produced phosphorescent materials with very high and tunable oxygen quenching efficiency; and a simple color-changing ratiometric oxygen sensor has been constructed.},
author = {Qi, Xiao-Lin and Liu, Si-Yang and Lin, Rui-Biao and Liao, Pei-Qin and Ye, Jia-Wen and Lai, Zhihui and Guan, Yanyan and Cheng, Xiao-Ning and Zhang, Jie-Peng and Chen, Xiao-Ming},
doi = {10.1039/c3cc43461c},
file = {:Users/marc/Library/Application Support/Mendeley Desktop/Downloaded/Qi et al. - 2013 - Phosphorescence doping in a flexible ultramicroporous framework for high and tunable oxygen sensing efficiency.pdf:pdf},
isbn = {8620841122},
issn = {1359-7345},
journal = {Chemical Communications},
number = {61},
pages = {6864},
pmid = {23787446},
title = {{Phosphorescence doping in a flexible ultramicroporous framework for high and tunable oxygen sensing efficiency}},
url = {http://xlink.rsc.org/?DOI=c3cc43461c},
volume = {49},
year = {2013}
}
@article{Xu2016a,
author = {Xu, Nong and Kim, Sung Su and Li, Anwu and Grace, John R and {Jim Lim}, C and Boyd, Tony},
doi = {10.1016/j.powtec.2015.08.037},
file = {:Users/marc/Library/Application Support/Mendeley Desktop/Downloaded/Xu et al. - 2016 - Investigation of the influence of tar-containing syngas from biomass gasification on dense Pd and Pd–Ru membranes.pdf:pdf},
isbn = {00325910},
journal = {Powder Technology},
pages = {132--140},
title = {{Investigation of the influence of tar-containing syngas from biomass gasification on dense Pd and Pd–Ru membranes}},
volume = {290},
year = {2016}
}
@article{Ferrando-Soria2012,
abstract = {On the road to chemical sensors: A novel 2D oxamato-based manganese(II)-copper(II) mixed-metal-organic framework (M'MOF) exhibiting non-linear fluorescence and long-range magnetic ordering is reported. This new luminescent nanoporous magnet features highly selective solvent-and gas sorption-induced optical switching behavior, opening the door to potential application for sensing of small molecules.},
author = {Ferrando-Soria, Jes??s and Khajavi, Hossein and Serra-Crespo, Pablo and Gascon, Jorge and Kapteijn, Freek and Julve, Miguel and Lloret, Francesc and Pas??n, Jorge and Ruiz-P??rez, Catalina and Journaux, Yves and Pardo, Emilio},
doi = {10.1002/adma.201201846},
file = {:Users/marc/Library/Application Support/Mendeley Desktop/Downloaded/Ferrando-Soria et al. - 2012 - Highly selective chemical sensing in a luminescent nanoporous magnet.pdf:pdf},
isbn = {0935-9648},
issn = {09359648},
journal = {Advanced Materials},
keywords = {chemosensors,luminescent properties,metal-organic frameworks,methylviologen,molecule-based magnets},
number = {41},
pages = {5625--5629},
pmid = {22887721},
title = {{Highly selective chemical sensing in a luminescent nanoporous magnet}},
volume = {24},
year = {2012}
}
@misc{Checa2007,
abstract = {To understand the relative importance of biological versus physicochemical control over biomineralization, we have tested if the chemical composition of the medium (i.e., the Mg/Ca ratio) can change the mineralogy of mollusk shells. The shells of mollusks are made of calcite and/or aragonite, which are by far the most common CaCO3 polymorphs. Several species of bivalves with predominantly calcitic shells have been cultivated in artificial seawater with a Mg/Ca molar ratio within the range of 8.3-9.2, well above the present value for seawater (5.2). Four out of six species used (the scallop Chlamys varia, the oyster Ostrea edulis, the saddle oyster Anomia ephippium and the mussel Mytilus edulis) survived long enough to secrete significant amounts of calcium carbonate. The deposits (sometimes extensive) formed on the interior shell surfaces were predominantly aragonitic. Three individuals of C. varia also increased their length by adding new shell at the margin. Contrary to the internal shell deposits, these margins were high-Mg calcite. This implies that the marginal mantle is able to exert a more strict control on the secreted mineral phase than the mantle facing the internal shell surface. This is the first report on an in vivo experimentally forced switch in bivalve shell mineralogy, from calcite to aragonite due to a change in water chemistry.},
author = {Checa, Antonio G and Jim{\'{e}}nez-L{\'{o}}pez, Concepci{\'{o}}n and Rodr{\'{i}}guez-Navarro, Alejandro and Machado, Jorge P},
booktitle = {Marine Biology},
doi = {10.1007/s00227-006-0411-4},
number = {5},
title = {{Precipitation of aragonite by calcitic bivalves in Mg-enriched marine waters}},
volume = {150},
year = {2007}
}
@article{Gielens2006c,
author = {Gielens, F C and Knibbeler, R J J and Duysinx, P F J and Tong, H D and Vorstman, M A G and Keurentjes, J T F},
doi = {10.1016/j.memsci.2005.12.002},
file = {:Users/marc/Library/Application Support/Mendeley Desktop/Downloaded/Gielens et al. - 2006 - Influence of steam and carbon dioxide on the hydrogen flux through thin PdAg and Pd membranes.pdf:pdf},
isbn = {03767388},
journal = {Journal of Membrane Science},
number = {1-2},
pages = {176--185},
title = {{Influence of steam and carbon dioxide on the hydrogen flux through thin Pd/Ag and Pd membranes}},
volume = {279},
year = {2006}
}
@article{Unemoto2007b,
author = {Unemoto, A and Kaimai, A and Sato, K and Otake, T and Yashiro, K and Mizusaki, J and Kawada, T and Tsuneki, T and Shirasaki, Y and Yasuda, I},
doi = {10.1016/j.ijhydene.2007.04.030},
file = {:Users/marc/Library/Application Support/Mendeley Desktop/Downloaded/Unemoto et al. - 2007 - Surface reaction of hydrogen on a palladium alloy membrane under co-existence of H2OH2O, CO, CO2CO2 or CH4CH4☆.pdf:pdf},
isbn = {03603199},
journal = {International Journal of Hydrogen Energy},
number = {16},
pages = {4023--4029},
title = {{Surface reaction of hydrogen on a palladium alloy membrane under co-existence of H2OH2O, CO, CO2CO2 or CH4CH4☆}},
volume = {32},
year = {2007}
}
@article{Nishimura2002a,
author = {Nishimura, C and Komaki, M and Hwang, S and Amano, M},
file = {:Users/marc/Library/Application Support/Mendeley Desktop/Downloaded/Nishimura et al. - 2002 - V-Ni alloy membranes for hydrogen purification.pdf:pdf},
journal = {Journal of Alloys and Compounds},
pages = {902--906},
title = {{V-Ni alloy membranes for hydrogen purification}},
volume = {330-332},
year = {2002}
}
@article{Li1999b,
author = {Li, Anwu and Liang, Weiqiang and Hughes, Ronald},
file = {:Users/marc/Library/Application Support/Mendeley Desktop/Downloaded/Li, Liang, Hughes - 1999 - Fabrication of defect-free PdAl2O3 composite membranes for hydrogen separation.pdf:pdf},
journal = {Thin Solid Films},
pages = {106--112},
title = {{Fabrication of defect-free Pd/Al2O3 composite membranes for hydrogen separation}},
volume = {350},
year = {1999}
}
@article{Sakamoto1996b,
author = {Sakamoto, Y and Chen, F L and Kinari, Y and Sakamoto, F},
file = {:Users/marc/Library/Application Support/Mendeley Desktop/Downloaded/Sakamoto et al. - 1996 - Effect of Carbon Monoxide on Hydrogen Permeation in some Palladium Based Alloy Membranes.pdf:pdf},
journal = {International Journal of Hydrogen Energy},
number = {11/12},
pages = {1017--1024},
title = {{Effect of Carbon Monoxide on Hydrogen Permeation in some Palladium Based Alloy Membranes}},
volume = {210},
year = {1996}
}
@misc{Buxbaum2008b,
address = {United States},
author = {Buxbaum, Robert E},
editor = {Patent, United States},
file = {:Users/marc/Library/Application Support/Mendeley Desktop/Downloaded/Buxbaum - 2008 - Space Group CP2 Alloys For The Use And Separation Of Hydrogen.pdf:pdf},
title = {{Space Group CP2 Alloys For The Use And Separation Of Hydrogen}},
volume = {US 7,323,0},
year = {2008}
}
@incollection{Peters2015a,
author = {Peters, T A and Stange, M and Bredesen, R},
booktitle = {Palladium Membrane Technology for Hydrogen Production, Carbon Capture and Other Application},
doi = {10.1533/9781782422419.1.25},
editor = {Doukelis, A and Panopoulos, K and Koumanakos, A and Kakaras, E},
file = {:Users/marc/Library/Application Support/Mendeley Desktop/Downloaded/Peters, Stange, Bredesen - 2015 - Fabrication of palladium-based membranes by magnetron sputtering.pdf:pdf},
pages = {25--41},
publisher = {Woodhead Publishing},
title = {{Fabrication of palladium-based membranes by magnetron sputtering}},
year = {2015}
}
@article{Catalano2010b,
author = {Catalano, Jacopo and {Giacinti Baschetti}, Marco and Sarti, Giulio C},
doi = {10.1016/j.memsci.2010.06.055},
file = {:Users/marc/Library/Application Support/Mendeley Desktop/Downloaded/Catalano, Giacinti Baschetti, Sarti - 2010 - Hydrogen permeation in palladium-based membranes in the presence of carbon monoxide.pdf:pdf},
isbn = {03767388},
journal = {Journal of Membrane Science},
number = {1-2},
pages = {221--233},
title = {{Hydrogen permeation in palladium-based membranes in the presence of carbon monoxide}},
volume = {362},
year = {2010}
}
@article{Sun2006a,
author = {Sun, G B and Hidajat, K and Kawi, S},
doi = {10.1016/j.memsci.2006.07.015},
file = {:Users/marc/Library/Application Support/Mendeley Desktop/Downloaded/Sun, Hidajat, Kawi - 2006 - Ultra thin Pd membrane on $\alpha$-Al2O3 hollow fiber by electroless plating High permeance and selectivity.pdf:pdf},
isbn = {03767388},
journal = {Journal of Membrane Science},
number = {1-2},
pages = {110--119},
title = {{Ultra thin Pd membrane on $\alpha$-Al2O3 hollow fiber by electroless plating: High permeance and selectivity}},
volume = {284},
year = {2006}
}
@article{Li2000a,
author = {Li, A and Liang, W and Highes, R},
file = {:Users/marc/Library/Application Support/Mendeley Desktop/Downloaded/Li, Liang, Highes - 2000 - The effect of carbon monoxide and steam on the hydrogen permeability of a Pdstainless steel membrane.pdf:pdf},
journal = {Journal of Membrane Science},
pages = {135--141},
title = {{The effect of carbon monoxide and steam on the hydrogen permeability of a Pd/stainless steel membrane}},
volume = {165},
year = {2000}
}
@article{LI2000b,
author = {LI, Anwu and Liang, Weiqiang and Hughes, Ronald},
file = {:Users/marc/Library/Application Support/Mendeley Desktop/Downloaded/LI, Liang, Hughes - 2000 - Fabrication of dense palladium composite membranes for hydrogen separation.pdf:pdf},
journal = {Catalysis today},
pages = {45--51},
title = {{Fabrication of dense palladium composite membranes for hydrogen separation}},
volume = {56},
year = {2000}
}
@article{Gao2004a,
author = {Gao, Huiyuan and Lin, Y S and Li, Yongdan and Zhang, Baoquan},
file = {:Users/marc/Library/Application Support/Mendeley Desktop/Downloaded/Gao et al. - 2004 - Chemical Stability and Its Improvement of Palladium-Based Metallic Membranes.pdf:pdf},
journal = {Ind. Eng. Chem. Res.},
pages = {6920--6930},
title = {{Chemical Stability and It's Improvement of Palladium-Based Metallic Membranes}},
volume = {43},
year = {2004}
}
@article{Braun2014a,
author = {Braun, Fernando and Tarditi, Ana M and Miller, James B and Cornaglia, Laura M},
doi = {10.1016/j.memsci.2013.09.026},
file = {:Users/marc/Library/Application Support/Mendeley Desktop/Downloaded/Braun et al. - 2014 - Pd-based binary and ternary alloy membranes Morphological and perm-selective characterization in the presence of H.pdf:pdf},
isbn = {03767388},
journal = {Journal of Membrane Science},
pages = {299--307},
title = {{Pd-based binary and ternary alloy membranes: Morphological and perm-selective characterization in the presence of H2S}},
volume = {450},
year = {2014}
}
@article{Li2008b,
author = {Li, Hui and Xu, Hengyong and Li, Wenzhao},
doi = {10.1016/j.memsci.2008.06.053},
file = {:Users/marc/Library/Application Support/Mendeley Desktop/Downloaded/Li, Xu, Li - 2008 - Study of n value and $\alpha$$\beta$ palladium hydride phase transition within the ultra-thin palladium composite membrane.pdf:pdf},
isbn = {03767388},
journal = {Journal of Membrane Science},
number = {1-2},
pages = {44--49},
title = {{Study of n value and $\alpha$/$\beta$ palladium hydride phase transition within the ultra-thin palladium composite membrane}},
volume = {324},
year = {2008}
}
@article{Boon2015a,
author = {Boon, Jurriaan and Pieterse, J A Z and van Berkel, F P F and van Delft, Y C and {van Sint Annaland}, M},
doi = {10.1016/j.memsci.2015.08.061},
file = {:Users/marc/Library/Application Support/Mendeley Desktop/Downloaded/Boon et al. - 2015 - Hydrogen permeation through palladium membranes and inhibition by carbon monoxide, carbon dioxide, and steam.pdf:pdf},
isbn = {03767388},
journal = {Journal of Membrane Science},
pages = {344--358},
title = {{Hydrogen permeation through palladium membranes and inhibition by carbon monoxide, carbon dioxide, and steam}},
volume = {496},
year = {2015}
}
@article{Chen2008a,
author = {Chen, S C and Tu, G C and Hung, Caryat C Y and Huang, C A and Rei, M H},
doi = {10.1016/j.memsci.2007.12.066},
file = {:Users/marc/Library/Application Support/Mendeley Desktop/Downloaded/Chen et al. - 2008 - Preparation of palladium membrane by electroplating on AISI 316L porous stainless steel supports and its use for me.pdf:pdf},
isbn = {03767388},
journal = {Journal of Membrane Science},
number = {1-2},
pages = {5--14},
title = {{Preparation of palladium membrane by electroplating on AISI 316L porous stainless steel supports and its use for methanol steam reformer}},
volume = {314},
year = {2008}
}
@article{Conde2016a,
author = {Conde, Julio J and Maro{\~{n}}o, Marta and S{\'{a}}nchez-Herv{\'{a}}s, Jos{\'{e}} Mar{\'{i}}a},
doi = {10.1080/15422119.2016.1212379},
file = {:Users/marc/Library/Application Support/Mendeley Desktop/Downloaded/Conde, Maro{\~{n}}o, S{\'{a}}nchez-Herv{\'{a}}s - 2016 - Pd-Based Membranes for Hydrogen Separation Review of Alloying Elements and Their Influence on.pdf:pdf},
isbn = {1542-2119
1542-2127},
journal = {Separation {\&} Purification Reviews},
number = {2},
pages = {152--177},
title = {{Pd-Based Membranes for Hydrogen Separation: Review of Alloying Elements and Their Influence on Membrane Properties}},
volume = {46},
year = {2016}
}
@article{Bhatt2013a,
author = {Bhatt, R and Bhattacharya, S and Basu, R and Singh, A and Deshpande, U and Surger, C and Basu, S and Aswal, D K and Gupta, S K},
doi = {10.1016/j.tsf.2013.04.143},
file = {:Users/marc/Library/Application Support/Mendeley Desktop/Downloaded/Bhatt et al. - 2013 - Growth of Pd4S, PdS and PdS2 films by controlled sulfurization of sputtered Pd on native oxide of Si.pdf:pdf},
isbn = {00406090},
journal = {Thin Solid Films},
pages = {41--46},
title = {{Growth of Pd4S, PdS and PdS2 films by controlled sulfurization of sputtered Pd on native oxide of Si}},
volume = {539},
year = {2013}
}
@article{David2011a,
author = {David, E and Kopac, J},
doi = {10.1016/j.ijhydene.2010.12.032},
file = {:Users/marc/Library/Application Support/Mendeley Desktop/Downloaded/David, Kopac - 2011 - Devlopment of palladiumceramic membranes for hydrogen separation.pdf:pdf},
isbn = {03603199},
journal = {International Journal of Hydrogen Energy},
number = {7},
pages = {4498--4506},
title = {{Devlopment of palladium/ceramic membranes for hydrogen separation}},
volume = {36},
year = {2011}
}
@article{Okazaki2008a,
author = {Okazaki, Junya and Tanaka, David Alfredo Pacheco and Tanco, Margot Anabel Llosa and Wakui, Yoshito and Ikeda, Takuji and Mizukami, Fujio and Suzuki, Toshishige M},
doi = {10.2320/matertrans.MBW200720},
file = {:Users/marc/Library/Application Support/Mendeley Desktop/Downloaded/Okazaki et al. - 2008 - Preparation and Hydrogen Permeation Properties of Thin Pd-Au Alloy Membranes Supported on Porous $\alpha$-Alumina Tube.pdf:pdf},
isbn = {1347-5320
1345-9678},
journal = {Materials Transactions},
number = {3},
pages = {449--452},
title = {{Preparation and Hydrogen Permeation Properties of Thin Pd-Au Alloy Membranes Supported on Porous $\alpha$-Alumina Tube}},
volume = {49},
year = {2008}
}
@article{Gielens2002,
abstract = {In this study, hydrogen selective membranes have been fabricated using microsystem technology. A 750 nm dense layer of Pd (77 wt{\%}) and Ag (23 wt{\%}) is deposited on a non-porous 1 mm thick silicon nitride layer by cosputtering of a Pd and a Ag target. After sputtering, openings of 5 ??m are made in the silicon nitride layer to create a clear passage to the Pd/Ag surface. As a result of the production method, these membranes are pinhole free and have a low resistance to mass transfer in the gas phase, as virtually no support layer is present. The membranes have been tested in a gas permeation system to determine the hydrogen permeability as a function of temperature, gas flow rate, and feed composition. In addition, the hydrogen selectivity over helium has been determined, which appears to be above 1500. At 0.2 bar partial hydrogen pressure in the feed, the hydrogen permeability of the membranes has been found to range from 0.02 to 0.95 mol.H2/m2xs at 350 and 450??C, respectively. It is expected that by improving the hydrodynamics and increasing the operation temperature, substantially higher fluxes will be attainable.},
author = {Gielens, F. C. and Tong, H. D. and {Van Rijn}, C. J M and Vorstman, M. A G and Keurentjes, J. T F},
doi = {10.1016/S0011-9164(02)00637-9},
file = {::},
isbn = {0011-9164},
issn = {00119164},
journal = {Desalination},
keywords = {Hydrogen permeation,Micro-membrane,Micro-reactor,Microsieve,Palladium alloys},
number = {1-3},
pages = {417--423},
title = {{High-flux palladium-silver alloy membranes fabricated by microsystem technology}},
volume = {147},
year = {2002}
}
@article{Ryi2011a,
abstract = {This study presents a new non-alloy Ru/Pd composite membrane fabricated by electroless plating for hydrogen separation. It shows that palladium and ruthenium can be deposited on an aluminum-oxide-modified porous Hastalloy by using our new EDTA-free plating bath at room temperature and 358 K, respectively. A 6.8 ??m thick non-alloy Ru/Pd membrane film could be plated and helium leak test confirmed that the membrane was free of defects. Hydrogen permeation test showed that the membrane had a hydrogen permeation flux of 4.5 ?? 10-1 mol m-2 s-1 at a temperature of 773 K and a pressure difference of 100 kPa. The hydrogen permeability normalized value with thickness of the membrane was 1.4 times higher than our pure Pd membrane having similar structure. The EDX profiles of the front and back side membrane, cross-sectional EDX line scanning and XRD profile show that there was no alloying progress between the palladium and ruthenium layer after hydrogen permeation test at 773 K. ?? 2010, Hydrogen Energy Publications, LLC. Published by Elsevier Ltd. All rights reserved.},
author = {Ryi, Shin Kun and Li, Anwu and Lim, C. Jim and Grace, John R.},
doi = {10.1016/j.ijhydene.2010.06.014},
file = {:Users/marc/Library/Application Support/Mendeley Desktop/Downloaded/Ryi et al. - 2011 - Novel non-alloy RuPd composite membrane fabricated by electroless plating for hydrogen separation.pdf:pdf},
isbn = {0360-3199},
issn = {03603199},
journal = {International Journal of Hydrogen Energy},
keywords = {Electroless plating,Hydrogen,Membrane,Pd,Porous substrate,Ru},
number = {15},
pages = {9335--9340},
publisher = {Elsevier Ltd},
title = {{Novel non-alloy Ru/Pd composite membrane fabricated by electroless plating for hydrogen separation}},
url = {http://dx.doi.org/10.1016/j.ijhydene.2010.06.014},
volume = {36},
year = {2011}
}
@article{AbuElHawa2015a,
author = {{Abu El Hawa}, Hani W and Lundin, Sean-Thomas B and Paglieri, Stephen N and Harale, Aadesh and {Douglas Way}, J},
doi = {10.1016/j.memsci.2015.07.021},
file = {:Users/marc/Library/Application Support/Mendeley Desktop/Downloaded/Abu El Hawa et al. - 2015 - The influence of heat treatment on the thermal stability of Pd composite membranes.pdf:pdf},
isbn = {03767388},
journal = {Journal of Membrane Science},
pages = {113--120},
title = {{The influence of heat treatment on the thermal stability of Pd composite membranes}},
volume = {494},
year = {2015}
}
@article{Gielens2006b,
author = {Gielens, F C and Knibbeler, R J J and Duysinx, P F J and Tong, H D and Vorstman, M A G and Keurentjes, J T F},
doi = {10.1016/j.memsci.2005.12.002},
file = {:Users/marc/Library/Application Support/Mendeley Desktop/Downloaded/Gielens et al. - 2006 - Influence of steam and carbon dioxide on the hydrogen flux through thin PdAg and Pd membranes.pdf:pdf},
isbn = {03767388},
journal = {Journal of Membrane Science},
number = {1-2},
pages = {176--185},
title = {{Influence of steam and carbon dioxide on the hydrogen flux through thin Pd/Ag and Pd membranes}},
volume = {279},
year = {2006}
}
@article{Chen2005a,
author = {Chen, Huey‐Ing and Chu, Chin‐Yi and Huang, Ting‐Chia},
doi = {10.1081/ss-120030789},
file = {:Users/marc/Library/Application Support/Mendeley Desktop/Downloaded/Chen, Chu, Huang - 2005 - Comprehensive Characterization and Permeation Analysis of Thin PdAl2O3Composite Membranes Prepared by Suctio.pdf:pdf},
isbn = {0149-6395
1520-5754},
journal = {Separation Science and Technology},
number = {7},
pages = {1461--1483},
title = {{Comprehensive Characterization and Permeation Analysis of Thin Pd/Al2O3Composite Membranes Prepared by Suction‐Assisted Electroless Deposition}},
volume = {39},
year = {2005}
}
@article{Chen1996a,
author = {Chen, F L and Kinari, Y and Sakamoto, F and Nakayama, Y and Sakamoto, Y},
file = {:Users/marc/Library/Application Support/Mendeley Desktop/Downloaded/Chen et al. - 1996 - Hydrogen permeation through palladium-based alloy membranes in mixtures of 10{\%} Methane and Ethylene in the Hydrogen.pdf:pdf},
journal = {International Journal of Hydrogen Energy},
number = {7},
pages = {555--561},
title = {{Hydrogen permeation through palladium-based alloy membranes in mixtures of 10{\%} Methane and Ethylene in the Hydrogen}},
volume = {21},
year = {1996}
}
@article{Lewis2013a,
abstract = {Palladium-platinum (Pd-Pt) alloy membranes have been fabricated by sequential, electroless deposition onto porous yttria-stablized zirconia supports manufactured by Praxair, Inc. Membranes were synthesized with thicknesses of 4-12??m that contained up to 27wt{\%} Pt. Pd-Pt alloy membranes had lower pure-gas hydrogen flux compared with pure Pd membranes of equal thickness. However, when tested at 673K under an identical synthetic water-gas shift feed gas mixture composed of H2, H2O, CO2, and CO at 689.5kPa total pressure, the Pd-Pt alloy membranes had over 25{\%} higher hydrogen fluxes than a pure Pd membrane of similar thickness. Membrane films were analyzed after testing with SEM, EDS, and ICP-AES to corroborate membrane thickness and alloy compositions estimated by mass gain. SEM thickness estimates on membrane cross sections were similar to those estimated by mass gain. ICP-AES analysis was performed on two membranes and confirmed the composition estimated by gravimetric analysis. In five of six membranes the film composition estimated by gravimetric analysis was consistent with the surface composition estimated by EDS which indicated that the membranes had been adequately annealed. ?? 2013 Elsevier B.V.},
author = {Lewis, A. E. and Kershner, D. C. and Paglieri, S. N. and Slepicka, M. J. and Way, J. D.},
doi = {10.1016/j.memsci.2013.02.056},
file = {:Users/marc/Library/Application Support/Mendeley Desktop/Downloaded/Lewis et al. - 2013 - Pd-PtYSZ composite membranes for hydrogen separation from synthetic water-gas shift streams.pdf:pdf},
issn = {03767388},
journal = {Journal of Membrane Science},
keywords = {Electroless deposition,Fabricating palladium-platinum membranes,Hydrogen separation,Palladium-platinum alloy,Water-gas shift reaction},
pages = {257--264},
title = {{Pd-Pt/YSZ composite membranes for hydrogen separation from synthetic water-gas shift streams}},
volume = {437},
year = {2013}
}
@article{Liguori2014a,
abstract = {The present work is focused on the investigation of the performance and long-term stability of two composite palladium membranes under different operating conditions. One membrane (Pd/porous stainless steel (PSS)) is characterized by a {\~{}}10 microm-thick palladium layer on a porous stainless steel substrate, which is pretreated by means of surface modification and oxidation; the other membrane (Pd/Al2O3) is constituted by a {\~{}}7 microm-thick palladium layer on an asymmetric microporous Al2O3 substrate. The operating temperature and pressure ranges, used for studying the performance of these two kinds of membranes, are 350-450 degrees C and 200-800 kPa, respectively. The H2 permeances and the H2/N2 selectivities of both membranes were investigated and compared with literature data. At 400 degrees C and 200 kPa as pressure difference, Pd/PSS and Pd/Al2O3 membranes exhibited an H2/N2 ideal selectivity equal to 11700 and 6200, respectively, showing stability for 600 h. Thereafter, H2/N2 selectivity of both membranes progressively decreased and after around 2000 h, dropped dramatically to 55 and 310 for the Pd/PSS and Pd/Al2O3 membranes, respectively. As evidenced by Scanning Electron Microscope (SEM) analyses, the pinholes appear on the whole surface of the Pd/PSS membrane and this is probably due to release of sulphur from the graphite seal rings.},
annote = {Liguori, Simona
Iulianelli, Adolfo
Dalena, Francesco
Pinacci, Pietro
Drago, Francesca
Broglia, Maria
Huang, Yan
Basile, Angelo
ENG
Switzerland
2014/06/25 06:00
Membranes (Basel). 2014 Mar 6;4(1):143-62. doi: 10.3390/membranes4010143.},
author = {Liguori, S and Iulianelli, A and Dalena, F and Pinacci, P and Drago, F and Broglia, M and Huang, Y and Basile, A},
doi = {10.3390/membranes4010143},
file = {:Users/marc/Library/Application Support/Mendeley Desktop/Downloaded/Liguori et al. - 2014 - Performance and Long-Term Stability of PdPSS and PdAl2O3 Membranes for Hydrogen Separation.pdf:pdf},
isbn = {2077-0375 (Linking)},
journal = {Membranes (Basel)},
number = {1},
pages = {143--162},
pmid = {24957126},
title = {{Performance and Long-Term Stability of Pd/PSS and Pd/Al2O3 Membranes for Hydrogen Separation}},
url = {http://www.ncbi.nlm.nih.gov/pubmed/24957126},
volume = {4},
year = {2014}
}
@article{Gallucci2007a,
author = {Gallucci, F and Chiaravalloti, F and Tosti, S and Drioli, E and Basile, A},
doi = {10.1016/j.ijhydene.2006.09.034},
file = {:Users/marc/Library/Application Support/Mendeley Desktop/Downloaded/Gallucci et al. - 2007 - The effect of mixture gas on hydrogen permeation through a palladium membrane Experimental study and theoretica.pdf:pdf},
isbn = {03603199},
journal = {International Journal of Hydrogen Energy},
number = {12},
pages = {1837--1845},
title = {{The effect of mixture gas on hydrogen permeation through a palladium membrane: Experimental study and theoretical approach}},
volume = {32},
year = {2007}
}
@article{Sakamoto1996a,
author = {Sakamoto, Y and Chen, F L and Kinari, Y and Sakamoto, F},
file = {:Users/marc/Library/Application Support/Mendeley Desktop/Downloaded/Sakamoto et al. - 1996 - Effect of Carbon Monoxide on Hydrogen Permeation in some Palladium Based Alloy Membranes.pdf:pdf},
journal = {International Journal of Hydrogen Energy},
number = {11/12},
pages = {1017--1024},
title = {{Effect of Carbon Monoxide on Hydrogen Permeation in some Palladium Based Alloy Membranes}},
volume = {210},
year = {1996}
}
@article{Tereschenko2007a,
author = {Tereschenko, G and Ermilova, M and Mordovin, V and Orekhova, N and Gryaznov, V and Iulianelli, A and Gallucci, F and Basile, A},
doi = {10.1016/j.ijhydene.2007.03.044},
file = {:Users/marc/Library/Application Support/Mendeley Desktop/Downloaded/Tereschenko et al. - 2007 - New Ti–Ni dense membranes with low palladium content.pdf:pdf},
isbn = {03603199},
journal = {International Journal of Hydrogen Energy},
number = {16},
pages = {4016--4022},
title = {{New Ti–Ni dense membranes with low palladium content}},
volume = {32},
year = {2007}
}
@article{JamesB.MillerBretH.HowardCaseyP.OBrienBryanD.Morreale2009a,
author = {{James B. Miller  Bret H. Howard, Casey P. O'Brien, Bryan D. Morreale}, Dominic R Alfonso},
chapter = {18800},
file = {:Users/marc/Library/Application Support/Mendeley Desktop/Downloaded/James B. Miller Bret H. Howard, Casey P. O'Brien, Bryan D. Morreale - 2009 - Hydrogen Dissociation on PdS4 Surfaces.pdf:pdf},
journal = {The Journal of Physical Chemistry C},
pages = {18800--18806},
title = {{Hydrogen Dissociation on PdS4 Surfaces}},
volume = {113},
year = {2009}
}
@article{Kato1979a,
author = {Kato, M.},
file = {:Users/marc/Library/Application Support/Mendeley Desktop/Downloaded/Kato - 1979 - Electroless Gold Plating Bath Using Ascorbic Acid as Reducing Agent - Recent Improvements.pdf:pdf},
issn = {0140-6736 (Print)},
journal = {Lancet (London, England)},
keywords = {Ascorbic Acid,Biological Availability,Common Cold,Dose-Response Relationship,Drug,Humans,administration {\&} dosage,metabolism,prevention {\&} control,therapeutic use},
number = {8116},
pages = {615},
pmid = {85208},
title = {{Electroless Gold Plating Bath Using Ascorbic Acid as Reducing Agent - Recent Improvements}},
volume = {1},
year = {1979}
}
@article{Wang2004b,
author = {Wang, D and Flanagan, Ted B and Shanahan, Kirk L},
doi = {10.1016/j.jallcom.2003.09.150},
file = {:Users/marc/Library/Application Support/Mendeley Desktop/Downloaded/Wang, Flanagan, Shanahan - 2004 - Permeation of hydrogen through pre-oxidized Pd membranes in the presence and absence of CO.pdf:pdf},
isbn = {09258388},
journal = {Journal of Alloys and Compounds},
number = {1-2},
pages = {158--164},
title = {{Permeation of hydrogen through pre-oxidized Pd membranes in the presence and absence of CO}},
volume = {372},
year = {2004}
}
@article{Catalano2010a,
author = {Catalano, Jacopo and {Giacinti Baschetti}, Marco and Sarti, Giulio C},
doi = {10.1016/j.memsci.2010.06.055},
file = {:Users/marc/Library/Application Support/Mendeley Desktop/Downloaded/Catalano, Giacinti Baschetti, Sarti - 2010 - Hydrogen permeation in palladium-based membranes in the presence of carbon monoxide.pdf:pdf},
isbn = {03767388},
journal = {Journal of Membrane Science},
number = {1-2},
pages = {221--233},
title = {{Hydrogen permeation in palladium-based membranes in the presence of carbon monoxide}},
volume = {362},
year = {2010}
}
@article{Guerreiro2016a,
author = {Guerreiro, Bruno Honrado and Martin, Manuel H and Rou{\'{e}}, Lionel and Guay, Daniel},
doi = {10.1016/j.memsci.2016.02.040},
file = {:Users/marc/Library/Application Support/Mendeley Desktop/Downloaded/Guerreiro et al. - 2016 - Hydrogen permeability of PdCuAu membranes prepared from mechanically-alloyed powders.pdf:pdf},
isbn = {03767388},
journal = {Journal of Membrane Science},
pages = {68--82},
title = {{Hydrogen permeability of PdCuAu membranes prepared from mechanically-alloyed powders}},
volume = {509},
year = {2016}
}
@article{Hara2000a,
author = {Hara, S and Sakaki, K and Itoh, N and Kimura, H.-M. and Asami, K and Inoue, A},
file = {:Users/marc/Library/Application Support/Mendeley Desktop/Downloaded/Hara et al. - 2000 - An amorphous alloy membrane without noble metals for gaseous hydrogen separation.pdf:pdf},
journal = {Journal of Membrane Science},
pages = {289--294},
title = {{An amorphous alloy membrane without noble metals for gaseous hydrogen separation}},
volume = {164},
year = {2000}
}
@article{Kozhakhmetov2015a,
author = {Kozhakhmetov, S and Sidorov, N and Piven, V and Sipatov, I and Gabis, I and Arinov, B},
doi = {10.1016/j.jallcom.2015.01.242},
file = {:Users/marc/Library/Application Support/Mendeley Desktop/Downloaded/Kozhakhmetov et al. - 2015 - Alloys based on Group 5 metals for hydrogen purification membranes.pdf:pdf},
isbn = {09258388},
journal = {Journal of Alloys and Compounds},
pages = {S36--S40},
title = {{Alloys based on Group 5 metals for hydrogen purification membranes}},
volume = {645},
year = {2015}
}
@article{Xu2014a,
author = {Xu, Nong and Kim, Sung Su and Li, Anwu and Grace, John R. and Lim, C. Jim and Boyd, Tony},
doi = {10.1002/cjce.21954},
file = {:Users/marc/Library/Application Support/Mendeley Desktop/Downloaded/Xu et al. - 2014 - Preparation and characterization of palladium-ruthenium composite membrane on alumina-modified PSS substrate.pdf:pdf},
issn = {1939019X},
journal = {Canadian Journal of Chemical Engineering},
keywords = {Composite membrane,Electroless plating,Hydrogen separation,Palladium (Pd),Ruthenium (Ru)},
number = {6},
pages = {1041--1047},
title = {{Preparation and characterization of palladium-ruthenium composite membrane on alumina-modified PSS substrate}},
volume = {92},
year = {2014}
}
@article{Musket1976a,
author = {Musket, R G},
file = {:Users/marc/Library/Application Support/Mendeley Desktop/Downloaded/Musket - 1976 - Effects of contamination on the interaction of hydrogen gas with palladium A Review.pdf:pdf},
journal = {Journal of Less Common Metals},
pages = {173--183},
title = {{Effects of contamination on the interaction of hydrogen gas with palladium: A Review}},
volume = {45},
year = {1976}
}
@article{Kitiwan2010c,
author = {Kitiwan, Mettaya and Atong, Duangduen},
file = {:Users/marc/Library/Application Support/Mendeley Desktop/Downloaded/Kitiwan, Atong - 2010 - Effects of Porous Alumina Support and Plating Time on Electroless Plating of Palladium Membrane.pdf:pdf},
journal = {J. Mater. Sci. Technol},
number = {12},
pages = {1148--1152},
title = {{Effects of Porous Alumina Support and Plating Time on Electroless Plating of Palladium Membrane}},
volume = {26},
year = {2010}
}
@article{NathanW.Ockwig2007a,
author = {{Nathan W. Ockwig}, Tina M Nenoff},
journal = {Chemical Reviews},
pages = {4078--4110},
title = {{Membranes for Hydrogen Separation}},
volume = {107},
year = {2007}
}
@article{Mejdell2008a,
author = {Mejdell, A L and Klette, H and Ramachandran, A and Borg, A and Bredesen, R},
doi = {10.1016/j.memsci.2007.09.024},
file = {:Users/marc/Library/Application Support/Mendeley Desktop/Downloaded/Mejdell et al. - 2008 - Hydrogen permeation of thin, free-standing PdAg23{\%} membranes before and after heat treatment in air.pdf:pdf},
isbn = {03767388},
journal = {Journal of Membrane Science},
number = {1},
pages = {96--104},
title = {{Hydrogen permeation of thin, free-standing Pd/Ag23{\%} membranes before and after heat treatment in air}},
volume = {307},
year = {2008}
}
@article{Didenko2016a,
author = {Didenko, L P and Savchenko, V I and Sementsova, L A and Bikov, L A},
doi = {10.1016/j.ijhydene.2015.10.107},
file = {:Users/marc/Library/Application Support/Mendeley Desktop/Downloaded/Didenko et al. - 2016 - Hydrogen flux through the membrane based on the Pd–In–Ru foil.pdf:pdf},
isbn = {03603199},
journal = {International Journal of Hydrogen Energy},
number = {1},
pages = {307--315},
title = {{Hydrogen flux through the membrane based on the Pd–In–Ru foil}},
volume = {41},
year = {2016}
}
@article{Dunbar2012a,
author = {Dunbar, Zachary W and Chu, Deryn},
doi = {10.1016/j.jpowsour.2012.05.044},
file = {:Users/marc/Library/Application Support/Mendeley Desktop/Downloaded/Dunbar, Chu - 2012 - Thin palladium membranes supported on microstructured nickel for purification of reformate gases.pdf:pdf},
isbn = {03787753},
journal = {Journal of Power Sources},
pages = {47--53},
title = {{Thin palladium membranes supported on microstructured nickel for purification of reformate gases}},
volume = {217},
year = {2012}
}
@article{Seund-EunNam2001a,
author = {{Seund-Eun Nam}, Kew-Ho Lee},
chapter = {177},
file = {:Users/marc/Library/Application Support/Mendeley Desktop/Downloaded/Seund-Eun Nam - 2001 - Hydrogen separation by Pd alloy composite membranes introduction of diffusion barrier.pdf:pdf},
journal = {Journal of Membrane Science},
pages = {177--185},
title = {{Hydrogen separation by Pd alloy composite membranes: introduction of diffusio barrier}},
volume = {192},
year = {2001}
}
@article{FernandoRoaJ.DouglasWay2002a,
author = {{Fernando Roa  J. Douglas Way}, Michael J Block},
chapter = {411},
file = {:Users/marc/Library/Application Support/Mendeley Desktop/Downloaded/Fernando Roa J. Douglas Way - 2002 - The infleuance of alloy composition on the H2 flux of composite Pd-Cu membranes.pdf:pdf},
journal = {Desalination},
pages = {411--416},
title = {{The infleuance of alloy composition on the H2 flux of composite Pd-Cu membranes}},
volume = {147},
year = {2002}
}
@article{Peters2016b,
author = {Peters, T A and Polfus, J M and Stange, M and Veenstra, P and Nijmeijer, A and Bredesen, R},
doi = {10.1016/j.fuproc.2016.06.012},
file = {:Users/marc/Library/Application Support/Mendeley Desktop/Downloaded/Peters et al. - 2016 - H2 flux inhibition and stability of Pd-Ag membranes under exposure to trace amounts of NH3.pdf:pdf},
isbn = {03783820},
journal = {Fuel Processing Technology},
pages = {259--265},
title = {{H2 flux inhibition and stability of Pd-Ag membranes under exposure to trace amounts of NH3}},
volume = {152},
year = {2016}
}
@article{Sun2014,
abstract = {We present an investigation of molecular permeation of gases through nanoporous graphene membranes via molecular dynamics simulations; four different gases are investigated, namely helium, hydrogen, nitrogen, and methane. We show that in addition to the direct (gas-kinetic) flux of molecules crossing from the bulk phase on one side of the graphene to the bulk phase on the other side, for gases that adsorb onto the graphene, significant contribution to the flux across the membrane comes from a surface mechanism by which molecules cross after being adsorbed onto the graphene surface. Our results quantify the relative contribution of the bulk and surface mechanisms and show that the direct flux can be described reasonably accurately using kinetic theory, provided the latter is appropriately modified assuming steric molecule-pore interactions, with gas molecules behaving as hard spheres of known kinetic diameters. The surface flux is negligible for gases that do not adsorb onto graphene (e.g., He and H2), while for gases that adsorb (e.g., CH4 and N2) it can be on the order of the direct flux or larger. Our results identify a nanopore geometry that is permeable to hydrogen and helium, is significantly less permeable to nitrogen, and is essentially impermeable to methane, thus validating previous suggestions that nanoporous graphene membranes can be used for gas separation. We also show that molecular permeation is strongly affected by pore functionalization; this observation may be sufficient to explain the large discrepancy between simulated and experimentally measured transport rates through nanoporous graphene membranes.},
author = {Sun, Chengzhen and Boutilier, Michael S H and Au, Harold and Poesio, Pietro and Bai, Bofeng and Karnik, Rohit and Hadjiconstantinou, Nicolas G.},
doi = {10.1021/la403969g},
file = {:Users/marc/Library/Application Support/Mendeley Desktop/Downloaded/Sun et al. - 2014 - Mechanisms of molecular permeation through nanoporous graphene membranes.pdf:pdf},
isbn = {0743-7463},
issn = {07437463},
journal = {Langmuir},
number = {2},
pages = {675--682},
pmid = {24364726},
title = {{Mechanisms of molecular permeation through nanoporous graphene membranes}},
volume = {30},
year = {2014}
}
@article{Chen1996b,
author = {Chen, F L and Kinari, Y and Sakamoto, F and Nakayama, Y and Sakamoto, Y},
file = {:Users/marc/Library/Application Support/Mendeley Desktop/Downloaded/Chen et al. - 1996 - Hydrogen permeation through palladium-based alloy membranes in mixtures of 10{\%} Methane and Ethylene in the Hydrogen.pdf:pdf},
journal = {International Journal of Hydrogen Energy},
number = {7},
pages = {555--561},
title = {{Hydrogen permeation through palladium-based alloy membranes in mixtures of 10{\%} Methane and Ethylene in the Hydrogen}},
volume = {21},
year = {1996}
}
@article{V.JayaramanM.PakalaR.Y.Lin1995a,
author = {{V. Jayaraman   M. Pakala, R.Y. Lin}, Y S Lin},
file = {:Users/marc/Library/Application Support/Mendeley Desktop/Downloaded/V. Jayaraman M. Pakala, R.Y. Lin - 1995 - Fabrication of ultrathin metallic membranes on ceramic supports by sputter deposition.pdf:pdf},
journal = {Journal of Membrane Science},
pages = {89--100},
title = {{Fabrication of ultrathin metallic membranes on ceramic supports by sputter deposition}},
volume = {99},
year = {1995}
}
@article{PachecoTanaka2005a,
author = {{Pacheco Tanaka}, David A and {Llosa Tanco}, Margot A and Niwa, Shu-ichi and Wakui, Yoshito and Mizukami, Fujio and Namba, Takemi and Suzuki, Toshishige M},
doi = {10.1016/j.memsci.2004.06.002},
file = {:Users/marc/Library/Application Support/Mendeley Desktop/Downloaded/Pacheco Tanaka et al. - 2005 - Preparation of palladium and silver alloy membrane on a porous $\alpha$-alumina tube via simultaneous electrole.pdf:pdf},
isbn = {03767388},
journal = {Journal of Membrane Science},
number = {1-2},
pages = {21--27},
title = {{Preparation of palladium and silver alloy membrane on a porous $\alpha$-alumina tube via simultaneous electroless plating}},
volume = {247},
year = {2005}
}
@article{Dunbar2012b,
author = {Dunbar, Zachary W and Chu, Deryn},
doi = {10.1016/j.jpowsour.2012.05.044},
file = {:Users/marc/Library/Application Support/Mendeley Desktop/Downloaded/Dunbar, Chu - 2012 - Thin palladium membranes supported on microstructured nickel for purification of reformate gases.pdf:pdf},
isbn = {03787753},
journal = {Journal of Power Sources},
pages = {47--53},
title = {{Thin palladium membranes supported on microstructured nickel for purification of reformate gases}},
volume = {217},
year = {2012}
}
@article{Wang2006a,
author = {Wang, W P and Thomas, S and Zhang, X L and Pan, X L and Yang, W S and Xiong, G X},
doi = {10.1016/j.seppur.2006.04.007},
file = {:Users/marc/Library/Application Support/Mendeley Desktop/Downloaded/Wang et al. - 2006 - H2N2 gaseous mixture separation in dense Pd$\alpha$-Al2O3 hollow fiber membranes Experimental and simulation studies.pdf:pdf},
isbn = {13835866},
journal = {Separation and Purification Technology},
number = {1},
pages = {177--185},
title = {{H2/N2 gaseous mixture separation in dense Pd/$\alpha$-Al2O3 hollow fiber membranes: Experimental and simulation studies}},
volume = {52},
year = {2006}
}
@article{Y.S.Cheng2001a,
author = {{Y.S. Cheng}, K L Yeung},
file = {:Users/marc/Library/Application Support/Mendeley Desktop/Downloaded/Y.S. Cheng - 2001 - Effects of electroless plating chemistry on the synthesis of palladium membranes.pdf:pdf},
journal = {Journal of Membrane Science},
pages = {195--203},
title = {{Effects of electroless plating chemistry on the synthesis of palladium membranes}},
volume = {182},
year = {2001}
}
@article{Li2015b,
author = {Li, Panyuan and Wang, Zhi and Qiao, Zhihua and Liu, Yanni and Cao, Xiaochang and Li, Wen and Wang, Jixiao and Wang, Shichang},
doi = {10.1016/j.memsci.2015.08.010},
file = {:Users/marc/Library/Application Support/Mendeley Desktop/Downloaded/Li et al. - 2015 - Recent developments in membranes for efficient hydrogen purification.pdf:pdf},
isbn = {03767388},
journal = {Journal of Membrane Science},
pages = {130--168},
title = {{Recent developments in membranes for efficient hydrogen purification}},
volume = {495},
year = {2015}
}
@article{She2014a,
author = {She, Ying and Emerson, Sean C and Magdefrau, Neal J and Opalka, Susanne M and Thibaud-Erkey, Catherine and Vanderspurt, Thomas H},
doi = {10.1016/j.memsci.2013.09.025},
file = {:Users/marc/Library/Application Support/Mendeley Desktop/Downloaded/She et al. - 2014 - Hydrogen permeability of sulfur tolerant Pd–Cu alloy membranes.pdf:pdf},
isbn = {03767388},
journal = {Journal of Membrane Science},
pages = {203--211},
title = {{Hydrogen permeability of sulfur tolerant Pd–Cu alloy membranes}},
volume = {452},
year = {2014}
}
@article{Shimpo2004,
abstract = {We examined the hydrogen permeation of a melt-spun Zr60Al15Co2.5Ni7.5Cu15glassy alloy. The hydrogen flow increased with increasing temperature or difference in the square-roots of hydrogen pressures across the membrane, $\Delta$p. At higher temperature or higher hydrogen pressure difference, the hydrogen flow is proportional to $\Delta$p. It is therefore interpreted that the diffusion through the membrane is a rate-controlling factor for the hydrogen permeation in this alloy. The hydrogen permeability of the Zr60Al15Co2.5Ni7.5Cu15glassy alloy was measured to be 1.13×10-8(mol/(msPa1/2)) at 673K, which is as high as that of pure Pd metal. The present result demonstrates the possibility of future practical use of the glassy alloys as a hydrogen permeable membrane. {\textcopyright} 2003 Elsevier B.V. All rights reserved.},
author = {Shimpo, Yoichiro and ichi Yamaura, Shin and Okouchi, Hitoshi and Nishida, Motonori and Kajita, Osamu and Kimura, Hisamichi and Inoue, Akihisa},
doi = {10.1016/j.jallcom.2003.08.102},
file = {::},
isbn = {0925-8388},
issn = {09258388},
journal = {Journal of Alloys and Compounds},
keywords = {Glassy alloy,Hydrogen permeation,Hydrogen separation,Melt-spinning,Zr-based alloy},
number = {1-2},
pages = {197--200},
title = {{Hydrogen permeation characteristics of melt-spun Zr60Al15Co2.5Ni7.5Cu15glassy alloy membrane}},
volume = {372},
year = {2004}
}
@article{Musket1976b,
author = {Musket, R G},
file = {:Users/marc/Library/Application Support/Mendeley Desktop/Downloaded/Musket - 1976 - Effects of contamination on the interaction of hydrogen gas with palladium A Review.pdf:pdf},
journal = {Journal of Less Common Metals},
pages = {173--183},
title = {{Effects of contamination on the interaction of hydrogen gas with palladium: A Review}},
volume = {45},
year = {1976}
}
@article{Chen2012b,
author = {Chen, Chee and Gobina, Edward},
doi = {10.1016/s0958-2118(12)70252-5},
file = {:Users/marc/Library/Application Support/Mendeley Desktop/Downloaded/Chen, Gobina - 2012 - Trial designs of ultra-thin palladium alloy membrane purifiers for high-density hydrogen production Pt. 1.pdf:pdf},
isbn = {09582118},
journal = {Membrane Technology},
number = {12},
pages = {7--12},
title = {{Trial designs of ultra-thin palladium alloy membrane purifiers for high-density hydrogen production Pt. 1}},
volume = {2012},
year = {2012}
}
@article{Ayturk2009a,
author = {Ayturk, M Engin and Ma, Yi Hua},
doi = {10.1016/j.memsci.2008.12.062},
file = {:Users/marc/Library/Application Support/Mendeley Desktop/Downloaded/Ayturk, Ma - 2009 - Electroless Pd and Ag deposition kinetics of the composite Pd and PdAg membranes synthesized from agitated plating b.pdf:pdf},
isbn = {03767388},
journal = {Journal of Membrane Science},
number = {1-2},
pages = {233--245},
title = {{Electroless Pd and Ag deposition kinetics of the composite Pd and Pd/Ag membranes synthesized from agitated plating baths}},
volume = {330},
year = {2009}
}
@article{Chen2016a,
author = {Chen, Che-Hsuan and Huang, Yu-Rewi and Liu, Chen-Wei and Wang, Kuan-Wen},
doi = {10.1016/j.tsf.2016.04.049},
file = {:Users/marc/Library/Application Support/Mendeley Desktop/Downloaded/Chen et al. - 2016 - Preparation and modification of PdAg membranes by electroless and electroplating process for hydrogen separation.pdf:pdf},
isbn = {00406090},
journal = {Thin Solid Films},
pages = {189--194},
title = {{Preparation and modification of PdAg membranes by electroless and electroplating process for hydrogen separation}},
volume = {618},
year = {2016}
}
@article{Ozaki2003a,
author = {Ozaki, Tetsuya and Zhang, Yi and Komaki, Masao and Nishimura, Chikashi},
file = {:Users/marc/Library/Application Support/Mendeley Desktop/Downloaded/Ozaki et al. - 2003 - Preparation of Palladium-coated V and V-15Ni membranes for hydrogen purification by electroless plating technique.pdf:pdf},
journal = {International Journal of Hydrogen Energy},
pages = {297--302},
title = {{Preparation of Palladium-coated V and V-15Ni membranes for hydrogen purification by electroless plating technique}},
volume = {28},
year = {2003}
}
@article{Dolan2009a,
author = {Dolan, Michael and Dave, Narendra and Morpeth, Leigh and Donelson, Richard and Liang, Daniel and Kellam, Michael and Song, Song},
doi = {10.1016/j.memsci.2008.10.030},
file = {:Users/marc/Library/Application Support/Mendeley Desktop/Downloaded/Dolan et al. - 2009 - Ni-based amorphous alloy membranes for hydrogen separation at 400°C.pdf:pdf},
isbn = {03767388},
journal = {Journal of Membrane Science},
number = {2},
pages = {549--555},
title = {{Ni-based amorphous alloy membranes for hydrogen separation at 400°C}},
volume = {326},
year = {2009}
}
@article{Li2008c,
author = {Li, Hui and Xu, Hengyong and Li, Wenzhao},
doi = {10.1016/j.memsci.2008.06.053},
file = {:Users/marc/Library/Application Support/Mendeley Desktop/Downloaded/Li, Xu, Li - 2008 - Study of n value and $\alpha$$\beta$ palladium hydride phase transition within the ultra-thin palladium composite membrane.pdf:pdf},
isbn = {03767388},
journal = {Journal of Membrane Science},
number = {1-2},
pages = {44--49},
title = {{Study of n value and $\alpha$/$\beta$ palladium hydride phase transition within the ultra-thin palladium composite membrane}},
volume = {324},
year = {2008}
}
@article{Lundin2016a,
author = {Lundin, Sean-Thomas B and Yamaguchi, Taichiro and Wolden, Colin A and Oyama, S Ted and Way, J Douglas},
doi = {10.1016/j.memsci.2016.04.048},
file = {:Users/marc/Library/Application Support/Mendeley Desktop/Downloaded/Lundin et al. - 2016 - The role (or lack thereof) of nitrogen or ammonia adsorption-induced hydrogen flux inhibition on palladium membra.pdf:pdf},
isbn = {03767388},
journal = {Journal of Membrane Science},
pages = {65--72},
title = {{The role (or lack thereof) of nitrogen or ammonia adsorption-induced hydrogen flux inhibition on palladium membrane performance}},
volume = {514},
year = {2016}
}
@article{Rei2009a,
author = {Rei, M H},
doi = {10.1016/j.jtice.2008.12.011},
file = {:Users/marc/Library/Application Support/Mendeley Desktop/Downloaded/Rei - 2009 - A decade's study and developments of palladium membrane in Taiwan.pdf:pdf},
isbn = {18761070},
journal = {Journal of the Taiwan Institute of Chemical Engineers},
number = {3},
pages = {238--245},
title = {{A decade's study and developments of palladium membrane in Taiwan}},
volume = {40},
year = {2009}
}
@misc{Buxbaum2008a,
address = {United States},
author = {Buxbaum, Robert E},
editor = {Patent, United States},
file = {:Users/marc/Library/Application Support/Mendeley Desktop/Downloaded/Buxbaum - 2008 - Space Group CP2 Alloys For The Use And Separation Of Hydrogen.pdf:pdf},
title = {{Space Group CP2 Alloys For The Use And Separation Of Hydrogen}},
volume = {US 7,323,0},
year = {2008}
}
@article{Seshimo2009a,
author = {Seshimo, Masahiro and Hirai, Takayuki and Rahman, Md Mizanur and Ozawa, Minoru and Sone, Masato and Sakurai, Makoto and Higo, Yakichi and Kameyama, Hideo},
doi = {10.1016/j.memsci.2009.07.007},
file = {:Users/marc/Library/Application Support/Mendeley Desktop/Downloaded/Seshimo et al. - 2009 - Functionally graded Pd$\gamma$-alumina composite membrane fabricated by electroless plating with emulsion of supercrit.pdf:pdf},
isbn = {03767388},
journal = {Journal of Membrane Science},
number = {1-2},
pages = {321--326},
title = {{Functionally graded Pd/$\gamma$-alumina composite membrane fabricated by electroless plating with emulsion of supercritical CO2}},
volume = {342},
year = {2009}
}
@article{D.T.Hughes1978a,
author = {{D.T. Hughes}, I R Harris},
chapter = {9},
file = {:Users/marc/Library/Application Support/Mendeley Desktop/Downloaded/D.T. Hughes - 1978 - A comparative study of hydrogen permeabilities and solubilities in some palladium solid solution alloys.pdf:pdf},
journal = {Journal of Less Common Metals},
pages = {9--21},
title = {{A comparative study of hydrogen permeabilities and solubilities in some palladium solid solution alloys}},
volume = {61},
year = {1978}
}
@incollection{Basile2011a,
author = {Basile, A and Iulianelli, A and Longo, T and Liguori, S and {De Falco}, Marcello},
booktitle = {Membrane Reactors for Hydrogen Production Processes},
doi = {10.1007/978-0-85729-151-6_2},
editor = {{De Falco  L.; laquaniello, G.}, Marcello; Marrelli},
file = {:Users/marc/Library/Application Support/Mendeley Desktop/Downloaded/Basile et al. - 2011 - Pd-based Selective Membrane State-of-the-Art.pdf:pdf},
pages = {21--55},
title = {{Pd-based Selective Membrane State-of-the-Art}},
year = {2011}
}
@article{Li2000b,
author = {Li, A and Liang, W and Highes, R},
file = {:Users/marc/Library/Application Support/Mendeley Desktop/Downloaded/Li, Liang, Highes - 2000 - The effect of carbon monoxide and steam on the hydrogen permeability of a Pdstainless steel membrane.pdf:pdf},
journal = {Journal of Membrane Science},
pages = {135--141},
title = {{The effect of carbon monoxide and steam on the hydrogen permeability of a Pd/stainless steel membrane}},
volume = {165},
year = {2000}
}
@article{Hara2002a,
author = {Hara, S and Hatakeyama, N and Itoh, N and Kimura, H -M. and Inoue, A},
file = {:Users/marc/Library/Application Support/Mendeley Desktop/Downloaded/Hara et al. - 2002 - Hydrogen permeation through palladium-coated amorphous Zr-M-Ni (M=Ti, Hf) alloy membranes.pdf:pdf},
journal = {Desalination},
pages = {115--120},
title = {{Hydrogen permeation through palladium-coated amorphous Zr-M-Ni (M=]Ti, Hf) alloy membranes}},
volume = {1444},
year = {2002}
}
@article{Peters2016c,
author = {Peters, T A and Stange, M and Veenstra, P and Nijmeijer, A and Bredesen, R},
doi = {10.1016/j.memsci.2015.10.031},
file = {:Users/marc/Library/Application Support/Mendeley Desktop/Downloaded/Peters et al. - 2016 - The performance of Pd–Ag alloy membrane films under exposure to trace amounts of H2S.pdf:pdf},
isbn = {03767388},
journal = {Journal of Membrane Science},
pages = {105--115},
title = {{The performance of Pd–Ag alloy membrane films under exposure to trace amounts of H2S}},
volume = {499},
year = {2016}
}
@article{LI2000a,
author = {LI, Anwu and Liang, Weiqiang and Hughes, Ronald},
file = {:Users/marc/Library/Application Support/Mendeley Desktop/Downloaded/LI, Liang, Hughes - 2000 - Fabrication of dense palladium composite membranes for hydrogen separation.pdf:pdf},
journal = {Catalysis today},
pages = {45--51},
title = {{Fabrication of dense palladium composite membranes for hydrogen separation}},
volume = {56},
year = {2000}
}
@phdthesis{Gil2015a,
address = {London},
author = {Gil, Ana Maria Gouveia},
booktitle = {Department of Chemical Engineering},
publisher = {Imperial College London},
title = {{Catalytic Hollow Fibre Membrane Reactors for H2 Production}},
volume = {PhD in Che},
year = {2015}
}
@article{Y.S.Cheng1999a,
author = {{Y.S. Cheng}, K L Yeung},
file = {:Users/marc/Library/Application Support/Mendeley Desktop/Downloaded/Y.S. Cheng - 1999 - Palladium-Silver composite membranes by electroless plating technique.pdf:pdf},
journal = {Journal of Membrane Science},
pages = {127--141},
title = {{Palladium-Silver composite membranes by electroless plating technique}},
volume = {158},
year = {1999}
}
@article{Lang1985a,
abstract = {The electrostatic interaction between two adsorbates, and, in particular, between an adsorbed atom and an adsorbed or adsorbing molecule is studied. Based on self-consistent calculations of the electrostatic potential around a series of atoms outside a jellium surface, it is shown that a simple electrostatic interaction can explain a large number of experimental observations concerning the influence of pre-adsorbed atoms on the adsorption rate, stability and adsorption configuration of simple molecules on metal surfaces. The role of pre-adsorbed alkalis as promoters and of electronegative atoms like P, S, Cl and O as poisons for the adsorption of electron acceptor molecules like H2, O2, N2 and CO is discussed, as well as the relative magnitude of the influence of the alkalis and the electronegative atoms. The peculiar effects that pre-adsorbed atoms have on molecules like H2O and NH3 are ascribed to the large intra-molecular electron transfer in these molecules. {\textcopyright} 1985.},
author = {Lang, N. D. and Holloway, S. and N{\o}rskov, J. K.},
doi = {10.1016/0039-6028(85)90208-0},
file = {:Users/marc/Library/Application Support/Mendeley Desktop/Downloaded/Lang, Holloway, N{\o}rskov - 1985 - Electrostatic adsorbate-adsorbate interactions The poisoning and promotion of the molecular adsorption.pdf:pdf},
isbn = {00396028 (ISSN)},
issn = {00396028},
journal = {Surface Science},
number = {1},
pages = {24--38},
title = {{Electrostatic adsorbate-adsorbate interactions: The poisoning and promotion of the molecular adsorption reaction}},
volume = {150},
year = {1985}
}
@article{Cheng2002a,
author = {Cheng, Y S and Pena, M A and Fierro, J L and Hui, D C W and Yeung, K L},
file = {:Users/marc/Library/Application Support/Mendeley Desktop/Downloaded/Cheng et al. - 2002 - Performance of alumina, zeolite, palladium, Pd–Ag alloy membranes for hydrogen separation from Towngas mixture.pdf:pdf},
journal = {Journal of Membrane Science},
pages = {329--340},
title = {{Performance of alumina, zeolite, palladium, Pd–Ag alloy membranes for hydrogen separation from Towngas mixture}},
volume = {204},
year = {2002}
}
@article{Ho2016a,
author = {Ho, W S Winston and Li, Kang},
doi = {10.1016/j.coche.2016.05.001},
file = {:Users/marc/Library/Application Support/Mendeley Desktop/Downloaded/Ho, Li - 2016 - Editorial overview Separation engineering Recent advances in separation science and technology.pdf:pdf},
isbn = {22113398},
journal = {Current Opinion in Chemical Engineering},
pages = {vii--xi},
title = {{Editorial overview: Separation engineering: Recent advances in separation science and technology}},
volume = {12},
year = {2016}
}
@article{Peng2009a,
author = {Peng, Lixia and Rao, Yongchu and Luo, Lizhu and Chen, Chang'An},
doi = {10.1016/j.jallcom.2009.06.158},
file = {:Users/marc/Library/Application Support/Mendeley Desktop/Downloaded/Peng et al. - 2009 - The poisoning of Pd–Y alloy membranes by carbon monoxide.pdf:pdf},
isbn = {09258388},
journal = {Journal of Alloys and Compounds},
number = {1-2},
pages = {74--77},
title = {{The poisoning of Pd–Y alloy membranes by carbon monoxide}},
volume = {486},
year = {2009}
}
@article{Hara2003,
abstract = {Amorphous-alloy membranes of a series of Zr36-xHfxNi64(0≤x≤36) were successfully prepared by the rapid-quenching method using a single roller; their amorphous structure and thermal behavior were characterized by XRD and DSC. All of the membranes, coated by palladium to provide activity to hydrogen dissociation and recombination, were sufficiently robust in a hydrogen atmosphere and showed stable permeability only to hydrogen at least in the range of 473-573K. On the other hand, over 573K, the permeation rate slowly decreased over time. Permeability was found to decrease with Hf substitute amount for Zr mainly due to increase in activation energy for permeation. {\textcopyright} 2002 Elsevier Science B.V. All rights reserved.},
author = {Hara, S. and Hatakeyama, N. and Itoh, N. and Kimura, H. M. and Inoue, A.},
doi = {10.1016/S0376-7388(02)00416-7},
file = {::},
isbn = {0376-7388},
issn = {03767388},
journal = {Journal of Membrane Science},
keywords = {Amorphous-alloys,Hydrogen diffusion,Hydrogen embrittlement,Hydrogen permeation,Zr-Hf-Ni alloys},
number = {1},
pages = {149--156},
pmid = {9979755},
title = {{Hydrogen permeation through amorphous-Zr36-xHfxNi64-alloy membranes}},
volume = {211},
year = {2003}
}
@article{Gielens2006a,
author = {Gielens, F C and Knibbeler, R J J and Duysinx, P F J and Tong, H D and Vorstman, M A G and Keurentjes, J T F},
doi = {10.1016/j.memsci.2005.12.002},
file = {:Users/marc/Library/Application Support/Mendeley Desktop/Downloaded/Gielens et al. - 2006 - Influence of steam and carbon dioxide on the hydrogen flux through thin PdAg and Pd membranes.pdf:pdf},
isbn = {03767388},
journal = {Journal of Membrane Science},
number = {1-2},
pages = {176--185},
title = {{Influence of steam and carbon dioxide on the hydrogen flux through thin Pd/Ag and Pd membranes}},
volume = {279},
year = {2006}
}
@article{She2014b,
author = {She, Ying and Emerson, Sean C and Magdefrau, Neal J and Opalka, Susanne M and Thibaud-Erkey, Catherine and Vanderspurt, Thomas H},
doi = {10.1016/j.memsci.2013.09.025},
file = {:Users/marc/Library/Application Support/Mendeley Desktop/Downloaded/She et al. - 2014 - Hydrogen permeability of sulfur tolerant Pd–Cu alloy membranes.pdf:pdf},
isbn = {03767388},
journal = {Journal of Membrane Science},
pages = {203--211},
title = {{Hydrogen permeability of sulfur tolerant Pd–Cu alloy membranes}},
volume = {452},
year = {2014}
}
@article{Pomerantz2011a,
author = {Pomerantz, Natalie and Ma, Yi Hua},
doi = {10.1016/j.memsci.2010.12.045},
file = {:Users/marc/Library/Application Support/Mendeley Desktop/Downloaded/Pomerantz, Ma - 2011 - Novel method for producing high H2 permeability Pd membranes with a thin layer of the sulfur tolerant PdCu fcc ph.pdf:pdf},
isbn = {03767388},
journal = {Journal of Membrane Science},
number = {1-2},
pages = {97--108},
title = {{Novel method for producing high H2 permeability Pd membranes with a thin layer of the sulfur tolerant Pd/Cu fcc phase}},
volume = {370},
year = {2011}
}
@article{Yamaura2001,
abstract = {We prepared the melt-spun (Ni0.6Nb0.4)100−xZrx (x=0 to 40 at{\%}) and other amorphous alloy membranes and examined the permeation of hydrogen through those alloy membranes. The interatomic spacing in the Ni–Nb–Zr amorphous structure increased with increasing Zr content. The crystallization temperature of the Ni–Nb–Zr amorphous alloys decreased with increasing Zr content. The hydrogen flow increased with an increase of the temperature or the difference in the square-roots of hydrogen pressures across the membrane, $\Delta$$\backslash$$\backslash$sqrtp. At relatively higher temperature up to 673 K or at relatively higher hydrogen pressure difference, $\Delta$$\backslash$$\backslash$sqrtp up to 550 Pa1⁄2, the hydrogen flow was more strictly proportional to $\Delta$$\backslash$$\backslash$sqrtp. This indicates that the diffusion of hydrogen through the membrane is a rate-controlling factor for hydrogen permeation. The permeability of the Ni–Nb–Zr amorphous alloys was strongly dependent on alloy compositions and increased with increasing Zr content. However, it was difficult to investigate the hydrogen permeability of the (Ni0.6Nb0.4)60Zr40 amorphous alloy in this work due to the embrittlement during the measurement. The maximum hydrogen permeability was 1.3×10−8 (mol{\textperiodcentered}m−1{\textperiodcentered}s−1{\textperiodcentered}Pa−1⁄2) at 673 K for the (Ni0.6Nb0.4)70Zr30 amorphous alloy. It is noticed that the hydrogen permeability of the (Ni0.6Nb0.4)70Zr30 amorphous alloy is higher than that of pure Pd metal. These permeation characteristics indicate the possibility of future practical use of the melt-spun amorphous alloys as a hydrogen permeable membrane.},
author = {Yamaura, Shin-ichi and Shimpo, Yoichiro (Fukuda Metal Foil {\&} Powder Co. Ltd. ) and Okouchi, Hitoshi (Fukuda Metal Foil {\&} Powder Co. Ltd. ) and Nishida, Motonori (Fukuda Metal Foil {\&} Powder Co. Ltd. ) and Kajita, Osamu (Fukuda Metal Foil {\&} Powder Co. Ltd. ) and Kimura, Hisamichi and Inoue, Akihisa},
doi = {10.2320/matertrans.42.1885},
file = {:Users/marc/Library/Application Support/Mendeley Desktop/Downloaded/Yamaura et al. - 2001 - Hydrogen Permeation Characteristics of Melt-Spun Ni-Nb-Zr Amorphous Alloy Membranes.pdf:pdf},
issn = {1345-9678},
journal = {Materials Transactions},
keywords = {amorphous alloy,hydrogen permeation,hydrogen separation,melt-spinning},
number = {9},
pages = {1885--1890},
title = {{Hydrogen Permeation Characteristics of Melt-Spun Ni-Nb-Zr Amorphous Alloy Membranes}},
url = {http://ci.nii.ac.jp/naid/130004451634/},
volume = {42},
year = {2001}
}
@article{Peters2008a,
author = {Peters, T A and Stange, M and Klette, H and Bredesen, R},
doi = {10.1016/j.memsci.2007.08.056},
file = {:Users/marc/Library/Application Support/Mendeley Desktop/Downloaded/Peters et al. - 2008 - High pressure performance of thin Pd–23{\%}Agstainless steel composite membranes in water gas shift gas mixtures i.pdf:pdf},
isbn = {03767388},
journal = {Journal of Membrane Science},
number = {1-2},
pages = {119--127},
title = {{High pressure performance of thin Pd–23{\%}Ag/stainless steel composite membranes in water gas shift gas mixtures; influence of dilution, mass transfer and surface effects on the hydrogen flux}},
volume = {316},
year = {2008}
}
@article{Wang2004a,
author = {Wang, D and Flanagan, Ted B and Shanahan, Kirk L},
doi = {10.1016/j.jallcom.2003.09.150},
file = {:Users/marc/Library/Application Support/Mendeley Desktop/Downloaded/Wang, Flanagan, Shanahan - 2004 - Permeation of hydrogen through pre-oxidized Pd membranes in the presence and absence of CO.pdf:pdf},
isbn = {09258388},
journal = {Journal of Alloys and Compounds},
number = {1-2},
pages = {158--164},
title = {{Permeation of hydrogen through pre-oxidized Pd membranes in the presence and absence of CO}},
volume = {372},
year = {2004}
}
@article{Yamaura2004,
abstract = {The (Ni/sub 0.6/Nb/sub 0.4/)/sub 45/Zr/sub 50/X/sub 5 /(X=Al, Co, Cu, P, Pd, Si, Sn, Ta or Ti) alloy ribbons were produced by the melt-spinning technique. All ribbon specimens were confirmed to have a single amorphous phase by XRD analysis. The crystallization temperature of the melt-spun (Ni/sub 0.6/Nb/sub 0.4 /)/sub 45/Zr/sub 50/X/sub 5/(X=Al, P, Pd, Si or Sn) amorphous alloys are higher than that of the (Ni/sub 0.6/Nb/sub 0.4 /)/sub 50/Zr/sub 50/ amorphous alloy (727 K). Although the hydrogen permeability of the (Ni/sub 0.6/Nb/sub 0.4/)/sub 45/Zr/sub 50/X/sub 5 /(X=Si, Sn, Ta or Ti) amorphous alloys could not be measured due to severe embrittlement during the permeation test, the (Ni/sub 0.6/Nb /sub 0.4/)/sub 45/Zr/sub 50/X/sub 5/(X=Al, Co, Cu, P or Pd) amorphous alloys had high ductility which was enough to measure the permeability. The hydrogen permeabilities of the (Ni/sub 0.6/Nb /sub 0.4/)/sub 45/Zr/sub 50/Co/sub 5/ and the (Ni /sub 0.6/Nb/sub 0.4/)/sub 45/Zr/sub 50/Cu/sub 5 / amorphous alloys were 2.46*10/sup -8/ and 2.34*10/sup -8/[mol middot m/sup -1/ middot s/sup -1 / middot Pa/sup -1/2/] at 673 K, respectively. The (Ni/sub 0.6 /Nb/sub 0.4/)/sub 45/Zr/sub 50/P/sub 5/ amorphous alloy possesses the lowest permeability of 1.36*10/sup -8/(mol middot m/sup -1/ middot s/sup -1 / middot Pa-1/2] at 673 K among the alloys where the permeability was measured. The reduction of the permeability in the (Ni/sub 0.6/Nb/sub 0.4/)/sub 45/Zr/sub 50/P/sub 5 / amorphous alloy is thought to be due to the preferential formation of Zr-P atomic pairs which may suppress the hydrogen solubility and hydrogen diffusivity in the alloy. Since the heat of mixing for Zr-P atomic pairs is negatively larger than that for other pairs such as Ni-P and Nb-P. It is concluded that the Ni-Nb-Zr-X(X=Co or Cu) amorphous alloys have high potential to hydrogen permeable membranes.},
author = {Yamaura, Shin Ichi and Shimpo, Yoichiro and Okouchi, Hitoshi and Nishida, Motonori and Kajita, Osamu and Inoue, Akihisa},
doi = {10.2320/jinstmet.68.1039},
file = {:Users/marc/Library/Application Support/Mendeley Desktop/Downloaded/Yamaura et al. - 2004 - The effect of additional elements on hydrogen permeation properties of melt-spun Ni-Nb-Zr amorphous alloys.pdf:pdf},
isbn = {0021-4876},
issn = {00214876},
journal = {Nippon Kinzoku Gakkaishi/Journal of the Japan Institute of Metals},
keywords = {Amorphous,Hydrogen permeation,Melt-spinning,Membrane,Separation},
number = {12},
pages = {1039--1042},
title = {{The effect of additional elements on hydrogen permeation properties of melt-spun Ni-Nb-Zr amorphous alloys}},
volume = {68},
year = {2004}
}
@article{Yun2011a,
author = {Yun, Samhun and {Ted Oyama}, S},
doi = {10.1016/j.memsci.2011.03.057},
file = {:Users/marc/Library/Application Support/Mendeley Desktop/Downloaded/Yun, Ted Oyama - 2011 - Correlations in palladium membranes for hydrogen separation A review.pdf:pdf},
isbn = {03767388},
journal = {Journal of Membrane Science},
number = {1-2},
pages = {28--45},
title = {{Correlations in palladium membranes for hydrogen separation: A review}},
volume = {375},
year = {2011}
}
@article{Chen1996c,
author = {Chen, F L and Kinari, Y and Sakamoto, F and Nakayama, Y and Sakamoto, Y},
file = {:Users/marc/Library/Application Support/Mendeley Desktop/Downloaded/Chen et al. - 1996 - Hydrogen permeation through palladium-based alloy membranes in mixtures of 10{\%} Methane and Ethylene in the Hydrogen.pdf:pdf},
journal = {International Journal of Hydrogen Energy},
number = {7},
pages = {555--561},
title = {{Hydrogen permeation through palladium-based alloy membranes in mixtures of 10{\%} Methane and Ethylene in the Hydrogen}},
volume = {21},
year = {1996}
}
@article{OBrien2011a,
author = {O'Brien, Casey P and Gellman, Andrew J and Morreale, Bryan D and Miller, James B},
doi = {10.1016/j.memsci.2011.01.044},
file = {:Users/marc/Library/Application Support/Mendeley Desktop/Downloaded/O'Brien et al. - 2011 - The hydrogen permeability of Pd4S.pdf:pdf},
isbn = {03767388},
journal = {Journal of Membrane Science},
number = {1-2},
pages = {263--267},
title = {{The hydrogen permeability of Pd4S}},
volume = {371},
year = {2011}
}
